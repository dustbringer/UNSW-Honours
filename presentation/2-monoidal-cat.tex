\begin{frame}
    \frametitle{Monoidal Categories}

    A \textit{monoidal category} is a category with an associative multiplication $\tensor$ for objects and morphisms, and a unit object $\mathbbb{1}$, such that the multiplication works well with composition. \newline

    Let $\mcal{C}$ be a monoidal category.

    % \only<1->{Testing}
    % \onslide<2>{Testing}
\end{frame}

% Basic
\begin{frame}
    \frametitle{Monoidal Categories}

    \begin{center}
        $f: a \to b$ \\ \vspace{0.5em}
        \tikz[vcenter, scale=0.4]{
    \squarecoord
    %
    \path
    (bot) edge[string=black] (top);
    \node[box] at (mid) {$f$};
    \node[above] at (top) {$b$};
    \node[below] at (bot) {$a$};
}
    \end{center}

\end{frame}

\begin{frame}
    \frametitle{Monoidal Categories}

    \begin{center}
        $f: a \to b \tensor c$ \\ \vspace{0.5em}
        \tikz[vcenter, scale=0.4]{
    \squarecoord
    %
    \path
    (bot) edge[string=black] (mid)
    (mid) edge[string=black] ($(top) - (1.5,0)$)
    (mid) edge[string=black] ($(top) + (1.5,0)$);
    \node[box] at (mid) {$f$};
    \node[above] at ($(top) - (1.5,0)$) {$b$};
    \node[above] at ($(top) + (1.5,0)$) {$c$};
    \node[below] at (bot) {$a$};
}
    \end{center}
\end{frame}

% Composition
\begin{frame}
    \frametitle{Monoidal Categories}

    \begin{center}
        $f: a \to b \tensor c$ \\
        $g: b \tensor c \to a \tensor c$ \\ \vspace{0.5em}

        \only<1>{
            \tikz[vcenter, scale=0.4]{
    \squarecoord
    % f
    \path
    (bot) edge[string=black] (mid)
    (mid) edge[string=black, round] ($(top) - (1.5,0)$)
    (mid) edge[string=black, round] ($(top) + (1.5,0)$);
    \node[box] at (mid) {$f$};
    \node[above] at ($(top) - (1.5,0)$) {$b$};
    \node[above] at ($(top) + (1.5,0)$) {$c$};
    \node[below] at (bot) {$a$};
}
            \quad , \quad
            \tikz[vcenter, scale=0.4]{
    \squarecoord
    % h
    \path
    ($(bot) + (-1.5,0)$) edge[string=black] ($(bot) + (-1.5,4)$)
    ($(bot) + (1.5,0)$) edge[string=black] ($(bot) + (1.5,4)$);
    \bigbox{$g$}{(mid) + (2,-0.75)}{(mid) + (-2,+0.75)};
    \node[above] at ($(top) + (-1.5,0)$) {$a$};
    \node[above] at ($(top) + (1.5,0)$) {$c$};
    \node[below] at ($(bot) - (1.5,0)$) {$b$};
    \node[below] at ($(bot) + (1.5,0)$) {$c$};
}
        }
        \only<2->{
            \tikz[vcenter, scale=0.4]{
    \squarecoord
    % f
    \path
    (bot) edge[string=black] (mid)
    (mid) edge[string=black, round] ($(top) - (1.5,0)$)
    (mid) edge[string=black, round] ($(top) + (1.5,0)$);
    \node[box] at (mid) {$f$};
    \coordinate[label=left:$b$] () at ($(top) - (1.5,0)$);
    \coordinate[label=left:$c$] () at ($(top) + (1.5,0)$);
    \coordinate[label=below:$a$] () at (bot);
    % h
    \path
    ($(top) + (-1.5,0)$) edge[string=black] ($(top) + (-1.5,4)$)
    ($(top) + (1.5,0)$) edge[string=black] ($(top) + (1.5,4)$);
    \bigbox{$h$}{(top) + (2,2-0.75)}{(top) + (-2,2+0.75)};
    \coordinate[label=$a$] () at ($(top) + (-1.5,4)$);
    \coordinate[label=$c$] () at ($(top) + (1.5,4)$);
}
        }
        \only<3->{
            = \tikz[vcenter, scale=0.4]{
    \squarecoord
    %
    \path
    (bot) edge[string=black] (mid)
    (mid) edge[string=black] ($(top) - (1.5,0)$)
    (mid) edge[string=black] ($(top) + (1.5,0)$);
    \node[box] at (mid) {$h \circ f$};
    \coordinate[label=$a$] () at ($(top) - (1.5,0)$);
    \coordinate[label=$c$] () at ($(top) + (1.5,0)$);
    \coordinate[label=below:$a$] () at (bot);
}
        }
    \end{center}
\end{frame}


% Identity
\begin{frame}
    \frametitle{Monoidal Categories}

    \begin{center}
        $\id_a: a \to a$ \\ \vspace{0.5em}
        \tikz[vcenter, scale=0.4]{
    \squarecoord
    %
    \path
    (bot) edge[string=black] (mid);
    \coordinate[label=$a$] () at (mid);
    \coordinate[label=below:$a$] () at (bot);
}
    \end{center}
\end{frame}

% Tensor
\begin{frame}
    \frametitle{Monoidal Categories}

    \begin{center}
        $f: a \to b \tensor c$ \\
        $h: x \to y$ \\ \vspace{0.5em}
        \only<1>{
            \tikz[vcenter, scale=0.4]{
    \squarecoord
    % f
    \path
    (bot) edge[string=black] (mid)
    (mid) edge[string=black] ($(top) - (1.5,0)$)
    (mid) edge[string=black] ($(top) + (1.5,0)$);
    \node[box] at (mid) {$f$};
    \node[above] at ($(top) - (1.5,0)$) {$b$};
    \node[above] at ($(top) + (1.5,0)$) {$c$};
    \node[below] at (bot) {$a$};
}
            \quad , \quad
            \tikz[vcenter, scale=0.4]{
    \squarecoord
    %g
    \path
    ($(bot) + (3,0)$) edge[string=black] ($(mid) + (3,0)$)
    ($(mid) + (3,0)$) edge[string=black] ($(top) + (3,0)$);
    \node[box] at ($(mid) + (3,0)$) {$h$};
    \node[above] at ($(top) + (3,0)$) {$y$};
    \node[below] at ($(bot) + (3,0)$) {$x$};
}
        }
        \only<2->{
            \tikz[vcenter, scale=0.4]{
    \squarecoord
    % f
    \path
    (bot) edge[string=black] (mid)
    (mid) edge[string=black] ($(top) - (1.5,0)$)
    (mid) edge[string=black] ($(top) + (1.5,0)$);
    \node[box] at (mid) {$f$};
    \coordinate[label=$b$] () at ($(top) - (1.5,0)$);
    \coordinate[label=$c$] () at ($(top) + (1.5,0)$);
    \coordinate[label=below:$a$] () at (bot);
    %g
    \path
    ($(bot) + (3,0)$) edge[string=black] ($(mid) + (3,0)$)
    ($(mid) + (3,0)$) edge[string=black] ($(top) + (3,0)$);
    \node[box] at ($(mid) + (3,0)$) {$g$};
    \coordinate[label=$y$] () at ($(top) + (3,0)$);
    \coordinate[label=below:$x$] () at ($(bot) + (3,0)$);
}
        }
        \only<3->{
            = \tikz[vcenter, scale=0.4]{
    \squarecoord
    %
    \path
    ($(bot) - (1,0)$) edge[string=black] (mid)
    ($(bot) + (1,0)$) edge[string=black] (mid)
    (mid) edge[string=black] ($(top) - (1.5,0)$)
    (mid) edge[string=black] (top)
    (mid) edge[string=black] ($(top) + (1.5,0)$);
    \node[box] at (mid) {$f \tensor h$};
    \node[above] at ($(top) - (1.5,0)$) {$b$};
    \node[above] at (top) {$c$};
    \node[above] at ($(top) + (1.5,0)$) {$y$};
    \node[below] at ($(bot) - (1,0)$) {$a$};
    \node[below] at ($(bot) + (1,0)$) {$x$};
}
        }
    \end{center}
\end{frame}

% Unit
\begin{frame}
    \frametitle{Monoidal Categories}

    \begin{center}
        \only<1>{
            $f_1: a \to \mathbbb{1}$ \\ \vspace{0.5em}
            \tikz[vcenter, scale=0.4]{
    \squarecoord
    \tikzfixsize{(2,0)}{(2,5)}
    %
    \path
    (bot) edge[string=black] (mid);
    \node[box] at (mid) {$f_1$};
    \coordinate[label=below:$a$] () at (bot);
}
        }
        \only<2>{
            $f_2: \mathbbb{1} \to b \tensor c$ \\ \vspace{0.5em}
            \tikz[vcenter, scale=0.4]{
    \squarecoord
    \tikzfixsize{(2,0)}{(2,4)}
    %
    \path
    (mid) edge[string=black] ($(top) - (1.5,0)$)
    (mid) edge[string=black] ($(top) + (1.5,0)$);
    \node[box] at (mid) {$f_2$};
    \node[above] at ($(top) - (1.5,0)$) {$b$};
    \node[above] at ($(top) + (1.5,0)$) {$c$};
    \node[below] at (bot) {\phantom{$a$}};
}
        }
    \end{center}
\end{frame}

% interchange law
\begin{frame}
    \frametitle{Monoidal Categories}

    \begin{center}
        \textit{Interchange Law} \\ \vspace{0.5em}
        \tikz[vcenter, scale=0.4]{
    \squarecoord
    % f
    \path
    ($(top) + (-1,0)$) edge[string=black] ($(bot) + (-1,0)$);
    \node[box] at ($(mid)+(-1,-.75)$) {$f$};
    \node[above] at ($(top) + (-1,0)$) {$b$};
    \node[below] at ($(bot) + (-1,0)$) {$a$};
    % g
    \path
    ($(top) + (1,0)$) edge[string=black] ($(bot) + (1,0)$);
    \node[box] at ($(mid)+(1,.75)$) {$g$};
    \node[above] at ($(top) + (1,0)$) {$d$};
    \node[below] at ($(bot) + (1,0)$) {$c$};
}
        = \tikz[vcenter, scale=0.4]{
    \squarecoord
    % f
    \path
    ($(top) + (-1,0)$) edge[string=black, round] ($(bot) + (-1,0)$);
    \node[box] at ($(mid)+(-1,0)$) {$f$};
    \coordinate[label=$b$] () at ($(top) + (-1,0)$);
    \coordinate[label=below:$a$] () at ($(bot) + (-1,0)$);
    % g
    \path
    ($(top) + (1,0)$) edge[string=black, round] ($(bot) + (1,0)$);
    \node[box] at ($(mid)+(1,0)$) {$g$};
    \coordinate[label=$d$] () at ($(top) + (1,0)$);
    \coordinate[label=below:$c$] () at ($(bot) + (1,0)$);
}
        = \tikz[vcenter, scale=0.4]{
    \squarecoord
    % f
    \path
    ($(top) + (-1,0)$) edge[string=black] ($(bot) + (-1,0)$);
    \node[box] at ($(mid)+(-1,.75)$) {$f$};
    \node[above] at ($(top) + (-1,0)$) {$b$};
    \node[below] at ($(bot) + (-1,0)$) {$a$};
    % g
    \path
    ($(top) + (1,0)$) edge[string=black] ($(bot) + (1,0)$);
    \node[box] at ($(mid)+(1,-.75)$) {$g$};
    \node[above] at ($(top) + (1,0)$) {$d$};
    \node[below] at ($(bot) + (1,0)$) {$c$};
}
    \end{center}
\end{frame}


% Frobenius Object %%%%%%%%%%%%%%%%%%%%%%%%%%%%%%%%%%%%%%%%%%

% \begin{frame}
%     \frametitle{Frobenius Object}

%     In a monoidal category, a \textit{Frobenius object} is an object $A$ with four maps
%     \only<1>{
%         \begin{center}
%             \tikz[vcenter, scale=0.4]{
    % \tikzfixsize{(0,0)}{(4,4)}
    % \node[below] (dom) at (-3,0) {$A \tensor A$};
    % \node[above] (cod) at (-3,4) {$A$};
    % \path[|->, >=stealth']
    % (dom) edge node[left] {$\mu$} (cod);
    %
    \path
    (0,0) edge[string=black] (1.5,2)
    (3,0) edge[string=black] (1.5,2)
    (1.5,2) edge[string=black] (1.5,4);
    \node[box] at (1.5,2) {$\mu$};
    % labels
    \node[below] at (0,0) {$A$};
    \node[below] at (3,0) {$A$};
    \node[above] at (1.5,4) {$A$};
}
%             \quad , \quad
%             \tikz[vcenter, scale=0.4]{
    \tikzfixsize{(0,0)}{(2,4)}
    %
    \path
    (1,2) edge[string=black] (1,4);
    \node[box] at (1,2) {$\eta$};
    % labels
    \node[above] at (1,4) {$A$};
    \node[below] at (1,0) {\phantom{$\mathbbb{1}$}};
}
%             \quad , \quad
%             \tikz[vcenter, scale=0.4]{
    % \tikzfixsize{(0,0)}{(4,4)}
    %
    \path
    (0,4) edge[string=black] (1.5,2)
    (3,4) edge[string=black] (1.5,2)
    (1.5,2) edge[string=black] (1.5,0);
    \node[box] at (1.5,2) {$\delta$};
    % labels
    \node[below] at (1.5,0) {$A$};
    \node[above] at (0,4) {$A$};
    \node[above] at (3,4) {$A$};
}
%             \quad , \quad
%             \tikz[vcenter, scale=0.4]{
    \tikzfixsize{(0,0)}{(2,4)}
    %
    \path
    (1,2) edge[string=black] (1,0);
    \node[box] at (1,2) {$\epsilon$};
    % labels
    \node[below] at (1,0) {$A$};
    \node[above] at (1,4) {\phantom{$\mathbbb{1}$}};
}.
%         \end{center}
%     }
%     \only<2>{
%         \begin{center}
%             \tikz[vcenter, scale=0.4]{
    % \tikzfixsize{(0,0)}{(4,4)}
    % 
    \path
    (0,0) edge[string=black] (2,2)
    (4,0) edge[string=black] (2,2)
    (2,2) edge[string=black] (2,4);
}
%             \quad , \quad
%             \tikz[vcenter, scale=0.4]{
    \tikzfixsize{(0,0)}{(2,4)}
    %
    \path
    (1,2) edge[string=black] (1,4);
    \node[enddot=black] at (1,2) {};
}
%             \quad , \quad
%             \tikz[vcenter, scale=0.4]{
    % \tikzfixsize{(0,0)}{(4,4)}
    %
    \path
    (0,4) edge[string=black] (2,2)
    (4,4) edge[string=black] (2,2)
    (2,2) edge[string=black] (2,0);
}
%             \quad , \quad
%             \tikz[vcenter, scale=0.4]{
    \tikzfixsize{(0,0)}{(2,4)}
    %
    \path
    (1,2) edge[string=black] (1,0);
    \node[enddot=black] at (1,2) {};
}.
%         \end{center}
%     }
%     satisfying the relations...
% \end{frame}

% % associativity
% \begin{frame}
%     \frametitle{Frobenius Object}

%     \begin{align*}
%         \tikz[vcenter, scale=0.3]{
    \tikzfixsize{(0,0)}{(6+.5,8)}
    %
    \path
    (0,0) edge[string=black] (2,2)
    (4,0) edge[string=black] (2,2)
    (2,2) edge[string=black] (2,4);
    \path
    (2,4) edge[string=black, round] (4,6)
    (6,4) edge[string=black, round] (4,6)
    (4,6) edge[string=black] (4,8);
    \path (6,0) edge[string=black] (6,4);
}
%         = & \tikz[vcenter, scale=0.3]{
    \tikzfixsize{(0-.5,0)}{(6,8)}
    %
    \path (0,0) edge[string=black] (0,4);
    \path
    (6,0) edge[string=black] (4,2)
    (2,0) edge[string=black] (4,2)
    (4,2) edge[string=black] (4,4);
    \path
    (4,4) edge[string=black, round] (2,6)
    (0,4) edge[string=black, round] (2,6)
    (2,6) edge[string=black] (2,8);
}   \\
%         \onslide<2>{
%             \tikz[vcenter, scale=0.3]{
    \tikzfixsize{(0,0)}{(6+.5,8)}
    %
    \path
    (0,8) edge[string=black] (2,6)
    (4,8) edge[string=black] (2,6)
    (2,6) edge[string=black] (2,4);
    \path
    (2,4) edge[string=black, round] (4,2)
    (6,4) edge[string=black, round] (4,2)
    (4,2) edge[string=black] (4,0);
    \path (6,8) edge[string=black] (6,4);
}
%         = & \tikz[vcenter, scale=0.3]{
    \tikzfixsize{(0-.5,0)}{(6,8)}
    %
    \path (0,8) edge[string=black] (0,4);
    \path
    (6,8) edge[string=black] (4,6)
    (2,8) edge[string=black] (4,6)
    (4,6) edge[string=black] (4,4);
    \path
    (4,4) edge[string=black, round] (2,2)
    (0,4) edge[string=black, round] (2,2)
    (2,2) edge[string=black] (2,0);
}
%         }
%     \end{align*}
% \end{frame}

% % unit
% \begin{frame}
%     \frametitle{Frobenius Object}

%     \begin{align*}
%         \tikz[vcenter, scale=0.3]{
    \tikzfixsize{(0-.5,0)}{(4+.5,8)}
    %
    \path (4,0) edge[string=black] (4,4);
    \path
    (0,2) edge[string=black] (0,4);
    \node[enddot=black] at (0,2) {};
    \path
    (0,4) edge[string=black, round] (2,6)
    (4,4) edge[string=black, round] (2,6)
    (2,6) edge[string=black] (2,8);
}
%         = & \tikz[vcenter, scale=0.3]{
    \tikzfixsize{(0,0)}{(2,8)}
    %
    \path
    (1,0) edge[string=black] (1,8);
}
%         = \tikz[vcenter, scale=0.3]{
    \tikzfixsize{(0-.5,0)}{(4,8)}
    %
    \path (0,0) edge[string=black] (0,4);
    \path
    (4,2) edge[string=black] (4,4);
    \node[enddot=black] at (4,2) {};
    \path
    (0,4) edge[string=black, round] (2,6)
    (4,4) edge[string=black, round] (2,6)
    (2,6) edge[string=black] (2,8);
}     \\
%         \onslide<2>{
%             \tikz[vcenter, scale=0.3]{
    \tikzfixsize{(0-.5,0)}{(4+.5,8)}
    %
    \path (4,8) edge[string=black] (4,4);
    \path
    (0,6) edge[string=black] (0,4);
    \node[enddot=black] at (0,6) {};
    \path
    (0,4) edge[string=black, round] (2,2)
    (4,4) edge[string=black, round] (2,2)
    (2,2) edge[string=black] (2,0);
}
%         = & \tikz[vcenter, scale=0.3]{
    \tikzfixsize{(0,0)}{(2,8)}
    %
    \path
    (1,0) edge[string=black] (1,8);
}
%             = \tikz[vcenter, scale=0.3]{
    \tikzfixsize{(0-.5,0)}{(4+.5,8)}
    %
    \path (0,8) edge[string=black] (0,4);
    \path
    (4,6) edge[string=black] (4,4);
    \node[enddot=black] at (4,6) {};
    \path
    (0,4) edge[string=black, round] (2,2)
    (4,4) edge[string=black, round] (2,2)
    (2,2) edge[string=black] (2,0);
}
%         }
%     \end{align*}
% \end{frame}

% % frob relation
% \begin{frame}
%     \frametitle{Frobenius Object}

%     \begin{center}
%         \tikz[vcenter, scale=0.35]{
    \tikzfixsize{(0,0)}{(6,6)}
    %
    \path (6,0) edge[string=black] (6,2);
    \path
    (0,4) edge[string=black, round] (2,2)
    (4,4) edge[string=black] (2,2)
    (2,2) edge[string=black] (2,0);
    %
    \path (0,4) edge[string=black] (0,6);
    \path
    % (2,2) edge[string=black] (4,4)
    (6,2) edge[string=black, round] (4,4)
    (4,4) edge[string=black] (4,6);
}
%         = \tikz[vcenter, scale=0.35]{
    \tikzfixsize{(0-.5,0)}{(4+.5,6)}
    %
    \path
    (0,0) edge[string=black] (2,2)
    (4,0) edge[string=black] (2,2)
    (2,2) edge[string=black] (2,4)
    (0,6) edge[string=black] (2,4)
    (4,6) edge[string=black] (2,4)
    ;
}
%         = \tikz[vcenter, scale=0.35]{
    \tikzfixsize{(0,0)}{(6,6)}
    %
    \path (6,6) edge[string=black] (6,4);
    \path
    (0,2) edge[string=black, round] (2,4)
    (4,2) edge[string=black] (2,4)
    (2,4) edge[string=black] (2,6);
    %
    \path (0,2) edge[string=black] (0,0);
    \path
    % (2,4) edge[string=black] (4,2)
    (6,4) edge[string=black, round] (4,2)
    (4,2) edge[string=black] (4,0);
}
%     \end{center}
% \end{frame}



% % isotopy

% % caps and cups
% \begin{frame}
%     \frametitle{Frobenius Object}

%     \begin{center}
%         \tikz[vcenter, scale=0.4]{
    % \tikzfixsize{(0,0)}{(4,4)}
    %
    \diagcap[black]{0,0}{2,2+.25}{4,0};
}
%         $\coloneqq$ \tikz[vcenter, scale=0.4]{
    % \tikzfixsize{(0,0)}{(4,4)}
    % 
    \path
    (0,0) edge[string=black] (2,2)
    (4,0) edge[string=black] (2,2)
    (2,2) edge[string=black] (2,3);
    \node[enddot=black] at (2,3) {};
} \\ \vspace{1em}
%         \tikz[vcenter, scale=0.4]{
    % \tikzfixsize{(0,0)}{(4,4)}
    %
    \diagcup[black]{0,4}{2,2-.25}{4,4};
}
%         $\coloneqq$ \tikz[vcenter, scale=0.4]{
    % \tikzfixsize{(0,0)}{(4,4)}
    %
    \path
    (0,4) edge[string=black] (2,2)
    (4,4) edge[string=black] (2,2)
    (2,2) edge[string=black] (2,1);
    \node[enddot=black] at (2,1) {};
}
%     \end{center}
% \end{frame}

% % isotopy
% \begin{frame}
%     \frametitle{Frobenius Object}

%     \begin{gather*}
%         \only<1>{
%             \tikz[vcenter, scale=0.35]{
    \tikzfixsize{(0,0)}{(6,6)}
    %
    \path (6,0) edge[string=black] (6,2);
    \path
    (0,4) edge[string=black] (2,2)
    (4,4) edge[string=black] (2,2)
    (2,2) edge[string=black] (2,0);
    %
    \path (0,4) edge[string=black] (0,6);
    \path
    % (2,2) edge[string=black] (4,4)
    (6,2) edge[string=black] (4,4)
    (4,4) edge[string=black] (4,6);
    % 
    \path
    (0,6) edge[string=black] (0,8)
    (4,6) edge[string=black] (4,7)
    (2,0) edge[string=black] (2,-1)
    (6,0) edge[string=black] (6,-2)
    ;
    \node[enddot=black] at (2,-1) {};
    \node[enddot=black] at (4,7) {};
    %
    \draw[dashed, black!50] (-.5,0) -- (6.5,0);
    \draw[dashed, black!50] (-.5,6) -- (6.5,6);
}
%             = \tikz[vcenter, scale=0.35]{
    \tikzfixsize{(0-.5,0)}{(4+.5,6)}
    %
    \path
    (0,0) edge[string=black] (2,2)
    (4,0) edge[string=black] (2,2)
    (2,2) edge[string=black] (2,4)
    (0,6) edge[string=black] (2,4)
    (4,6) edge[string=black] (2,4)
    ;
    \path
    (0,6) edge[string=black] (0,8)
    (0,0) edge[string=black] (0,-1)
    (4,6) edge[string=black] (4,7)
    (4,0) edge[string=black] (4,-2)
    ;
    \node[enddot=black] at (0,-1) {};
    \node[enddot=black] at (4,7) {};
    % 
    \draw[dashed, black!50] (-.5,0) -- (4.5,0);
    \draw[dashed, black!50] (-.5,6) -- (4.5,6);
}
%         }
%         \only<2>{
%             \tikz[vcenter, scale=0.4]{
    \squarecoord
    %
    \draw[string=black] (topl)
    to ++(0,-2)
    to[out=270,in=180,looseness=1] ++(1,-1)
    to[out=0,in=270,looseness=1] ++(1,1)
    to[out=90,in=180,looseness=1] ++(1,1)
    to[out=0,in=90,looseness=1] ++(1,-1)
    to ++(0,-2)
    ;
}
%             \quad = \tikz[vcenter, scale=0.4]{
    \squarecoord
    \tikzfixsize{(1,0)}{(3,4)}
    %
    \path (top) edge[string=black] (bot);
}
%         }
%     \end{gather*}
% \end{frame}

\begin{frame}
    \frametitle{Isotopy}

    \begin{gather*}
        \tikz[vcenter, scale=0.4]{
    \squarecoord
    %
    \draw[string=black] (topl)
    to ++(0,-2)
    to[out=270,in=180,looseness=1] ++(1,-1)
    to[out=0,in=270,looseness=1] ++(1,1)
    to[out=90,in=180,looseness=1] ++(1,1)
    to[out=0,in=90,looseness=1] ++(1,-1)
    to ++(0,-2)
    ;
}
        \quad = \tikz[vcenter, scale=0.4]{
    \squarecoord
    \tikzfixsize{(1,0)}{(3,4)}
    %
    \path (top) edge[string=black] (bot);
}
        = \quad \tikz[vcenter, scale=0.4]{
    \squarecoord
    %
    \draw[string=black] (topr)
    to ++(0,-2)
    to[out=270,in=0,looseness=1] ++(-1,-1)
    to[out=180,in=270,looseness=1] ++(-1,1)
    to[out=90,in=0,looseness=1] ++(-1,1)
    to[out=180,in=90,looseness=1] ++(-1,-1)
    to ++(0,-2)
    ;
}
    \end{gather*}
\end{frame}

% \begin{frame}
%     \frametitle{Isotopy}

%     \begin{gather*}
%         \tikz[vcenter, scale=0.4]{
    \squarecoord
    %
    \path
    (2,0) edge[string=black] (2,2)
    (0,2) edge[string=black] (0,1.5)
    ;
    \diagcap[black]{0,2}{1,3}{2,2};
    \node[enddot=black] at (0,1.5) {};
}
%         \quad = \quad
%         \tikz[vcenter, scale=0.4]{
    \tikzfixsize{(2,0)}{(2,3)}
    \squarecoord
    %
    \path
    (2,0) edge[string=black] (2,2)
    ;
    \node[enddot=black] at (2,2) {};
}
%         \quad = \quad
%         \tikz[vcenter, scale=0.4]{
    \squarecoord
    %
    \path
    (2,0) edge[string=black] (2,2)
    (4,2) edge[string=black] (4,1.5)
    ;
    \diagcap[black]{2,2}{3,3}{4,2};
    \node[enddot=black] at (4,1.5) {};
} \\
%         \tikz[vcenter, scale=0.4]{
    \squarecoord
    %
    \path
    (2,4) edge[string=black] (2,2)
    (0,2) edge[string=black] (0,2.5)
    ;
    \diagcup[black]{0,2}{1,1}{2,2};
    \node[enddot=black] at (0,2.5) {};
}
%         \quad = \quad
%         \tikz[vcenter, scale=0.4]{
    \tikzfixsize{(2,1)}{(2,4)}
    \squarecoord
    %
    \path
    (2,4) edge[string=black] (2,2)
    ;
    \node[enddot=black] at (2,2) {};
}
%         \quad = \quad
%         \tikz[vcenter, scale=0.4]{
    \squarecoord
    %
    \path
    (2,4) edge[string=black] (2,2)
    (4,2) edge[string=black] (4,2.5)
    ;
    \diagcup[black]{2,2}{3,1}{4,2};
    \node[enddot=black] at (4,2.5) {};
} \\
%         \tikz[vcenter, scale=0.4]{
    % \tikzfixsize{(0,0)}{(6+.5,8)}
    %
    \path
    (2,0) edge[string=black] (2,2)
    (2,2) edge[string=black] (1.5,2.5)
    (2,2) edge[string=black] (4,4)
    ;
    \draw[string=black] (1.5,2.5)
    to[out=135,in=90,looseness=1.4] (0,2)
    ;
    \path (0,0) edge[string=black] (0,2) ;
}
%         \quad = \quad
%         \tikz[vcenter, scale=0.4]{
    % \tikzfixsize{(0-.5,0)}{(6,8)}
    %
    \path
    (4,0) edge[string=black, round] (2,2)
    (0,0) edge[string=black, round] (2,2)
    (2,2) edge[string=black] (2,4);
}
%         \quad = \quad
%         \tikz[vcenter, scale=0.4]{
    % \tikzfixsize{(0,0)}{(6+.5,8)}
    %
    \path
    (2,0) edge[string=black] (2,2)
    (2,2) edge[string=black] (2.5,2.5)
    (2,2) edge[string=black] (0,4)
    ;
    \draw[string=black] (2.5,2.5)
    to[out=45,in=90,looseness=1.4] (4,2)
    ;
    \path (4,0) edge[string=black] (4,2) ;
} \\
%         \tikz[vcenter, scale=0.4]{
    % \tikzfixsize{(0,0)}{(6+.5,8)}
    %
    \path
    (2,4) edge[string=black] (2,2)
    (2,2) edge[string=black] (1.5,1.5)
    (2,2) edge[string=black] (4,0)
    ;
    \draw[string=black] (1.5,1.5)
    to[out=-135,in=270,looseness=1.4] (0,2)
    ;
    \path (0,4) edge[string=black] (0,2) ;
}
%         \quad = \quad
%         \tikz[vcenter, scale=0.4]{
    % \tikzfixsize{(0-.5,0)}{(6,8)}
    %
    \path
    (4,4) edge[string=black, round] (2,2)
    (0,4) edge[string=black, round] (2,2)
    (2,2) edge[string=black] (2,0);
}
%         \quad = \quad
%         \tikz[vcenter, scale=0.4]{
    % \tikzfixsize{(0,0)}{(6+.5,8)}
    %
    \path
    (2,4) edge[string=black] (2,2)
    (2,2) edge[string=black] (2.5,1.5)
    (2,2) edge[string=black] (0,0)
    ;
    \draw[string=black] (2.5,1.5)
    to[out=-45,in=270,looseness=1.4] (4,2)
    ;
    \path (4,4) edge[string=black] (4,2) ;
} \\
%     \end{gather*}
% \end{frame}


\begin{frame}
    \frametitle{$\Z$-linear Monoidal Category}

    \begin{gather*}
        3 \tikz[vcenter, scale=0.4]{
    \squarecoord
    %
    \path
    (bot) edge[string=black] (mid)
    (mid) edge[string=black] ($(top) - (1.5,0)$)
    (mid) edge[string=black] ($(top) + (1.5,0)$);
    \node[box] at (mid) {$f$};
    \node[above] at ($(top) - (1.5,0)$) {$b$};
    \node[above] at ($(top) + (1.5,0)$) {$c$};
    \node[below] at (bot) {$a$};
} - 2 \tikz[vcenter, scale=0.4]{
    \squarecoord
    \tikzfixsize{(1,0)}{(4,4)}
    %
    \path
    (bot) edge[string=black] (mid);
    \node[box] at (mid) {$f_1$};
    \node[below] at (bot) {$a$};
    \node[above] at (top) {\phantom{$a$}};
    %
    \path
    ($(mid)+(2,0)$) edge[string=black] ($(top) + (-1.5+2,0)$)
    ($(mid)+(2,0)$) edge[string=black] ($(top) + (1.5+2,0)$);
    \node[box] at ($(mid)+(2,0)$) {$f_2$};
    \node[above] at ($(top) + (-1.5+2,0)$) {$b$};
    \node[above] at ($(top) + (1.5+2,0)$) {$c$};
    \node[below] at ($(bot)+(2,0)$) {\phantom{$a$}};
}
    \end{gather*}
\end{frame}
