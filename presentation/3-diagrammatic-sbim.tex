\begin{frame}
    \frametitle{Diagrammatic Soergel Bimodules}
    Let $\cD$ be the (diagrammatic) $\Z$-linear monoidal category with:

    Generating object $\dRed{\mathsf{I}}$.

    Generating morphisms
    \begin{gather*}
        \tikz[vcenter, scale=0.4]{
    \tikzfixsize{(0,0)}{(4,4)}
    \squarecoord
    %
    \path
    (bot) edge[string=dRed] (mid);
    \node [enddot=dRed] at (mid) {};
}
        \quad , \quad
        \tikz[vcenter, scale=0.4]{
    \tikzfixsize{(0,0)}{(4,4)}
    \squarecoord
    %
    \path
    (mid) edge[string=dRed] (top)
    (mid) edge[string=dRed] (botl)
    (mid) edge[string=dRed] (botr);
}
        \quad , \quad
        \tikz[vcenter, scale=0.4]{
    \tikzfixsize{(0,0)}{(4,4)}
    \squarecoord
    %
    \path
    (bot) edge[string=dBlue] (mid);
    \node [enddot=dBlue] at (mid) {};
}
        \quad , \quad
        \tikz[vcenter, scale=0.4]{
    \tikzfixsize{(0,0)}{(4,4)}
    \squarecoord
    %
    \path
    (mid) edge[string=dRed] (bot)
    (mid) edge[string=dRed] (topl)
    (mid) edge[string=dRed] (topr);
}
    \end{gather*}
    and local relations...
\end{frame}

\begin{frame}
    \frametitle{Relations}

    \begin{align*}
        \tikz[vcenter, scale=0.4]{
    \tikzfixsize{(1.5,0)}{(3.5,4)}
    \squarecoord
    %
    \path
    (top) edge[string=dRed] (bot)
    (mid) edge[string=dRed] ($(mid) + (1,0)$);
    \node [enddot=dRed] at ($(mid) + (1,0)$) {};
} = \tikz[vcenter, scale=0.4]{
    \tikzfixsize{(1,0)}{(3,4)}
    \squarecoord
    %
    \path
    (top) edge[string=dRed] (bot);
} = \input{tikz/3/relation1-frob-unit2.tex}
        \quad & ,\quad\quad
        \tikz[vcenter, scale=0.4]{
    \tikzfixsize{(0,0)}{(4,4)}
    \squarecoord
    %
    \path
    (topl) edge[string=red] ($(mid)+(0,.75)$)
    (topr) edge[string=red] ($(mid)+(0,.75)$)
    (botl) edge[string=red] ($(mid)+(0,-.75)$)
    (botr) edge[string=red] ($(mid)+(0,-.75)$)
    ($(mid)+(0,.75)$) edge[string=red] ($(mid)+(0,-.75)$);
} = \tikz[vcenter, scale=0.4]{
    \tikzfixsize{(0,0)}{(4,4)}
    \squarecoord
    %
    \path
    (topl) edge[string=red] (midl)
    (topr) edge[string=red] (midr)
    (botl) edge[string=red] (midl)
    (botr) edge[string=red] (midr)
    (midl) edge[string=red] (midr);
}, \\
        \tikz[vcenter, scale=0.4]{
    \tikzfixsize{(0.25,0)}{(3.25,4)}
    \squarecoord
    %
    \path
    ($(mid)+(0,1)$) edge[string=dRed] (top)
    ($(mid)-(0,1)$) edge[string=dRed] (bot);
    \draw[string=dRed] ($(mid)+(0,1)$) arc (90 : 360+90 : 1);
    % circle with radius half distance from (mid + (0,1)) to (mid - (0,1))
} = 0
        \quad & ,\quad\quad
        \input{tikz/3/relation4-left-bar.tex} = 2 \tikz[vcenter, scale=0.4]{
    \tikzfixsize{(1,0)}{(3,4)}
    \squarecoord
    %
    \path
    ($(mid)+(0,.75)$) edge[string=dRed] (top)
    ($(mid)+(0,-.75)$) edge[string=dRed] (bot);
    \node [enddot=dRed] at ($(mid)+(0,.75)$) {};
    \node [enddot=dRed] at ($(mid)+(0,-.75)$) {};
} - \tikz[vcenter, scale=0.4]{
    \tikzfixsize{(0,0)}{(4,4)}
    \squarecoord
    %
    \path
    (top) edge[string=red] (bot)
    ($(midt)+(1,0)$) edge[string=red] ($(midb)+(1,0)$);
    \node [enddot=red] at ($(midt)+(1,0)$) {};
    \node [enddot=red] at ($(midb)+(1,0)$) {};
}
    \end{align*}
\end{frame}

% Cups and caps
% \begin{frame}
%     \frametitle{Relations}

%     \begin{gather*}
%         \input{tikz/3/cap1.tex}
%         \quad \coloneqq \quad
%         \tikz[vcenter, scale=0.4]{
    % \tikzfixsize{(0,0)}{(4,4)}
    % 
    \path
    (0,0) edge[string=dRed] (2,2)
    (4,0) edge[string=dRed] (2,2)
    (2,2) edge[string=dRed] (2,3);
    \node[enddot=dRed] at (2,3) {};
}
%     \end{gather*}
%     \begin{gather*}
%         \tikz[vcenter, scale=0.4]{
    % \tikzfixsize{(0,0)}{(4,4)}
    %
    \diagcup[dRed]{0,4}{2,2-.25}{4,4};
}
%         \quad \coloneqq \quad
%         \tikz[vcenter, scale=0.4]{
    % \tikzfixsize{(0,0)}{(4,4)}
    %
    \path
    (0,4) edge[string=dRed] (2,2)
    (4,4) edge[string=dRed] (2,2)
    (2,2) edge[string=dRed] (2,1);
    \node[enddot=dRed] at (2,1) {};
}
%     \end{gather*}
% \end{frame}

\begin{frame}
    \frametitle{Example 1}

    \begin{align*}
        \only<1>{
            \input{tikz/3/relation-example1-big.tex}
        }
        \only<2->{
            \onslide<2->{
                \tikz[vcenter, scale=0.4]{
    % \tikzfixsize{(0,0)}{(4,4)}
    \squarecoord
    %
    \path
    % Top half
    (4,6) edge[string=dRed] (4,2)
    (4,4) edge[string=dRed] (1,4)
    (2,4) edge[string=dRed] (2,4.5)
    (2,4.5) edge[string=dRed] (2.5,5)
    (2,4.5) edge[string=dRed] (1.5,5)
    % Barbells
    (3.25,2) edge[string=dRed] (3.25,3)
    (1.75,1) edge[string=dRed] (1.75,2)
    ;
    % Bottom curve
    \draw[string=dRed] (4,2)
    to[out=270,in=0] (3.25,1)
    to[out=180,in=270] (2.5,2)
    to[out=90,in=0] (1.75,3)
    to[out=180,in=90] (1,2);
    \path
    (3.25,1) edge[string=dRed] (3.25,0)
    (1,2) edge[string=dRed] (1,0)
    ;
    %
    \node [enddot=dRed] at (1,4) {};
    \node [enddot=dRed] at (2.5,5) {};
    \node [enddot=dRed] at (1.5,5) {};
    \node [enddot=dRed] at (3.25,2) {};
    \node [enddot=dRed] at (3.25,3) {};
    \node [enddot=dRed] at (1.75,1) {};
    \node [enddot=dRed] at (1.75,2) {};
    %
    \wallr{6}{6}{1.5}
}
                = \tikz[vcenter, scale=0.6]{
    \tikzfixsize{(0,0)}{(4,4)}
    \squarecoord
    %
    \path
    (top) edge[string=dRed] ($(mid) + (0,.75)$)
    (2,0.75) edge[string=dRed] (2,1.75)
    (.5,0.75) edge[string=dRed] (.5,1.75)
    ;
    \draw[string=dRed] ($(bot) - (1,0)$)
    to[out=90,in=200] ($(mid) + (0,.75)$)
    to[out=-20,in=90] ($(bot) + (1,0)$)
    ;
    %
    \node [enddot=dRed] at (2,0.75) {};
    \node [enddot=dRed] at (2,1.75) {};
    \node [enddot=dRed] at (.5,0.75) {};
    \node [enddot=dRed] at (.5,1.75) {};
}
            }
            \onslide<4->{
         & = 2 \tikz[vcenter, scale=0.4]{
    \tikzfixsize{(0,0)}{(3,4)}
    \squarecoord
    %
    \path
    ($(top)-(1,0)$) edge[string=red] ($(midt)-(1,0)$)
    ($(bot)-(1,0)$) edge[string=red] ($(midb)-(1,0)$)
    (bot) edge[string=red] (midb)
    ;
    %
    \node [enddot=red] at ($(midt)-(1,0)$) {};
    \node [enddot=red] at ($(midb)-(1,0)$) {};
    \node [enddot=red] at ($(midb)$) {};
    %
    \wallr{4}{4}{1.25}
} - \tikz[vcenter, scale=0.4]{
    \tikzfixsize{(0,0)}{(4,4)}
    \squarecoord
    %
    \path
    ($(top)-(1,0)$) edge[string=dRed] ($(mid)+(-1,.75)$)
    ($(bot)-(1,0)$) edge[string=dRed] ($(mid)+(-1,-.75)$)
    ($(mid)+(1,.75)$) edge[string=dRed] ($(mid)+(1,-.75)$)
    (bot) edge[string=dRed] ($(mid)+(0,-.75)$)
    ;
    %
    \node [enddot=dRed] at ($(mid)+(-1,.75)$) {};
    \node [enddot=dRed] at ($(mid)+(-1,-.75)$) {};
    \node [enddot=dRed] at ($(mid)+(0,-.75)$) {};
    \node [enddot=dRed] at ($(mid)+(1,.75)$) {};
    \node [enddot=dRed] at ($(mid)+(1,-.75)$) {};
    %
    \wallr{5}{4}{1.25}
} \\
            }
            \only<6->{
         & = 2 \tikz[vcenter, scale=0.6]{
    \tikzfixsize{(0,0)}{(4,4)}
    \squarecoord
    %
    \path
    (top) edge[string=dRed] (bot)
    ($(mid)+(-1,.75)$) edge[string=dRed] ($(mid)+(-1,-.75)$)
    (3,0) edge[string=dRed] ($(mid)+(1,-.75)$)
    ;
    %
    \node [enddot=dRed] at ($(mid)+(-1,.75)$) {};
    \node [enddot=dRed] at ($(mid)+(1,-.75)$) {};
    \node [enddot=dRed] at ($(mid)+(-1,-.75)$) {};
} - \tikz[vcenter, scale=0.4]{
    \tikzfixsize{(0,0)}{(4,4)}
    \squarecoord
    %
    \path
    (top) edge[string=dRed] (mid)
    ($(mid)+(1.25,.75)$) edge[string=dRed] ($(mid)+(1.25,-.75)$)
    ($(mid)+(-1.25,.75)$) edge[string=dRed] ($(mid)+(-1.25,-.75)$)
    ;
    \draw[string=dRed] ($(bot) - (0.75,0)$)
    to[out=90,in=180+20,looseness=0.8] (mid)
    to[out=-20,in=90,looseness=0.8] ($(bot) + (0.75,0)$)
    ;
    %
    \node [enddot=dRed] at ($(mid)+(1.25,.75)$) {};
    \node [enddot=dRed] at ($(mid)+(1.25,-.75)$) {};
    \node [enddot=dRed] at ($(mid)+(-1.25,.75)$) {};
    \node [enddot=dRed] at ($(mid)+(-1.25,-.75)$) {};
}
            }
        }
    \end{align*}
    \only<2>{
        \vspace{-4em}
        \begin{mdframed}[style=MyFrame,nobreak=true,align=center,userdefinedwidth=12em]
            \vspace{-1em}
            \begin{gather*}
                \tikz[vcenter, scale=0.4]{
    \tikzfixsize{(1.5,0)}{(3.5,4)}
    \squarecoord
    %
    \path
    (top) edge[string=dRed] (bot)
    (mid) edge[string=dRed] ($(mid) + (1,0)$);
    \node [enddot=dRed] at ($(mid) + (1,0)$) {};
} = \tikz[vcenter, scale=0.4]{
    \tikzfixsize{(1,0)}{(3,4)}
    \squarecoord
    %
    \path
    (top) edge[string=dRed] (bot);
} = \input{tikz/3/relation1-frob-unit2.tex}
            \end{gather*}
        \end{mdframed}
    }
    \only<3-4>{
        \vspace{-4em}
        \begin{mdframed}[style=MyFrame,nobreak=true,align=center,userdefinedwidth=12em]
            \vspace{-1em}
            \begin{gather*}
                \input{tikz/3/relation4-left-bar.tex} = 2 \tikz[vcenter, scale=0.4]{
    \tikzfixsize{(1,0)}{(3,4)}
    \squarecoord
    %
    \path
    ($(mid)+(0,.75)$) edge[string=dRed] (top)
    ($(mid)+(0,-.75)$) edge[string=dRed] (bot);
    \node [enddot=dRed] at ($(mid)+(0,.75)$) {};
    \node [enddot=dRed] at ($(mid)+(0,-.75)$) {};
} - \tikz[vcenter, scale=0.4]{
    \tikzfixsize{(0,0)}{(4,4)}
    \squarecoord
    %
    \path
    (top) edge[string=red] (bot)
    ($(midt)+(1,0)$) edge[string=red] ($(midb)+(1,0)$);
    \node [enddot=red] at ($(midt)+(1,0)$) {};
    \node [enddot=red] at ($(midb)+(1,0)$) {};
}
            \end{gather*}
        \end{mdframed}
    }
    \only<5>{
        \vspace{-4em}
        \begin{mdframed}[style=MyFrame,nobreak=true,align=center,userdefinedwidth=12em]
            \vspace{-1em}
            \begin{gather*}
                \tikz[vcenter, scale=0.4]{
    \tikzfixsize{(1.5,0)}{(3.5,4)}
    \squarecoord
    %
    \path
    (top) edge[string=dRed] (bot)
    (mid) edge[string=dRed] ($(mid) + (1,0)$);
    \node [enddot=dRed] at ($(mid) + (1,0)$) {};
} = \tikz[vcenter, scale=0.4]{
    \tikzfixsize{(1,0)}{(3,4)}
    \squarecoord
    %
    \path
    (top) edge[string=dRed] (bot);
} = \input{tikz/3/relation1-frob-unit2.tex}
            \end{gather*}
        \end{mdframed}
    }
\end{frame}

\begin{frame}
    \frametitle{Example 2}

    \begin{gather*}
        \only<1>{
            \input{tikz/3/bubbles-example1.tex}
        }
        \only<2>{
            \tikz[vcenter, scale=0.4]{
    % barbell
    \path
    (2,1.6) edge[string=dRed] (2,2.4);
    \node[enddot=dRed] at (2,1.6) {};
    \node[enddot=dRed] at (2,2.4) {};
    % bubbles
    \path
    (2,3) edge[string=dRed] (2,4)
    (3.1,-.1) edge[string=dRed] (3.1,-.7)
    % (2.7,.8) edge[string=dRed] (2.5,1.12)
    ;
    \node[enddot=dRed] at (2,4) {};
    \node[enddot=dRed] at (3.1,-.7) {};
    \draw[string=dRed] (2,3) arc (90 : 360+90 : 1);
    \draw[string=dRed] (3,2) arc (100 : -163 : .7);
    \draw[string=dRed] (2.7,.74) arc (-220 : 35 : .5);
    % indicator
    \node (pivot) at (2.5,.7) {};
    \draw[black!50, dashed, thick, rotate around={50:(pivot)}]
    (pivot) rectangle ++(1.7,.7)
    ;
}
        }
        \only<3->{
            \tikz[vcenter, scale=0.4]{
    % barbell
    \path
    (2,1.6) edge[string=dRed] (2,2.4);
    \node[enddot=dRed] at (2,1.6) {};
    \node[enddot=dRed] at (2,2.4) {};
    % bubbles
    \path
    (2,3) edge[string=dRed] (2,4)
    (3.1,-.1) edge[string=dRed] (3.1,-.7)
    % (2.7,.8) edge[string=dRed] (2.5,1.12)
    ;
    \node[enddot=dRed] at (2,4) {};
    \node[enddot=dRed] at (3.1,-.7) {};
    \draw[string=dRed] (2,3) arc (90 : 360+90 : 1);
    \draw[string=dRed] (3,2) arc (100 : -163 : .7);
    \draw[string=dRed] (2.7,.74) arc (-220 : 35 : .5);
    % indicator
    \node (pivot) at (2.5,.7) {};
    \draw[black!50, dashed, thick, rotate around={50:(pivot)}]
    (pivot) rectangle ++(1.7,.7)
    ;
}
            \quad = \quad
            \only<3>{
                \tikz[vcenter, scale=0.4]{
    % barbell
    \path
    (2,1.6) edge[string=dRed] (2,2.4);
    \node[enddot=dRed] at (2,1.6) {};
    \node[enddot=dRed] at (2,2.4) {};
    % bubbles
    \path
    (2,3) edge[string=dRed] (2,3.5)
    (2.5,-.9) edge[string=dRed] (2.5,-1.4)
    % (2.7,.8) edge[string=dRed] (2.5,1.12)
    (2,3-2) edge[string=dRed] (2,3-2.4)
    ;
    \node[enddot=dRed] at (2,3.5) {};
    \node[enddot=dRed] at (2.5,-1.4) {};
    \draw[string=dRed] (2,3) arc (90 : 360+90 : 1);
    \draw[string=dRed] (2,.6) arc (90 : 360+90 : .5);
    \draw[string=dRed] (2,-.4) arc (-180 : 90 : .5);
    % indicator
    \node (pivot) at (2.1,-.9) {};
    \draw[black!50, dashed, thick, rotate around={50:(pivot)}]
    (pivot) rectangle ++(1.3,.6)
    ;
}
            }
            \only<4->{
                \tikz[vcenter, scale=0.6]{
    % barbell
    \path
    (2,1.6) edge[string=red] (2,2.4);
    \node[enddot=red] at (2,1.6) {};
    \node[enddot=red] at (2,2.4) {};
    % bubbles
    \path
    (2,3) edge[string=dRed] (2,4)
    (3.1,-.1) edge[string=dRed] (3.1,-.7)
    % (2.7,.8) edge[string=dRed] (2.5,1.12)
    ;
    \node[enddot=red] at (2,4) {};
    \node[enddot=red] at (3.1,-.7) {};
    \draw[string=dRed] (2,3) arc (90 : 360+90 : 1);
    \draw[string=dRed] (3,2) arc (100 : -163 : .7);
    \draw[string=dRed] (2.7,.74) arc (-220 : 35 : .5);
    % indicator
    \draw[black!50, dashed, very thick]
    (2,1.1) -- ++(1.1,1.2)
    -- ++(.35,-.38)
    -- ++(-1.1,-1.2)
    -- cycle;
    ;
}
            }
            \only<5>{
                \quad = \quad
                \tikz[vcenter, scale=0.6]{
    % barbell
    \path
    (2,1.8) edge[string=red] (2,2.5);
    \node[enddot=red] at (2,1.8) {};
    \node[enddot=red] at (2,2.5) {};
    % bubbles
    \path
    (2,3) edge[string=dRed] (2,4)
    (2.5,-.5) edge[string=dRed] (2.5,-1)
    % (2.7,.8) edge[string=dRed] (2.5,1.12)
    (2,3-1.62) edge[string=dRed] (2,3-2)
    ;
    \node[enddot=red] at (2,4) {};
    \node[enddot=red] at (2.5,-1) {};
    \draw[string=dRed] (2,3) arc (90 : 360+90 : .81);
    \draw[string=dRed] (2,1) arc (90 : 360+90 : .5);
    \draw[string=dRed] (2,0) arc (-180 : 90 : .5);
    % indicator
    % \draw[black!50, dashed, very thick]
    % (1.6,0) -- ++(.9,.85)
    % -- ++(.35,-.38)
    % -- ++(-.9,-.85)
    % -- cycle;
    % ;
}
            }
            \only<6->{
                \quad = \quad
                \input{tikz/3/bubbles-example6.tex}
            }
            \only<7->{
                \quad = \quad
                \tikz[vcenter, scale=0.6]{
    % barbell
    \path
    (2,1.6) edge[string=dRed] (2,2.4);
    \node[enddot=dRed] at (2,1.6) {};
    \node[enddot=dRed] at (2,2.4) {};
    % bubbles
    \path
    (2,3) edge[string=dRed] (2,3.5)
    (2,1) edge[string=dRed] (2,.75)
    (2,-.25) edge[string=dRed] (2,-.5)
    (2,-1.5) edge[string=dRed] (2,-2)
    ;
    \node[enddot=dRed] at (2,3.5) {};
    \node[enddot=dRed] at (2,-2) {};
    \draw[string=dRed] (2,3) arc (90 : 360+90 : 1);
    \draw[string=dRed] (2,.75) arc (90 : 360+90 : .5);
    \draw[string=dRed] (2,-.5) arc (90 : 360+90 : .5);
}
            }
            \only<9->{
                = 0
            }
        }
    \end{gather*}

    \only<2-7>{
        \begin{mdframed}[style=MyFrame,nobreak=true,align=center,userdefinedwidth=12em]
            \vspace{-3em}
            \begin{gather*}
                \\ \tikz[vcenter, scale=0.4]{
    \tikzfixsize{(0,0)}{(4,4)}
    \squarecoord
    %
    \path
    (topl) edge[string=red] ($(mid)+(0,.75)$)
    (topr) edge[string=red] ($(mid)+(0,.75)$)
    (botl) edge[string=red] ($(mid)+(0,-.75)$)
    (botr) edge[string=red] ($(mid)+(0,-.75)$)
    ($(mid)+(0,.75)$) edge[string=red] ($(mid)+(0,-.75)$);
}
                = \tikz[vcenter, scale=0.4]{
    \tikzfixsize{(0,0)}{(4,4)}
    \squarecoord
    %
    \path
    (topl) edge[string=red] (midl)
    (topr) edge[string=red] (midr)
    (botl) edge[string=red] (midl)
    (botr) edge[string=red] (midr)
    (midl) edge[string=red] (midr);
}
            \end{gather*}
        \end{mdframed}
    }
    \only<8->{
        \begin{mdframed}[style=MyFrame,nobreak=true,align=center,userdefinedwidth=7em]
            \vspace{-3em}
            \begin{gather*}
                \\ \tikz[vcenter, scale=0.4]{
    \tikzfixsize{(0.25,0)}{(3.25,4)}
    \squarecoord
    %
    \path
    ($(mid)+(0,1)$) edge[string=dRed] (top)
    ($(mid)-(0,1)$) edge[string=dRed] (bot);
    \draw[string=dRed] ($(mid)+(0,1)$) arc (90 : 360+90 : 1);
    % circle with radius half distance from (mid + (0,1)) to (mid - (0,1))
} = 0
            \end{gather*}
        \end{mdframed}
    }
\end{frame}


\begin{frame}
    \frametitle{Soergel Bimodules}

    \begin{theorem}[Elias--Williamson, 2013]
        The additive Karoubi envelope of $\cD$ is equivalent to the category of Soergel Bimodules $\sbim$ over $S_2$ as (idempotent complete) graded additive $\C$-linear monoidal categories.
    \end{theorem}
\end{frame}


\begin{frame}
    \frametitle{Generalisations}

    \begin{itemize}
        \item Over general Coxeter groups e.g. $S_n$, $D_n$
        \[
            \tikz[vcenter, scale=0.45]{
    \squarecoord
    \tikzfixsize{(0,0)}{(4,4)}
    % \node (top) at ($(top)+(0,1)$) {};
    \node (mid) at ($(mid)+(0,.5)$) {};
    % centre blue
    \path[string=dBlue]
    ($(mid) + (-.5,0)$) edge ($(mid) + (.5,0)$)
    ($(mid) + (-.5,0)$) edge ($(top) + (-1,0)$)
    ($(mid) + (.5,0)$) edge ($(top) + (1,0)$)
    ($(mid) + (-.5,0)$) edge ($(mid) + (-.8,-1)$)
    ($(mid) + (.5,0)$) edge ($(mid) + (.7,-.75)$);
    \draw[string=dBlue] (bot)
    to[out=90,in=195] ($(mid) + (.7,-.75)$)
    to[out=10,in=120] ($(mid) + (.7,-.75) + (.75,-.35)$);
    \node[enddot=dBlue] at ($(mid) + (-.8,-1)$) {};
    \node[enddot=dBlue] at ($(mid) + (.7,-.75) + (.75,-.35)$) {};
    % red cap
    \draw[string=dRed] ($(bot) + (.4,0)$)
    to[out=90,in=180] ($(bot) + (.85,1.3)$)
    to[out=0,in=90] ($(bot) + (1.3,0)$);
    % stalactite and stalagmite
    \path
    ($(bot) + (-1.15,0)$) edge[string=dBlue] ($(bot) + (-1,.7)$)
    (top) edge[string=dRed] ($(top) + (0,-.75)$)
    ($(bot) + (.85,0)$) edge[string=dBlue] ($(bot) + (.85,.7)$);
    \node[enddot=dBlue] at ($(bot) + (-1,.7)$) {};
    \node[enddot=dRed] at ($(top) + (0,-.75)$) {};
    \node[enddot=dBlue] at ($(bot) + (.85,.7)$) {};
    % left barbells
    \path[string=dBlue] (0,1.5) edge (0,2.5);
    \draw[string=dRed] (-.65,1.5)
    to[out=270,in=180] (0,.9)
    to[out=0,in=270] (.65,1.5)
    to (.65,2.5)
    to[out=90,in=0] (0,3.1)
    to[out=180,in=90] (-.65,2.5);
    \node[enddot=dBlue] at (0,1.5) {};
    \node[enddot=dBlue] at (0,2.5) {};
    \node[enddot=dRed] at (-.65,1.5) {};
    \node[enddot=dRed] at (-.65,2.5) {};
}
        \]
        \item Diagrammatics for other categories of representations...
    \end{itemize}
    % Put some examples and dot points
\end{frame}


\begin{frame}
    \frametitle{Further Applications}

    \begin{itemize}
        \item Diagrammatics for BGG Category $\cO$, Tilting modules
        \item Changing scalars to fields of characteristic $p$
    \end{itemize}
\end{frame}

