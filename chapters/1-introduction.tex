\chapter{Introduction}

Visual interpretations of data and objects in mathematics are a tool that aids us in calculations and often provides insights into the mathematics they encode. This diagrammatic philosophy takes form in various settings, and can be defined precisely for algebraic objects to help us understand them better. A simple example are \textit{string diagrams} for permutations of a \textit{symmetric group}. A permutation can be drawn as strings between two copies of a set determining how the objects are permuted. For example, the permutation $(12354)$ in $S_5$ has the string diagram (reading from bottom to top)
\begin{center}
    \inputtikz{s5-example}.
\end{center}
Compositions of these permutations is the operation of joining corresponding strings start to end in order to create a larger string diagram representing their product. Related are \textit{(Artin) braid groups}, whose elements can be depicted similarly to the symmetric group, but where each crossing of strings has a choice of going over or under. As suggested by the name, these string diagrams resemble braids, and are important in knot theory.

A significant example are \textit{planar algebras} in the work of Vaughan Jones and many others. These are certain algebras of planar diagrams that describe operators. The study of the Temperley--Lieb--Jones (planar\footnote{Many diagrammatic versions of this algebra were independently discovered, for example by Rumer--Teller--Weyl \cite{weyl-diagrammatics} and Kauffman \cite{kauffman-tl-planar-algebra}}) algebra lead to the discovery of an important invariant in knot theory \cite{jones-knot-polynomial} in 1983, which we know now as the Jones polynomial. For this and surrounding works Jones received a Fields medal in 1990. This technology of planar algebras have been since used to study subfactors in functional analysis \cite{jones-planar-algebra}\footnote{Originally from 1999, and was recently published.} with consequences in statistical mechanics and mathematical physics. Although diagrammatics have been around before Jones' work on subfactors (prominent examples by Rumer--Teller--Weyl \cite{weyl-diagrammatics} and Brauer \cite{brauer-diagrammatics}), the diagrammatics of subfactors kick-started diagrammatics as a field with the birth of quantum topology.

In representation theory, our main motivational example is given by the proof of the Kazhdan--Lusztig conjecture through the diagrammatics of Soergel bimodules. This conjecture relates Kazhdan--Lusztig polynomials, arising from the Weyl group associated with a Lie algebra, to Jordan--H\"older multiplicities of particular representations of Lie algebras called Verma modules. Proofs were discovered independently by Beilinson--Bernstein and Brylinski--Kashiwara in 1981, both using geometric tools. However these methods had no clear generalisation to general Coxeter groups for variations of the original conjecture. Around ten years later, Soergel was working toward an algebraic proof using Soergel bimodules, however Soergel hit a technical road block. In 2010's, Elias and Williamson developed planar diagrams for morphisms on Soergel bimodules (see \cite{hodge-theory-of-soergel-bimodules} and \cite{diagrammatics-for-soergel-categories}) and were able to overcome the technical point where Soergel got stuck, to prove the conjecture diagrammatically. The diagrams provide an intuitive visual language that serve to simplify potentially diffucult algebraic calculations. Moreover, the diagrammatic category can be considered independently from algebraic Soergel bimodules. We explore this diagrammatic category for the symmetric group $S_2$ in \autoref{sec:one-col-sbim}. Let us stress that these diagrammatics can also be defined for any Coxeter group, including symmetric groups and dihedral groups. A general definition can be found in \cite{intro-soergel-bimodules} along with an introduction to the category of algebraic Soergel bimodules $\sbim$. Soergel had also shown that $\sbim$ is linked to other categories of representations, such as the Bernstein-Gelfand-Gelfand category $\mathcal{O}$ in \cite{soergel-category-O}. By this, a diagrammatic version of this category of representations can be defined. We see example in more detail in \autoref{sec:diag-osl2}.

One of the advantages of the diagrammatic Soergel bimodules is that it can be defined over $\Z$ and extended to fields of characteristic $p$ where classical Soergel bimodules are ill-behaved. Characters in the category of tilting modules (certain representations of a Lie group or quantum group) can be calculated via Kazhdan--Lusztig polynomials in characteristic zero. However, these polynomials were unknown in characteristic $p$. Riche and Williamson in \cite{riche-williamson-tilt-modules-p-canon-basis} were able to construct these characteristic $p$ Kazhdan--Lusztig polynomials by considering diagrammatic Soergel bimodules in characteristic $p$.

In this thesis we present an exposition for existing constructions of diagrammatics in representation theory. The first chapter gives an introduction to diagrammatics for monoidal categories, provides a diagrammatic description of Frobenius objects in monoidal categories, then defines module categories and some mechanisms to form an additive idempotent complete category. In \autoref{chapter:one-col-diagrammatics} we define the category of diagrammatic Soergel bimodules associated with the symmetric group $S_2$, construct a basis for its morphism spaces and state the theorem for its equivalence to the category of algebraic Soergel bimodules. We use this diagrammatic category to construct a diagrammatic module category with an extra relation, then prove its equivalence to the category of projective objects in the principle block of the category $\cO$. In \autoref{chapter:two-col-diagrammatics} we consider the affine symmetric group $\wtilde{S}_2$ to define the diagrammatic Soergel bimodules associated it, construct a basis for its morphism spaces and state the theorem for its equivalence to the category of algebraic Soergel bimodules. The extra generator in $\wtilde{S}_2$ compared with $S_2$ provides some additional complexity to the structure of the category. We then form a module category with two extra relations and provide a proof of its equivalence to the category of tilting modules for $\fraksl_2$. In the last chapter we discuss the consequences of diagrammatics in relation to \autoref{chapter:one-col-diagrammatics} and \autoref{chapter:two-col-diagrammatics}, mention some generalisations and further areas of interest.

Note that one of the advantages of diagrammatics is that we don't need to understand these algebraic categories in representation theory to study them. For this reason, we will defer some details in the proofs involving category $\cO$ and tilting modules to other sources.

The contents of this thesis are for honours students and future readers who are interested in this topic. The reader is assumed to have some familiarity with undergraduate algebra (such as groups, rings, algebras and fields), basic ideas in representation theory (such as the action of a group or algebra and the equivalence with modules), and basic category theory, including some knowledge of monoidal categories.
