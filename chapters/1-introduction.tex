\chapter{Introduction}

Visual interpretations of functions simplify calculations and often provides insights into the mathematical objects they encode. \red{More detail about what this is} This general philosophy takes form in various settings. A simple example are string diagrams for permutations. A permutation can be drawn as strings between two copies of a set determining how the objects are permuted. Compositions of these permutations is the operation of joining corresponding strings start to end in order to create a larger string diagram representing their product. Another example are \textit{(Artin) braid groups}, whose elements can be depicted similarly to the symmetric group, but where each crossing of strings has a choice of going over or under. As suggested by the name, these string diagrams resemble braids.

A significant example are \textit{planar algebras} in the work of Jones. These are certain algebras of planar diagrams that describe operators. His study of the Temperley-Lieb-Jones (planar\footnote{The algebra was later presented as a diagram algebra by Kauffman in \cite{kauffman-tl-planar-algebra}}) algebra lead to the discovery of an important invariant in knot theory \cite{jones-knot-polynomial} in 1983, which we know now as the Jones polynomial. For this and surrounding works he received a Fields medal. This technology of planar algebras have been since used to study subfactors in functional analysis \cite{jones-planar-algebra}\footnote{Originally from 1999, and was recently published.} and have consequences in for example statistical mechanics and mathematical physics.

In representation theory, our main motivational example is given by the proof of the Kazhdan--Lusztig conjecture through the diagrammatics of Soergel bimodules. The conjecture relates Kazhdan--Lusztig polynomials, arising from the Weyl group associated with a Lie algebra, to Jordan--H\"older multiplicities of particular representations of Lie algebras called Verma modules. Proofs were discovered independently by Beilinson--Bernstein and Brylinski--Kashiwara in 1981 but by geometric methods, which was unsatisfying to many. Around this time, Soergel was working toward an algebraic proof by Soergel bimodules, however he hit a technical road block. In 2010's, Elias and Williamson \red{Ref?} developed planar diagrams for morphisms on Soergel bimodules and were able to overcome the technical point where Soergel got stuck, to prove the conjecture diagrammatically. The diagrams can greatly simplify algebraic calculations and the diagrammatic category can be considered independently from Soergel bimodules. We explore this diagrammatics for $S_2$ in \autoref{sec:one-col-sbim}.

One of the advantages of the diagrammatic Soergel bimodules is that it can be defined over $\Z$ and extended to fields of characteristic $p$ where classical Soergel bimodules are ill-behaved. Characters in the category of tilting modules (certain representations of a Lie algebra) can be calculated via Kazhdan--Lusztig polynomial in characteristic zero, however these polynomials were unknown in characteristic $p$. Riche and Williamson in \cite{riche-williamson-tilt-modules-p-canon-basis} were able to construct these characteristic $p$ Kazhdan--Lusztig polynomials by considering diagrammatic Soergel bimodules in characteristic $p$.


% TODO: Talk about versatility of diagrammatics - about the same but can describe lots of different things

% TODO: Talk about Jones with planar algebra for subfactor theory
% Talk about EW proof of soergel's conjecture
% Talk about Riche-Williamson on p-canonical basis

% TODO: Mention the origins of this stuff, see EW, Temperley-Lieb-Jones
% see also https://ncatlab.org/nlab/show/string+diagram for other
% see also survey on drawing monoidal category

In this paper we give an introduction to drawing morphisms in monoidal categories, \red{put more here} and define some mechanisms to form an additive and idempotent complete category. In \autoref{chapter:one-col-diagrammatics} we define diagrammatic Soergel bimodules associated with the symmetric group $S_2$, construct a basis for its morphism spaces and state the theorem for its equivalence to the category of algebraic Soergel bimodules. We use this diagrammatic category to construct a diagrammatic module category with an extra relation, then prove its equivalence to the category of projective objects in the principle block of the category $\cO$. In \autoref{chapter:two-col-diagrammatics} we consider the affine symmetric group $\wtilde{S}_2$ to define the diagrammatic Soergel bimodules associated it, construct a basis for its morphism spaces and state the theorem for its equivalence to the category of algebraic Soergel bimodules. The extra generator in $\wtilde{S}_2$ compared with $S_2$ provides some additional complexity to the structure of the category. We then form a module category with two extra relations and provide a proof of its equivalence to the category of tilting modules for $\fraksl_2$. In the last chapter we discuss the consequences of diagrammatics in relation to \autoref{chapter:one-col-diagrammatics} and \autoref{chapter:two-col-diagrammatics}, mention generalisations of the results \red{and further areas of interest}.

The contents of this thesis are for honours students and future readers that are interested in this topic. The reader is assumed to have some familiarity with undergraduate algebra (such as groups, rings, algebras and fields), basic ideas in representation theory, basic category theory and monoidal categories.

\red{Talk about we dont need to know about category $\cO$ and $\Tilt$. One of the advantages of diagrammatics is that we don't need to understand these complex categories in representation theory to study them. For this reason, we will defer the details in the proofs involving them to other sources.}


\red{Talk about introduction to soergel bimodules}

% Elias, Ben; Khovanov, Mikhail. Diagrammatics for Soergel categories. Int. J. Math. Math. Sci. 2010, Art. ID 978635, 58 pp. MR3095655

% Elias, Ben; Williamson, Geordie. The Hodge theory of Soergel bimodules. Ann. of Math. (2) 180 (2014), no. 3, 1089--1136. MR3245013 Add to clipboard

