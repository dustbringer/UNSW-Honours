\section{Additive Karoubi Envelope}
\label{sec:additive-karoubi}

\red{Something about why this is needed}

\red{Mention that this technical and mostly a formal process}

\subsection*{Additive and Karoubian Categories}
% tilt 4.1

\begin{definition}
    A \textit{preadditive category} is a category enriched over the category of abelian groups. That is, for objects $A$ and $B$, $\Hom(A,B)$ has the structure of an abelian group and the composition of morphisms is bilinear (over the abelian group operation).
\end{definition}

\begin{remark}
    In particular $R$-linear categories are preadditive because $R$-modules are defined over abelian groups.
\end{remark}

\begin{definition}
    A \textit{biproduct} of objects of a category is both a product and a coproduct.
    An \textit{additive category} is a preadditive category that admits all finite biproducts.
\end{definition}

Biproducts are a generalisation of direct sums of modules, so we often write $\oplus$ and say ``direct sum''. In other words, additive categories are preadditive categories containing all direct sums.

\begin{definition}
    An \textit{idempotent} is a endomorphism $e$ such that $e \circ e = e$.
    We say that a preadditive category is \textit{Karoubian} or \textit{idempotent complete} if for every idempotent $e: X \to X$ there is a direct sum decomposition $X \cong Y \oplus Z$ such that $e$ is a projection onto $Y$.
\end{definition}

This is a formal way to say that a category contains all direct sums, as every direct summand is an image of an idempotent given by projection.




\subsection*{Additive Closure and Karoubi Envelope}

We can formally add direct sums and direct summands into a preadditive category, by the additive closure and the Karoubi envelope.

\begin{definition}
    Let $\mcal{C}$ be a preadditive category. The \textit{additive closure} $\mcal{C}^\oplus$ of $\mcal{C}$ is the category where objects are finite (possibly empty) formal direct sums $\bigoplus_{i=1}^n A_i$ for $A_i \in \op{ob}(\mcal{C})$. We call the empty direct sum the \textit{zero object} $0$. A morphism $f$ of $\Hom_{\mcal{C}^\oplus}(\bigoplus_{i=1}^n A_i, \bigoplus_{i=1}^m B_i)$ is an $m \times n$ matrix $f = (f_{j,i})$ of morphisms $f_{j,i} \in \Hom_\mcal{C}(A_i, B_j)$.
\end{definition}

% The additive closure is sometimes denoted by $\cat{Mat}(\mcal{C})$.
It is clear that $\mcal{C}$ is a category that embeds in $\mcal{C}^\oplus$ and $\mcal{C}^\oplus$ is additive. In the case where $\mcal{C}$ is monoidal, $\mcal{C}^\oplus$ is monoidal by extending the monoidal product to be an additive functor in each input. If $\mcal{C}$ is $R$-linear, then $\mcal{C}$ is an $R$-linear category by assuming that the $R$-action on morphisms applies componentwise. If $\mcal{C}$ is a $\mcal{M}$-module category, then $\mcal{C}$ is a $\mcal{M}$-module category by additionally assuming that the module action applies componentwise.

\begin{lemma}
    The additive closure satisfies the following universal property. For every preadditive functor $F: \mcal{C} \to \mcal{D}$ where $\mcal{D}$ is an additive category, there is a unique additive functor $F': \mcal{C}^\oplus \to \mcal{D}$ such that the composition $\mcal{C} \hookrightarrow \mcal{C}^\oplus \xto{F'} \mcal{D}$ is $F$.
\end{lemma}

This is a classical result so we will not provide a proof. It can be observed by extending $F$ to a functor $F^\oplus: \mcal{C}^\oplus \to \mcal{D}^\oplus$ defined componentwise with $F$.

\begin{definition}
    Let $\mcal{C}$ be a category. The \textit{Karoubi envelope} $\op{Kar}(\mcal{C})$ of $\mcal{C}$ is the category where objects are ordered pairs $(A,e)$ for an object $A$ in $\mcal{C}$ and an idempotent $e \in \End_\mcal{C}(A)$.  Morphisms $f: (A, e) \to (A', e')$ are morphisms $f:A \to A'$ in $\mcal{C}$ such that $f = f \circ e = e' \circ f$, where composition is composition in $\mcal{C}$. Equivalently, morphisms $f: (A, e) \to (A', e')$ are of the form $e'\circ f \circ e$ for some (not necessarily unique) morphism $f: A \to A'$. The identity morphism on $(A,e)$ is $e$.
\end{definition}

The objects $(A,e)$ should be seen as ``the image of $e$''. This is sometimes called the \textit{Karoubian closure} or \textit{idempotent completion}. The \textit{additive Karoubi envelope} of a category $\mcal{C}$ is $\op{Kar}(\mcal{C}^\oplus)$ which we may denote $\Kar(\mcal{C})$.

\begin{proposition}
    The Karoubi envelope $\op{Kar}(\mcal{C})$ is Karoubian.
\end{proposition}

A proof can be found at \cite[Lemma 11.17]{intro-soergel-bimodules}.

\begin{lemma}
    Every functor $F: \mcal{C} \to \mcal{D}$ where $\mcal{D}$ is Karoubian, extends uniquely (up to isomorphism) to a functor $F': \op{Kar}(\mcal{C}) \to \mcal{D}$.
\end{lemma}

This is another classical result. See \cite[Proposition 6.5.9 (1)]{borceux-categorical-algebra} for a proof.


\red{What happens when we do this on monoidal, $R$-linear or additive categories?}

\red{Talk about diagrammatics, $\mcal{C}^\oplus$ is easy to describe diagrammatically (just matrices of diagrams) but $\op{Kar}$ is not easy to describe in general (we need to find idempotents and put them before and after a diagram)}

