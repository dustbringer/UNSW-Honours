\section{Drawing Monoidal Categories}
\label{sec:monoidal-cat}

Monoidal categories are the main context in which we consider diagrammatics. For more details about monoidal categories the reader may refer to \cite{tensor-categories}, and a helpful survey of diagrams for various types of monoidal categories at \cite{selinger-survey-of-graphical-monoidal-categories}.

\begin{definition}
    A \textit{monoidal category} $\mathcal{C}$ is a category equipped with a bifunctor $\tensor: \mathcal{C} \times \mathcal{C} \to \mathcal{C}$ and a unit object $\mathbbb{1}$, such that certain associativity and unit relations hold, see \cite[Definition 2.1.1, 2.2.8]{tensor-categories}. The bifunctor $\tensor$ is called the \textit{tensor} or \textit{monoidal product}. A monoidal category is \textit{strict} if we have equalities $A \tensor (B \tensor C) = (A \tensor B) \tensor C$ and $A = \mathbbb{1} \tensor A = A \tensor \mathbbb{1}$ for objects and similarly for morphisms.
\end{definition}

The functoriality of $\tensor$ means that the monoidal product commutes with composition in both variables.

\begin{definition}
    A functor $F: \mcal{C} \to \mcal{D}$ between monoidal categories is called \textit{(strict) monoidal} if it preserves the monoidal product, i.e. $F(A \tensor B) = F(A) \tensor F(B)$. Structure preserving functors for other types of categories can be defined in a similar way.
\end{definition}

For this thesis, we will assume that monoidal categories and monoidal functors are strict\footnote{We can also define diagrammatics for non-strict monoidal categories, but drawing isomorphisms composed with each morphism is cumbersome.}. This does not pose any problems since all monoidal categories are monoidally equivalent to a strict one\footnote{See \cite[\nopp VII.2]{maclane-category-theory} or \cite[Theorem 2.8.5]{tensor-categories}}, and a similar strictification\footnote{See \cite{power-general-coherence}.} can be applied to the functor. In this context, the details in the coherence relations are trivial.

The morphisms of a monoidal category $\mcal{C}$ can be drawn as string diagrams embedded in a planar strip. We fix the convention that a diagram is a morphism when read from bottom to top; that is, the domain is on the bottom of the strip and the codomain on the top. Morphisms that make up a diagram are drawn as tokens or boxes. For example
\begin{center}
    \inputtikz{2.1/string-diagram-example}
\end{center}
depicts a morphism $f: a \to b \tensor c$. Notice here that tensor products of objects have its factors displayed horizontally. The compositions of morphisms is the vertical stacking of diagrams whenever labels on domains and codomains match. For example, the composition $g \circ f: a \to b \tensor c \to a \tensor c$ of $f: a \to b \tensor c$ with $g: b \tensor c \to a \tensor c$ has the diagram
\begin{center}
    \inputtikz{2.1/string-diagram-example-composition1}
    =
    \inputtikz{2.1/string-diagram-example-composition2}.
\end{center}
For identity morphisms we just draw a vertical line, so $\id_a$ is the diagram
\begin{center}
    \inputtikz{2.1/string-diagram-example-identity}.
\end{center}
This is a sensible choice since composition with the identity should not change the diagram, which is clear diagrammatically. The tensor product of morphisms is the horizontal concatenation of diagrams, such that strings from separate diagrams don't interact. For example, given $h:x \to y$, the tensor product $f \tensor h: a \tensor x \to b \tensor c \tensor y$ is drawn as
\begin{center}
    \inputtikz{2.1/string-diagram-example-tensor1}
    =
    \inputtikz{2.1/string-diagram-example-tensor2}.
\end{center}
We let the monoidal unit $\mathbbb{1}$ be blank and unlabelled, and strings that would join to $\mathbbb{1}$ are blank. Particularly, $\id_\mathbbb{1}$ is an empty diagram. It makes sense to display $\mathbbb{1}$ in this way since tensoring with $\mathbbb{1}$ (in a strict monoidal category) does nothing to objects and tensoring with $\id_\mathbbb{1}$ does nothing to morphisms. By this convention, we also have diagrams such as
\begin{center}
    \inputtikz{2.1/string-diagram-example-unit1}
    \quad and \quad
    \inputtikz{2.1/string-diagram-example-unit2}
\end{center}
for morphisms $f_1: a \to \mathbbb{1}$ and $f_2: \mathbbb{1} \to b \tensor c$.

The bifunctoriality of $\tensor$ implies the following \textit{interchange law}. For morphisms $f: a \to b$ and $g: c \to d$, we have $(\id_b \tensor g) \circ (f \tensor \id_c) = f \tensor g = (f \tensor \id_d) \circ (\id_a \tensor g)$. In other words the following diagram commutes.

\begin{center}
    \begin{mytikzcd}[sep=large]
        a \tensor c \arrow[r,"f \tensor \id_c"] \arrow[d, "\id_a \tensor g"'] \arrow[dr, "f \tensor g"]
        & b \tensor c \arrow[d, "\id_b \tensor g"] \\
        a \tensor d \arrow[r,"f \tensor \id_d"]
        & b \tensor d
    \end{mytikzcd}
\end{center}
Written with string diagrams, this is
\begin{center}
    \inputtikz{2.1/interchange-law1}
    = \inputtikz{2.1/interchange-law2}
    = \inputtikz{2.1/interchange-law3}
\end{center}
which holds up to vertical deformation of the diagram. This is a small taste of isotopy, but only in the vertical direction.

Before looking at an example of a diagrammatic monoidal category, we just mention some definitions.

\begin{definition}
    For a commutative ring $R$, an \textit{$R$-linear category} is a category enriched over the category of $R$-modules. That is, for objects $a,b$, the set of morphisms $\Hom(a,b)$ is an $R$-module and the composition of morphisms is $R$-bilinear. An \textit{$R$-linear monoidal category} is a category that is both monoidal and $R$-linear such that the monoidal product on morphisms is $R$-bilinear. A ($\Z$-)\textit{graded $R$-linear category} is a category where $\Hom(A,B)$ is a $\Z$-graded $R$-module. That is, $\Hom(A,B) = \bigoplus_{i \in \Z} \Hom^i(A,B)$ where $\Hom^i(A,B)$ is the homogeneous component of degree $i$, and for $f \in \Hom^i(A,B)$ and $g \in \Hom^j(B,C)$, the composition $g \circ f$ is in $\Hom^{i+j}(A,C)$.
\end{definition}

By the bilinearity of composition and tensor products, $0 \tensor f = (0 + 0) \tensor f = 0 \tensor f + 0 \tensor f$ and similarly with composition, so composition and tensors with $0$ are zero.

\begin{example}
    The category of vector spaces over a field $\Bbbk$, $\cat{Vect}_\Bbbk$, is a $\Bbbk$-linear monoidal category given by the usual tensor product of vector spaces and linear maps.
\end{example}

\begin{definition}
    A monoidal category $\mcal{C}$ is \textit{generated} by a set $G_o$ of objects and $G_m$ of morphisms, when all non-unit objects are finite tensor products of objects in $G_o$ and all non-identity morphisms are finite combinations of tensors and compositions of morphisms in $G_m$. Similarly, we may define \textit{generated} $R$-linear monoidal categories such that we also allow $R$-linear combinations of morphisms.
\end{definition}

\begin{example}
    \label{eg:tl-category}
    The Temperley--Lieb--Jones category $\mcal{TL}$ is a (diagrammatic) strict $\Z$-linear monoidal category whose objects are generated by the vertical line $\mathsf{I}$ and morphisms generated by the cup $\cup: \mathbbb{1} \to \mathsf{I} \tensor \mathsf{I}$ and cap $\cap: \mathsf{I} \tensor \mathsf{I} \to \mathbbb{1}$, with the relation
    \begin{center}\label{rel:tl-relation}
        \inputtikz{2.1/tl-relation1}
        = \inputtikz{2.1/tl-relation2}
        = \inputtikz{2.1/tl-relation3},
    \end{center}
    where the composition and tensor product are vertical and horizontal concatenation. This is an isotopy relation allowing us to ``straighten out zig-zags''.

    \begin{remark}
        The generating object $\mathsf{I}$ is self-dual with adjunction maps $\cup$ and $\cap$ satisfying the adjoint relation given by \eqref{rel:tl-relation}. In other words, $\mcal{TL}$ is a free monoidal category on a self-dual object.
    \end{remark}

    The morphisms in this category are $\Z$-linear combinations of diagrams such as
    \begin{center}
        \inputtikz{2.1/tl-example}.
    \end{center}
    The diagrams in this category are crossingless matchings, i.e. each generator is connected to exactly one line and lines don't cross, and possibly have floating circles.

    The standard definition of the Temperley--Lieb--Jones category has an extra relation that identifies circles with $\delta \id_\mathbbb{1}$, for some fixed constant $\delta \in \Z$. For example, consider the quotient of $\mcal{TL}$ by the relation
    \begin{gather*}
        \inputtikz{2.1/tl-circle} = -2 \id_\mathbbb{1},
    \end{gather*}
    where $\id_\mathbbb{1}$ is the blank diagram.
    In this quotient, the above diagram is
    \begin{center}
        4 \inputtikz{2.1/tl-example-no-circle}.
    \end{center}
    \begin{remark}
        This category controls the representation theory of $\fraksl_2$.
    \end{remark}
\end{example}

















