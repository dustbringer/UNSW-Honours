\section{Drawing Monoidal Categories}

A monoidal category $\mathcal{C}$ is a category equipped with a bifunctor $\tensor: \mathcal{C} \times \mathcal{C} \to \mathcal{C}$ and a unit object $\mathbbb{1}$, such that certain associativity and unit relations hold\footnote{For more details see \cite{tensor-categories}.}. We will assume that monoidal categories are strict, since all monoidal categories are monoidally equivalent to a strict one \red{[Reference?]}. The morphisms of $C$ can be drawn as string diagrams. We can think of maps from bottom to top, and functions that make up the morphism are boxes. For example
\begin{center}
    \inputtikz{2.1/string-diagram-example}
\end{center}
describes a morphism $f: a \to b \tensor c$. For identity morphisms we drop the box and only draw a vertical line, so $\id_a$ is the diagram
\begin{center}
    \inputtikz{2.1/string-diagram-example-identity}.
\end{center}
The tensor product of morphisms is the horizontal concatenation of diagrams, such that strings from separate functions don't interact. For example a morphism $f \tensor g: a \tensor x \to b \tensor c \tensor y$ is drawn as
\begin{center}
    \inputtikz{2.1/string-diagram-example-tensor1}
    = 
    \inputtikz{2.1/string-diagram-example-tensor2}.
\end{center}
By convention, $\mathbbb{1}$ is blank and unlabelled, and strings that would join to $\mathbbb{1}$ are blank. Particularly, $\id_\mathbbb{1}$ is a blank diagram, and we have diagrams such as
\begin{center}
    \inputtikz{2.1/string-diagram-example-unit1}: $a \to \mathbbb{1}$
    \quad and \quad
    \inputtikz{2.1/string-diagram-example-unit2}: $\mathbbb{1} \to b \tensor c$.
\end{center}
The compositions of morphisms is the vertical stacking of diagrams where domains and codomains match. For example a composition $h \circ f: a \to b \tensor c \to a \tensor c$ has the diagram
\begin{center}
    \inputtikz{2.1/string-diagram-example-composition1}
    =
    \inputtikz{2.1/string-diagram-example-composition2}.
\end{center}




% \begin{center}
%     \inputtikz{2.1/string-diagram-example}
% \end{center}



