\section{One-colour Diagrammatic Hecke Category}

The first one-colour diagrammatic we explore is the one-colour (diagrammatic) Hecke category $\cH(S_2)$ for the symmetric group $S_2 = \angl{s \mathrel{|} s^2 = 1}$. This is a monoidal category with a presentation in terms of generators and relations.
% Afterwards we will see that this diagrammatic category leads to the category of Soergel Bimodules, which categorifies the Hecke algebra.

\begin{remark}
    All the following diagrams could be defined without the language of categories, as planar pictures with appropriate vertical stacking and horizontal concatenation. Nevertheless, we define them in a category because they will eventually be seen as diagrammatic versions of important categories in representation theory.
\end{remark}

The objects of $\cH(S_2)$ are generated by taking formal tensor products of the non-identity element $s \in S_2$. We will write these objects as words, e.g. $s$, $ssss \eqqcolon s^4$, $sssssss \eqqcolon s^7$, where the tensor product is just concatenation. The empty tensor product (or empty word) will be denoted by $\vn \eqqcolon s^0$.


% TODO: mention isotopy in the background section
The morphisms are generated, up to isotopy, by univalent and trivalent vertices
\begin{align} \label{eq:one-col-gen}
    \inputtikz{3.1/generator1}
    \quad , \quad
    \inputtikz{3.1/generator2}
\end{align}
which are maps $s \to \vn$ and $ss \to s$ respectively. Note that we put a large dot on univalent vertices to signify that the line stops abruptly and does not connect to the top. The composition of such diagrams is appropriate vertical stacking, and the tensor product is horizontal concatenation (without intersection). Additionally\footnote{\green{Pun intended.}}, we allow formal sums of diagrams by putting a $\Z$-module structure on each morphism space $\Hom(s^n, s^m)$, for non-negative integers $n,m$. Composition or tensor with the zero morphism $0$ in this $\Z$-module result in $0$. To abuse notation, the empty diagram $\vn \to \vn$ will be denoted $\vn$. The identity morphism in $\Hom(s^n, s^n)$ is the diagram consisting of $n$ vertical lines.
% \begin{gather}
%     \inputtikz{3.1/identity-diagram}
% \end{gather}


% TODO: Something about the line being red

Such diagrams are subject to the following local relations.
\begin{subequations} \label{eq:one-col-sbim-rel}
    \begin{gather}
        % Unit and counit
        \label{rel:one-col-frob-unit}
        \inputtikz{3.1/relation1-frob-unit} = \inputtikz{3.1/relation1-identity}
        \\
        % Frobenius associativity
        \label{rel:one-col-frob-ass}
        \inputtikz{3.1/relation2-frob-ass1} = \inputtikz{3.1/relation2-frob-ass2}
        \\
        % Needle annihilation
        \label{rel:one-col-needle}
        \inputtikz{3.1/relation3-needle} = 0
        \\
        % Polynomial forcing for S_2
        \label{rel:one-col-poly-forcing}
        \inputtikz{3.1/relation4-left-bar} = 2 \inputtikz{3.1/relation4-split-line} - \inputtikz{3.1/relation4-right-bar}
    \end{gather}
\end{subequations}

\begin{remark}
    The object $s$ is a Frobenius algebra object in $\cH(S_2)$. The generators \eqref{eq:one-col-gen} and their horizontal reflections are the unit, multiplication, counit and comultiplication maps. The unit, associativity and Frobenius associativity axioms are satisfied by the relations \eqref{rel:one-col-frob-unit} and \eqref{rel:one-col-frob-ass}.
\end{remark}

\begin{example}
    \label{eg:one-col-relations}
    Let us use the relations in \eqref{eq:one-col-sbim-rel} to simplify the following morphism in $\Hom((s,s),(s))$.
    \begin{align*}
        \inputtikz{3.1/relation-example1}
         & = \inputtikz{3.1/relation-example2}
        \\ & = 2 \inputtikz{3.1/relation-example3.1} - \inputtikz{3.1/relation-example3.2}
        \\ & = 2 \inputtikz{3.1/relation-example4.1} - \inputtikz{3.1/relation-example4.2}
    \end{align*}
\end{example}

There is a right (or left) $\Z[\barbell]$-basis for $\Hom(s^n, s^m)$ described in \cite{elias-williamson-soergel-calculus} called the Double Leaves basis. To define this basis we must first look at morphisms known as Light Leaves. Given a word $w = s^n$, a subexpression is a binary string of length $n$. For example, $0000, 0110$ and $1011$ are subexpressions of $s^4=ssss$. Given a subexpression $e$ of an object $w$, we can apply it to produce an element $w(e) \in S_2$, e.g. $ssss(1011) = s*1*s*s = s$. Each term of the subexpression is a decision of whether to include the corresponding $s$ in the word, where excluding an $s$ amounts to multiplying by $1$.
\red{Talk about how $s$ could mean an object of $\cH$ or an object of $S_2$.}

For a subexpression $e$ of an expression $w$, we can label each term by $U_0,U_1,D_0$ or $D_1$. The label is $U_*$ if the partial subexpression up to the current term evaluates to $1 \in S_2$ and $D_*$ if it evaluates to $s \in S_2$, where the subscript is the corresponding term in $e$.

\begin{example} \label{eg:one-col-light-leaf}
    For the object $ssss$ and subexpression $0101$, we can find the labels:
    \begin{center}
        \begin{tabular}{ |r||p{6em}|p{6em}|p{6em}|p{6em}| }
            \hline
            Choice         & 1     & 2         & 3             & 4                 \\ \hline
            Partial $w$    & $s$   & $ss$      & $sss$         & $ssss$            \\ \hline
            Partial $e$    & $0$   & $01$      & $010$         & $0101$            \\ \hline
            Partial $w(e)$ & $1$   & $1*s = s$ & $1*s*1=s$     & $1*s*1*s=1$       \\ \hline
            Labels         & $U_0$ & $U_0 U_1$ & $U_0 U_1 D_0$ & $U_0 U_1 D_0 D_1$ \\ \hline
        \end{tabular}
    \end{center}
\end{example}

The light leaf $LL_{w,e} \in \Hom(w, w(e))$ \red{$w(e)$ here is an element of $S_2$ that we identify with one of the objects $\vn$ or $s$}, corresponding to the object $w$ and subexpression $e$, is defined iteratively as follows. Let $LL_{\vn,\vn} = \vn$ be the empty diagram. Given $LL_{w',e'}$ and $i \in \{0,1\}$, $LL_{w's,e'i}$ is one of
\begin{equation}
    \inputtikz{3.1/light-leaves-def-u0} \:,\:
    \inputtikz{3.1/light-leaves-def-u1} \:,\:
    \inputtikz{3.1/light-leaves-def-d0} \:,\:
    \inputtikz{3.1/light-leaves-def-d1}
\end{equation}
depending on the next label, where $w'$ and $e'$ are appropriate subwords of $w$ and $e$. Observe that the codomain of a light leaf $LL_{w,e}$ is the object corresponding to the evaluation $w(e) \in S_2$ of the subexpression. The recursive definition is consistent, since if the next label is $U_*$ then the codomain of $LL_{w',e'}$ (the evaluation of the partial subexpression $w'(e')$ up to the label) is $1$, and when the next label is $D_*$ the codomain of $LL_{w',e'}$ is $s$. \red{Rewrite this to make sense.}
\red{Do we need to talk about 'degree' of light leaves?}

\begin{example}
    Following from Example \eqref{eg:one-col-light-leaf} for $w = ssss$ and $e = 0101$, we have labels $U_0 U_1 D_0 D_1$ so the light leaf $LL_{w,e}$ is built as follows.
    \[
        \vn
        \to \inputtikz{3.1/light-leaves-example1}
        \to \inputtikz{3.1/light-leaves-example2}
        \to \inputtikz{3.1/light-leaves-example3}
        \to \inputtikz{3.1/light-leaves-example4}
    \]
\end{example}

Let $\ol{LL}_{w,e}$ denote the vertical reflection of $LL_{w,e}$. A \textit{double leaf} associated to expressions $w,y$ is a composition
\[
    \mathbb{LL}_{f,e} \coloneqq \ol{LL}_{y,f} \circ LL_{w,e} : w \to y
\]
for subexpressions $e$ of $w$ and $f$ of $y$ such that $w(e) = f(y)$. Visually this looks like a morphism from $w$ to $y$ factoring through $w(e) = y(f) \in \{1,s\}$,
\[
    \inputtikz{3.1/double-leaves}.
\]
% TODO: Maybe give some indication of why we use trapeziums

\begin{example}
    Let $w = ssss$ and $y = sss$. Let $e=0111$ be a subexpression of $w$, and $f=010$ be a subexpression of $y$. The corresponding light leaves are
    \[
        LL_{w,e} = \inputtikz{3.1/double-leaves-example1.1}
        \text{ and }
        LL_{y,f} = \inputtikz{3.1/double-leaves-example1.2}.
    \]
    Then the double leaf $\mathbb{LL}_{f,e} = \ol{LL}_{y,f} \circ LL_{w,e} : w \to y$, factoring through $s$, is
    \[
        \inputtikz{3.1/double-leaves-example2}.
    \]
\end{example}

Notice that these double leaves have no floating diagrams such as $\barbell$. In order for these double leaves to be a basis for a morphism space, we insert these floating diagrams by taking linear combinations as a left $\Z[\barbell]$-module. Here, the left $\barbell$-action on a diagram is just concatenation by $\barbell$ on the left. Since we can move barbells to the right via. the relation \eqref{rel:one-col-poly-forcing}, we can equivalently act by $\Z[\barbell]$ on the right. This leads us to the following theorem.

\begin{theorem}[Elias-Williamson \cite{elias-williamson-soergel-calculus}, Theorem 1.2]
    \label{thm:one-col-double-leaves-basis}
    Given objects $w,y \in \cH(S_2)$, let \green{$\mathbb{LL}_{w,y}$} \footnote{\green{this can be confused with the double leaves themselves, maybe write $\mathbb{LL}(w,y)$}} be the collection of double leaves $\mathbb{LL}_{f,e}$ for subexpressions $e$ of $w$ and $f$ of $y$, such that $w(e) = y(f)$. Then \green{$\mathbb{LL}_{w,y}$} is a left (or right) $\Z[\barbell]$-module basis for $\Hom(w,y)$.
\end{theorem}
A purely diagrammatic proof (of a more general theorem) can be found in \cite{elias-williamson-soergel-calculus}.

\begin{remark}
    The above light leaves and double leaves, introduced in \cite{elias-williamson-soergel-calculus}, are diagrammatic analogues of Libedinsky's work in \cite{libedinsky-lightleavesbasis}.
\end{remark}

The morphisms can be graded such that the univalent vertices has degree $1$ and trivalent vertices have degree $-1$. The degree of a general diagram is the sum of the degrees of the generators that appear in it. \red{How do you do degree of a $\Z$-linear combination?}

\red{Put example}

The double leaves bases allow us to show that the Karoubi envelope of $\cH(S_2)$ is equivalent to the category of Soergel Bimodules $\sbim$ over $S_2$ as monoidal categories.

\begin{theorem}[Elias-Williamson \cite{elias-williamson-soergel-calculus}, Theorem 6.30] \label{thm:one-col-sbim-equiv}
    There is an equivalence of categories between $\Kar(\cH(S_2))$ and the category of Soergel Bimodules $\sbim$ over $S_2$.
\end{theorem}
The proof gives an equivalence of categories $\cH(S_2) \cong \bsbim$ of Bott-Samelson bimodules over $S_2$, by comparing the dimensions of morphism spaces using double leaves. Since $\Kar(\bsbim) \cong \sbim$ and Karoubi envelope preserves equivalences, we obtain $\Kar(\cH(S_2)) \cong \sbim$.

