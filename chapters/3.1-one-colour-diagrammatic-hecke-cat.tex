\section{One-colour Diagrammatic Hecke Category}

The first one-colour diagrammatic we explore is the one-colour diagrammatic Hecke category $\blue\cH$ for the symmetric group $S_2 = \angl{s \mathrel{|} s^2 = 1}$. % Afterwards we will see that this diagrammatic category leads to the category of Soergel Bimodules, which categorifies the Hecke algebra.

The objects of this category are generated by taking formal tensor products of the non-identity element $s \in S_2$. For example the tensor product of four $s$'s which denote with the expression $(s,s,s,s)$.

The morphisms in this category have a presentation in terms of generators and relations. For convenience, we will describe them up to isotopy. % TODO: mention isotopy in the background section
The generators are the following univalent and trivalent vertices, which can be rotated and flipped vertically using isotopy.
\begin{align}
    \tikz[vcenter, scale=0.4]{
    \coordinate (mid) at (2,2);
    \coordinate (bot) at (2,0);
    %
    \path
    (bot) edge[string=red] (mid);
    %
    \node [enddot=red] at (mid) {};
    \draw[rounded corners, draw=black!10, very thick] (-0.25,0) rectangle (4.25,4);
}
    \quad , \quad
    \tikz[vcenter, scale=0.4]{
    \coordinate (mid) at (2,2);
    \coordinate (bot) at (2,0);
    \coordinate (top) at (2,4);
    \coordinate (botl) at (0,0);
    \coordinate (botr) at (4,0);
    %
    \path
    (mid) edge[string=red] (top)
    (mid) edge[string=red] (botl)
    (mid) edge[string=red] (botr);
    %
    \draw[rounded corners, draw=black!10, very thick] (-0.25,0) rectangle (4.25,4);
}
\end{align}
These morphisms are subject to the following local relations.
\begin{subequations} \label{eq:one-col-sbim-rel}
    \begin{gather}
        % Unit and counit
        \label{subeqn:one-col-frob-unit}
        \tikz[vcenter, scale=0.4]{
            \squarecoord
            %
            \path
            (top) edge[string=red] (bot)
            (mid) edge[string=red] (midr2);
            %
            \node [enddot=red] at (midr2) {};
            \draw[rounded corners, draw=black!10, very thick] (-0.25,0) rectangle (4.25,4);
        }
        =
        \tikz[vcenter, scale=0.4]{
            \squarecoord
            %
            \path
            (top) edge[string=red] (bot);
            %
            \draw[rounded corners, draw=black!10, very thick] (-0.25,0) rectangle (4.25,4);
        }
        \\
        % Frobenius associativity
        \label{subeqn:one-col-frob-ass}
        \tikz[vcenter, scale=0.4]{
            \squarecoord
            %
            \path
            (topl) edge[string=red] (midt)
            (topr) edge[string=red] (midt)
            (botl) edge[string=red] (midb)
            (botr) edge[string=red] (midb)
            (midt) edge[string=red] (midb);
            %
            \draw[rounded corners, draw=black!10, very thick] (-0.25,0) rectangle (4.25,4);
        }
        =
        \tikz[vcenter, scale=0.4]{
            \squarecoord
            %
            \path
            (topl) edge[string=red] (midl)
            (topr) edge[string=red] (midr)
            (botl) edge[string=red] (midl)
            (botr) edge[string=red] (midr)
            (midl) edge[string=red] (midr);
            %
            \draw[rounded corners, draw=black!10, very thick] (-0.25,0) rectangle (4.25,4);
        }
        \\
        % Barbell
        \tikz[vcenter, scale=0.4]{
            \squarecoord
            %
            \path
            (midt) edge[string=red] (midb);
            %
            \node [enddot=red] at (midt) {};
            \node [enddot=red] at (midb) {};
            \draw [rounded corners, draw=black!10, very thick] (-0.25,0) rectangle (4.25,4);
        }
        =
        \tikz[vcenter, scale=0.4]{
            \squarecoord
            %
            \node[box] at (mid) {$\alpha$};
            \draw [rounded corners, draw=black!10, very thick] (-0.25,0) rectangle (4.25,4);
        }
        \\
        % Needle annihilation
        \tikz[vcenter, scale=0.4]{
            \squarecoord
            %
            \path
            (midt2) edge[string=red] (top)
            (midb2) edge[string=red] (bot);
            \draw[string=red] (midt2) let \p1 = ($0.5*(midt2) - 0.5*(midb2)$) in arc (90 : 360+90 : ({veclen(\x1,\y1)});); % circle with radius half distance from midt2 to midb2
            %
            \draw [rounded corners, draw=black!10, very thick] (-0.25,0) rectangle (4.25,4);
        }
        = 0
        \\
        % Polynomial forcing for S_2
        \tikz[vcenter, scale=0.4]{
            \squarecoord
            %
            \path
            (top) edge[string=red] (bot)
            ($(midt)+(-1,0)$) edge[string=red] ($(midb)+(-1,0)$);
            %
            \node [enddot=red] at ($(midt)+(-1,0)$) {};
            \node [enddot=red] at ($(midb)+(-1,0)$) {};
            \draw [rounded corners, draw=black!10, very thick] (-0.25,0) rectangle (4.25,4);
        }
        =
        2 \ \tikz[vcenter, scale=0.4]{
            \squarecoord
            %
            \path
            (midt) edge[string=red] (top)
            (midb) edge[string=red] (bot);
            %
            \node [enddot=red] at (midt) {};
            \node [enddot=red] at (midb) {};
            \draw [rounded corners, draw=black!10, very thick] (-0.25,0) rectangle (4.25,4);
        }
        -
        \tikz[vcenter, scale=0.4]{
            \squarecoord
            %
            \path
            (top) edge[string=red] (bot)
            ($(midt)+(1,0)$) edge[string=red] ($(midb)+(1,0)$);
            %
            \node [enddot=red] at ($(midt)+(1,0)$) {};
            \node [enddot=red] at ($(midb)+(1,0)$) {};
            \draw [rounded corners, draw=black!10, very thick] (-0.25,0) rectangle (4.25,4);
        }
    \end{gather}
\end{subequations}

To simplify the boxes, we can write
\begin{align}
    \tikz[vcenter, scale=0.4]{
        \squarecoord
        %
        \node[box] at (midl) {$f$};
        \node[box] at (midr) {$g$};
        \draw [rounded corners, draw=black!10, very thick] (-0.25,0) rectangle (4.25,4);
    }
    =
    \tikz[vcenter, scale=0.4]{
        \squarecoord
        %
        \node[box] at (mid) {$fg$};
        \draw [rounded corners, draw=black!10, very thick] (-0.25,0) rectangle (4.25,4);
    }
\end{align}
where $f,g$ and $fg$ are polynomials in $\alpha$. \red{Polynomials with what coefficients?}

The first four generators with the relations \eqref{subeqn:one-col-frob-unit} and \eqref{subeqn:one-col-frob-ass} describes a Frobenius algebra object structure on the object $s$. Here the generators correspond to the unit, counit, multiplication and comultiplication maps respectively.

\begin{example}
    Let us use the relations in \eqref{eq:one-col-sbim-rel} to simplify the following morphism in $\Hom((s,s),(s))$.
    \begin{align*}
        \tikz[vcenter, scale=0.4]{
            \squarecoord
            %
            \path
            % Top half
            (0,6) edge[string=red] (0,2)
            (0,4) edge[string=red] (3,4)
            (2,4) edge[string=red] (2,4.5)
            (2,4.5) edge[string=red] (1.5,5)
            (2,4.5) edge[string=red] (2.5,5)
            % Barbells
            (.75,2) edge[string=red] (.75,3)
            (2.25,1) edge[string=red] (2.25,2)
            ;
            % Bottom curve
            \draw[string=red] (0,2)
            to[out=270,in=180] (.75,1)
            to[out=0,in=270] (1.5,2)
            to[out=90,in=180] (2.25,3)
            to[out=0,in=90] (3,2);
            \path
            (0.75,1) edge[string=red] (0.75,0)
            (3,2) edge[string=red] (3,0)
            ;
            %
            \node [enddot=red] at (3,4) {};
            \node [enddot=red] at (1.5,5) {};
            \node [enddot=red] at (2.5,5) {};
            \node [enddot=red] at (.75,2) {};
            \node [enddot=red] at (.75,3) {};
            \node [enddot=red] at (2.25,1) {};
            \node [enddot=red] at (2.25,2) {};
            \draw [rounded corners, draw=black!10, very thick] (-0.5,0) rectangle (3.5,6);
        }
         & =
        \tikz[vcenter, scale=0.4]{
            \squarecoord
            %
            \path
            (top) edge[string=red] (midt)
            (2,0.75) edge[string=red] (2,1.75)
            (3.5,0.75) edge[string=red] (3.5,1.75)
            ;
            \draw[string=red] ($(bot) - (1,0)$)
            to[out=90,in=200] (midt)
            to[out=-20,in=90] ($(bot) + (1,0)$)
            ;
            %
            \node [enddot=red] at (2,0.75) {};
            \node [enddot=red] at (2,1.75) {};
            \node [enddot=red] at (3.5,0.75) {};
            \node [enddot=red] at (3.5,1.75) {};
            \draw [rounded corners, draw=black!10, very thick] (-0.25,0) rectangle (4.25,4);
        }    \\
         & =
        2 \ \tikz[vcenter, scale=0.4]{
            \squarecoord
            %
            \path
            (top) edge[string=red] (midt)
            (3.5,0.75) edge[string=red] (3.5,1.75)
            ;
            \draw[string=red] (1.1,1.75)
            to[out=80,in=200] (midt)
            to[out=-20,in=90] ($(bot) + (1,0)$)
            ;
            \path ($(bot) - (1,0)$) edge[string=red] (1,0.75);
            %
            \node [enddot=red] at (1,0.75) {};
            \node [enddot=red] at (1.1,1.75) {};
            \node [enddot=red] at (3.5,0.75) {};
            \node [enddot=red] at (3.5,1.75) {};
            \draw [rounded corners, draw=black!10, very thick] (-0.25,0) rectangle (4.25,4);
        }
        -
        \tikz[vcenter, scale=0.4]{
            \squarecoord
            %
            \path
            (top) edge[string=red] (midt)
            (0.5,0.75) edge[string=red] (0.5,1.75)
            (3.5,0.75) edge[string=red] (3.5,1.75)
            ;
            \draw[string=red] ($(bot) - (1,0)$)
            to[out=90,in=200] (midt)
            to[out=-20,in=90] ($(bot) + (1,0)$)
            ;
            %
            \node [enddot=red] at (0.5,0.75) {};
            \node [enddot=red] at (0.5,1.75) {};
            \node [enddot=red] at (3.5,0.75) {};
            \node [enddot=red] at (3.5,1.75) {};
            \draw [rounded corners, draw=black!10, very thick] (-0.25,0) rectangle (4.25,4);
        }    \\
         & =
        2 \ \tikz[vcenter, scale=0.4]{
            \squarecoord
            %
            \path
            (top) edge[string=red] (bot)
            ($(midt)+(1,0)$) edge[string=red] ($(midb)+(1,0)$)
            (1,0) edge[string=red] ($(midb)-(1,0)$)
            ;
            %
            \node [enddot=red] at ($(midt)+(1,0)$) {};
            \node [enddot=red] at ($(midb)+(1,0)$) {};
            \node [enddot=red] at ($(midb)-(1,0)$) {};
            \draw [rounded corners, draw=black!10, very thick] (-0.25,0) rectangle (4.25,4);
        }
        -
        \tikz[vcenter, scale=0.4]{
            \squarecoord
            %
            \path
            (top) edge[string=red] (mid)
            ($(midt)+(1.25,0)$) edge[string=red] ($(midb)+(1.25,0)$)
            ($(midt)-(1.25,0)$) edge[string=red] ($(midb)-(1.25,0)$)
            ;
            \draw[string=red] ($(bot) - (1,0)$)
            to[out=90,in=200] (mid)
            to[out=-20,in=90] ($(bot) + (1,0)$)
            ;
            %
            \node [enddot=red] at ($(midt)+(1.25,0)$) {};
            \node [enddot=red] at ($(midb)+(1.25,0)$) {};
            \node [enddot=red] at ($(midt)-(1.25,0)$) {};
            \node [enddot=red] at ($(midb)-(1.25,0)$) {};
            \draw [rounded corners, draw=black!10, very thick] (-0.25,0) rectangle (4.25,4);
        }.
    \end{align*}
\end{example}