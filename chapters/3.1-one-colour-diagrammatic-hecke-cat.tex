\section{One-colour Diagrammatic Hecke Category}

The first one-colour diagrammatic we explore is the one-colour diagrammatic Hecke category $\cH(S_2)$ for the symmetric group $S_2 = \angl{s \mathrel{|} s^2 = 1}$. % Afterwards we will see that this diagrammatic category leads to the category of Soergel Bimodules, which categorifies the Hecke algebra.

The objects of this category are generated by taking formal tensor products of the non-identity element $s \in S_2$. For example the tensor product of four $s$'s which denote with the expression $(s,s,s,s)$.

The morphisms in this category have a presentation in terms of generators and relations. For convenience, we will describe them up to isotopy. % TODO: mention isotopy in the background section
The generators are the following univalent and trivalent vertices, which can be rotated and flipped vertically using isotopy.
\begin{align} \label{eq:one-col-gen}
    \tikz[vcenter, scale=0.4]{
    \tikzfixsize{(0,0)}{(4,4)}
    \squarecoord
    %
    \path
    (bot) edge[string=red] (mid);
    \node [enddot=red] at (mid) {};
}
    \quad , \quad
    \tikz[vcenter, scale=0.4]{
    \tikzfixsize{(0,0)}{(4,4)}
    \squarecoord
    %
    \path
    (mid) edge[string=red] (top)
    (mid) edge[string=red] (botl)
    (mid) edge[string=red] (botr);
}
\end{align}
These morphisms are subject to the following local relations.
\begin{subequations} \label{eq:one-col-sbim-rel}
    \begin{gather}
        % Unit and counit
        \label{rel:one-col-frob-unit}
        \tikz[vcenter, scale=0.4]{
    \tikzfixsize{(0,0)}{(4,4)}
    \squarecoord
    %
    \path
    (top) edge[string=dRed] (bot)
    (mid) edge[string=dRed] ($(mid) + (1,0)$);
    \node [enddot=dRed] at ($(mid) + (1,0)$) {};
} = \tikz[vcenter, scale=0.4]{
    \tikzfixsize{(0,0)}{(4,4)}
    \squarecoord
    %
    \path
    (top) edge[string=red] (bot);
}
        \\
        % Frobenius associativity
        \label{rel:one-col-frob-ass}
        \tikz[vcenter, scale=0.4]{
    \tikzfixsize{(0,0)}{(4,4)}
    \squarecoord
    %
    \path
    (topl) edge[string=red] (midt)
    (topr) edge[string=red] (midt)
    (botl) edge[string=red] (midb)
    (botr) edge[string=red] (midb)
    (midt) edge[string=red] (midb);
} = \tikz[vcenter, scale=0.4]{
    \tikzfixsize{(0,0)}{(4,4)}
    \squarecoord
    %
    \path
    (topl) edge[string=red] ($(mid)+(-.75,0)$)
    (topr) edge[string=red] ($(mid)+(.75,0)$)
    (botl) edge[string=red] ($(mid)+(-.75,0)$)
    (botr) edge[string=red] ($(mid)+(.75,0)$)
    ($(mid)+(-.75,0)$) edge[string=red] ($(mid)+(.75,0)$);
}
        \\
        % Needle annihilation
        \label{rel:one-col-needle}
        \tikz[vcenter, scale=0.4]{
    \tikzfixsize{(0,0)}{(4,4)}
    \squarecoord
    %
    \path
    ($(mid)+(0,1)$) edge[string=red] (top)
    ($(mid)-(0,1)$) edge[string=red] (bot);
    \draw[string=red] ($(mid)+(0,1)$) arc (90 : 360+90 : 1);
    % circle with radius half distance from (mid + (0,1)) to (mid - (0,1))
} = 0
        \\
        % Polynomial forcing for S_2
        \label{rel:one-col-poly-forcing}
        \tikz[vcenter, scale=0.4]{
    \tikzfixsize{(0,0)}{(4,4)}
    \squarecoord
    %
    \path
    (top) edge[string=red] (bot)
    ($(mid)+(-1,.75)$) edge[string=red] ($(mid)+(-1,-.75)$);
    \node [enddot=red] at ($(mid)+(-1,.75)$) {};
    \node [enddot=red] at ($(mid)+(-1,-.75)$) {};
} = 2 \tikz[vcenter, scale=0.4]{
    \tikzfixsize{(0,0)}{(4,4)}
    \squarecoord
    %
    \path
    ($(mid)+(0,.75)$) edge[string=dRed] (top)
    ($(mid)+(0,-.75)$) edge[string=dRed] (bot);
    \node [enddot=dRed] at ($(mid)+(0,.75)$) {};
    \node [enddot=dRed] at ($(mid)+(0,-.75)$) {};
} - \tikz[vcenter, scale=0.4]{
    \tikzfixsize{(0,0)}{(4,4)}
    \squarecoord
    %
    \path
    (top) edge[string=red] (bot)
    ($(midt)+(1,0)$) edge[string=red] ($(midb)+(1,0)$);
    \node [enddot=red] at ($(midt)+(1,0)$) {};
    \node [enddot=red] at ($(midb)+(1,0)$) {};
}
    \end{gather}
\end{subequations}

\red{Mention that the morphisms are enriched over the category of $\Z$-modules. What is $0$? the "zero module"? Also is $\identity$ the identity morphism? in what context?}

\begin{remark}
    The object $s$ is a Frobenius algebra object in $\cH(S_2)$. The generators \eqref{eq:one-col-gen} and their horizontal reflections are the unit, multiplication, counit and comultiplication maps. The unit, associativity and Frobenius associativity axioms are satisfied by the relations \eqref{rel:one-col-frob-unit} and \eqref{rel:one-col-frob-ass}.
\end{remark}

\begin{example}
    \label{eg:one-col-relations}
    Let us use the relations in \eqref{eq:one-col-sbim-rel} to simplify the following morphism in $\Hom((s,s),(s))$.
    \begin{align*}
        \tikz[vcenter, scale=0.4]{
    \tikzfixsize{(0,0)}{(4,4)}
    \squarecoord
    %
    \path
    % Top half
    (4,6) edge[string=red] (4,2)
    (4,4) edge[string=red] (1,4)
    (2,4) edge[string=red] (2,4.5)
    (2,4.5) edge[string=red] (2.5,5)
    (2,4.5) edge[string=red] (1.5,5)
    % Barbells
    (3.25,2) edge[string=red] (3.25,3)
    (1.75,1) edge[string=red] (1.75,2)
    ;
    % Bottom curve
    \draw[string=red] (4,2)
    to[out=270,in=0] (3.25,1)
    to[out=180,in=270] (2.5,2)
    to[out=90,in=0] (1.75,3)
    to[out=180,in=90] (1,2);
    \path
    (3.25,1) edge[string=red] (3.25,0)
    (1,2) edge[string=red] (1,0)
    ;
    %
    \node [enddot=red] at (1,4) {};
    \node [enddot=red] at (2.5,5) {};
    \node [enddot=red] at (1.5,5) {};
    \node [enddot=red] at (3.25,2) {};
    \node [enddot=red] at (3.25,3) {};
    \node [enddot=red] at (1.75,1) {};
    \node [enddot=red] at (1.75,2) {};
}
         & = \tikz[vcenter, scale=0.4]{
    \tikzfixsize{(0,0)}{(4,4)}
    \squarecoord
    %
    \path
    (top) edge[string=dRed] ($(mid) + (0,.75)$)
    (2,0.75) edge[string=dRed] (2,1.75)
    (.5,0.75) edge[string=dRed] (.5,1.75)
    ;
    \draw[string=dRed] ($(bot) - (1,0)$)
    to[out=90,in=200] ($(mid) + (0,.75)$)
    to[out=-20,in=90] ($(bot) + (1,0)$)
    ;
    %
    \node [enddot=dRed] at (2,0.75) {};
    \node [enddot=dRed] at (2,1.75) {};
    \node [enddot=dRed] at (.5,0.75) {};
    \node [enddot=dRed] at (.5,1.75) {};
}
        \\ & = 2 \tikz[vcenter, scale=0.4]{
    \tikzfixsize{(0,0)}{(4,4)}
    \squarecoord
    %
    \path
    (top) edge[string=red] (midt)
    (.5,0.75) edge[string=red] (.5,1.75)
    ;
    \draw[string=red] ($(bot) - (1,0)$)
    to[out=90,in=200] (midt)
    to[out=-20,in=90] (2.9,1.75)
    ;
    \path ($(bot) + (1,0)$) edge[string=red] (3,0.75);
    %
    \node [enddot=red] at (3,0.75) {};
    \node [enddot=red] at (2.9,1.75) {};
    \node [enddot=red] at (.5,0.75) {};
    \node [enddot=red] at (.5,1.75) {};
} - \tikz[vcenter, scale=0.4]{
    \tikzfixsize{(0,0)}{(4,4)}
    \squarecoord
    %
    \path
    (top) edge[string=red] ($(mid) + (0,.75)$)
    (0.5,0.75) edge[string=red] (0.5,1.75)
    (3.5,0.75) edge[string=red] (3.5,1.75)
    ;
    \draw[string=red] ($(bot) - (1,0)$)
    to[out=90,in=200] ($(mid) + (0,.75)$)
    to[out=-20,in=90] ($(bot) + (1,0)$)
    ;
    %
    \node [enddot=red] at (0.5,0.75) {};
    \node [enddot=red] at (0.5,1.75) {};
    \node [enddot=red] at (3.5,0.75) {};
    \node [enddot=red] at (3.5,1.75) {};
}
        \\ & = 2 \tikz[vcenter, scale=0.4]{
    \tikzfixsize{(0,0)}{(4,4)}
    \squarecoord
    %
    \path
    (top) edge[string=red] (bot)
    ($(midt)-(1,0)$) edge[string=red] ($(midb)-(1,0)$)
    (3,0) edge[string=red] ($(midb)+(1,0)$)
    ;
    %
    \node [enddot=red] at ($(midt)-(1,0)$) {};
    \node [enddot=red] at ($(midb)+(1,0)$) {};
    \node [enddot=red] at ($(midb)-(1,0)$) {};
} - \tikz[vcenter, scale=0.4]{
    \tikzfixsize{(0,0)}{(4,4)}
    \squarecoord
    %
    \path
    (top) edge[string=red] (mid)
    ($(mid)+(1.25,.75)$) edge[string=red] ($(mid)+(1.25,-.75)$)
    ($(mid)+(-1.25,.75)$) edge[string=red] ($(mid)+(-1.25,-.75)$)
    ;
    \draw[string=red] ($(bot) - (0.75,0)$)
    to[out=90,in=180+20,looseness=0.8] (mid)
    to[out=-20,in=90,looseness=0.8] ($(bot) + (0.75,0)$)
    ;
    %
    \node [enddot=red] at ($(mid)+(1.25,.75)$) {};
    \node [enddot=red] at ($(mid)+(1.25,-.75)$) {};
    \node [enddot=red] at ($(mid)+(-1.25,.75)$) {};
    \node [enddot=red] at ($(mid)+(-1.25,-.75)$) {};
}
    \end{align*}
\end{example}

There is a right (or left) $\Z[\barbell]$-basis for $\Hom(s^n, s^m)$ described in \cite{elias-williamson-soergel-calculus} called the Double Leaves basis. To define this basis we must first look at morphisms known as Light leaves. Given a word $w = s^n$, a subexpression is a binary string of length $n$. For example, $0000, 0110$ and $1011$ are subexpressions of $s^4=ssss$. Given a subexpression $e$ of an object $w$, we can apply it to produce an element $w^e \in S_2$, e.g. $ssss^{1011} = s*1*s*s = s$. \red{Maybe use subscript here to avoid confusion with $s^n = ss...s$}. Each term of the subexpression is a decision of whether to include the corresponding $s$ in the word, where the decision to exclude an $s$ amounts to multiplying by $1$.

For a subexpression $e$ of an expression $w$, we can label each term by $U_0,U_1,D_0$ or $D_1$. The label is $U_*$ if the partial subexpression up to the current term evaluates to $1 \in S_2$ and $D_*$ if it evaluates to $s \in S_2$, where the subscript corresponds to the term in $e$.

\begin{example} \label{eg:one-col-light-leaf}
    For the object $ssss$ and subexpression $0101$, we can find the labels:
    \begin{center}
        \begin{tabular}{ |r||p{6em}|p{6em}|p{6em}|p{6em}| }
            \hline
            Choice        & 1     & 2         & 3             & 4                 \\ \hline
            Partial $w$   & $s$   & $ss$      & $sss$         & $ssss$            \\ \hline
            Partial $e$   & $0$   & $01$      & $010$         & $0101$            \\ \hline
            Partial $w^e$ & $1$   & $1*s = s$ & $1*s*1=s$     & $1*s*1*s=1$       \\ \hline
            Labels        & $U_0$ & $U_0 U_1$ & $U_0 U_1 D_0$ & $U_0 U_1 D_0 D_1$ \\ \hline
        \end{tabular}
    \end{center}
\end{example}

The light leaf $LL_{w,e} \in \Hom(w, w^e)$, corresponding to the object $w$ and subexpression $e$, is defined iteratively as follows. Let $LL_{\varnothing,\varnothing} = \varnothing$ be the empty diagram. Given $LL_{w',e'}$ and $i \in \{0,1\}$, $LL_{w's,e'i}$ is one of
\begin{equation}
    \tikz[vcenter, scale=0.4]{
    \tikzfixsize{(0,0)}{(4,4)}
    %
    \path
    (0.5,0) edge[string=dRed] (0.5,1)
    (3,0) edge[string=dRed] (3,1)
    (4,0) edge[string=dRed] (4,1.5)
    ;
    \node[enddot=dRed] at (4,1.5) {};
    \node[dRed] at (1.85, 0.5) {\dots};
    \node[below] at (4,0) {\small${U_0}$};
    %
    \draw[draw=black] (0,1) rectangle ++(3.5,2) node[pos=.5] {$LL_{w',e'}$};
} \:,\:
    \tikz[vcenter, scale=0.4]{
    \tikzfixsize{(0,0)}{(4,4)}
    %
    \path
    (0.5,0) edge[string=dRed] (0.5,1)
    (3,0) edge[string=dRed] (3,1)
    (4,0) edge[string=dRed] (4,4)
    ;
    \node[dRed] at (1.85, 0.5) {\dots};
    \node[below] at (4,0) {\small${U_1}$};
    %
    \draw[draw=black] (0,1) rectangle ++(3.5,2) node[pos=.5] {$LL_{w',e'}$};
} \:,\:
    \tikz[vcenter, scale=0.4]{
    \tikzfixsize{(0,0)}{(4,4)}
    %
    \path
    (0.5,0) edge[string=dRed] (0.5,0.75)
    (3,0) edge[string=dRed] (3,0.75)
    (4,0) edge[string=dRed] (4,2.75)
    ;
    \diagmult[dRed]{3,2.75}{3.5,3.25}{4,2.75}{3.5,4};
    ;
    \node[dRed] at (1.85, 0.375) {\,\dots};
    \node[below] at (4,0) {\small${D_0}$};
    %
    \draw[draw=black] (0,0.75) rectangle ++(3.5,2) node[pos=.5] {$LL_{w',e'}$};
} \:,\:
    \tikz[vcenter, scale=0.4]{
    \tikzfixsize{(0,0)}{(4,4)}
    %
    \path
    (0.5,0) edge[string=dRed] (0.5,0.75)
    (3,0) edge[string=dRed] (3,0.75)
    (4,0) edge[string=dRed] (4,2.75)
    ;
    \diagcap[dRed]{3,2.75}{3.5,3.25}{4,2.75};
    ;
    \node[dRed] at (1.75, 0.375) {\,\dots};
    \coordinate[label=below:\small${D_1}$] () at (4,0);
    %
    \draw[draw=black] (0,0.75) rectangle ++(3.5,2) node[pos=.5] {$LL_{w',e'}$};
}
\end{equation}
depending on the next label, where $w'$ and $e'$ are appropriate subwords of $w$ and $e$. Observe that the codomain of a light leaf $LL_{w,e}$ corresponds to the evaluation $w^e \in S_2$ of the subexpression. The recursive definition is consistent, since if the next label is $U_*$ then the codomain of $LL_{w',e'}$ (the evaluation of the partial subexpression $w'^{e'}$ up to the label) is $1$, and when the next label is $D_*$ the codomain of $LL_{w',e'}$ is $s$. \red{Rewrite this to make sense.}
\red{Do we need to talk about 'degree' of light leaves?}

\begin{example}
    Following from Example \eqref{eg:one-col-light-leaf} for $w = ssss$ and $e = 0101$, we have labels $U_0 U_1 D_0 D_1$ so the light leaf $LL_{w,e}$ is built as follows.
    \[
        \varnothing
        \to \tikz[vcenter, scale=0.4]{
    \tikzfixsize{(0,0)}{(4,4)}
    %
    \path
    (2,0) edge[string=dRed] (2,2)
    ;
    \node[enddot=dRed] at (2,2) {};
    \coordinate[label=below:\small${U_0}$] () at (2,0);
}
        \to \tikz[vcenter, scale=0.4]{
    \tikzfixsize{(0,0)}{(4,4)}
    %
    \path
    (1.5,0) edge[string=red] (1.5,2)
    (2.5,0) edge[string=red] (2.5,4)
    ;
    \node[enddot=red] at (1.5,2) {};
    \coordinate[label=below:\small${U_1}$] () at (2.5,0);
}
        \to \tikz[vcenter, scale=0.4]{
    \tikzfixsize{(0,0)}{(4,4)}
    %
    \path
    (1,0) edge[string=dRed] (1,1.75)
    (2,0) edge[string=dRed] (2,2)
    (3,0) edge[string=dRed] (3,2)
    ;
    \diagmult[dRed]{2,2}{2.5,2.5}{3,2}{2.5,4};
    \node[enddot=dRed] at (1,1.75) {};
    \node[below] at (3,0) {\small${D_0}$};
}
        \to \tikz[vcenter, scale=0.4]{
    \tikzfixsize{(0,0)}{(5,4)}
    %
    \path
    (1,0) edge[string=dRed] (1,1.5)
    (2,0) edge[string=dRed] (2,1.75)
    (3,0) edge[string=dRed] (3,1.75)
    (4,0) edge[string=dRed] (4,3)
    ;
    \diagmult[dRed]{2,1.75}{2.5,2.25}{3,1.75}{2.5,3};
    \diagcap[dRed]{2.5,3}{3.25, 3.75}{4,3};
    \node[enddot=dRed] at (1,1.5) {};
    \node[below] at (4,0) {\small${D_1}$};
}
    \]
\end{example}

Let $\ol{LL}_{w,e}$ denote the vertical reflection of $LL_{w,e}$. A double leaf associated to expressions $w,y$ is a composition
\[
    \mathbb{LL}_{f,e} \coloneqq \ol{LL}_{y,f} \circ LL_{w,e} : w \to y
\]
for subexpressions $e$ of $w$ and $f$ of $y$ such that $w^e = f^y$. Visually this looks like a morphism from $w$ to $y$ factoring through $w^e = y^f \in \{1,s\}$,
\[
    \tikz[vcenter, scale=0.4]{
    % \tikzfixsize{(0,0)}{(4,4)}
    %
    \draw (0,0) -- ++(5,0) -- ++(-1,2) -- ++(-3,0) -- cycle;
    \draw (0,4) -- ++(5,0) -- ++(-1,-2) -- ++(-3,0) -- cycle;
    \node at (2.5,1) {$LL_{w,e}$};
    \node at (2.5,3) {$\ol{LL}_{y,f}$};
    \node[right] at (5,4) {\small$y$};
    \node[right] at (5,0) {\small$w$};
    \node[right] at (4,2) {\small$\psi\phi(w(e)) = \psi\phi(y(f))$};
}.
\]
\red{Maybe give some indication of why we use trapeziums}

\begin{example}
    Let $w = ssss$ and $y = sss$. Let $e=0111$ be a subexpression of $w$, and $f=010$ be a subexpression of $y$. The corresponding light leaves are
    \[
        LL_{w,e} = \tikz[vcenter, scale=0.4]{
    \tikzfixsize{(0,0)}{(5,4)}
    %
    \path
    (1,0) edge[string=red] (1,2)
    (2,0) edge[string=red] (2,2)
    (3,0) edge[string=red] (3,2)
    (4,0) edge[string=red] (4,4)
    ;
    \diagcap[red]{2,2}{2.5,2.5}{3,2};
    \node[enddot=red] at (1,2) {};
    \coordinate[label=below:\tiny${U_0}$] () at (1,0);
    \coordinate[label=below:\tiny${U_1}$] () at (2,0);
    \coordinate[label=below:\tiny${D_1}$] () at (3,0);
    \coordinate[label=below:\tiny${U_1}$] () at (4,0);
}
        \text{ and }
        LL_{y,f} = \tikz[vcenter, scale=0.4]{
    \tikzfixsize{(0,0)}{(4,4)}
    %
    \path
    (1,0) edge[string=red] (1,2)
    (2,0) edge[string=red] (2,2)
    (3,0) edge[string=red] (3,2)
    ;
    \diagmult[red]{2,2}{2.5,2.5}{3,2}{2.5,4};
    \node[enddot=red] at (1,2) {};
    \coordinate[label=below:\tiny${U_0}$] () at (1,0);
    \coordinate[label=below:\tiny${U_1}$] () at (2,0);
    \coordinate[label=below:\tiny${D_0}$] () at (3,0);
}.
    \]
    Then the double leaf $\mathbb{LL}_{f,e} = \ol{LL}_{y,f} \circ LL_{w,e} : w \to y$, factoring through $s$, is
    \[
        \tikz[vcenter, scale=0.4]{
    \tikzfixsize{(0,0)}{(3.5,4)}
    %
    % Bottom
    \path
    (0,0) edge[string=dRed] (0,1)
    (1,0) edge[string=dRed] (1,1)
    (2,0) edge[string=dRed] (2,1)
    (3,0) edge[string=dRed] (3,2)
    ;
    \diagcap[dRed]{1,1}{1.5,1.5}{2,1};
    \node[enddot=dRed] at (0,1) {};
    % Top
    \path
    (1.5,4) edge[string=dRed] (1.5,3)
    (2.5,4) edge[string=dRed] (2.5,3)
    (3.5,4) edge[string=dRed] (3.5,3)
    ;
    \diagcomult[dRed]{2.5,3}{3,2.5}{3.5,3}{3,2};
    \node[enddot=dRed] at (1.5,3) {};
    % Text
    \node at (-2,3) {\small$\overline{LL}_{y,f}$};
    \node at (-2,1) {\small$LL_{w,e}$};
    % \coordinate[label={[yshift=-3pt]center:\tiny$\dRed{s}$}] () at (0,0);
    % \coordinate[label={[yshift=-3pt]center:\tiny$\dRed{s}$}] () at (1,0);
    % \coordinate[label={[yshift=-3pt]center:\tiny$\dRed{s}$}] () at (2,0);
    % \coordinate[label={[yshift=-3pt]center:\tiny$\dRed{s}$}] () at (3,0);
    % \coordinate[label={[yshift=3pt]center:\tiny$\dRed{s}$}] () at (1.5,4);
    % \coordinate[label={[yshift=3pt]center:\tiny$\dRed{s}$}] () at (3.5,4);
    % \coordinate[label={[yshift=3pt]center:\tiny$\dRed{s}$}] () at (2.5,4);
    %
    \draw[dashed, black!50] (-0.5,2) -- (4,2);
}.
    \]
\end{example}

Notice that these double leaves have no floating diagrams such as $\barbell$. In order for these double leaves to be a basis for a morphism space, we insert these floating diagrams by taking linear combinations as a right $\Z[\barbell]$-module. Here, the right $\barbell$-action on a diagram is just concatenation by $\barbell$ on the right. Since we can move barbells to the left, via. the relation \eqref{rel:one-col-poly-forcing}, we can equivalently act by $\Z[\barbell]$ on the left. \red{Why do we default to right module?} This leads us to the following theorem.

\begin{theorem}[Elias-Williamson \cite{elias-williamson-soergel-calculus}, Theorem 1.2]
    \label{thm:one-col-double-leaves-basis}
    Given objects $w,y$ in $\cH(S_2)$, let \blue{$\mathbb{LL}_{w,y}$} \footnote{\blue{this can be confused with the double leaves themselves, maybe write $\mathbb{LL}(w,y)$}} be the collection of double leaves $\mathbb{LL}_{f,e}$ for subexpressions $e$ of $w$ and $f$ of $y$, such that $w^e = y^f$. Then \blue{$\mathbb{LL}_{w,y}$} is a right (or left) $\Z[\barbell]$-module basis for $\Hom(w,y)$.
\end{theorem}
The purely diagrammatic proof (of a more general theorem) can be found in \cite{elias-williamson-soergel-calculus}.
\red{What is it used for?}

\begin{remark}
    The above light leaves and double leaves, introduced in \cite{elias-williamson-soergel-calculus}, are diagrammatic analogues of Libedinsky's work in \cite{libedinsky-lightleavesbasis} and \cite{libedinsky-doubleleavesbasis}.
\end{remark}

