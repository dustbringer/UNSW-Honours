\section{One-colour Diagrammatic Hecke Category}

The first one-colour diagrammatic we explore is the one-colour diagrammatic Hecke category $\cH(S_2)$ for the symmetric group $S_2 = \angl{s \mathrel{|} s^2 = 1}$. % Afterwards we will see that this diagrammatic category leads to the category of Soergel Bimodules, which categorifies the Hecke algebra.

The objects of this category are generated by taking formal tensor products of the non-identity element $s \in S_2$. For example the tensor product of four $s$'s which denote with the expression $(s,s,s,s)$.

The morphisms in this category have a presentation in terms of generators and relations. For convenience, we will describe them up to isotopy. % TODO: mention isotopy in the background section
The generators are the following univalent and trivalent vertices, which can be rotated and flipped vertically using isotopy.
\begin{align} \label{eq:one-col-gen}
    \tikz[vcenter, scale=0.4]{
    \tikzfixsize{(0,0)}{(4,4)}
    \squarecoord
    %
    \path
    (bot) edge[string=red] (mid);
    \node [enddot=red] at (mid) {};
}
    \quad , \quad
    \tikz[vcenter, scale=0.4]{
    \tikzfixsize{(0,0)}{(4,4)}
    \squarecoord
    %
    \path
    (mid) edge[string=red] (top)
    (mid) edge[string=red] (botl)
    (mid) edge[string=red] (botr);
}
\end{align}
These morphisms are subject to the following local relations.
\begin{subequations} \label{eq:one-col-sbim-rel}
    \begin{gather}
        % Unit and counit
        \label{subeqn:one-col-frob-unit}
        \tikz[vcenter, scale=0.4]{
    \tikzfixsize{(0,0)}{(4,4)}
    \squarecoord
    %
    \path
    (top) edge[string=dRed] (bot)
    (mid) edge[string=dRed] ($(mid) + (1,0)$);
    \node [enddot=dRed] at ($(mid) + (1,0)$) {};
} = \tikz[vcenter, scale=0.4]{
    \tikzfixsize{(0,0)}{(4,4)}
    \squarecoord
    %
    \path
    (top) edge[string=red] (bot);
}
        \\
        % Frobenius associativity
        \label{subeqn:one-col-frob-ass}
        \tikz[vcenter, scale=0.4]{
    \tikzfixsize{(0,0)}{(4,4)}
    \squarecoord
    %
    \path
    (topl) edge[string=red] (midt)
    (topr) edge[string=red] (midt)
    (botl) edge[string=red] (midb)
    (botr) edge[string=red] (midb)
    (midt) edge[string=red] (midb);
} = \tikz[vcenter, scale=0.4]{
    \tikzfixsize{(0,0)}{(4,4)}
    \squarecoord
    %
    \path
    (topl) edge[string=red] ($(mid)+(-.75,0)$)
    (topr) edge[string=red] ($(mid)+(.75,0)$)
    (botl) edge[string=red] ($(mid)+(-.75,0)$)
    (botr) edge[string=red] ($(mid)+(.75,0)$)
    ($(mid)+(-.75,0)$) edge[string=red] ($(mid)+(.75,0)$);
}
        \\
        % Needle annihilation
        \tikz[vcenter, scale=0.4]{
    \tikzfixsize{(0,0)}{(4,4)}
    \squarecoord
    %
    \path
    ($(mid)+(0,1)$) edge[string=red] (top)
    ($(mid)-(0,1)$) edge[string=red] (bot);
    \draw[string=red] ($(mid)+(0,1)$) arc (90 : 360+90 : 1);
    % circle with radius half distance from (mid + (0,1)) to (mid - (0,1))
} = 0
        \\
        % Polynomial forcing for S_2
        \tikz[vcenter, scale=0.4]{
    \tikzfixsize{(0,0)}{(4,4)}
    \squarecoord
    %
    \path
    (top) edge[string=red] (bot)
    ($(mid)+(-1,.75)$) edge[string=red] ($(mid)+(-1,-.75)$);
    \node [enddot=red] at ($(mid)+(-1,.75)$) {};
    \node [enddot=red] at ($(mid)+(-1,-.75)$) {};
} = 2 \tikz[vcenter, scale=0.4]{
    \tikzfixsize{(0,0)}{(4,4)}
    \squarecoord
    %
    \path
    ($(mid)+(0,.75)$) edge[string=dRed] (top)
    ($(mid)+(0,-.75)$) edge[string=dRed] (bot);
    \node [enddot=dRed] at ($(mid)+(0,.75)$) {};
    \node [enddot=dRed] at ($(mid)+(0,-.75)$) {};
} - \tikz[vcenter, scale=0.4]{
    \tikzfixsize{(0,0)}{(4,4)}
    \squarecoord
    %
    \path
    (top) edge[string=red] (bot)
    ($(midt)+(1,0)$) edge[string=red] ($(midb)+(1,0)$);
    \node [enddot=red] at ($(midt)+(1,0)$) {};
    \node [enddot=red] at ($(midb)+(1,0)$) {};
}
    \end{gather}
\end{subequations}

\begin{remark}
    The object $s$ is a Frobenius algebra object in $\cH(S_2)$. The generators \eqref{eq:one-col-gen} and their horizontal reflections are the unit, multiplication, counit and comultiplication maps. The unit, associativity and Frobenius associativity axioms are satisfied by the relations \eqref{subeqn:one-col-frob-unit} and \eqref{subeqn:one-col-frob-ass}.
\end{remark}

\begin{example}
    Let us use the relations in \eqref{eq:one-col-sbim-rel} to simplify the following morphism in $\Hom((s,s),(s))$.
    \begin{align*}
        \tikz[vcenter, scale=0.4]{
    \tikzfixsize{(0,0)}{(4,4)}
    \squarecoord
    %
    \path
    % Top half
    (4,6) edge[string=red] (4,2)
    (4,4) edge[string=red] (1,4)
    (2,4) edge[string=red] (2,4.5)
    (2,4.5) edge[string=red] (2.5,5)
    (2,4.5) edge[string=red] (1.5,5)
    % Barbells
    (3.25,2) edge[string=red] (3.25,3)
    (1.75,1) edge[string=red] (1.75,2)
    ;
    % Bottom curve
    \draw[string=red] (4,2)
    to[out=270,in=0] (3.25,1)
    to[out=180,in=270] (2.5,2)
    to[out=90,in=0] (1.75,3)
    to[out=180,in=90] (1,2);
    \path
    (3.25,1) edge[string=red] (3.25,0)
    (1,2) edge[string=red] (1,0)
    ;
    %
    \node [enddot=red] at (1,4) {};
    \node [enddot=red] at (2.5,5) {};
    \node [enddot=red] at (1.5,5) {};
    \node [enddot=red] at (3.25,2) {};
    \node [enddot=red] at (3.25,3) {};
    \node [enddot=red] at (1.75,1) {};
    \node [enddot=red] at (1.75,2) {};
}
         & = \tikz[vcenter, scale=0.4]{
    \tikzfixsize{(0,0)}{(4,4)}
    \squarecoord
    %
    \path
    (top) edge[string=dRed] ($(mid) + (0,.75)$)
    (2,0.75) edge[string=dRed] (2,1.75)
    (.5,0.75) edge[string=dRed] (.5,1.75)
    ;
    \draw[string=dRed] ($(bot) - (1,0)$)
    to[out=90,in=200] ($(mid) + (0,.75)$)
    to[out=-20,in=90] ($(bot) + (1,0)$)
    ;
    %
    \node [enddot=dRed] at (2,0.75) {};
    \node [enddot=dRed] at (2,1.75) {};
    \node [enddot=dRed] at (.5,0.75) {};
    \node [enddot=dRed] at (.5,1.75) {};
}
        \\ & = 2 \tikz[vcenter, scale=0.4]{
    \tikzfixsize{(0,0)}{(4,4)}
    \squarecoord
    %
    \path
    (top) edge[string=red] (midt)
    (.5,0.75) edge[string=red] (.5,1.75)
    ;
    \draw[string=red] ($(bot) - (1,0)$)
    to[out=90,in=200] (midt)
    to[out=-20,in=90] (2.9,1.75)
    ;
    \path ($(bot) + (1,0)$) edge[string=red] (3,0.75);
    %
    \node [enddot=red] at (3,0.75) {};
    \node [enddot=red] at (2.9,1.75) {};
    \node [enddot=red] at (.5,0.75) {};
    \node [enddot=red] at (.5,1.75) {};
} - \tikz[vcenter, scale=0.4]{
    \tikzfixsize{(0,0)}{(4,4)}
    \squarecoord
    %
    \path
    (top) edge[string=red] ($(mid) + (0,.75)$)
    (0.5,0.75) edge[string=red] (0.5,1.75)
    (3.5,0.75) edge[string=red] (3.5,1.75)
    ;
    \draw[string=red] ($(bot) - (1,0)$)
    to[out=90,in=200] ($(mid) + (0,.75)$)
    to[out=-20,in=90] ($(bot) + (1,0)$)
    ;
    %
    \node [enddot=red] at (0.5,0.75) {};
    \node [enddot=red] at (0.5,1.75) {};
    \node [enddot=red] at (3.5,0.75) {};
    \node [enddot=red] at (3.5,1.75) {};
}
        \\ & = 2 \tikz[vcenter, scale=0.4]{
    \tikzfixsize{(0,0)}{(4,4)}
    \squarecoord
    %
    \path
    (top) edge[string=red] (bot)
    ($(midt)-(1,0)$) edge[string=red] ($(midb)-(1,0)$)
    (3,0) edge[string=red] ($(midb)+(1,0)$)
    ;
    %
    \node [enddot=red] at ($(midt)-(1,0)$) {};
    \node [enddot=red] at ($(midb)+(1,0)$) {};
    \node [enddot=red] at ($(midb)-(1,0)$) {};
} - \tikz[vcenter, scale=0.4]{
    \tikzfixsize{(0,0)}{(4,4)}
    \squarecoord
    %
    \path
    (top) edge[string=red] (mid)
    ($(mid)+(1.25,.75)$) edge[string=red] ($(mid)+(1.25,-.75)$)
    ($(mid)+(-1.25,.75)$) edge[string=red] ($(mid)+(-1.25,-.75)$)
    ;
    \draw[string=red] ($(bot) - (0.75,0)$)
    to[out=90,in=180+20,looseness=0.8] (mid)
    to[out=-20,in=90,looseness=0.8] ($(bot) + (0.75,0)$)
    ;
    %
    \node [enddot=red] at ($(mid)+(1.25,.75)$) {};
    \node [enddot=red] at ($(mid)+(1.25,-.75)$) {};
    \node [enddot=red] at ($(mid)+(-1.25,.75)$) {};
    \node [enddot=red] at ($(mid)+(-1.25,-.75)$) {};
}.
    \end{align*}
\end{example}


There is a $\Z[\barbell]$-basis for $\Hom(s^n, s^m)$ introduced by Libedinsky \cite{libedinsky-doubleleavesbasis} called the Double Leaves basis. We first define morphisms known as Light leaves. Given a word $w = s^n$, a subexpression is a binary string of length $n$. For example, $0000, 0110$ and $1011$ are subexpressions of $s^4=ssss$. Given a subexpression $e$ of an object $w$, we can apply it for an element $w^e \in S_2$, e.g. $ssss^{1011} = s*1*s*s = s$. \red{Maybe use subscript here to avoid confusion with $s^n = ss...s$}. Each term of the subexpression is a decision of whether to include the corresponding $s$ in the word, where the decision to exclude an $s$ amounts to multiplying by $1$.

For a subexpression $e$ of an expression $w$, we can label each term by $U_0,U_1,D_0$ or $D_1$. The label is $U_*$ if the partial subexpression up to the current term evaluates to $1 \in S_2$ and $D_*$ if it evaluates to $s \in S_2$, where the subscript corresponds to the term in $e$.

\begin{example}
    For the object $ssss$ and subexpression $0101$, we can find the labels.
    \begin{center}
        \begin{tabular}{ |r||p{6em}|p{6em}|p{6em}|p{6em}| }
            \hline
            Choice        & 1     & 2         & 3             & 4                 \\ \hline
            Partial $w$   & $s$   & $ss$      & $sss$         & $ssss$            \\ \hline
            Partial $e$   & $0$   & $01$      & $010$         & $0101$            \\ \hline
            Partial $w^e$ & $1$   & $1*s = s$ & $1*s*1=s$     & $1*s*1*s=1$       \\ \hline
            Labels        & $U_0$ & $U_0 U_1$ & $U_0 U_1 D_0$ & $U_0 U_1 D_0 D_1$ \\ \hline
        \end{tabular}
    \end{center}
\end{example}

The light leaf $LL_{w,e} \in \Hom(w, w^e)$, corresponding to the object $w$ and subexpression $e$, is defined iteratively as follows. Let $LL_{\varnothing,\varnothing} = \varnothing$ be the empty diagram. Given $LL_{w',e'}$ and $i \in \{0,1\}$, $LL_{w's,e'i}$ is one of
\begin{equation}
    \input{tikz/3.1/light-leaves-u0.tex} \:,\:
    \input{tikz/3.1/light-leaves-u1.tex} \:,\:
    \input{tikz/3.1/light-leaves-d0.tex} \:,\:
    \input{tikz/3.1/light-leaves-d1.tex}
\end{equation}
depending on the next label, where $w'$ and $e'$ are appropriate subwords of $w$ and $e$.





