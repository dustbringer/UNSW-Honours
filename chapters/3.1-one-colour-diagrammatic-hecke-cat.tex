\chapter{One-colour Diagrammatics}

\section{One-colour Diagrammatic Hecke Category}

In this section, we describe one-colour diagrammatics for morphisms in $\bsbim$. The morphisms in this category have a presentation in terms of generators and relations. % TODO: mention isotopy at the end, and stop putting in isotopy relations

The generators are the following univalent and trivalent vertices, along with boxes where $f$ is a homogeneous polynomial in $R$.
\begin{align}
    \tikz[vcenter, scale=0.4]{
        \coordinate (mid) at (2,2);
        \coordinate (bot) at (2,0);
        %
        \path
        (bot) edge[very thick, red] (mid);
        %
        \node [enddot=red] at (mid) {};
        \draw[rounded corners, draw=black!10, very thick] (-0.25,0) rectangle (4.25,4);
    }
    \quad , \quad
    \tikz[vcenter, scale=0.4]{
        \coordinate (mid) at (2,2);
        \coordinate (top) at (2,4);
        %
        \path
        (top) edge[very thick, red] (mid);
        %
        \node [enddot=red] at (mid) {};
        \draw [rounded corners, draw=black!10, very thick] (-0.25,0) rectangle (4.25,4);
    }
    \quad , \quad
    \tikz[vcenter, scale=0.4]{
        \coordinate (mid) at (2,2);
        \coordinate (bot) at (2,0);
        \coordinate (top) at (2,4);
        \coordinate (topl) at (0,4);
        \coordinate (topr) at (4,4);
        %
        \path
        (mid) edge[very thick, red] (bot)
        (mid) edge[very thick, red] (topl)
        (mid) edge[very thick, red] (topr);
        %
        \draw[rounded corners, draw=black!10, very thick] (-0.25,0) rectangle (4.25,4);
    }
    \quad , \quad
    \tikz[vcenter, scale=0.4]{
        \coordinate (mid) at (2,2);
        \coordinate (bot) at (2,0);
        \coordinate (top) at (2,4);
        \coordinate (botl) at (0,0);
        \coordinate (botr) at (4,0);
        %
        \path
        (mid) edge[very thick, red] (top)
        (mid) edge[very thick, red] (botl)
        (mid) edge[very thick, red] (botr);
        %
        \draw[rounded corners, draw=black!10, very thick] (-0.25,0) rectangle (4.25,4);
    }\quad , \quad
    \tikz[vcenter, scale=0.4]{
        \coordinate (mid) at (2,2);
        %
        \node[draw, rounded corners] at (mid) {$f$};
        \draw [rounded corners, draw=black!10, very thick] (-0.25,0) rectangle (4.25,4);
    }
\end{align}

These are the unit, multiplication, counit and comultiplication maps from the Frobenius algebra structure of $B_s \in \bsbim$.

