\section{One-colour Diagrammatic Hecke Category}

The first one-colour diagrammatic we explore is the \textit{one-colour (diagrammatic) Hecke category} $\cH(S_2)$ for the symmetric group $S_2 = \angl{s \mathrel{|} s^2 = e}$. At the end of this section, we see that this diagrammatic category is equivalent to the category of Soergel Bimodules under additive Karoubian closure.

\begin{remark}
    All diagrammatics below and in \autoref{chapter:two-col-diagrammatics} can be defined in the language of planar algebras, without the additional structure of categories, e.g. in \cite{jones-planar-algebra}. Nevertheless, we define them in the context of categories as we will see them as diagrammatic versions of important categories in representation theory.
\end{remark}

\begin{definition}
    The \textit{one-colour (diagrammatic) Hecke category} $\cH(S_2)$ is a $\C$-linear monoidal category with the following presentation.

    The objects are generated by taking formal tensor products of the non-identity element $s \in S_2$. We will write these objects as words, e.g. $s$, $ssss \eqqcolon s^4$, $sssssss \eqqcolon s^7$, where the tensor product is concatenation. The empty tensor product, i.e. the monoidal identity, will be denoted $\vn \eqqcolon s^0$.


    % TODO: mention isotopy in the background section
    The morphisms are generated, up to isotopy, by univalent and trivalent vertices
    \begin{align} \label{eq:one-col-gen}
        \inputtikz{3.1/generator1}
        \quad , \quad
        \inputtikz{3.1/generator2}
    \end{align}
    that are maps $s \to \vn$ and $ss \to s$ respectively. Note that we put a large dot on univalent vertices to signify that the line stops abruptly and does not connect to the top. The composition of such diagrams is appropriate vertical stacking, and the tensor product is horizontal concatenation (without intersection). The free $\C$-module structure on each morphism space $\Hom(s^n, s^m)$ produces $\C$-linear combinations of such diagrams. \red{Something about composition/tensor and addition commuting} Then, composition or tensors with the zero morphism $0$ result in $0$. To abuse notation, the empty diagram $\vn \to \vn$ will be denoted $\vn$. The identity morphism in $\Hom(s^n, s^n)$ is the diagram consisting of $n$ (red) vertical lines
    \begin{gather}
        \inputtikz{3.1/identity-diagram},
    \end{gather}
    which we may identify with $s^n$.


    % TODO: Something about the line being red, and not labelling domain and codomain (using identity strand as the label)

    Such diagrams are subject to the following local relations
    \begin{subequations} \label{eq:one-col-hecke-rel}
        \begin{gather}
            % Unit and counit
            \label{rel:one-col-frob-unit}
            \inputtikz{3.1/relation1-frob-unit} = \inputtikz{3.1/relation1-identity},
            \\
            % Frobenius associativity
            \label{rel:one-col-frob-ass}
            \inputtikz{3.1/relation2-frob-ass1} = \inputtikz{3.1/relation2-frob-ass2},
            \\
            % Needle annihilation
            \label{rel:one-col-needle}
            \inputtikz{3.1/relation3-needle} = 0,
            \\
            % Polynomial forcing for S_2
            \label{rel:one-col-barbell-forcing}
            \inputtikz{3.1/relation4-left-bar} = 2 \inputtikz{3.1/relation4-split-line} - \inputtikz{3.1/relation4-right-bar}.
        \end{gather}
    \end{subequations}
\end{definition}


\begin{remark}
    The object $s$ is a Frobenius object in $\cH(S_2)$. The generators \eqref{eq:one-col-gen} and their horizontal reflections are the unit, multiplication, counit and comultiplication maps. The unit, associativity and Frobenius associativity axioms are satisfied by the relations \eqref{rel:one-col-frob-unit} and \eqref{rel:one-col-frob-ass}.
\end{remark}
\red{Put a definition of frob object in intro}

\begin{example}
    \label{eg:one-col-relations}
    Using the relations in \eqref{eq:one-col-hecke-rel} we can simplify the morphism in $\Hom(ss,s)$,
    \begin{align*}
        \inputtikz{3.1/relation-example1}
         & = \inputtikz{3.1/relation-example2}
        \\ & = 2 \inputtikz{3.1/relation-example3.1} - \inputtikz{3.1/relation-example3.2}
        \\ & = 2 \inputtikz{3.1/relation-example4.1} - \inputtikz{3.1/relation-example4.2}.
    \end{align*}
\end{example}

\red{Add example of using frob associativity}

The morphism space $\Hom(s^n, s^m)$ has a left (or right) $\C[\barbell]$-basis called the \textit{double leaves} basis, as described in \cite{elias-williamson-soergel-calculus}. To define this basis, we must first define morphisms known as \textit{light leaves}.


Define $\phi: (\op{ob}(\cH(S_2)), \tensor) \to (S_2, *)$ to be the monoid homomorphism\footnote{A map that preserves the monoidal product and identity element.} mapping $s \mapsto s$ and $\vn \mapsto 1$, and $\psi: S_2 \to \op{ob}(\cH(S_2))$ to be the function that maps $s \to s$ and $1 \to \vn$. \red{Should this be a definition?}
The maps $\phi$ and $\psi$ allow words $w = s^n$ to be seen as elements of $S_2$, and $1,s \in S_2$ to be seen as the objects $\vn, s \in \cH(S_2)$. Clearly, $\phi \psi$ is the identity map on $S_2$, and the map $\psi \phi: \cH(S_2) \to \cH(S_2)$ takes objects to one of $\vn$ or $s$ in $\cH(S_2)$ by considering them as elements in $S_2$.

\begin{definition}(Subexpression for $S_2$)
    Given a word $w = s^n$, a \textit{subexpression} $e$ is a binary string of length $n$. We can \textit{apply} a subexpression to produce an object $w(e) \in \cH(S_2)$, which is $w$ where terms corresponding to $0$ in $e$ are replaced with $\vn$.
\end{definition}
% TODO: \red{Maybe define expressions and subexpressions for general groups}

For example, $0000, 0110$ and $1011$ are subexpressions of $s^4=ssss$. Applying the third subexpression gives $ssss(1011) = s\vn ss = sss$, by strictness of the monoidal category. Here, each term of the subexpression is a decision to include or exclude the corresponding $s$ in the word, where excluding an $s$ amounts to tensoring with $\vn$. 

For a word $w$ and subexpression $e$, we label each term by $U_0,U_1,D_0$ or $D_1$. A term is labelled $U_*$ if $\phi$ applied to the partial subexpression up to the current term is $1 \in S_2$, and labelled $D_*$ if it evaluates to $s \in S_2$. The label's subscript is the corresponding term in $e$.

\begin{example} \label{eg:one-col-light-leaf}
    For the object $ssss$ and subexpression $0101$, we can find the labels:
    \begin{center}
        \begin{tabular}{ |r||p{6em}|p{6em}|p{6em}|p{6em}| }
            \hline
            Choice         & 1     & 2         & 3             & 4                 \\ \hline
            Partial $w$    & $s$   & $ss$      & $sss$         & $ssss$            \\ \hline
            Partial $e$    & $0$   & $01$      & $010$         & $0101$            \\ \hline
            Partial $w(e)$ & $\vn$   & $\vn s = s$ & $\vn s \vn=s$     & $\vn s \vn s=ss$       \\ \hline
            Labels         & $U_0$ & $U_0 U_1$ & $U_0 U_1 D_0$ & $U_0 U_1 D_0 D_1$ \\ \hline
        \end{tabular}
    \end{center}
\end{example}

\begin{definition}
    The \textit{light leaf} $LL_{w,e} \in \Hom(w, \psi\phi(w(e)))$ for a word $w$ and subexpression $e$, is defined iteratively as follows. Let $LL_{\vn,\vn} = \vn$ be the empty diagram. Given $LL_{w',e'}$ and $i \in \{0,1\}$, the light leaf $LL_{w's,e'i}$ is one of
    
\end{definition}
\begin{equation}
    \inputtikz{3.1/light-leaves-def-u0} \:,\:
    \inputtikz{3.1/light-leaves-def-u1} \:,\:
    \inputtikz{3.1/light-leaves-def-d0} \:,\:
    \inputtikz{3.1/light-leaves-def-d1}
\end{equation}
corresponding to the next label, where $w'$ and $e'$ are appropriate subwords\footnote{A word with some letters removed.} of $w$ and $e$ respectively. Here, the codomain of a light leaf $LL_{w,e}$ is the object $\psi\phi(w(e))$. So if the next label is $U_*$ then the codomain of $LL_{w',e'}$ is $\vn$, and when the next label is $D_*$ the codomain of $LL_{w',e'}$ is $s$. This implies that the recursive definition is consistent.

\begin{example}
    Following from Example \eqref{eg:one-col-light-leaf} for $w = ssss$ and $e = 0101$, we have labels $U_0 U_1 D_0 D_1$ so the light leaf $LL_{w,e}$ is built as follows.
    \[
        \vn
        \to \inputtikz{3.1/light-leaves-example1}
        \to \inputtikz{3.1/light-leaves-example2}
        \to \inputtikz{3.1/light-leaves-example3}
        \to \inputtikz{3.1/light-leaves-example4}
    \]
\end{example}

\begin{definition}
    Let $\ol{LL}_{w,e}$ denote the vertical reflection of $LL_{w,e}$. The \textit{double leaf} for words $w,y$ is a composition
    \[
        \mathbb{LL}_{f,e} \coloneqq \ol{LL}_{y,f} \circ LL_{w,e} : w \to y
    \]
    for subexpressions $e$ of $w$ and $f$ of $y$ such that $w(e) = f(y)$.
\end{definition}
Visually this looks like a morphism from $w$ to $y$ factoring through $w(e) = y(f) \in \{\vn,s\}$,
\[
    \inputtikz{3.1/double-leaves}.
\]
% TODO: Maybe give some indication of why we use trapeziums

\begin{example}
    Let $w = ssss$ and $y = sss$. Let $e=0111$ be a subexpression of $w$, and $f=010$ be a subexpression of $y$. The corresponding light leaves are
    \[
        LL_{w,e} = \inputtikz{3.1/double-leaves-example1.1}
        \text{ and }
        LL_{y,f} = \inputtikz{3.1/double-leaves-example1.2}.
    \]
    Then the double leaf $\mathbb{LL}_{f,e} = \ol{LL}_{y,f} \circ LL_{w,e} : w \to y$, factoring through $s$, is
    \[
        \inputtikz{3.1/double-leaves-example2}.
    \]
\end{example}

Note that these double leaves have no floating diagrams such as $\barbell$. In order for these double leaves to be a basis for a morphism space, we insert these floating diagrams by taking linear combinations as a left $\C[\barbell]$-module, where the (left) $\barbell$-action is left concatenation by $\barbell$. Since we can move barbells to the right via. the relation \eqref{rel:one-col-barbell-forcing} and double leaves cut down the middle are double leaves factoring through $\varnothing$, we can equivalently act by $\C[\barbell]$ on the right. This leads us to the following theorem.

\begin{theorem}[Elias-Williamson {\cite[Theorem 1.2]{elias-williamson-soergel-calculus}}]
    \label{thm:one-col-double-leaves-basis}
    Given objects $w,y \in \cH(S_2)$, let $\mathbb{LL}(w,y)$ be the collection of double leaves $\mathbb{LL}_{f,e}$ for subexpressions $e$ of $w$ and $f$ of $y$, such that $w(e) = y(f)$. Then $\mathbb{LL}(w,y)$ is a basis for $\Hom(w,y)$ as a left (or right) $\C[\barbell]$-module.
\end{theorem}
A purely diagrammatic proof (of a more general theorem) can be found in \cite{elias-williamson-soergel-calculus}.

\begin{remark}
    The above light leaves and double leaves, introduced in \cite{elias-williamson-soergel-calculus}, are diagrammatic analogues of Libedinsky's construction in \cite{libedinsky-lightleavesbasis}.
\end{remark}

The morphisms in this category can be graded such that the univalent vertices has degree $1$ and trivalent vertices have degree $-1$. The degree of a general diagram is the sum of the degrees of the generators that appear in it.

\red{Put example}

The double leaves bases allow us to show that the Karoubi envelope of $\cH(S_2)$ is equivalent to the category of Soergel Bimodules $\sbim$ over $S_2$ as monoidal categories.

\begin{theorem}[Elias-Williamson {\cite[Theorem 6.30]{elias-williamson-soergel-calculus}}] \label{thm:one-col-sbim-equiv}
    The category $\Kar(\cH(S_2))$ and the category of Soergel Bimodules $\sbim$ over $S_2$ are equivalent as graded $\C$-linear monoidal categories.
\end{theorem}
The proof in \cite{elias-williamson-soergel-calculus} gives an equivalence of graded $\C$-linear monoidal categories $\cH(S_2) \cong \bsbim$ where $\bsbim$ is the category of Bott-Samelson bimodules over $S_2$. This was done by comparing the graded dimensions of morphism spaces using double leaves bases. Since $\Kar(\bsbim) \cong \sbim$ and Karoubi envelope preserves equivalences, we obtain $\Kar(\cH(S_2)) \cong \sbim$.

