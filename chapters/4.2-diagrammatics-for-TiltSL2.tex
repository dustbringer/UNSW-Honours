\section{Diagrammatic $\Tilt(\fraksl_2)$}

\red{Something something about Tilt}

\red{Something something about extending $\cH(\wtilde{S}_2)$ from $\Z$ to $\C$.}

Although need to work over $\C$ for $\Tilt(\fraksl_2)$, the diagrammatic category below c an be defined more simply over $\Z$.

\begin{definition}
    \label{def:DT}
    Let $\mcal{DT} \coloneqq \mcal{DT}(\fraksl_2)$ be the $\Z$-linear left $\cH(\wtilde{S}_2)$-module category with elements generated by the monoidal identity $\vn$ of $\cH(\wtilde{S}_2)$, and morphisms generated by the empty diagram $\vn$. The action of $\cH(\wtilde{S}_2)$ on the left is left concatenation for objects and morphisms. The relations on diagrams in $\cH(\wtilde{S}_2)$ follow through to diagrams in $\mcal{DT}(\fraksl_2)$. Additionally, we imagine a wall on the right of diagrams and impose the local wall-annihilation relations
    \begin{equation}
        \label{rel:two-col-wall}
        \inputtikz{4.2/relation-red-barbell-wall}
        = \inputtikz{4.2/relation-blue-wall}
        = 0.
    \end{equation}
\end{definition}


\begin{example}
    The morphism in \autoref{eg:two-col-relations} collapses to $0$ because all the diagrams have either blue or barbell on the right.

    \red{TODO: Another example clarifying 'blue on the right'}
\end{example}

The objects of this category are identical to objects in $\cH(\wtilde{S}_2)$ and the morphisms are the same modulo the wall relations \eqref{rel:two-col-wall}. Naturally, we wonder whether double leaves form bases for the morphism spaces in $\mcal{DT}$. It is easy to see that double leaves appear in $\mcal{DT}$ by acting on $\vn$ by double leaves in $\cH(\wtilde{S}_2)$. All morphisms in $\mcal{DT}$ are morphisms in $\cH(\wtilde{S}_2)$ so they can be written as $\Z[\barbell, \barbell[dBlue]]$-linear combinations of double leaves, though some of these leaves have collapsed to $0$. This makes it clear that double leaves span the morphism spaces of $\mcal{DT}$ as (left) $\Z[\barbell, \barbell[dBlue]]$-modules. However they may not be linearly independent as neither left nor right modules as with the one-colour case. Although double leaves are not always a basis for its respective morphism space as $\Z[\barbell, \barbell[dBlue]]$-modules, it turns out a subset of them are a basis over $\Z$.

\begin{lemma}
    \label{lem:DT-double-leaves}
    Let $\pi: \op{mor}(\cH(\wtilde{S}_2)) \to \op{mor}(\mcal{DT})$ be the projection map which takes a morphism to the result of its action on the empty diagram $\vn$. Then the image $\pi(\mathbb{LL}(w,y))$ without zero morphisms is a basis for $\Hom_{\mcal{DT}}(w,y)$ as a $\C$-module.
\end{lemma}

\begin{proof}
    We consider morphisms $\Hom(w,y)$ in $\mcal{DT}$ for fixed objects $w,y$, and write $\mathbb{LL} \coloneqq \pi(\mathbb{LL}(w,y))$ for the set of double leaves in $\mcal{DT}$. Any diagram in $\mcal{DT}$ can be written as a $\C$-linear combination of morphisms without floating diagrams by pulling floating diagrams to the right with \eqref{rel:one-col-barbell-forcing} and \eqref{rel:two-col-barbell-forcing} then applying the wall relation \eqref{rel:two-col-wall}. We can write each of these as a $\Z[\barbell, \barbell[dBlue]]$-linear combination of double leaves with a right action, and reduce it to a $\Z$-linear combination by \eqref{rel:two-col-wall}. This implies that $\mathbb{LL}$ spans $\Hom(w,y)$ as a $\Z$-module. Now $\mathbb{LL}$ may not be linearly independent because the two-colour wall relation \eqref{rel:two-col-wall} reduces all diagrams factoring through a word ending with $t$ to $0$. The set of light leaves after removing morphisms killed by \eqref{rel:two-col-wall}, i.e. $\mathbb{LL} \setminus \{0\}$, still spans $\Hom(w,y)$ by the argument above. This set is linearly independent since, by construction, \eqref{rel:two-col-wall} has no effect on $\Z$-linear combinations of $\mathbb{LL} \setminus \{0\}$. Then it follows from linear independence over $\Z[\barbell, \barbell[dBlue]]$ that this set is linearly independent over $\Z$ in $\mcal{DT}$.
\end{proof}

We aim to show that this diagrammatic category is equivalent to $\Tilt(\fraksl_2)$. So, from now on, we write $\mcal{DT}$ for the $\C$-linear $\cH_\C(\wtilde{S}_2)$-module category obtained by replacing $\Z$ with $\C$ and $\cH(\wtilde{S}_2)$ with $\cH_\C(\wtilde{S}_2)$ in \autoref{def:DT}. The above discussion and \autoref{lem:DT-double-leaves} still apply to $\mcal{DT}$ over $\C$.

Since $\cH_\C(S_2)$ appears inside $\cH_\C(\wtilde{S}_2)$ for each colour, \autoref{lem:ss-equal-2s} provides explicit isomorphisms $ss \cong s \oplus s$ and $tt \cong t \oplus t$ in the additive closure of $\cH_\C(\wtilde{S}_2)$.

\begin{definition}
    Let $F: \Kar(\mcal{DT}) \to \Tilt(\fraksl_2)$ to be the additive $\C$-linear $\cH_\C(\wtilde{S}_2)$-module functor defined as follows \red{What about gradings?}. Map the empty word $\vn$ to the \red{check:} trivial module $T(\vn)$. Given a general word $s_1 ... s_n$ in $\mcal{DT}$, for $s_i \in \{s,t\}$, map $F(s_1 ... s_n) = \Theta_{s_n} ... \Theta_{s_1} T(\vn)$ where $\Theta_s, \Theta_t$ are translation functors associated to generators of $\wtilde{S}_2$. On morphisms, map identity to the corresponding identity, $\counit$ to the inclusion map $i_0: T(\vn) \to \Theta_s T(\vn)$ and $\unit$ to the projection map $p_0: \Theta_s T(\vn) \to T(\vn)$. Consider diagrams of the form
    \begin{center}
        \inputtikz{4.2/tilt-inclusion}
        \quad , \quad
        \inputtikz{4.2/tilt-projection}
    \end{center}
    where the colours of the diagram alternate, ``$\dots$'' stands for zero or more identity strands and the purple is either red or blue such that the colours alternate. The functor $F$ maps them to the inclusion $i_k: \Theta_x ... \Theta_s T(\vn) \to \Theta_y ... \Theta_s T(\vn)$ and projection $p_k: \Theta_y ... \Theta_s T(\vn) \to \Theta_x ... \Theta_s T(\vn)$, respectively, where $x, y \in \{s,t\}$ are different and $k$ is the number of identity strands. The mappings are summarised in the picture below.
    \begin{equation}
        \label{img:tilt-functor}
        \inputtikz{4.2/tilt-proof-functor}
    \end{equation}
    The right wall on each diagram is not shown to reduce clutter. We extend the functor by additivity, composition and linearity. \red{Do we need to define where JW-projectors go?}
\end{definition}

\begin{lemma}
    The functor $F$ is well defined.
\end{lemma}
\begin{proof} \red{CHECK}
    Consider the additive closure $\mcal{DT}^\oplus$ of $\mcal{DT}$. For red and blue, the isomorphisms given in \autoref{lem:ss-equal-2s} imply that any diagram in $\mcal{DT}$ is isomorphic to a matrix of $\C$-linear combinations of diagrams where the domain and codomain have alternating colours. Due to \eqref{rel:two-col-wall} non-trivial diagrams do not have blue strands next to the wall, so the alternating colours must end in red, i.e. the domain and codomain end in $s$. We constructed $F$ to be additive, so it suffices to consider morphisms between indecomposable summands of words alternating $s,t$ and ending in $s$. Recall that these alternating words have idempotents given by their Jones-Wenzl projectors. The discussion in \cite[Section 5.4.2]{elias-dihedral-cathedral} states that the images of Jones-Wenzl projectors give all the indecomposables \red{Is this right? Is the ref right--maybe see book thm9.22} and \cite[Corollary 4.21]{anderson-tubbenhauer-tilt} shows that all morphisms not killed by Jones-Wenzl Projectors are of the form
    \begin{center}
        \inputtikz{4.2/tilt-projection}
        \quad , \quad
        \inputtikz{4.2/tilt-inclusion}
        \quad , \quad
        \inputtikz{4.2/tilt-proj-incl}
    \end{center}
    with alternating colours, purple being either red or blue, and ``...'' are zero or more identity strands. Therefore the morphisms in the picture \eqref{img:tilt-functor} are enough to define a mapping on $\mcal{DT}$ by going through $\Kar(\mcal{DT})$ \red{Word this better}.

    Next, we check that all relations are preserved. From \cite[Proposition 5.84 and Lemma 5.87]{mazorchuk-lectures-sl2-modules} \red{Is this the right ref?}, we know that $\Theta_s$ is a Frobenius object in the category of endofunctors of $\cO$ and there are unit, counit, multiplication and comultiplication natural transformations from the Frobenius object structure. Applying these to $T(\vn)$ result in the same relations in $\Tilt(\fraksl_2)$ for $T(\vn), \Theta_s T(\vn)$ and $\Theta_s^2(T(\vn))$. Note that the projection and inclusion maps in the picture \eqref{img:tilt-functor} are exactly the unit and counit of $\Theta_s$ evaluated at $T(\vn)$, and the trivalent vertices provided by projecting the isomorphisms in \autoref{lem:ss-equal-2s} map exactly to the multiplication and comultiplication maps. Furthermore, in \cite[Section 2.4]{soergel-category-O} \red{Ref?} we see that $p_0 \circ i_0 = 0$ and \red{Is this right:} $\id_{\Theta_t(T(\vn))} = 0$ in $\Tilt(\fraksl_2)$ which is analogous to the two-colour wall relations \eqref{rel:two-col-wall}. Furthermore there is a relation $i_k \circ p_k = p_{k+1} \circ i_{k+1}$ up to a scalar multiple \red{ref?} which is analogous to [\red{PUT THE RELATION HERE}]. \red{What about $p_{k+1} \circ p_k = 0$?} Hence all the relations in $\mcal{DT}$ are preserved by $F$. By construction, $F$ preserves direct sums, $\C$-linear combinations and the Soregel module structure in \cite{soergel-category-O}, so $F$ is well defined as a functor between $\C$-linear $\cH(S_2)$-module categories.

    \red{Talk about how extending by composition makes sense.}
\end{proof}

\red{Say something here?}

The following result states that $\mcal{DT}$ is indeed a diagrammatic incarnation of $\Tilt(\fraksl_2)$.

\red{Be clear that I don't understand $\Tilt$ very well.}



\begin{theorem}[\red{???}]
    The diagrammatic category $\Kar(\mcal{DT}(\fraksl_2))$ and $\Tilt(\fraksl_2)$ are equivalent as $\C$-linear $\cH_\C(\wtilde{S}_2)$-module categories.
\end{theorem}

\begin{proof}
    As a shorthand, we write $\cT$ for $\Tilt(\fraksl_2)$.
    \grey{
        We now prove that $F$ is fully faithful. It follows from \autoref{lem:ss-equal-2s} and the description of $P(\vn)$ and $P(s)$ in \cite[Section 5.2]{mazorchuk-lectures-sl2-modules} that the image of $\unit$ and $\counit$ generate all morphisms of the form $\Theta_s^n(P(\vn)) \to \Theta_s^m(P(\vn))$. Hence $F$ is full. Now the mapping of $F$ on all morphism spaces are determined by those depicted in the above picture. So, for faithfulness, it suffices to compare the $\C$-dimensions of morphism spaces between objects shown in the picture. By \autoref{lem:DO_0-double-leaves}, $\Hom(\vn, \vn)$ has a basis $\{\vn = \id_\vn\}$, $\Hom(s, \vn)$ has a basis $\{\unit\}$, $\Hom(\vn, s)$ has a basis $\{\counit\}$, and $\Hom(s, s)$ has a basis $\{\id_s, \counit \circ \unit\}$. The bases for the corresponding morphism spaces in $\proj(\cO_0)$ are exactly those in the image \red{Ref? - that these are actually the bases of the hom spaces}, so these dimensions coincide. Therefore $F$ is fully faithful.

        All objects in $\proj(\cO_0)$ appear as direct sums and direct summands of the elements $\Theta_s^n(P(\vn))$ for non-negative integers $n$. Therefore the additive Karoubi envelope induces an equivalence $\Kar(\mcal{DO}_0) \cong \proj(\cO_0)$ as $\C$-linear left $\cH(S_2)$-module categories.
    }
\end{proof}

\red{Comment on grading?}

\red{Comment on consequences}


