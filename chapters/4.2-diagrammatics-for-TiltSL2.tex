\section{Diagrammatic $\Tilt(\fraksl_2)$}

\red{Something something about Tilt}

\red{Something something about extending $\cH(\wtilde{S}_2)$ from $\Z$ to $\C$.}


\begin{definition}
    Let $\mcal{DT}(\fraksl_2)$ be the $\C$-linear left $\cH(\wtilde{S}_2)$-module category with elements generated by the monoidal identity $\vn$ of $\cH(\wtilde{S}_2)$, and morphisms generated by the empty diagram $\vn$. The action of $\cH(\wtilde{S}_2)$ on the left is left concatenation for objects and morphisms. The relations on diagrams in $\cH(\wtilde{S}_2)$ follow through to diagrams in $\mcal{DT}(\fraksl_2)$. Additionally, we imagine a wall on the right of diagrams and impose the wall-annihilation relations
    \begin{equation}
        \label{rel:two-col-wall}
        \inputtikz{4.2/relation-red-barbell-wall}
        = \inputtikz{4.2/relation-blue-wall}
        = 0.
    \end{equation}
\end{definition}

In this section we just write $\mcal{DT}$ for this category.

\begin{example}
    The morphism in \autoref{eg:two-col-relations} collapses to $0$ because all the diagrams have either blue or barbell on the right.

    \red{TODO: Another example clarifying 'blue on the right'}
\end{example}

The objects of this category are identical to objects in $\cH(\wtilde{S}_2)$ and the morphisms are the same modulo the wall relations \eqref{rel:two-col-wall}. Naturally, we wonder whether double leaves form bases for the morphism spaces in $\mcal{DT}$. It is easy to see that double leaves appear in $\mcal{DT}$ by acting on $\vn$ by double leaves in $\cH(\wtilde{S}_2)$. All morphisms in $\mcal{DT}$ are morphisms in $\cH(\wtilde{S}_2)$ so they can be written as $\C[\barbell, \barbell[dBlue]]$-linear combinations of double leaves, though some of these leaves have collapsed to $0$. This makes it clear that double leaves span the morphism spaces of $\mcal{DT}$ as (left) $\C[\barbell, \barbell[dBlue]]$-modules. However they may not be linearly independent as neither left nor right modules as with the one-colour case. Although double leaves are not always a basis for its respective morphism space as $\C[\barbell, \barbell[dBlue]]$-modules, it turns out a subset of them are a basis over $\C$.

\begin{lemma}
    \label{lem:DT-double-leaves}
    Let $\pi: \op{mor}(\cH(\wtilde{S}_2)) \to \op{mor}(\mcal{DT})$ be the projection map which takes a morphism $f$ to the result of its action on the empty diagram $\vn$. Then the image $\pi(\mathbb{LL}(w,y))$ without zero morphisms is a basis for $\Hom_{\mcal{DT}}(w,y)$ as a $\C$-module.
\end{lemma}

\begin{proof}
    We consider morphisms $\Hom(w,y)$ in $\mcal{DT}$ for fixed objects $w,y$, and write $\mathbb{LL} \coloneqq \pi(\mathbb{LL}(w,y))$ for the set of double leaves in $\mcal{DT}$. Any diagram in $\mcal{DT}$ can be written as a $\C$-linear combination of morphisms without floating diagrams by pulling floating diagrams to the right with \eqref{rel:one-col-barbell-forcing} and \eqref{rel:two-col-barbell-forcing} then applying the wall relation \eqref{rel:two-col-wall}. We can write each of these as a $\C[\barbell, \barbell[dBlue]]$-linear combination of double leaves with a right action, and reduce it to a $\C$-linear combination by \eqref{rel:two-col-wall}. This implies that $\mathbb{LL}$ spans $\Hom(w,y)$ as a $\C$-module. Now $\mathbb{LL}$ may not be linearly independent because the two-colour wall relation \eqref{rel:two-col-wall} reduces all diagrams factoring through a word ending with $t$ to $0$. The set of light leaves after removing morphisms killed by \eqref{rel:two-col-wall}, i.e. $\mathbb{LL} \setminus \{0\}$, still spans $\Hom(w,y)$ by the argument above. This set is linearly independent since, by construction, \eqref{rel:two-col-wall} has no effect on $\C$-linear combinations of $\mathbb{LL} \setminus \{0\}$. Then it follows from linear independence over $\C[\barbell, \barbell[dBlue]]$ that this set is linearly independent over $\C$ in $\mcal{DT}$.
\end{proof}

