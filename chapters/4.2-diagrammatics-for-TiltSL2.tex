\section{Diagrammatic \texorpdfstring{$\Tilt(\fraksl_2)$}{Tilt(sl2)}}
\label{sec:diag-tiltsl2}

With the two-colour diagrammatic category, we can construct diagrammatics for the category of tilting modules $\Tilt(\fraksl_2)$. We give a brief overview of its structure.

For a fixed $\ell \in \Z_{\geq 2}$, consider quantum\footnote{A generalisation of Lie algebras.} $\fraksl_2$ at a primitive complex $2\ell$-th root of unity, where we follow conventions from \cite[Section 2]{anderson-tubbenhauer-tilt}. An indecomposable module of this algebra is called \textit{tilting} if it appears as a direct summand of a tensor product of the defining two-dimensional representation of quantum $\fraksl_2$. A general tilting module is a finite direct sum of indecomposable tilting modules.

Let $\Tilt(\fraksl_2)$ be the category of tilting modules of quantum $\fraksl_2$ at a primitive complex $2\ell$-th root of unity. According to, for example, \cite[Lemma 2.26]{anderson-tubbenhauer-tilt} $\Tilt(\fraksl_2)$ splits into a direct sum $\Tilt(\fraksl_2) \cong \bigoplus_{i \in {-1, ..., \ell-1}} \Tilt_i (\fraksl_2)$ such that the categories for indexes $-1$ and $\ell-1$ are semisimple, and all other categories are equivalent. We can thus focus on $\Tilt_0(\fraksl_2)$. This category is additive, idempotent complete , Krull--Schmidt and has indecomposables indexed by elements of $\wtilde{S}_2$ (see for example \cite[Lemma 2.26]{anderson-tubbenhauer-tilt}).

Although need to be over $\C$ for $\Tilt(\fraksl_2)$, the following diagrammatic category can be defined more simply over $\Z$.

\begin{definition}
    \label{def:DT}
    Let $\mcal{DT}_0 \coloneqq \mcal{DT}_0(\fraksl_2)$ be the $\Z$-linear (left) $\cH(\wtilde{S}_2)$-module category with elements generated by the monoidal identity $\vn$ of $\cH(\wtilde{S}_2)$, and morphisms generated by the empty diagram $\vn$. The action of $\cH(\wtilde{S}_2)$ on the left is left concatenation for objects and morphisms. The relations on diagrams in $\mcal{DT}_0(\fraksl_2)$ are inherited from those in $\cH(\wtilde{S}_2)$. Additionally, we imagine a wall on the right of diagrams and impose the local wall-annihilation relations
    \begin{equation} \tag{W2} \label{rel:two-col-wall}
        \inputtikz{4.2/relation-red-barbell-wall}
        = \inputtikz{4.2/relation-blue-wall}
        = 0.
    \end{equation}
    In other words, if a red barbell or blue string can come close to the wall without anything in between, then the diagram is $0$. Note that local relations in \eqref{rel:two-col-wall} involve the wall.
\end{definition}

Similar to $\mcal{DO}_0$, this is not a monoidal category due to the new relations \eqref{rel:two-col-wall}.

\begin{example}
    The morphism in \autoref{eg:two-col-relations} collapses to $0$ because all the diagrams have either blue or barbell on the right.

    \red{TODO: Another example clarifying 'blue on the right'}
\end{example}


\begin{proposition}\label{prop:inj-proj-and-proj-inj}
    In the following diagrams, the domain and codomain alternate colours and we only depict the case for odd $k$. For even $k$, just swap the colours red and blue on the left of the ellipsis. For integers $k \geq 1$ 
    \begin{subequations}
        \begin{gather}
            \tag{\ref{prop:inj-proj-and-proj-inj}a}
            \label{rel:barbell-diff-is-split}
            \inputtikz{4.2/inj-proj/barbell-diff} = -2 \inputtikz{4.2/inj-proj/split}
            \\
            \tag{\ref{prop:inj-proj-and-proj-inj}b}
            \label{rel:barbell-same-is-split}
            \inputtikz{4.2/inj-proj/barbell-same} = 2 \inputtikz{4.2/inj-proj/split}
        \end{gather}
    \end{subequations}
    and for $k \geq 3$
    \begin{subequations}
        \begin{gather}
            \tag{\ref{prop:inj-proj-and-proj-inj}c}
            \label{rel:id-split-id-is-0}
            \inputtikz{4.2/inj-proj/id-split-id} = 0.
        \end{gather}
    \end{subequations}
\end{proposition}

\begin{proof}
    For $k \in \{1,2\}$, we check the second two relations by hand. For $k=1$, pulling the barbell through the line using \eqref{rel:one-col-barbell-forcing} and \eqref{rel:two-col-barbell-forcing}, then applying \eqref{rel:two-col-wall} gives us
    \begin{equation*}
        \inputtikz{4.2/inj-proj/proof-k1.1-1}
        = \inputtikz{4.2/inj-proj/proof-k1.1-2} + 2 \inputtikz{4.2/inj-proj/proof-k1.1-3} - 2 \inputtikz{4.2/inj-proj/proof-k1.1-4}
        = - 2 \inputtikz{4.2/inj-proj/proof-k1.1-4}
    \end{equation*}
    and
    \begin{equation*}
        \inputtikz{4.2/inj-proj/proof-k1.2-1}
        = 2 \inputtikz{4.2/inj-proj/proof-k1.2-2} - \inputtikz{4.2/inj-proj/proof-k1.2-3}
        = 2 \inputtikz{4.2/inj-proj/proof-k1.2-2}.
    \end{equation*}
    By a similar proof, using the $k=1$ relations locally, we have for $k=2$,
    \begin{align*}
        \inputtikz{4.2/inj-proj/proof-k2.1-1}
         & = \inputtikz{4.2/inj-proj/proof-k2.1-2} + 2 \inputtikz{4.2/inj-proj/proof-k2.1-3} - 2 \inputtikz{4.2/inj-proj/proof-k2.1-4}
        \\ & \overset{(k=1)}{=} 2 \inputtikz{4.2/inj-proj/proof-k2.1-5} + 2 (-2)\inputtikz{4.2/inj-proj/proof-k2.1-5} - 2 \inputtikz{4.2/inj-proj/proof-k2.1-4} = - 2 \inputtikz{4.2/inj-proj/proof-k2.1-4}
    \end{align*}
    and
    \begin{align*}
        \inputtikz{4.2/inj-proj/proof-k2.2-1}
         & = 2 \inputtikz{4.2/inj-proj/proof-k2.2-2} - \inputtikz{4.2/inj-proj/proof-k2.2-3}
        \\ & \overset{(k=1)}{=} 2 \inputtikz{4.2/inj-proj/proof-k2.2-2} - (-2)\inputtikz{4.2/inj-proj/proof-k2.2-4} = 2 \inputtikz{4.2/inj-proj/proof-k2.2-2}.
    \end{align*}

    Now we proceed by induction on $k$. For $k=3$ we first show \eqref{rel:id-split-id-is-0}. By a similar argument to \eqref{rel:id-id-to-H} we have
    \begin{align*}
        \inputtikz{4.2/inj-proj/proof-k3.1-1}
         = \inputtikz{4.2/inj-proj/proof-k3.1-2} & = \frac{1}{2} \inputtikz{4.2/inj-proj/proof-k3.1-3} + \frac{1}{2} \inputtikz{4.2/inj-proj/proof-k3.1-4}
        \\ & \overset{(k=1)}{=} \frac{2}{2} \inputtikz{4.2/inj-proj/proof-k3.1-5} + \frac{2}{2} \inputtikz{4.2/inj-proj/proof-k3.1-6} = 0
    \end{align*}
    since the wall is accessible by the blue dot. Then
    \begin{align*}
        \inputtikz{4.2/inj-proj/proof-k3.2-1}
         & = \inputtikz{4.2/inj-proj/proof-k3.2-2} + 2 \inputtikz{4.2/inj-proj/proof-k3.2-3} - 2\inputtikz{4.2/inj-proj/proof-k3.2-4}
        \\ & \overset{(k=2)}{=} 2 \inputtikz{4.2/inj-proj/proof-k3.2-5} + 2 (-2) \inputtikz{4.2/inj-proj/proof-k3.2-5} - 2\inputtikz{4.2/inj-proj/proof-k3.2-4}
        \\ & = - 2\inputtikz{4.2/inj-proj/proof-k3.2-4}
    \end{align*}
    and
    \begin{align*}
        \inputtikz{4.2/inj-proj/proof-k3.3-1}
         & = 2 \inputtikz{4.2/inj-proj/proof-k3.3-2} - \inputtikz{4.2/inj-proj/proof-k3.3-3}
        \\ & \overset{(k=2)}{=} 2 \inputtikz{4.2/inj-proj/proof-k3.3-2} - (-2)\inputtikz{4.2/inj-proj/proof-k3.3-4}
        \\ & = 2 \inputtikz{4.2/inj-proj/proof-k3.3-2}.
    \end{align*}

    Let $k \geq 4$ and assume the relations hold for diagrams with $k-1,k-2,...,1$. We will depict the diagrams with odd $k$, where the even $k$ case can be retrieved by swapping red and blue to the left of the ellipsis. Again, the argument to \eqref{rel:id-id-to-H} implies
    \begin{align*}
        \inputtikz{4.2/inj-proj/proof-k4.1-1}
         & = \frac{1}{2} \inputtikz{4.2/inj-proj/proof-k4.1-2} + \frac{1}{2} \inputtikz{4.2/inj-proj/proof-k4.1-3}
        \\ & \overset{ind.}{=} \frac{2}{2} \inputtikz{4.2/inj-proj/proof-k4.1-4} + \frac{2}{2} \inputtikz{4.2/inj-proj/proof-k4.1-5}
        \\ & = \inputtikz{4.2/inj-proj/proof-k4.1-6} + \inputtikz{4.2/inj-proj/proof-k4.1-7}
        \\ & \overset{ind.}{=} 0
    \end{align*}
    where the string to directly left of the ellipsis is the right red string when $k=4$. Furthermore, we have
    \begin{align*}
        \inputtikz{4.2/inj-proj/proof-k4.2-1}
         & = \inputtikz{4.2/inj-proj/proof-k4.2-2} + 2 \inputtikz{4.2/inj-proj/proof-k4.2-3} - 2 \inputtikz{4.2/inj-proj/proof-k4.2-4}
        \\ & \overset{ind.}{=} 2 \inputtikz{4.2/inj-proj/proof-k4.2-5} + 2 (-2) \inputtikz{4.2/inj-proj/proof-k4.2-5} - 2 \inputtikz{4.2/inj-proj/proof-k4.2-4}
        \\ & = - 2 \inputtikz{4.2/inj-proj/proof-k4.2-4}
    \end{align*}
    and
    \begin{align*}
        \inputtikz{4.2/inj-proj/proof-k4.3-1}
         & = 2 \inputtikz{4.2/inj-proj/proof-k4.3-2} - \inputtikz{4.2/inj-proj/proof-k4.3-3}
        \\ & \overset{ind.}{=} 2 \inputtikz{4.2/inj-proj/proof-k4.3-2} - (-2) \inputtikz{4.2/inj-proj/proof-k4.3-4}
        \\ & = 2 \inputtikz{4.2/inj-proj/proof-k4.3-2}.
    \end{align*}
\end{proof}


The objects of this category are identical to objects in $\cH(\wtilde{S}_2)$ and the morphisms are the same modulo the wall relations \eqref{rel:two-col-wall}. Naturally, we wonder whether double leaves form bases for the morphism spaces in $\mcal{DT}_0$. It is easy to see that double leaves appear in $\mcal{DT}_0$ by acting on $\vn$ by double leaves in $\cH(\wtilde{S}_2)$. All morphisms in $\mcal{DT}_0$ are morphisms in $\cH(\wtilde{S}_2)$ so they can be written as $\Z[\barbell, \barbell[dBlue]]$-linear combinations of double leaves, though some of these leaves have collapsed to $0$. This makes it clear that double leaves span the morphism spaces of $\mcal{DT}_0$ as (left) $\Z[\barbell, \barbell[dBlue]]$-modules. However they may not be linearly independent as neither left nor right modules as with the one-colour case. Although double leaves are not always a basis for its respective morphism space as $\Z[\barbell, \barbell[dBlue]]$-modules, it turns out a subset of them are a basis over $\Z$.

\begin{lemma}
    \label{lem:DT-double-leaves}
    Let $\pi: \op{mor}(\cH(\wtilde{S}_2)) \to \op{mor}(\mcal{DT}_0)$ be the projection map which takes a morphism to the result of its action on the empty diagram $\vn$. Then the image $\pi(\mathbb{LL}(w,y))$ without zero morphisms is a basis for $\Hom_{\mcal{DT}_0}(w,y)$ as a $\C$-module.
\end{lemma}

\begin{proof}
    Consider morphisms $\Hom(w,y)$ in $\mcal{DT}_0$ for fixed objects $w,y$, and write $\mathbb{LL} \coloneqq \pi(\mathbb{LL}(w,y))$ for the set of double leaves in $\mcal{DT}_0$. Any diagram in $\mcal{DT}_0$ can be written as a $\C$-linear combination of morphisms without floating diagrams by pulling floating diagrams to the right with \eqref{rel:one-col-barbell-forcing} and \eqref{rel:two-col-barbell-forcing} then applying the wall relation \eqref{rel:two-col-wall}. We can write each of these as a $\Z[\barbell, \barbell[dBlue]]$-linear combination of double leaves with a right action, and reduce it to a $\Z$-linear combination by \eqref{rel:two-col-wall}. This implies that $\mathbb{LL}$ spans $\Hom(w,y)$ as a $\Z$-module. Now $\mathbb{LL}$ may not be linearly independent because the two-colour wall relation \eqref{rel:two-col-wall} reduces all diagrams factoring through a word ending with $t$ to $0$. The set of light leaves after removing morphisms killed by \eqref{rel:two-col-wall}, i.e. $\mathbb{LL} \setminus \{0\}$, still spans $\Hom(w,y)$ by the argument above. This set is linearly independent since, by construction, \eqref{rel:two-col-wall} has no effect on $\Z$-linear combinations of $\mathbb{LL} \setminus \{0\}$. Then it follows from linear independence over $\Z[\barbell, \barbell[dBlue]]$ that this set is linearly independent over $\Z$ in $\mcal{DT}_0$.
\end{proof}

Since there exists light leaves with unbroken red strands on the right, this lemma implies that our category does not collapse by adding the module category structure and the wall relation \eqref{rel:two-col-wall}. Unlike \autoref{sec:diag-osl2}, we will not be using this result to prove the equivalence of categories.

\subsection*{Equivalence with $\Tilt_0(\fraksl_2)$}

We aim to show that the additive Karoubi envelope of this diagrammatic category is equivalent to $\Tilt_0(\fraksl_2)$. From now on, we write $\mcal{DT}_0$ for the $\C$-linear $\cH_\C(\wtilde{S}_2)$-module category obtained by extending scalars from $\Z$ with $\C$. All the above discussion and results still apply to $\mcal{DT}_0$ over $\C$. For brevity we may also write $\cT_0$ for $\Tilt_0(\fraksl_2)$.

Since $\cH_\C(S_2)$ appears inside $\cH_\C(\wtilde{S}_2)$ for each colour, \autoref{lem:ss-equal-2s} provides explicit isomorphisms $ss \cong s \oplus s$ and $tt \cong t \oplus t$ in the additive closure of $\cH_\C(\wtilde{S}_2)$.

\begin{definition}
    Let $F: \mcal{DT}_0^\oplus \to \Tilt_0(\fraksl_2)$ to be the additive $\C$-linear $\cH_\C(\wtilde{S}_2)$-module functor defined as follows. Map the empty word $\vn$ to the trivial module $T(\vn)$. Given a general word $s_n \dots s_1$ in $\mcal{DT}_0$, for $s_i \in \{s,t\}$, map $F(s_n \dots s_1) = \Theta_{s_n} \dots \Theta_{s_1} T(\vn)$ where $\Theta_s, \Theta_t$ are translation functors associated to generators of $\wtilde{S}_2$.

    On morphisms, we define $F$ recursively. Note that we only have red strands on the right since \eqref{rel:two-col-wall} reduces right blue strands to 0. For $k \geq 0$, define for odd $k$
    \begin{gather*}
        \id^d_k \coloneqq \inputtikz{4.2/tilt-identity-k}
        \quad , \quad
        i^d_k \coloneqq \inputtikz{4.2/tilt-inclusion-k}
        \quad , \quad
        p^d_k \coloneqq \inputtikz{4.2/tilt-projection-k}
    \end{gather*}
    where colours alternate and a red strand on the right when $k \neq 0$. For even $k$, we define these similarly with colours to the left of the ellipsis swapped. Similarly for $k \geq 0$, we define $\id_k: \Theta_x \dots \Theta_s(T(\vn)) \to \Theta_x \dots \Theta_s(T(\vn))$, $i_k: \Theta_x \dots \Theta_s(T(\vn)) \to \Theta_y \Theta_x\dots \Theta_s(T(\vn))$ and $p_k: \Theta_y \Theta_x \dots \Theta_s(T(\vn)) \to \Theta_x\dots \Theta_s(T(\vn))$ to be the identity, inclusion and projection maps in $\cT_0$, where the subscripts alternate $s,t$ and $\Theta_x \dots \Theta_s$ is a composition of $k$ translation functors. Further we write $\tilde{p}_k \coloneqq (-1)^{k+1}\frac{1}{2^{k+1}} p_k$. Let $F(\id^d_k) = \id_k$. On the generators \eqref{eq:two-col-gen} of $\mcal{DT}_0$, map

    \begin{align*}
        \inputtikz{4.2/tilt-gen-identity}
         & \xmapsto{F}
        \begin{cases}
            \id_{k+1},  & \text{if $k$ even}, \\
            \pvec{\id_k & 0                   \\ 0 & \id_k}, &\text{if $k$ odd},
        \end{cases}
        \\
        \inputtikz{4.2/tilt-gen-inclusion}
         & \xmapsto{F}
        \begin{cases}
            \tilde{p}_k, & \text{if $k$ even},  \\
            \pvec{i_{k-1} \circ \tilde{p}_{k-1} \\ \id_k}, &\text{if $k$ odd},
        \end{cases}
        \\
        \inputtikz{4.2/tilt-gen-projection}
         & \xmapsto{F}
        \begin{cases}
            i_k,        & \text{if $k$ even},                          \\
            \pvec{\id_k & i_{k-1} \circ \tilde{p}_{k-1}}, & \text{if $k$ odd},
        \end{cases}
        \\
        \inputtikz{4.2/tilt-gen-multiplication}
         & \xmapsto{F}
        \begin{cases}
            \pvec{0 & \id_{k+1}}, & \text{if $k$ even},     \\
            \pvec{0 & 0           & \id_k               & 0 \\ 0 & 0 & 0 & \id_k}, &\text{if $k$ odd},
        \end{cases}
        \\
        \inputtikz{4.2/tilt-gen-division}
         & \xmapsto{F}
        \begin{cases}
            \pvec{\id_{k+1} \\ 0}, & \text{if $k$ even},     \\
            \pvec{\id_k & 0 \\ 0 & \id_k \\ 0 & 0 \\ 0 & 0}, &\text{if $k$ odd},
        \end{cases}
    \end{align*}
    where black strands can be any colour and each entry in the matrix are matrices themselves. For blue generators, the definition is the same with the words even and odd swapped. Putting a red (resp. blue) identity strands on the left of a diagram is applying $\Theta_s$ (resp. $\Theta_t$) to the output morphism. Pictorially, for a morphism $f$ in $\mcal{DT}_0$,
    \begin{gather*}
        \inputtikz{4.2/tilt-identity-on-left} \xmapsto {F} \Theta_s F(f).
    \end{gather*}
    We extend the functor by composition, additivity and linearity.
    
    The mappings that don't involve matrices are summarised in the picture below.
    \begin{equation}
        \label{img:tilt-functor}
        \inputtikz{4.2/tilt-proof-functor}
    \end{equation}
    The right wall on each diagram is not shown to reduce clutter.
\end{definition}

The definition on generators is a consequence of the isomorphism $\Theta_s \Theta_s \cong \Theta_s \oplus \Theta_s$ analogous to $ss \cong s \oplus s$ (and respectively for $t$) from \autoref{lem:ss-equal-2s}.

\begin{remark} \label{rk:tilt-functor-def}
    The action of an arbitrary morphism of $\cH(\wtilde{S}_2)$ on the left of a morphism in $\mcal{DT}_0$ is sent to the Godement product\footnote{The horizontal composition of natural transformations.} of the natural transformations underlying the image of morphisms under $F$. Taking the Godement product of natural transformations $\Theta_x \dots \Theta_s \to \Theta_y \dots \Theta_s$, when viewed as diagrams in $\mcal{DT}_0$, is just a left tensor of the corresponding diagrams. Visually, the construction of looks like putting identity morphisms on the left of one morphism on the right of the other, so that the codomains align, and then composing them. In $\cT_0$, this is the Kronecker product of matrices.
\end{remark}


\begin{lemma}
    The functor $F$ is well defined as a functor between additive $\C$-linear $\cH(S_2)$-module categories.
\end{lemma}
\begin{proof}
    By \autoref{rk:tilt-functor-def}, the definition of $F$ defines an action of every morphism in $\mcal{DT}_0^\oplus$. It remains to check that all relations are preserved. It follows from \cite[Proposition 2.34]{anderson-tubbenhauer-tilt} that the translation functors $\Theta_s, \Theta_t$ are Frobenius objects in the category of endofunctors of $\cT$ and there are unit, counit, multiplication and comultiplication natural transformations and corresponding relations from the Frobenius object structure. Applying these to $T(\vn)$ result in Frobenius object relations in $\cT_0$ for $T(\vn), \Theta_s T(\vn)$ and $\Theta_s^2(T(\vn))$, and similarly with $\Theta_t$. Note that $\unit$ and $\counit$ map to $i_0$ and $\tilde{p}_0$ \red{Is the scaling here right?} which are exactly the unit and counit of $\Theta_s$ evaluated at $T(\vn)$ (up to scaling), and the trivalent vertices defined with $\id^d_0$ are mapped exactly to the multiplication and comultiplication maps. The isomorphism \autoref{lem:ss-equal-2s} we use to reduce domain and codomain has an analogue $\Theta_s \circ \Theta_s \cong \Theta_s \oplus \Theta_s$ as in \cite[Corollary 2.35(a)]{anderson-tubbenhauer-tilt}, and similarly for $t$. Furthermore, in \cite[Proposition 2.30]{anderson-tubbenhauer-tilt} we see that $p_0 \circ i_0 = 0$, $p_{k+1} \circ i_{k+1} = i_k \circ p_k$ that are analogous to the relations in \autoref{prop:inj-proj-and-proj-inj}, up to an adjusting scalar given in the definition. From \cite[Corollary 2.35]{anderson-tubbenhauer-tilt} the translation functors satisfy properties analogous to the two-colour wall relations \eqref{rel:two-col-wall}. Checking that the remaining relations \eqref{rel:one-col-needle}, \eqref{rel:one-col-barbell-forcing}, \eqref{rel:two-col-barbell-forcing} and \eqref{rel:two-col-wall} hold in $\cT_0$ is straightforward (see \cite[Lemma 4.26]{anderson-tubbenhauer-tilt}). Therefore all the relations in $\mcal{DT}_0$ are preserved by $F$. By construction, $F$ preserves direct sums, $\C$-linear combinations and the Soergel module structure, so $F$ is well defined as a functor between additive $\C$-linear $\cH(S_2)$-module categories.
\end{proof}

\begin{theorem}[Andersen--Tubbenhauer, {\cite[Theorem 4.27]{anderson-tubbenhauer-tilt}}]
    The diagrammatic category $\Kar(\mcal{DT}_0(\fraksl_2))$ and $\Tilt_0(\fraksl_2)$ are equivalent as additive $\C$-linear $\cH_\C(\wtilde{S}_2)$-module categories.
\end{theorem}

\begin{proof}
    Since $\cT_0$ is additive and Karoubian, our functor $F$ extends uniquely to an additive functor $F': \Kar(\mcal{DT}_0) \to \cT_0$. By the argument in \cite[Theorem 4.27]{anderson-tubbenhauer-tilt}, every element in $\cT_0$ is isomorphic to $F'$ applied to a direct sum of images of Jones--Wenzl projectors, so $F'$ is essentially surjective. Particularly, this shows that the images of JW-projectors map exactly to the indecomposable ``leading'' tilting modules.
    
    By \autoref{lem:ss-equal-2s} and \eqref{rel:two-col-wall}, we just consider words with alternating generators and ending with $s$. Write $T(...ts)$ for the leading indecomposable summand of $...\Theta_t \Theta_s (T(\vn))$ in $\cT_0$, and write $b_{...ts}$ for the image of $\JW_{...ts}$. By \cite[Section 5.4.2]{elias-dihedral-cathedral}, Jones--Wenzl projectors are primitive idempotents and their images are all the indecomposables in $\mcal{DT}_0$, and as mentioned above they map to the leading indecomposables in $\cT_0$. For full and faithfulness, it is sufficient to check that the dimensions of the morphism spaces between indecomposables $\Hom_{\mcal{DT}_0}(b_{x...ts}, b_{y...ts})$ and $\Hom_{\cT_0}(T(x...ts), T(y...ts))$ coincide. On the diagrammatic side, a morphism $b_{x...ts} \to b_{y...ts}$ is given by $\JW_{y...ts} f \JW_{x...ts}$ where $f: x...ts \to y... ts$. Since morphisms can be written as a linear combination of double leaves, we consider $f$ to be a double leaf. By \autoref{prop:JW-pitchfork}, all double leaves in which pitchfork appear on the top or bottom of the diagram are killed. Since the domain and codomain alternate colours, the remaining diagrams are a tensor and composition of $\unit$, $\counit$ and identity strands. Notice that we have the relation
    \begin{align*}
        \inputtikz{4.2/tilt-dot-line-dot-line-id1.1}
        = \inputtikz{4.2/tilt-dot-line-dot-line-id1.2}
        & = \frac{1}{2} \inputtikz{4.2/tilt-dot-line-dot-line-id1.3} + \frac{1}{2} \inputtikz{4.2/tilt-dot-line-dot-line-id1.4} \\
        & = \frac{1}{2} \inputtikz{4.2/tilt-dot-line-dot-line-id1.3} + \frac{2}{2} \inputtikz{4.2/tilt-dot-line-dot-line-id1.5}
        = \frac{1}{2} \inputtikz{4.2/tilt-dot-line-dot-line-id1.3}.
    \end{align*}
    Proceeding inductively on the number of identity strands on the right, we have
    \begin{align*}
        \inputtikz{4.2/tilt-dot-line-dot-line-id2.1}
        = \inputtikz{4.2/tilt-dot-line-dot-line-id2.2}
        & = \frac{1}{2} \inputtikz{4.2/tilt-dot-line-dot-line-id2.3} + \frac{1}{2} \inputtikz{4.2/tilt-dot-line-dot-line-id2.4} \\
        & = \frac{1}{2} \inputtikz{4.2/tilt-dot-line-dot-line-id2.3} + \frac{2}{2} \inputtikz{4.2/tilt-dot-line-dot-line-id2.5} \\
        & = \frac{1}{2} \inputtikz{4.2/tilt-dot-line-dot-line-id2.3} + \inputtikz{4.2/tilt-dot-line-dot-line-id2.6}
    \end{align*}
    for even length domain, where the third equality follows from \autoref{prop:inj-proj-and-proj-inj}. Swapping blue and red strands left of the ellipsis gives us the odd case. By induction the second term is a linear combination of diagrams with pitchforks, hence this diagram is a linear combinations of diagrams with pitchforks. Particularly, these are killed by Jones--Wenzl projectors. The same holds for the vertically reflected diagram. Along with \autoref{prop:inj-proj-and-proj-inj}, we conclude that the only double leaves we should consider are $\id_k$, $i^d_k$, $p^d_k$ and their composition $i^d_k \circ p^d_k$. This is informally summarised by the diagram below (similar to \eqref{img:tilt-functor}).
    \begin{equation}
        \label{img:tilt-functor-indecomp}
        \inputtikz{4.2/tilt-proof-functor-indecomp}
    \end{equation}
    Although not drawn, all the diagrams are flanked by Jones--Wenzl projectors, and the matching morphisms in $\cT_0$ are pre and post-composed with the idempotents corresponding to the appropriate JW-projectors. Putting JW-projectors above and below any of these diagrams clearly do not result in zero. Moreover, in the endomorphism space of each non-trivial indecomposable, the morphisms $\id^d_k$ and $i^d_{k-1} \circ p^d_{k-1}$, with JW-projectors before and after, can easily be checked to be linearly independent. Hence the bases for the spaces can be read off the picture \eqref{img:tilt-functor-indecomp}. In $\cT_0$, the analogous bases for the morphism spaces in \cite[Corollary 2.3.1]{anderson-tubbenhauer-tilt} have matching dimensions, hence $F'$ is fully faithful. Therefore the categories $\Kar(\mcal{DT}_0)$ and $\Tilt_0(\fraksl_2)$ are equivalent as (idempotent complete) additive $\C$-linear $\cH_\C(\wtilde{S}_2)$-module categories.
\end{proof}

This functor is defined similarly to that for $\proj(\cO_0)$ in \autoref{sec:diag-osl2}. However this is not apparent since Jones--Wenzl projectors for that category are trivial (just the red identity strand).


