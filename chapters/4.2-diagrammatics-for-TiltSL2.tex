\section{Diagrammatic $\Tilt(\fraksl_2)$}

\red{Something something about Tilt}

\red{Something something about extending $\cH(\wtilde{S}_2)$ from $\Z$ to $\C$.}


\begin{definition}
    Let $\mcal{DT}(\fraksl_2)$ be the $\C$-linear \red{Maybe have this $\Z$} left $\cH(\wtilde{S}_2)$-module category with elements generated by the monoidal identity $\vn$ of $\cH(\wtilde{S}_2)$, and morphisms generated by the empty diagram $\vn$. The action of $\cH(\wtilde{S}_2)$ on the left is left concatenation for objects and morphisms. The relations on diagrams in $\cH(\wtilde{S}_2)$ follow through to diagrams in $\mcal{DT}(\fraksl_2)$. Additionally, we imagine a wall on the right of diagrams and impose the local wall-annihilation relations
    \begin{equation}
        \label{rel:two-col-wall}
        \inputtikz{4.2/relation-red-barbell-wall}
        = \inputtikz{4.2/relation-blue-wall}
        = 0.
    \end{equation}
\end{definition}

In this section we just write $\mcal{DT}$ for this category.

\begin{example}
    The morphism in \autoref{eg:two-col-relations} collapses to $0$ because all the diagrams have either blue or barbell on the right.

    \red{TODO: Another example clarifying 'blue on the right'}
\end{example}

The objects of this category are identical to objects in $\cH(\wtilde{S}_2)$ and the morphisms are the same modulo the wall relations \eqref{rel:two-col-wall}. Naturally, we wonder whether double leaves form bases for the morphism spaces in $\mcal{DT}$. It is easy to see that double leaves appear in $\mcal{DT}$ by acting on $\vn$ by double leaves in $\cH(\wtilde{S}_2)$. All morphisms in $\mcal{DT}$ are morphisms in $\cH(\wtilde{S}_2)$ so they can be written as $\C[\barbell, \barbell[dBlue]]$-linear combinations of double leaves, though some of these leaves have collapsed to $0$. This makes it clear that double leaves span the morphism spaces of $\mcal{DT}$ as (left) $\C[\barbell, \barbell[dBlue]]$-modules. However they may not be linearly independent as neither left nor right modules as with the one-colour case. Although double leaves are not always a basis for its respective morphism space as $\C[\barbell, \barbell[dBlue]]$-modules, it turns out a subset of them are a basis over $\C$.

\begin{lemma}
    \label{lem:DT-double-leaves}
    Let $\pi: \op{mor}(\cH(\wtilde{S}_2)) \to \op{mor}(\mcal{DT})$ be the projection map which takes a morphism to the result of its action on the empty diagram $\vn$. Then the image $\pi(\mathbb{LL}(w,y))$ without zero morphisms is a basis for $\Hom_{\mcal{DT}}(w,y)$ as a $\C$-module.
\end{lemma}

\begin{proof}
    We consider morphisms $\Hom(w,y)$ in $\mcal{DT}$ for fixed objects $w,y$, and write $\mathbb{LL} \coloneqq \pi(\mathbb{LL}(w,y))$ for the set of double leaves in $\mcal{DT}$. Any diagram in $\mcal{DT}$ can be written as a $\C$-linear combination of morphisms without floating diagrams by pulling floating diagrams to the right with \eqref{rel:one-col-barbell-forcing} and \eqref{rel:two-col-barbell-forcing} then applying the wall relation \eqref{rel:two-col-wall}. We can write each of these as a $\C[\barbell, \barbell[dBlue]]$-linear combination of double leaves with a right action, and reduce it to a $\C$-linear combination by \eqref{rel:two-col-wall}. This implies that $\mathbb{LL}$ spans $\Hom(w,y)$ as a $\C$-module. Now $\mathbb{LL}$ may not be linearly independent because the two-colour wall relation \eqref{rel:two-col-wall} reduces all diagrams factoring through a word ending with $t$ to $0$. The set of light leaves after removing morphisms killed by \eqref{rel:two-col-wall}, i.e. $\mathbb{LL} \setminus \{0\}$, still spans $\Hom(w,y)$ by the argument above. This set is linearly independent since, by construction, \eqref{rel:two-col-wall} has no effect on $\C$-linear combinations of $\mathbb{LL} \setminus \{0\}$. Then it follows from linear independence over $\C[\barbell, \barbell[dBlue]]$ that this set is linearly independent over $\C$ in $\mcal{DT}$.
\end{proof}

Since $\cH(S_2)$ appears inside $\cH(\wtilde{S}_2)$ as both colours, \autoref{lem:ss-equal-2s} gives explicit isomorphisms $s \tensor s \cong s \oplus s$ and $t \tensor t \cong t \oplus t$.

\red{Say something here?}

The following result states that $\mcal{DT}$ is indeed a diagrammatic incarnation of $\Tilt(\fraksl_2)$.

\red{Be clear that I don't understand $\Tilt$ very well.}

\begin{theorem}[\red{???}]
    The diagrammatic category $\Kar(\mcal{DT}(\fraksl_2))$ and $\Tilt(\fraksl_2)$ are equivalent as $\C$-linear $\cH(\wtilde{S}_2)$-module categories.
\end{theorem}


\red{Check all of this \& Put precise references}

\red{Maybe write description as a soergel module outside the proof}

\begin{proof}
    As a shorthand, we write $\cT$ for $\Tilt(\fraksl_2)$.
    \grey{The work of Soergel in \cite[Section 2.4]{soergel-category-O} shows that $\proj(\cO_0)$ is a Soergel module, i.e. it has a left action of the category of Soergel bimodules defined by applications of the translation functors $\Theta_\vn, \Theta_s \in \End(\cO)$ (corresponding to elements in $S_2$). \red{Explains what this means, how its related to the $\cH(S_2)$ module category} We will construct a functor that will map faithfully into a full subcategory of $\proj(\cO_0)$, which will become the whole projective category under the additive Karoubi envelope. This mimics the strategy in the proof for \autoref{thm:one-col-sbim-equiv}.
    
    Define the functor $F: \mcal{DO}_0 \to \proj(\cO_0)$ that sends the empty object $\vn$ to the trivial module $P(\vn)$, and the Soergel module action corresponding to $s$ to the translation functor $\Theta_s$. Then the object $s$ maps to $\Theta_s(P(\vn)) \eqqcolon P(s)$, and for example $s^3$ maps to $\Theta_s^3(P(\vn)) = \Theta_s\Theta_s\Theta_s(P(\vn))$. In order for $F$ to be functorial, it must map identity diagrams $s^n \to s^n$ to $\id_{\Theta_s^n(P(\vn))}$. On non-identity maps, we let $F(\counit) = i$ be the inclusion $P(\vn) \to P(s)$ and $F(\unit) = p$ be the projection $P(s) \to P(\vn)$. The mapping of $F$ is depicted by the following picture.
    \begin{center}
        \inputtikz{4.2/tiltsl2-proof-functor}
    \end{center}
    Now \cite[Proposition 5.90]{mazorchuk-lectures-sl2-modules} shows that there is a natural isomorphism $\Theta_s \Theta_s \cong \Theta_s \oplus \Theta_s$ analogous to the isomorphism $s \tensor s \cong s \oplus s$ given in the proof of \autoref{lem:ss-equal-2s}. We will eventually take the additive closure of $\mcal{DO}_0$, so it does not hurt to use these isomorphisms. Given a morphism in $\mcal{DO}_0$ from $s^n$ to $s^m$, repeated precomposition and postcomposition with $s \to s \oplus s$ and $s \oplus s \to s$ from \autoref{lem:ss-equal-2s} results in a matrix of morphisms with domain and codomain in $\{\vn, s\}$. By \eqref{rel:one-col-barbell-forcing} and \eqref{rel:barbell-wall} we can draw the entries of the matrix without floating diagrams, so the only diagrams are $\counit \circ \unit$ and $\dRed{\mathsf{I}}$ up to linear combinations. Therefore, extending by linearity, the picture above completely describes the image of $F$. We can similarly pull back the matrices of morphisms in $\proj(\cO_0)$ to a morphism between $\Theta_s^n(P(\vn))$ and $\Theta_s^m(P(\vn))$ via the analogous maps defining $\Theta_s \Theta_s \cong \Theta_s \oplus \Theta_s$.

    From classical results e.g. \cite[Proposition 5.84 and Lemma 5.87]{mazorchuk-lectures-sl2-modules}, it follows that $\Theta_s$ is a Frobenius object in the category of endofunctors of $\cO$. Then there are unit, counit, multiplication and comultiplication natural transformations satisfying coherence relations in the Frobenius object structure. Applying these to $P(\vn)$ result in the same relations in $\proj(\cO_0)$ for $P(\vn), P(s)$ and $\Theta_s^2(P(\vn))$. Note that the projection and inclusion maps above are exactly the unit and counit of $\Theta_s$ evaluated at $P(\vn)$, and the trivalent vertices provided by projecting the isomorphisms in \autoref{lem:ss-equal-2s} are exactly the multiplication and comultiplication maps. Furthermore, in \cite[Section 2.4]{soergel-category-O} we see that $p \circ i = 0$ in $\proj(\cO_0)$ which is analogous\footnote{This relation extends to the analogue of the local barbell-wall relation, as all `barbell on the right' morphisms in $\proj(\cO_0)$ are linear combinations of applications of $\Theta_s$ to $p \circ i$, which is $0$.} to the barbell-wall relation \eqref{rel:barbell-wall}. Hence all the relations in $\mcal{DO}_0$ are preserved by $F$. By construction, $F$ preserves $\C$-linear combinations and the Soregel module structure in \cite{soergel-category-O}, so $F$ is well defined as a functor between $\C$-linear $\cH(S_2)$-module categories.

    We now prove that $F$ is fully faithful. It follows from \autoref{lem:ss-equal-2s} and the description of $P(\vn)$ and $P(s)$ in \cite[Section 5.2]{mazorchuk-lectures-sl2-modules} that the image of $\unit$ and $\counit$ generate all morphisms of the form $\Theta_s^n(P(\vn)) \to \Theta_s^m(P(\vn))$. Hence $F$ is full. Now the mapping of $F$ on all morphism spaces are determined by those depicted in the above picture. So, for faithfulness, it suffices to compare the $\C$-dimensions of morphism spaces between objects shown in the picture. By \autoref{lem:DO_0-double-leaves}, $\Hom(\vn, \vn)$ has a basis $\{\vn = \id_\vn\}$, $\Hom(s, \vn)$ has a basis $\{\unit\}$, $\Hom(\vn, s)$ has a basis $\{\counit\}$, and $\Hom(s, s)$ has a basis $\{\id_s, \counit \circ \unit\}$. The bases for the corresponding morphism spaces in $\proj(\cO_0)$ are exactly those in the image \red{Ref? - that these are actually the bases of the hom spaces}, so these dimensions coincide. Therefore $F$ is fully faithful.

    All objects in $\proj(\cO_0)$ appear as direct sums and direct summands of the elements $\Theta_s^n(P(\vn))$ for non-negative integers $n$. Therefore the additive Karoubi envelope induces an equivalence $\Kar(\mcal{DO}_0) \cong \proj(\cO_0)$ as $\C$-linear left $\cH(S_2)$-module categories.
    }
\end{proof}

\red{Comment on grading?}