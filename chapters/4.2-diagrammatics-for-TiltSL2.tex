\section{Diagrammatic $\Tilt(\fraksl_2)$}
\label{sec:4.2}

\red{Something something about Tilt}

\red{Something something about extending $\cH(\wtilde{S}_2)$ from $\Z$ to $\C$.}

Although need to work over $\C$ for $\Tilt(\fraksl_2)$, the diagrammatic category below can be defined more simply over $\Z$.

\begin{definition}
    \label{def:DT}
    Let $\mcal{DT} \coloneqq \mcal{DT}(\fraksl_2)$ be the $\Z$-linear left $\cH(\wtilde{S}_2)$-module category with elements generated by the monoidal identity $\vn$ of $\cH(\wtilde{S}_2)$, and morphisms generated by the empty diagram $\vn$. The action of $\cH(\wtilde{S}_2)$ on the left is left concatenation for objects and morphisms. The relations on diagrams in $\mcal{DT}(\fraksl_2)$ are inherited from those in $\cH(\wtilde{S}_2)$. Additionally, we imagine a wall on the right of diagrams and impose the local wall-annihilation relations
    \begin{equation} \tag{W2} \label{rel:two-col-wall}
        \inputtikz{4.2/relation-red-barbell-wall}
        = \inputtikz{4.2/relation-blue-wall}
        = 0.
    \end{equation}
    In other words, if a red barbell or blue string can come close to the wall without anything in between, then the diagram is $0$. Note that these local relations involve the wall.
\end{definition}


\begin{example}
    The morphism in \autoref{eg:two-col-relations} collapses to $0$ because all the diagrams have either blue or barbell on the right.

    \red{TODO: Another example clarifying 'blue on the right'}
\end{example}


Using \eqref{rel:two-col-wall} we can extend \eqref{rel:two-col-barbell-forcing}.
\begin{proposition}
    For integers $k \geq 3$ the following relations hold, where the domain and codomain alternate colours, and \dPink{pink} and \dPurp{purple} represent different colours in \{\dRed{red}, \dBlue{blue}\}.
    \addtocounter{equation}{-1}
    \begin{subequations}
        \begin{gather}
            \label{rel:id-split-id-is-0}
            \inputtikz{4.2/inj-proj/id-split-id} = 0
            \\
            \label{rel:barbell-diff-is-split}
            \inputtikz{4.2/inj-proj/barbell-diff} = -2 \inputtikz{4.2/inj-proj/split}
            \\
            \label{rel:barbell-same-is-split}
            \inputtikz{4.2/inj-proj/barbell-same} = 2 \inputtikz{4.2/inj-proj/split}
        \end{gather}
    \end{subequations}
\end{proposition}

\begin{proof}
    The second two relations also hold for $k\in \{1,2\}$. For $k=1$, pulling the barbell through the line using \eqref{rel:one-col-barbell-forcing} and \eqref{rel:two-col-barbell-forcing}, then applying \eqref{rel:two-col-wall} gives us
    \begin{equation*}
        \inputtikz{4.2/inj-proj/proof-k1.1-1}
        = \inputtikz{4.2/inj-proj/proof-k1.1-2} + 2 \inputtikz{4.2/inj-proj/proof-k1.1-3} - 2 \inputtikz{4.2/inj-proj/proof-k1.1-4}
        = - 2 \inputtikz{4.2/inj-proj/proof-k1.1-4}
    \end{equation*}
    and
    \begin{equation*}
        \inputtikz{4.2/inj-proj/proof-k1.2-1}
        = 2 \inputtikz{4.2/inj-proj/proof-k1.2-2} - \inputtikz{4.2/inj-proj/proof-k1.2-3}
        = 2 \inputtikz{4.2/inj-proj/proof-k1.2-2}.
    \end{equation*}
    By a similar proof, using the $k=1$ relations locally, we have for $k=2$,
    \begin{align*}
        \inputtikz{4.2/inj-proj/proof-k2.1-1}
         & = \inputtikz{4.2/inj-proj/proof-k2.1-2} + 2 \inputtikz{4.2/inj-proj/proof-k2.1-3} - 2 \inputtikz{4.2/inj-proj/proof-k2.1-4}
        \\ & \overset{(k=1)}{=} 2 \inputtikz{4.2/inj-proj/proof-k2.1-5} + 2 (-2)\inputtikz{4.2/inj-proj/proof-k2.1-5} - 2 \inputtikz{4.2/inj-proj/proof-k2.1-4}
        \\ & = - 2 \inputtikz{4.2/inj-proj/proof-k2.1-4}
    \end{align*}
    and
    \begin{align*}
        \inputtikz{4.2/inj-proj/proof-k2.2-1}
         & = 2 \inputtikz{4.2/inj-proj/proof-k2.2-2} - \inputtikz{4.2/inj-proj/proof-k2.2-3}
        \\ & \overset{(k=1)}{=} 2 \inputtikz{4.2/inj-proj/proof-k2.2-2} - (-2)\inputtikz{4.2/inj-proj/proof-k2.2-4}
        \\ & = 2 \inputtikz{4.2/inj-proj/proof-k2.2-2}.
    \end{align*}

    Now we proceed by induction on $k$. For $k=3$ we first show \eqref{rel:id-split-id-is-0}. By a similar argument to \eqref{rel:id-id-to-H} we have
    \begin{align*}
        \inputtikz{4.2/inj-proj/proof-k3.1-1}
         & = \inputtikz{4.2/inj-proj/proof-k3.1-2}
        \\ & = \frac{1}{2} \inputtikz{4.2/inj-proj/proof-k3.1-3} + \frac{1}{2} \inputtikz{4.2/inj-proj/proof-k3.1-4}
        \\ & \overset{(k=1)}{=} \frac{2}{2} \inputtikz{4.2/inj-proj/proof-k3.1-5} + \frac{2}{2} \inputtikz{4.2/inj-proj/proof-k3.1-6}
        \\ & = 0
    \end{align*}
    since the wall is accessible by the blue dot. Then
    \begin{align*}
        \inputtikz{4.2/inj-proj/proof-k3.2-1}
         & = \inputtikz{4.2/inj-proj/proof-k3.2-2} + 2 \inputtikz{4.2/inj-proj/proof-k3.2-3} - 2\inputtikz{4.2/inj-proj/proof-k3.2-4}
        \\ & \overset{(k=2)}{=} 2 \inputtikz{4.2/inj-proj/proof-k3.2-5} + 2 (-2) \inputtikz{4.2/inj-proj/proof-k3.2-5} - 2\inputtikz{4.2/inj-proj/proof-k3.2-4}
        \\ & = - 2\inputtikz{4.2/inj-proj/proof-k3.2-4}
    \end{align*}
    and
    \begin{align*}
        \inputtikz{4.2/inj-proj/proof-k3.3-1}
         & = 2 \inputtikz{4.2/inj-proj/proof-k3.3-2} - \inputtikz{4.2/inj-proj/proof-k3.3-3}
        \\ & \overset{(k=2)}{=} 2 \inputtikz{4.2/inj-proj/proof-k3.3-2} - (-2)\inputtikz{4.2/inj-proj/proof-k3.3-4}
        \\ & = 2 \inputtikz{4.2/inj-proj/proof-k3.3-2}.
    \end{align*}

    Let $k \geq 4$ and assume the relations hold for diagrams with $k-1,k-2,...,1$. Again, the argument to \eqref{rel:id-id-to-H} implies
    \begin{align*}
        \inputtikz{4.2/inj-proj/proof-k4.1-1}
         & = \frac{1}{2} \inputtikz{4.2/inj-proj/proof-k4.1-2} + \frac{1}{2} \inputtikz{4.2/inj-proj/proof-k4.1-3}
        \\ & \overset{ind.}{=} \frac{2}{2} \inputtikz{4.2/inj-proj/proof-k4.1-4} + \frac{2}{2} \inputtikz{4.2/inj-proj/proof-k4.1-5}
        \\ & = \inputtikz{4.2/inj-proj/proof-k4.1-6} + \inputtikz{4.2/inj-proj/proof-k4.1-7}
        \\ & \overset{ind.}{=} 0
    \end{align*}
    where the rightmost pink string is the right red string when $k=4$. Furthermore, we have
    \begin{align*}
        \inputtikz{4.2/inj-proj/proof-k4.2-1}
         & = \inputtikz{4.2/inj-proj/proof-k4.2-2} + 2 \inputtikz{4.2/inj-proj/proof-k4.2-3} - 2 \inputtikz{4.2/inj-proj/proof-k4.2-4}
        \\ & \overset{ind.}{=} 2 \inputtikz{4.2/inj-proj/proof-k4.2-5} + 2 (-2) \inputtikz{4.2/inj-proj/proof-k4.2-5} - 2 \inputtikz{4.2/inj-proj/proof-k4.2-4}
        \\ & = - 2 \inputtikz{4.2/inj-proj/proof-k4.2-4}
    \end{align*}
    and
    \begin{align*}
        \inputtikz{4.2/inj-proj/proof-k4.3-1}
         & = 2 \inputtikz{4.2/inj-proj/proof-k4.3-2} - \inputtikz{4.2/inj-proj/proof-k4.3-3}
        \\ & \overset{ind.}{=} 2 \inputtikz{4.2/inj-proj/proof-k4.3-2} - (-2) \inputtikz{4.2/inj-proj/proof-k4.3-4}
        \\ & = 2 \inputtikz{4.2/inj-proj/proof-k4.3-2}.
    \end{align*}
\end{proof}


The objects of this category are identical to objects in $\cH(\wtilde{S}_2)$ and the morphisms are the same modulo the wall relations \eqref{rel:two-col-wall}. Naturally, we wonder whether double leaves form bases for the morphism spaces in $\mcal{DT}$. It is easy to see that double leaves appear in $\mcal{DT}$ by acting on $\vn$ by double leaves in $\cH(\wtilde{S}_2)$. All morphisms in $\mcal{DT}$ are morphisms in $\cH(\wtilde{S}_2)$ so they can be written as $\Z[\barbell, \barbell[dBlue]]$-linear combinations of double leaves, though some of these leaves have collapsed to $0$. This makes it clear that double leaves span the morphism spaces of $\mcal{DT}$ as (left) $\Z[\barbell, \barbell[dBlue]]$-modules. However they may not be linearly independent as neither left nor right modules as with the one-colour case. Although double leaves are not always a basis for its respective morphism space as $\Z[\barbell, \barbell[dBlue]]$-modules, it turns out a subset of them are a basis over $\Z$.

\begin{lemma}
    \label{lem:DT-double-leaves}
    Let $\pi: \op{mor}(\cH(\wtilde{S}_2)) \to \op{mor}(\mcal{DT})$ be the projection map which takes a morphism to the result of its action on the empty diagram $\vn$. Then the image $\pi(\mathbb{LL}(w,y))$ without zero morphisms is a basis for $\Hom_{\mcal{DT}}(w,y)$ as a $\C$-module.
\end{lemma}

\begin{proof}
    We consider morphisms $\Hom(w,y)$ in $\mcal{DT}$ for fixed objects $w,y$, and write $\mathbb{LL} \coloneqq \pi(\mathbb{LL}(w,y))$ for the set of double leaves in $\mcal{DT}$. Any diagram in $\mcal{DT}$ can be written as a $\C$-linear combination of morphisms without floating diagrams by pulling floating diagrams to the right with \eqref{rel:one-col-barbell-forcing} and \eqref{rel:two-col-barbell-forcing} then applying the wall relation \eqref{rel:two-col-wall}. We can write each of these as a $\Z[\barbell, \barbell[dBlue]]$-linear combination of double leaves with a right action, and reduce it to a $\Z$-linear combination by \eqref{rel:two-col-wall}. This implies that $\mathbb{LL}$ spans $\Hom(w,y)$ as a $\Z$-module. Now $\mathbb{LL}$ may not be linearly independent because the two-colour wall relation \eqref{rel:two-col-wall} reduces all diagrams factoring through a word ending with $t$ to $0$. The set of light leaves after removing morphisms killed by \eqref{rel:two-col-wall}, i.e. $\mathbb{LL} \setminus \{0\}$, still spans $\Hom(w,y)$ by the argument above. This set is linearly independent since, by construction, \eqref{rel:two-col-wall} has no effect on $\Z$-linear combinations of $\mathbb{LL} \setminus \{0\}$. Then it follows from linear independence over $\Z[\barbell, \barbell[dBlue]]$ that this set is linearly independent over $\Z$ in $\mcal{DT}$.
\end{proof}

We will show that the additive Karoubi envelope of this diagrammatic category is equivalent to $\Tilt(\fraksl_2)$. From now on, we write $\mcal{DT}$ for the $\C$-linear $\cH_\C(\wtilde{S}_2)$-module category obtained by replacing $\Z$ with $\C$ and $\cH(\wtilde{S}_2)$ with $\cH_\C(\wtilde{S}_2)$ in \autoref{def:DT}. The all the above discussion and results still apply to $\mcal{DT}$ over $\C$. Also write $\cT$ for $\Tilt(\fraksl_2)$.

Since $\cH_\C(S_2)$ appears inside $\cH_\C(\wtilde{S}_2)$ for each colour, \autoref{lem:ss-equal-2s} provides explicit isomorphisms $ss \cong s \oplus s$ and $tt \cong t \oplus t$ in the additive closure of $\cH_\C(\wtilde{S}_2)$.

\begin{definition}
    Let $F: \mcal{DT}^\oplus \to \Tilt(\fraksl_2)$ to be the additive $\C$-linear $\cH_\C(\wtilde{S}_2)$-module functor defined as follows \red{What about gradings?}. Map the empty word $\vn$ to the trivial module $T(\vn)$ \red{Is that right to say?}. Given a general word $s_n \dots s_1$ in $\mcal{DT}$, for $s_i \in \{s,t\}$, map $F(s_n \dots s_1) = \Theta_{s_n} \dots \Theta_{s_1} T(\vn)$ where $\Theta_s, \Theta_t$ are translation functors associated to generators of $\wtilde{S}_2$.

    On morphisms, we define $F$ recursively. Note that we only have red strands on the right since otherwise \eqref{rel:two-col-wall} reduces it to 0. For $k \geq 0$, define
    \begin{gather*}
        \id^d_k \coloneqq \inputtikz{4.2/tilt-identity-k}
        \quad , \quad
        i^d_k \coloneqq \inputtikz{4.2/tilt-inclusion-k}
        \quad , \quad
        p^d_k \coloneqq \inputtikz{4.2/tilt-projection-k}
    \end{gather*}
    where colours alternate and a red strand on the right when $k \neq 0$. Similarly we define $\id_k: \Theta_x \dots \Theta_s(T(\vn)) \to \Theta_x \dots \Theta_s(T(\vn))$, $i_k: \Theta_x \dots \Theta_s(T(\vn)) \to \Theta_y \Theta_x\dots \Theta_s(T(\vn))$ and $p_k: \Theta_y \Theta_x \dots \Theta_s(T(\vn)) \to \Theta_x\dots \Theta_s(T(\vn))$ to be the identity, inclusion and projection maps in $\cT$, where the subscripts alternate $s,t$ and $\Theta_x \dots \Theta_s$ is a composition of $k$ functors. Let $F(\id^d_k) = \id_k$. On the generators \eqref{eq:two-col-gen} of $\mcal{DT}$, map

    \begin{align*}
        \inputtikz{4.2/tilt-gen-identity}
         & \xmapsto{F}
        \begin{cases}
            \id_{k+1},  & \text{if $k$ even}, \\
            \pvec{\id_k & 0                   \\ 0 & \id_k}, &\text{if $k$ odd},
        \end{cases}
        \\
        \inputtikz{4.2/tilt-gen-inclusion}
         & \xmapsto{F}
        \begin{cases}
            p_k, & \text{if $k$ even},  \\
            \pvec{i_{k-1} \circ p_{k-1} \\ \id_k}, &\text{if $k$ odd},
        \end{cases}
        \\
        \inputtikz{4.2/tilt-gen-projection}
         & \xmapsto{F}
        \begin{cases}
            i_k,        & \text{if $k$ even},                          \\
            \pvec{\id_k & i_{k-1} \circ p_{k-1}}, & \text{if $k$ odd},
        \end{cases}
        \\
        \inputtikz{4.2/tilt-gen-multiplication}
         & \xmapsto{F}
        \begin{cases}
            \pvec{0 & \id_{k+1}}, & \text{if $k$ even},     \\
            \pvec{0 & 0           & \id_k               & 0 \\ 0 & 0 & 0 & \id_k}, &\text{if $k$ odd},
        \end{cases}
        \\
        \inputtikz{4.2/tilt-gen-division}
         & \xmapsto{F}
        \begin{cases}
            \pvec{\id_{k+1} \\ 0}, & \text{if $k$ even},     \\
            \pvec{\id_k & 0 \\ 0 & \id_k \\ 0 & 0 \\ 0 & 0}, &\text{if $k$ odd},
        \end{cases}
    \end{align*}
    where each entry in the matrix are matrices themselves. \red{CHECK THE SCALING!!!!!!!} For \dBlue{blue} generators, the definition is the same with the words even and odd swapped. Putting a \dRed{red} (resp. \dBlue{blue}) identity strands on the left of a diagram is applying $\Theta_s$ (resp. $\Theta_t$) to the output morphism. Pictorially, for a morphism $f$ in $\mcal{DT}$,
    \begin{gather*}
        \inputtikz{4.2/tilt-identity-on-left} \xmapsto {F} \Theta_s F(f).
    \end{gather*}
     We extend the functor by additivity, composition and linearity.
    
    The mappings that don't involve matrices are summarised in the picture below.
    \begin{equation}
        \label{img:tilt-functor}
        \inputtikz{4.2/tilt-proof-functor}
    \end{equation}
    The right wall on each diagram is not shown to reduce clutter.
\end{definition}

The definition on generators is a consequence of the isomorphism $\Theta_s \Theta_s \cong \Theta_s \oplus \Theta_s$ analogous to $ss \cong s \oplus s$ (and respectively for $t$) from \autoref{lem:ss-equal-2s}.

\begin{remark}
    The action of an arbitrary morphism of $\cH(\wtilde{S}_2)$ on the left of a morphism in $\mcal{DT}$ is sent to the Godement product\footnote{The horizontal composition of natural transformations.} of the natural transformations underlying the image of morphisms under $F$. Taking the Godement product of natural transformations $\Theta_x \dots \Theta_s \to \Theta_y \dots \Theta_s$, when viewed as diagrams in $\mcal{DT}$, is just a left tensor. Diagrammatically, the construction of looks like putting identity morphisms on the left of one morphism on the right of the other, so that the codomains align, and then composing them. In $\cT$, this is the Kronecker product of matrices.
\end{remark}


\begin{lemma}
    The functor $F$ is well defined.
\end{lemma}
\begin{proof} \red{CHECK}
    For red and blue, the isomorphisms given in \autoref{lem:ss-equal-2s} imply that any diagram in $\mcal{DT}$ is isomorphic to a matrix of $\C$-linear combinations of diagrams where the domain and codomain have alternating colours. Due to \eqref{rel:two-col-wall} non-trivial diagrams do not have blue strands next to the wall, so the alternating colours must end in red, i.e. the domain and codomain end in $s$. We constructed $F$ to be additive, so it suffices to consider morphisms between indecomposable summands of words alternating $s,t$ and ending in $s$. Recall that these alternating words have idempotents given by their Jones-Wenzl projectors. The discussion in \cite[Section 5.4.2]{elias-dihedral-cathedral} states that the images of Jones-Wenzl projectors give all the indecomposables \red{Is this right? Is the ref right--maybe see book thm9.22} and \cite[Corollary 4.21]{anderson-tubbenhauer-tilt} shows that all morphisms not killed by Jones-Wenzl Projectors are of the form
    \begin{center}
        \inputtikz{4.2/tilt-projection}
        \quad , \quad
        \inputtikz{4.2/tilt-inclusion}
        \quad , \quad
        \inputtikz{4.2/tilt-proj-incl}
    \end{center}
    with alternating colours, purple being either red or blue, and ``...'' are zero or more identity strands. Therefore the morphisms in the picture \eqref{img:tilt-functor} are enough to define a mapping on $\mcal{DT}$ by going through $\Kar(\mcal{DT})$ \red{Word this better}.

    Next, we check that all relations are preserved. From \cite[Proposition 5.84 and Lemma 5.87]{mazorchuk-lectures-sl2-modules} \red{Is this the right ref?}, we know that $\Theta_s$ is a Frobenius object in the category of endofunctors of $\cO$ and there are unit, counit, multiplication and comultiplication natural transformations from the Frobenius object structure. Applying these to $T(\vn)$ result in the same relations in $\Tilt(\fraksl_2)$ for $T(\vn), \Theta_s T(\vn)$ and $\Theta_s^2(T(\vn))$. Note that the projection and inclusion maps in the picture \eqref{img:tilt-functor} are exactly the unit and counit of $\Theta_s$ evaluated at $T(\vn)$, and the trivalent vertices provided by projecting the isomorphisms in \autoref{lem:ss-equal-2s} map exactly to the multiplication and comultiplication maps. Furthermore, in \cite[Section 2.4]{soergel-category-O} \red{Ref?} we see that $p_0 \circ i_0 = 0$ and \red{Is this right:} $\id_{\Theta_t(T(\vn))} = 0$ in $\Tilt(\fraksl_2)$ which is analogous to the two-colour wall relations \eqref{rel:two-col-wall}. Furthermore there is a relation $i_k \circ p_k = p_{k+1} \circ i_{k+1}$ up to a scalar multiple \red{ref?} which is analogous to [\red{PUT THE RELATION HERE}]. \red{What about $p_{k+1} \circ p_k = 0$?} Hence all the relations in $\mcal{DT}$ are preserved by $F$. By construction, $F$ preserves direct sums, $\C$-linear combinations and the Soregel module structure in \cite{soergel-category-O}, so $F$ is well defined as a functor between $\C$-linear $\cH(S_2)$-module categories.

    \red{Talk about how extending by composition makes sense.}
\end{proof}

\red{Say something here?}

The following result states that $\mcal{DT}$ is indeed a diagrammatic incarnation of $\Tilt(\fraksl_2)$.

\red{Be clear that I don't understand $\Tilt$ very well.}



\begin{theorem}[\red{???}]
    The diagrammatic category $\Kar(\mcal{DT}(\fraksl_2))$ and $\Tilt(\fraksl_2)$ are equivalent as $\C$-linear $\cH_\C(\wtilde{S}_2)$-module categories.
\end{theorem}

\begin{proof}
    As a shorthand, we write $\cT$ for $\Tilt(\fraksl_2)$.
    \grey{
        We now prove that $F$ is fully faithful. It follows from \autoref{lem:ss-equal-2s} and the description of $P(\vn)$ and $P(s)$ in \cite[Section 5.2]{mazorchuk-lectures-sl2-modules} that the image of $\unit$ and $\counit$ generate all morphisms of the form $\Theta_s^n(P(\vn)) \to \Theta_s^m(P(\vn))$. Hence $F$ is full. Now the mapping of $F$ on all morphism spaces are determined by those depicted in the above picture. So, for faithfulness, it suffices to compare the $\C$-dimensions of morphism spaces between objects shown in the picture. By \autoref{lem:DO_0-double-leaves}, $\Hom(\vn, \vn)$ has a basis $\{\vn = \id_\vn\}$, $\Hom(s, \vn)$ has a basis $\{\unit\}$, $\Hom(\vn, s)$ has a basis $\{\counit\}$, and $\Hom(s, s)$ has a basis $\{\id_s, \counit \circ \unit\}$. The bases for the corresponding morphism spaces in $\proj(\cO_0)$ are exactly those in the image \red{Ref? - that these are actually the bases of the hom spaces}, so these dimensions coincide. Therefore $F$ is fully faithful.

        All objects in $\proj(\cO_0)$ appear as direct sums and direct summands of the elements $\Theta_s^n(P(\vn))$ for non-negative integers $n$. Therefore the additive Karoubi envelope induces an equivalence $\Kar(\mcal{DO}_0) \cong \proj(\cO_0)$ as $\C$-linear left $\cH(S_2)$-module categories.
    }
\end{proof}

\red{Comment on grading?}

\red{Comment on consequences}


