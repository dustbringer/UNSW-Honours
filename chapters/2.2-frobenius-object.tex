\section{Frobenius Objects}
\label{sec:frob-object}

The structure of Frobenius objects gives rise to useful diagrammatics that can be defined up to isotopy. This section gives some background to the objects we will encounter in \autoref{sec:one-col-sbim} and beyond.

Let $\mcal{C}$ be a (strict) monoidal category.

\begin{definition}
    A \textit{monoid object} in $\mcal{C}$ is a triple $(M,\mu,\eta)$ for an object $M \in \mcal{C}$, a \textit{multiplication} map $\mu: M \tensor M \to M$ and a \textit{unit} map $\eta: \mathbbb{1} \to M$, such that
    \begin{gather*}
        \begin{mytikzcd}
    & M \tensor M \tensor M \arrow[dl,"\mu \tensor \id_M"'] \arrow[dr, "\id_M \tensor \mu"] \\
    M \tensor M \arrow[dr, "\mu"']
    && M \tensor M \arrow[dl, "\mu"] \\
    & M
\end{mytikzcd}
    \end{gather*}
    and
    \begin{gather*}
        \begin{mytikzcd}
    \mathbbb{1} \tensor M
    \arrow[r, "\eta \tensor \id_M"]
    \arrow[dr, "\id_M"']
    & M \tensor M
    \arrow[d, "\mu"]
    & M \tensor \mathbbb{1}
    \arrow[l, "\id_M \tensor \eta"']
    \arrow[dl, "\id_M"]
    \\
    & M
\end{mytikzcd}
    \end{gather*}
    commute. The first diagram is the \textit{associativity} relation $\mu \circ (\mu \tensor \id_M) = \mu \circ (\id_M \tensor \mu)$ and the second diagram is the \textit{unit} relation $\id_M = \mu \circ (\eta \tensor \id_M) = \mu \circ (\id_M \tensor \eta)$.

    Dually, a \textit{comonoid object} in $\mcal{C}$ is a triple $(M,\delta,\epsilon)$ for an object $M \in \mcal{C}$, a \textit{comultiplication} map $\delta: M \to M \tensor M$ and a \textit{counit} map $\epsilon: M \to \mathbbb{1}$, satisfying the \textit{coassociativity} relation
    \begin{gather*}
        \begin{mytikzcd}[arrows=<-]
    & M \tensor M \tensor M \arrow[dl,"\delta \tensor \id_M"'] \arrow[dr, "\id_M \tensor \delta"] \\
    M \tensor M \arrow[dr, "\delta"']
    && M \tensor M \arrow[dl, "\delta"] \\
    & M
\end{mytikzcd}
    \end{gather*}
    and \textit{counit} relation
    \begin{gather*}
        \begin{mytikzcd}[arrows=<-]
    \mathbbb{1} \tensor M
    \arrow[r, "\epsilon \tensor \id_M"]
    \arrow[dr, "\id_M"']
    & M \tensor M
    \arrow[d, "\delta"]
    & M \tensor \mathbbb{1}
    \arrow[l, "\id_M \tensor \epsilon"']
    \arrow[dl, "\id_M"]
    \\
    & M
\end{mytikzcd}.
    \end{gather*}
\end{definition}

Monoid objects generalise monoids in algebra, i.e. sets with an identity equipped with an associative binary operation.

\begin{definition}
    A \textit{Frobenius object} in $\mcal{C}$ is a quintuple $(A,\mu,\eta,\delta,\epsilon)$ such that $(A,\mu,\eta)$ is a monoid object, $(A,\delta,\epsilon)$ is a comonoid object, and the maps satisfy the \textit{Frobenius relations}
    \begin{gather*}
        \begin{mytikzcd}
    & A \tensor A
    \arrow[dl, "\delta \tensor \id_A"']
    \arrow[d, "\mu"]
    \arrow[dr, "\id_A \tensor \delta"]
    \\
    A \tensor A \tensor A
    \arrow[dr, "\id_A \tensor \mu"']
    & A
    \arrow[d, "\delta"]
    & A \tensor A \tensor A
    \arrow[dl, "\mu \tensor \id_A"]
    \\
    & A \tensor A
\end{mytikzcd},
    \end{gather*}
    that is $(\id_A \tensor \mu) \circ (\delta \tensor \id_A) = \delta \circ \mu = (\mu \tensor \id_A) \circ (\id_A \tensor \delta)$.
\end{definition}


The maps and relations for a Frobenius object $(A,\mu,\eta,\delta,\epsilon)$ have a pleasant description via the diagrams given in \autoref{sec:monoidal-cat}. The structure maps are drawn as
\begin{gather*}
    \inputtikz{2.2/frob-multiplication}
    \quad , \quad
    \inputtikz{2.2/frob-unit}
    \quad , \quad
    \inputtikz{2.2/frob-comultiplication}
    \quad , \quad
    \inputtikz{2.2/frob-counit}.
\end{gather*}

For the rest of this section, we will only work with the Frobenius object $A$ and $\mathbbb{1}$. We can stop putting the label $A$ by identifying $A$ with the identity strand $\mathsf{I} = \id_A$. Diagrammatically, the associativity relation $\mu \circ (\mu \tensor \id_M) = \mu \circ (\id_M \tensor \mu)$ is
\begin{gather*}
    \inputtikz{2.2/frob-relation-associativity1}
    =
    \inputtikz{2.2/frob-relation-associativity2},
\end{gather*}
the coassociativity relation $(\delta \tensor \id_A) \circ \delta = (\id_A \tensor \delta) \circ \delta$ is
\begin{gather*}
    \inputtikz{2.2/frob-relation-coassociativity1}
    =
    \inputtikz{2.2/frob-relation-coassociativity2},
\end{gather*}
the unit relation $\id_A = \mu \circ (\eta \tensor \id_A) = \mu \circ (\id_A \tensor \eta)$ is
\begin{gather*}
    \inputtikz{2.2/frob-relation-unit1}
    =
    \inputtikz{2.2/frob-relation-unit2}
    =
    \inputtikz{2.2/frob-relation-unit3},
\end{gather*}
the counit relation $\id_A = (\epsilon \tensor \id_A) \circ \delta = (\id_A \tensor \epsilon) \circ \delta$ is
\begin{gather*}
    \inputtikz{2.2/frob-relation-counit1}
    =
    \inputtikz{2.2/frob-relation-counit2}
    =
    \inputtikz{2.2/frob-relation-counit3},
\end{gather*}
and the Frobenius relation $(\id_A \tensor \mu) \circ (\delta \tensor \id_A) = \delta \circ \mu = (\mu \tensor \id_A) \circ (\id_A \tensor \delta)$ is
\begin{gather*}
    \inputtikz{2.2/frob-relation-frob1}
    =
    \inputtikz{2.2/frob-relation-frob2}
    =
    \inputtikz{2.2/frob-relation-frob3}.
\end{gather*}


To further simply the diagrams, we stop labelling the morphisms and draw the structure maps as univalent and trivalent vertices
\begin{gather*}
    \tag{FrobGen}\label{def:frob-generators}
    \inputtikz{2.2/frob-simple-multiplication}
    \quad , \quad
    \inputtikz{2.2/frob-simple-unit}
    \quad , \quad
    \inputtikz{2.2/frob-simple-comultiplication}
    \quad , \quad
    \inputtikz{2.2/frob-simple-counit},
\end{gather*}
where the large dot on the unit and counit indicates that the string stops before reaching the other end. Then the relations become
\begin{gather}
    \tag{Frob1}\label{rel:frob-associativity}
    \inputtikz{2.2/frob-simple-relation-associativity1}
    = \inputtikz{2.2/frob-simple-relation-associativity2}
    \quad , \quad
    \inputtikz{2.2/frob-simple-relation-coassociativity1}
    = \inputtikz{2.2/frob-simple-relation-coassociativity2},
    \\
    \tag{Frob2}\label{rel:frob-unit}
    \inputtikz{2.2/frob-simple-relation-unit1}
    = \inputtikz{2.2/frob-simple-relation-unit2}
    = \inputtikz{2.2/frob-simple-relation-unit3}
    \quad , \quad
    \inputtikz{2.2/frob-simple-relation-counit1}
    = \inputtikz{2.2/frob-simple-relation-counit2}
    = \inputtikz{2.2/frob-simple-relation-counit3},
\end{gather}
and
\begin{gather}
    \tag{Frob3}\label{rel:frob-relation}
    \inputtikz{2.2/frob-simple-relation-frob1}
    =
    \inputtikz{2.2/frob-simple-relation-frob2}
    =
    \inputtikz{2.2/frob-simple-relation-frob3}.
\end{gather}

If we write cups and caps for the diagrams
\begin{gather} \label{def:frob-cup-cap}
    \inputtikz{2.2/frob-isotopy-cap1}
    \coloneqq
    \inputtikz{2.2/frob-isotopy-cap2}
    \quad , \quad
    \inputtikz{2.2/frob-isotopy-cup1}
    \coloneqq
    \inputtikz{2.2/frob-isotopy-cup2},
\end{gather}
then the Frobenius object relations admit a more familiar form of (planar) isotopy by the relations
\begin{gather}
    \tag{Iso1}
    \label{rel:frob-isotopy1}
    \inputtikz{2.2/frob-isotopy-bend1}
    = \inputtikz{2.2/frob-isotopy-bend2}
    = \inputtikz{2.2/frob-isotopy-bend3},
\end{gather}
which we saw in for the Temperley-Lieb-Jones category. For instance the first equality follows from \eqref{rel:frob-relation} and \eqref{rel:frob-unit},
\begin{gather*}
    \inputtikz{2.2/frob-isotopy-bend1}
    = \inputtikz{2.2/frob-isotopy-bend-eg1}
    = \inputtikz{2.2/frob-isotopy-bend-eg2}
    = \inputtikz{2.2/frob-isotopy-bend-eg3}.
\end{gather*}

\begin{remark} \label{rk:frob-self-dual}
    This implies that Frobenius objects $A$ are dualisable and self-dual, with the unit of duality given by the cap $A \tensor A \to \mathbbb{1}$ and the counit given by the cup $\mathbbb{1} \to A \tensor A$ above. The triangle identities for duality are exactly the relation \eqref{rel:frob-isotopy1}, which is sometimes called the zig-zag relation. Alternatively, this corresponds to the left tensor functor $A \tensor -$ being self-adjoint by a similar argument.
\end{remark}

We can similarly deduce more isotopy relations
\begin{gather}
    \tag{Iso2}\label{rel:frob-isotopy2}
    \inputtikz{2.2/frob-isotopy-univalent-rot1.1}
    = \inputtikz{2.2/frob-isotopy-univalent-rot1.2}
    = \inputtikz{2.2/frob-isotopy-univalent-rot1.3}
    \quad , \quad
    \inputtikz{2.2/frob-isotopy-univalent-rot2.1}
    = \inputtikz{2.2/frob-isotopy-univalent-rot2.2}
    = \inputtikz{2.2/frob-isotopy-univalent-rot2.3}
    \\
    \tag{Iso3}\label{rel:frob-isotopy3}
    \inputtikz{2.2/frob-isotopy-trivalent-rot1.1}
    = \inputtikz{2.2/frob-isotopy-trivalent-rot1.2}
    = \inputtikz{2.2/frob-isotopy-trivalent-rot1.3}
    \quad , \quad
    \inputtikz{2.2/frob-isotopy-trivalent-rot2.1}
    = \inputtikz{2.2/frob-isotopy-trivalent-rot2.2}
    = \inputtikz{2.2/frob-isotopy-trivalent-rot2.3}
\end{gather}
which can be thought of as ``rotating vertices''. Using these identities, we can rotate entire diagrams by putting caps and cups around it.
\begin{example}
    The unit relation can be rotated to the counit map
    \begin{align*}
        \inputtikz{2.2/example-rotation1.1}
        \leadsto
        \inputtikz{2.2/example-rotation1.2}
         & = \inputtikz{2.2/example-rotation1.3}.
    \end{align*}
    where the equality follows from \eqref{rel:frob-isotopy2}.
\end{example}

\begin{example}
    The comultiplication map can be rotated to the multiplication map
    \begin{align*}
        \inputtikz{2.2/example-rotation2.1}
        \leadsto
        \inputtikz{2.2/example-rotation2.2}
         & = \inputtikz{2.2/example-rotation2.3}
        \\ & = \inputtikz{2.2/example-rotation2.4}
        \\ & = \inputtikz{2.2/example-rotation2.5}
        \\ & = \inputtikz{2.2/example-rotation2.6}.
    \end{align*}
    where the equality follows from applying \eqref{rel:frob-isotopy3} three times then \eqref{rel:frob-isotopy1}.
\end{example}

In fact, Frobenius objects satisfy all possible isotopy relations\footnote{This is a consequence of the well known connections to 2dTQFTs, see for example \cite{kock_frob_algebra}.}. We can therefore consider the diagrams generated by concatenations of Frobenius structure maps up to planar isotopy. That is, we equate two diagrams if one diagram can be continuously deformed to the other in the plane without crossing. In this way, we can just use our visual intuition in place of applying any specific isotopy relations from \eqref{rel:frob-isotopy1}-\eqref{rel:frob-isotopy3}.

The Frobenius object relations \eqref{rel:frob-associativity}, \eqref{rel:frob-unit}, \eqref{rel:frob-relation} can be simplified as following. The unit and counit relations are
\begin{gather*}
    \inputtikz{2.2/relation1.1}
    = \inputtikz{2.2/relation1.2}
    \left(
    = \inputtikz{2.2/relation1.3}
    \right),
\end{gather*}
where the second equality follows from rotating the first one with cups and caps. Here the horizontal line has no innate meaning in the category but isotopically asserts equality between the ``bent up'' and ``bent down'' diagrams in \eqref{rel:frob-unit}.

Note that allowing isotopy, the Frobenius relation \eqref{rel:frob-relation} implies the associativity and coassociativity relations \eqref{rel:frob-associativity}. For instance, we have
\begin{gather*}
    \inputtikz{2.2/frob-rel-to-associativity1}
    = \inputtikz{2.2/frob-rel-to-associativity2}
    = \inputtikz{2.2/frob-rel-to-associativity3}
    = \inputtikz{2.2/frob-rel-to-associativity4},
\end{gather*}
where the second equality is the Frobenius relation. For completeness, this calculation shows the trivalent rotations \eqref{rel:frob-isotopy3}, but the reader is encouraged to think of the first and third equalities as isotopic deformations.

Therefore, up to isotopy, the Frobenius object relations are summed by the unit and Frobenius relation
\begin{gather}
    \tag{FrobRel} \label{rel:frob-universal}
    \inputtikz{2.2/relation1.1}
    = \inputtikz{2.2/relation1.2}
    \quad , \quad
    \inputtikz{2.2/relation2.1}
    = \inputtikz{2.2/relation2.2}.
\end{gather}

\begin{remark}
    Let $\op{Frob}$ be the monoidal category with objects generated by an object $\mathsf{I}$ and morphisms generated by the diagrams \eqref{def:frob-generators} with relations \eqref{rel:frob-universal}. For any monoidal category $\mcal{C}$ with a Frobenius object $A$, there exists a functor\footnote{This functor need not be full nor faithful, as there may be more morphisms in the target category which could satisfy more relations.} $\op{Frob} \to \mcal{C}$ mapping $\mathsf{I} \mapsto A$ and the morphisms in \eqref{def:frob-generators} to the corresponding Frobenius structure maps. In other words, $\op{Frob}$ is the free monoidal category generated by a Frobenius object.
\end{remark}

\begin{remark}
    Noting \autoref{rk:frob-self-dual}, if we define the generating object $\mathsf{I}$ to be self-dual, then we automatically get cups and caps as adjunction morphisms\footnote{The cups and caps align with the generators by \eqref{def:frob-cup-cap}.} satisfying the relation \eqref{rel:frob-isotopy1}. In this case, we can rotate diagrams so we would only need the first two generators in \eqref{def:frob-generators}.
    % This kind of thing is a case of a pivotal category
\end{remark}

These objects will appear again in the context of diagrammatic Soergel bimodules in \autoref{sec:one-col-sbim}.
