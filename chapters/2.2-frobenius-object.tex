\section{Frobenius Objects}
\label{sec:2.2}

\red{Something something about }
\red{Many relations in categorical structures can be written in diagrammatic terms - adjunctions, monoid}

\red{Something about isotopy}

Let $\mcal{C}$ be a (strict) monoidal category. We can define the following objects.

\begin{definition}
    A \textit{monoid object} in $\mcal{C}$ is a triple $(M,\mu,\eta)$ for an object $M \in \mcal{C}$, a \textit{multiplication} map $\mu: M \tensor M \to M$ and a \textit{unit} map $\eta: \mathbbb{1} \to M$, such that
    \begin{center}
        \begin{mytikzcd}
    & M \tensor M \tensor M \arrow[dl,"\mu \tensor \id_M"'] \arrow[dr, "\id_M \tensor \mu"] \\
    M \tensor M \arrow[dr, "\mu"']
    && M \tensor M \arrow[dl, "\mu"] \\
    & M
\end{mytikzcd}
    \end{center}
    and
    \begin{center}
        \begin{mytikzcd}
    \mathbbb{1} \tensor M
    \arrow[r, "\eta \tensor \id_M"]
    \arrow[dr, "\id_M"']
    & M \tensor M
    \arrow[d, "\mu"]
    & M \tensor \mathbbb{1}
    \arrow[l, "\id_M \tensor \eta"']
    \arrow[dl, "\id_M"]
    \\
    & M
\end{mytikzcd}
    \end{center}
    commute. The first diagram is the \textit{associativity} relation $\mu \circ (\mu \tensor \id_M) = \mu \circ (\id_M \tensor \mu)$ and the second diagram is the \textit{unit} relation $\id_M = \mu \circ (\eta \tensor \id_M) = \mu \circ (\id_M \tensor \eta)$.

    Dually, a \textit{comonoid object} in $\mcal{C}$ is a triple $(M,\delta,\epsilon)$ for an object $M \in \mcal{C}$, a \textit{comultiplication} map $\delta: M \to M \tensor M$ and a \textit{counit} map $\epsilon: M \to \mathbbb{1}$, satisfying the \textit{coassociativity} relation
    \begin{center}
        \begin{mytikzcd}[arrows=<-]
    & M \tensor M \tensor M \arrow[dl,"\delta \tensor \id_M"'] \arrow[dr, "\id_M \tensor \delta"] \\
    M \tensor M \arrow[dr, "\delta"']
    && M \tensor M \arrow[dl, "\delta"] \\
    & M
\end{mytikzcd}
    \end{center}
    and \textit{counit} relation
    \begin{center}
        \begin{mytikzcd}[arrows=<-]
    \mathbbb{1} \tensor M
    \arrow[r, "\epsilon \tensor \id_M"]
    \arrow[dr, "\id_M"']
    & M \tensor M
    \arrow[d, "\delta"]
    & M \tensor \mathbbb{1}
    \arrow[l, "\id_M \tensor \epsilon"']
    \arrow[dl, "\id_M"]
    \\
    & M
\end{mytikzcd}.
    \end{center}
\end{definition}

Monoid objects generalise monoids, i.e. sets with an identity equipped with an associative binary operation.

\begin{definition}
    A \textit{Frobenius object} in $\mcal{C}$ is a quintuple $(A,\mu,\eta,\delta,\epsilon)$ such that $(A,\mu,\eta)$ is a monoid object, $(A,\delta,\epsilon)$ is a comonoid object, and the maps satisfy the \textit{Frobenius relations}
    \begin{center}
        \begin{mytikzcd}
    & A \tensor A
    \arrow[dl, "\delta \tensor \id_A"']
    \arrow[d, "\mu"]
    \arrow[dr, "\id_A \tensor \delta"]
    \\
    A \tensor A \tensor A
    \arrow[dr, "\id_A \tensor \mu"']
    & A
    \arrow[d, "\delta"]
    & A \tensor A \tensor A
    \arrow[dl, "\mu \tensor \id_A"]
    \\
    & A \tensor A
\end{mytikzcd},
    \end{center}
    that is $(\id_A \tensor \mu) \circ (\delta \tensor \id_A) = \delta \circ \mu = (\mu \tensor \id_A) \circ (\id_A \tensor \delta)$.
\end{definition}


The maps and relations for a Frobenius object $(A,\mu,\eta,\delta,\epsilon)$ have a nice description with the diagrams given in \autoref{sec:2.1}. The structure maps are drawn as
\begin{center}
    \inputtikz{2.2/frob-multiplication}
    \quad , \quad
    \inputtikz{2.2/frob-unit}
    \quad , \quad
    \inputtikz{2.2/frob-comultiplication}
    \quad , \quad
    \inputtikz{2.2/frob-counit}.
\end{center}

For the rest of this section, we only work with the Frobenius object $A$ and $\mathbbb{1}$. We can stop putting the label $A$ by identifying $A$ with the identity strand $\mathsf{I} = \id_A$. Diagrammatically, the associativity relation $\mu \circ (\mu \tensor \id_M) = \mu \circ (\id_M \tensor \mu)$ is
\begin{center}
    \inputtikz{2.2/frob-relation-associativity1}
    =
    \inputtikz{2.2/frob-relation-associativity2},
\end{center}
the coassociativity relation $(\delta \tensor \id_A) \circ \delta = (\id_A \tensor \delta) \circ \delta$ is
\begin{center}
    \inputtikz{2.2/frob-relation-coassociativity1}
    =
    \inputtikz{2.2/frob-relation-coassociativity2},
\end{center}
the unit relation $\id_A = \mu \circ (\eta \tensor \id_A) = \mu \circ (\id_A \tensor \eta)$ is
\begin{center}
    \inputtikz{2.2/frob-relation-unit1}
    =
    \inputtikz{2.2/frob-relation-unit2}
    =
    \inputtikz{2.2/frob-relation-unit3},
\end{center}
the counit relation $\id_A = (\epsilon \tensor \id_A) \circ \delta = (\id_A \tensor \epsilon) \circ \delta$ is
\begin{center}
    \inputtikz{2.2/frob-relation-counit1}
    =
    \inputtikz{2.2/frob-relation-counit2}
    =
    \inputtikz{2.2/frob-relation-counit3},
\end{center}
and the Frobenius relation $(\id_A \tensor \mu) \circ (\delta \tensor \id_A) = \delta \circ \mu = (\mu \tensor \id_A) \circ (\id_A \tensor \delta)$ is
\begin{center}
    \inputtikz{2.2/frob-relation-frob1}
    =
    \inputtikz{2.2/frob-relation-frob2}
    =
    \inputtikz{2.2/frob-relation-frob3}.
\end{center}


To simplify the diagrams, we stop labelling the functions and draw the structure maps as
\begin{center}
    \inputtikz{2.2/frob-simple-multiplication}
    \quad , \quad
    \inputtikz{2.2/frob-simple-unit}
    \quad , \quad
    \inputtikz{2.2/frob-simple-comultiplication}
    \quad , \quad
    \inputtikz{2.2/frob-simple-counit}.
\end{center}
So the relations become
\begin{center}
    \inputtikz{2.2/frob-simple-relation-associativity1}
    =
    \inputtikz{2.2/frob-simple-relation-associativity2}, \\

    \inputtikz{2.2/frob-simple-relation-coassociativity1}
    =
    \inputtikz{2.2/frob-simple-relation-coassociativity2}, \\

    \inputtikz{2.2/frob-simple-relation-unit1}
    =
    \inputtikz{2.2/frob-simple-relation-unit2}
    =
    \inputtikz{2.2/frob-simple-relation-unit3}, \\

    \inputtikz{2.2/frob-simple-relation-counit1}
    =
    \inputtikz{2.2/frob-simple-relation-counit2}
    =
    \inputtikz{2.2/frob-simple-relation-counit3},
\end{center}
and
\begin{center}
    \inputtikz{2.2/frob-simple-relation-frob1}
    =
    \inputtikz{2.2/frob-simple-relation-frob2}
    =
    \inputtikz{2.2/frob-simple-relation-frob3}.
\end{center}




\red{Talk about isotopy}


\red{Maybe something about the (diagrammatic?) category Frob, capturing the data of a frobenius object}

