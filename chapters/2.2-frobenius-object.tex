\section{Frobenius Objects}
\label{sec:2.2}

\red{Something something about }
\red{Many relations in categorical structures can be written in diagrammatic terms - adjunctions, monoid}

\red{Something about isotopy}

Let $\mcal{C}$ be a (strict) monoidal category. We can define the following objects.

\begin{definition}
    A \textit{monoid object} in $\mcal{C}$ is a triple $(M,\mu,\eta)$ for an object $M \in \mcal{C}$, a \textit{multiplication} map $\mu: M \tensor M \to M$ and a \textit{unit} map $\eta: \mathbbb{1} \to M$, such that
    \begin{center}
        \begin{mytikzcd}
    & M \tensor M \tensor M \arrow[dl,"\mu \tensor \id_M"'] \arrow[dr, "\id_M \tensor \mu"] \\
    M \tensor M \arrow[dr, "\mu"']
    && M \tensor M \arrow[dl, "\mu"] \\
    & M
\end{mytikzcd}
    \end{center}
    and
    \begin{center}
        \begin{mytikzcd}
    \mathbbb{1} \tensor M
    \arrow[r, "\eta \tensor \id_M"]
    \arrow[dr, "\id_M"']
    & M \tensor M
    \arrow[d, "\mu"]
    & M \tensor \mathbbb{1}
    \arrow[l, "\id_M \tensor \eta"']
    \arrow[dl, "\id_M"]
    \\
    & M
\end{mytikzcd}
    \end{center}
    commute. The first diagram is the \textit{associativity} relation $\mu \circ (\mu \tensor \id_M) = \mu \circ (\id_M \tensor \mu)$ and the second diagram is the \textit{unit} relation $\id_M = \mu \circ (\eta \tensor \id_M) = \mu \circ (\id_M \tensor \eta)$.

    Dually, a \textit{comonoid object} in $\mcal{C}$ is a triple $(M,\delta,\epsilon)$ for an object $M \in \mcal{C}$, a \textit{comultiplication} map $\delta: M \to M \tensor M$ and a \textit{counit} map $\epsilon: M \to \mathbbb{1}$, satisfying the \textit{coassociativity} relation
    \begin{center}
        \begin{mytikzcd}[arrows=<-]
    & M \tensor M \tensor M \arrow[dl,"\delta \tensor \id_M"'] \arrow[dr, "\id_M \tensor \delta"] \\
    M \tensor M \arrow[dr, "\delta"']
    && M \tensor M \arrow[dl, "\delta"] \\
    & M
\end{mytikzcd}
    \end{center}
    and \textit{counit} relation
    \begin{center}
        \begin{mytikzcd}[arrows=<-]
    \mathbbb{1} \tensor M
    \arrow[r, "\epsilon \tensor \id_M"]
    \arrow[dr, "\id_M"']
    & M \tensor M
    \arrow[d, "\delta"]
    & M \tensor \mathbbb{1}
    \arrow[l, "\id_M \tensor \epsilon"']
    \arrow[dl, "\id_M"]
    \\
    & M
\end{mytikzcd}.
    \end{center}
\end{definition}

Monoid objects generalise monoids, i.e. sets with an identity equipped with an associative binary operation.

\begin{definition}
    A \textit{Frobenius object} in $\mcal{C}$ is a quintuple $(A,\mu,\eta,\delta,\epsilon)$ such that $(A,\mu,\eta)$ is a monoid object, $(A,\delta,\epsilon)$ is a comonoid object, and the maps satisfy the \textit{Frobenius relations}
    \begin{center}
        \begin{mytikzcd}
    & A \tensor A
    \arrow[dl, "\delta \tensor \id_A"']
    \arrow[d, "\mu"]
    \arrow[dr, "\id_A \tensor \delta"]
    \\
    A \tensor A \tensor A
    \arrow[dr, "\id_A \tensor \mu"']
    & A
    \arrow[d, "\delta"]
    & A \tensor A \tensor A
    \arrow[dl, "\mu \tensor \id_A"]
    \\
    & A \tensor A
\end{mytikzcd},
    \end{center}
    that is $(\id_A \tensor \mu) \circ (\delta \tensor \id_A) = \delta \circ \mu = (\mu \tensor \id_A) \circ (\id_A \tensor \delta)$.
\end{definition}


The maps and relations for a Frobenius object $(A,\mu,\eta,\delta,\epsilon)$ have a nice description with the diagrams given in \autoref{sec:2.1}. The structure maps are drawn as
\begin{center}
    \tikz[vcenter, scale=0.4]{
    % \tikzfixsize{(0,0)}{(4,4)}
    % \node[below] (dom) at (-3,0) {$A \tensor A$};
    % \node[above] (cod) at (-3,4) {$A$};
    % \path[|->, >=stealth']
    % (dom) edge node[left] {$\mu$} (cod);
    %
    \path
    (0,0) edge[string=black] (1.5,2)
    (3,0) edge[string=black] (1.5,2)
    (1.5,2) edge[string=black] (1.5,4);
    \node[box] at (1.5,2) {$\mu$};
    % labels
    \node[below] at (0,0) {$A$};
    \node[below] at (3,0) {$A$};
    \node[above] at (1.5,4) {$A$};
}
    \quad , \quad
    \tikz[vcenter, scale=0.4]{
    \tikzfixsize{(0,0)}{(2,4)}
    %
    \path
    (1,2) edge[string=black] (1,4);
    \node[box] at (1,2) {$\eta$};
    % labels
    \node[above] at (1,4) {$A$};
    \node[below] at (1,0) {\phantom{$\mathbbb{1}$}};
}
    \quad , \quad
    \tikz[vcenter, scale=0.4]{
    % \tikzfixsize{(0,0)}{(4,4)}
    %
    \path
    (0,4) edge[string=black] (1.5,2)
    (3,4) edge[string=black] (1.5,2)
    (1.5,2) edge[string=black] (1.5,0);
    \node[box] at (1.5,2) {$\delta$};
    % labels
    \node[below] at (1.5,0) {$A$};
    \node[above] at (0,4) {$A$};
    \node[above] at (3,4) {$A$};
}
    \quad , \quad
    \tikz[vcenter, scale=0.4]{
    \tikzfixsize{(0,0)}{(2,4)}
    %
    \path
    (1,2) edge[string=black] (1,0);
    \node[box] at (1,2) {$\epsilon$};
    % labels
    \node[below] at (1,0) {$A$};
    \node[above] at (1,4) {\phantom{$\mathbbb{1}$}};
}.
\end{center}

For the rest of this section, we only work with the Frobenius object $A$ and $\mathbbb{1}$. We can stop putting the label $A$ by identifying $A$ with the identity strand $\mathsf{I} = \id_A$. Diagrammatically, the associativity relation $\mu \circ (\mu \tensor \id_M) = \mu \circ (\id_M \tensor \mu)$ is
\begin{center}
    \tikz[vcenter, scale=0.3]{
    \tikzfixsize{(0,0)}{(6+.5,8)}
    %
    \path
    (0,0) edge[string=black] (2,2)
    (4,0) edge[string=black] (2,2)
    (2,2) edge[string=black] (2,4);
    \node[box] at (2,2) {$\mu$};
    \path
    (2,4) edge[string=black, round] (4,6)
    (6,4) edge[string=black, round] (4,6)
    (4,6) edge[string=black] (4,8);
    \node[box] at (4,6) {$\mu$};
    \path (6,0) edge[string=black] (6,4);
}
    =
    \tikz[vcenter, scale=0.3]{
    \tikzfixsize{(0-.5,0)}{(6,8)}
    %
    \path (0,0) edge[string=black] (0,4);
    \path
    (6,0) edge[string=black] (4,2)
    (2,0) edge[string=black] (4,2)
    (4,2) edge[string=black] (4,4);
    \node[box] at (4,2) {$\mu$};
    \path
    (4,4) edge[string=black, round] (2,6)
    (0,4) edge[string=black, round] (2,6)
    (2,6) edge[string=black] (2,8);
    \node[box] at (2,6) {$\mu$};
},
\end{center}
the coassociativity relation $(\delta \tensor \id_A) \circ \delta = (\id_A \tensor \delta) \circ \delta$ is
\begin{center}
    \tikz[vcenter, scale=0.3]{
    \tikzfixsize{(0,0)}{(6+.5,8)}
    %
    \path
    (0,8) edge[string=black] (2,6)
    (4,8) edge[string=black] (2,6)
    (2,6) edge[string=black] (2,4);
    \node[box] at (2,6) {$\delta$};
    \path
    (2,4) edge[string=black, round] (4,2)
    (6,4) edge[string=black, round] (4,2)
    (4,2) edge[string=black] (4,0);
    \node[box] at (4,2) {$\delta$};
    \path (6,8) edge[string=black] (6,4);
}
    =
    \tikz[vcenter, scale=0.3]{
    \tikzfixsize{(0-.5,0)}{(6,8)}
    %
    \path (0,8) edge[string=black] (0,4);
    \path
    (6,8) edge[string=black] (4,6)
    (2,8) edge[string=black] (4,6)
    (4,6) edge[string=black] (4,4);
    \node[box] at (4,6) {$\delta$};
    \path
    (4,4) edge[string=black, round] (2,2)
    (0,4) edge[string=black, round] (2,2)
    (2,2) edge[string=black] (2,0);
    \node[box] at (2,2) {$\delta$};
},
\end{center}
the unit relation $\id_A = \mu \circ (\eta \tensor \id_A) = \mu \circ (\id_A \tensor \eta)$ is
\begin{center}
    \tikz[vcenter, scale=0.3]{
    \tikzfixsize{(0,0)}{(2,8)}
    %
    \path
    (1,0) edge[string=black] (1,8);
}
    =
    \tikz[vcenter, scale=0.3]{
    \tikzfixsize{(0-.5,0)}{(4+.5,8)}
    %
    \path (4,0) edge[string=black] (4,4);
    \path
    (0,2) edge[string=black] (0,4);
    \node[box] at (0,2) {$\eta$};
    \path
    (0,4) edge[string=black, round] (2,6)
    (4,4) edge[string=black, round] (2,6)
    (2,6) edge[string=black] (2,8);
    \node[box] at (2,6) {$\mu$};
}
    =
    \tikz[vcenter, scale=0.3]{
    \tikzfixsize{(0-.5,0)}{(4,8)}
    %
    \path (0,0) edge[string=black] (0,4);
    \path
    (4,2) edge[string=black] (4,4);
    \node[box] at (4,2) {$\eta$};
    \path
    (0,4) edge[string=black, round] (2,6)
    (4,4) edge[string=black, round] (2,6)
    (2,6) edge[string=black] (2,8);
    \node[box] at (2,6) {$\mu$};
},
\end{center}
the counit relation $\id_A = (\epsilon \tensor \id_A) \circ \delta = (\id_A \tensor \epsilon) \circ \delta$ is
\begin{center}
    \tikz[vcenter, scale=0.3]{
    \tikzfixsize{(0,0)}{(2,8)}
    %
    \path
    (1,0) edge[string=black] (1,8);
}
    =
    \tikz[vcenter, scale=0.3]{
    \tikzfixsize{(0-.5,0)}{(4+.5,8)}
    %
    \path (4,8) edge[string=black] (4,4);
    \path
    (0,6) edge[string=black] (0,4);
    \node[box] at (0,6) {$\epsilon$};
    \path
    (0,4) edge[string=black, round] (2,2)
    (4,4) edge[string=black, round] (2,2)
    (2,2) edge[string=black] (2,0);
    \node[box] at (2,2) {$\delta$};
}
    =
    \tikz[vcenter, scale=0.3]{
    \tikzfixsize{(0-.5,0)}{(4+.5,8)}
    %
    \path (0,8) edge[string=black] (0,4);
    \path
    (4,6) edge[string=black] (4,4);
    \node[box] at (4,6) {$\epsilon$};
    \path
    (0,4) edge[string=black, round] (2,2)
    (4,4) edge[string=black, round] (2,2)
    (2,2) edge[string=black] (2,0);
    \node[box] at (2,2) {$\delta$};
},
\end{center}
and the Frobenius relation $(\id_A \tensor \mu) \circ (\delta \tensor \id_A) = \delta \circ \mu = (\mu \tensor \id_A) \circ (\id_A \tensor \delta)$ is
\begin{center}
    \tikz[vcenter, scale=0.35]{
    \tikzfixsize{(0,0)}{(6,6)}
    %
    \path (6,0) edge[string=black] (6,2);
    \path
    (0,4) edge[string=black] (2,2)
    (4,4) edge[string=black] (2,2)
    (2,2) edge[string=black] (2,0);
    \node[box] at (2,2) {$\delta$};
    %
    \path (0,4) edge[string=black] (0,6);
    \path
    % (2,2) edge[string=black] (4,4)
    (6,2) edge[string=black] (4,4)
    (4,4) edge[string=black] (4,6);
    \node[box] at (4,4) {$\mu$};
}
    =
    \tikz[vcenter, scale=0.35]{
    \tikzfixsize{(0-.5,0)}{(4+.5,7)}
    %
    \path
    (0,0) edge[string=black] (2,2)
    (4,0) edge[string=black] (2,2)
    (2,2) edge[string=black] (2,5)
    (0,7) edge[string=black] (2,5)
    (4,7) edge[string=black] (2,5)
    ;
    \node[box] at (2,2) {$\mu$};
    \node[box] at (2,5) {$\delta$};
}
    =
    \tikz[vcenter, scale=0.35]{
    \tikzfixsize{(0,0)}{(6,6)}
    %
    \path (6,6) edge[string=black] (6,4);
    \path
    (0,2) edge[string=black] (2,4)
    (4,2) edge[string=black] (2,4)
    (2,4) edge[string=black] (2,6);
    \node[box] at (2,4) {$\mu$};
    %
    \path (0,2) edge[string=black] (0,0);
    \path
    % (2,4) edge[string=black] (4,2)
    (6,4) edge[string=black] (4,2)
    (4,2) edge[string=black] (4,0);
    \node[box] at (4,2) {$\delta$};
}.
\end{center}


If we stop labelling the functions and draw the structure maps as
\begin{center}
    \tikz[vcenter, scale=0.4]{
    % \tikzfixsize{(0,0)}{(4,4)}
    % 
    \path
    (0,0) edge[string=black] (2,2)
    (4,0) edge[string=black] (2,2)
    (2,2) edge[string=black] (2,4);
}
    \quad , \quad
    \tikz[vcenter, scale=0.4]{
    \tikzfixsize{(0,0)}{(2,4)}
    %
    \path
    (1,2) edge[string=black] (1,4);
    \node[enddot=black] at (1,2) {};
}
    \quad , \quad
    \tikz[vcenter, scale=0.4]{
    % \tikzfixsize{(0,0)}{(4,4)}
    %
    \path
    (0,4) edge[string=black] (2,2)
    (4,4) edge[string=black] (2,2)
    (2,2) edge[string=black] (2,0);
}
    \quad , \quad
    \tikz[vcenter, scale=0.4]{
    \tikzfixsize{(0,0)}{(2,4)}
    %
    \path
    (1,2) edge[string=black] (1,0);
    \node[enddot=black] at (1,2) {};
},
\end{center}
then the relations become...






\red{Talk about isotopy}


\red{Maybe something about the (diagrammatic?) category Frob, capturing the data of a frobenius object}

