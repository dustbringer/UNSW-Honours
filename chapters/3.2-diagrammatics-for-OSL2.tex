\section{Diagrammatic $\mathcal{O}(\op{SL}(2))$}

With the diagrammatic category $\cH(S_2)$, we can describe diagrammatics for the category $\mathcal{O}(\op{SL}(2))$. In particular, we define a modular category $\red{[what~do~we~call~this~cat?]}$ over $\cH(S_2)$. % TODO: Check the wording and notation here

This module category has elements copied from $\cH(S_2)$ and morphisms are generated by the empty diagram $\varnothing$, with $\cH(S_2)$ acting on the left by left concatenation on objects and morphisms. Additionally, the morphisms have one new relation, where diagrams collapse to $0$ when there are barbells on the right. To depict this we add a wall on the right of the diagram, i.e. embedding the diagrams in the one-sided strip $[0,1] \times \R_{\geq 0}$ instead of in the double-sided strip $[0,1] \times \R$. For example a morphism may be
\begin{equation*}
    \tikz[vcenter, scale=0.4]{
    \tikzfixsize{(0,0)}{(4,4)}
    \squarecoord
    %
    \path
    (midl2) edge[string=red] (topl)
    (midl2) edge[string=red] (mid)
    (mid) edge[string=red] (top)
    (mid) edge[string=red] (midb2)
    (midl2) edge[string=red] (botl);
    %
    \node [enddot=red] at (midb2) {};
    \wallr{4}{4}{1.5}
}
\end{equation*}
Then, diagrams are related to the wall by
\begin{equation}
    \label{rel:right-barbell}
    \tikz[vcenter, scale=0.4]{
    \squarecoord
    %
    \path
    ($(midt) - (0.5,0)$) edge[string=red] ($(midb) - (0.5,0)$);
    \node[enddot=red] at ($(midt) - (0.5,0)$) {};
    \node[enddot=red] at ($(midb) - (0.5,0)$) {};
    \wallr{4}{4}{1.5}
} = 0.
\end{equation}

\red{What happens when i have 0 concatenated to a diagram? Is it also 0 (its a tensor product)?}

Notice that all the morphisms in $\cH(S_2)$ appear in this modular category, although they may have been annihilated by \eqref{rel:right-barbell}.

\begin{example}
    We use the new relation \eqref{rel:right-barbell} to further simplify the morphism in Example \eqref{eg:one-col-relations}.
    \begin{align*}
        \tikz[vcenter, scale=0.4]{
    % \tikzfixsize{(0,0)}{(4,4)}
    \squarecoord
    %
    \path
    % Top half
    (4,6) edge[string=dRed] (4,2)
    (4,4) edge[string=dRed] (1,4)
    (2,4) edge[string=dRed] (2,4.5)
    (2,4.5) edge[string=dRed] (2.5,5)
    (2,4.5) edge[string=dRed] (1.5,5)
    % Barbells
    (3.25,2) edge[string=dRed] (3.25,3)
    (1.75,1) edge[string=dRed] (1.75,2)
    ;
    % Bottom curve
    \draw[string=dRed] (4,2)
    to[out=270,in=0] (3.25,1)
    to[out=180,in=270] (2.5,2)
    to[out=90,in=0] (1.75,3)
    to[out=180,in=90] (1,2);
    \path
    (3.25,1) edge[string=dRed] (3.25,0)
    (1,2) edge[string=dRed] (1,0)
    ;
    %
    \node [enddot=dRed] at (1,4) {};
    \node [enddot=dRed] at (2.5,5) {};
    \node [enddot=dRed] at (1.5,5) {};
    \node [enddot=dRed] at (3.25,2) {};
    \node [enddot=dRed] at (3.25,3) {};
    \node [enddot=dRed] at (1.75,1) {};
    \node [enddot=dRed] at (1.75,2) {};
    %
    \wallr{6}{6}{1.5}
}
         & = 2 \tikz[vcenter, scale=0.4]{
    \tikzfixsize{(0,0)}{(4,4)}
    \squarecoord
    %
    \path
    (top) edge[string=dRed] (bot)
    ($(mid)+(-1,.75)$) edge[string=dRed] ($(mid)+(-1,-.75)$)
    (3,0) edge[string=dRed] ($(mid)+(1,-.75)$)
    ;
    %
    \node [enddot=dRed] at ($(mid)+(-1,.75)$) {};
    \node [enddot=dRed] at ($(mid)+(1,-.75)$) {};
    \node [enddot=dRed] at ($(mid)+(-1,-.75)$) {};
    %
    \wallr{5}{4}{1.25}
} - \tikz[vcenter, scale=0.4]{
    \tikzfixsize{(0,0)}{(4,4)}
    \squarecoord
    %
    \path
    (top) edge[string=red] (mid)
    ($(midt)+(1.25,0)$) edge[string=red] ($(midb)+(1.25,0)$)
    ($(midt)-(1.25,0)$) edge[string=red] ($(midb)-(1.25,0)$)
    ;
    \draw[string=red] ($(bot) - (0.75,0)$)
    to[out=90,in=180+20,looseness=0.8] (mid)
    to[out=-20,in=90,looseness=0.8] ($(bot) + (0.75,0)$)
    ;
    %
    \node [enddot=red] at ($(midt)+(1.25,0)$) {};
    \node [enddot=red] at ($(midb)+(1.25,0)$) {};
    \node [enddot=red] at ($(midt)-(1.25,0)$) {};
    \node [enddot=red] at ($(midb)-(1.25,0)$) {};
    %
    \wallr{5.25}{4}{1.25}
}
         \\ & = 2 \left( 2 \tikz[vcenter, scale=0.4]{
    \tikzfixsize{(0,0)}{(3,4)}
    \squarecoord
    %
    \path
    ($(top)-(1,0)$) edge[string=red] ($(midt)-(1,0)$)
    ($(bot)-(1,0)$) edge[string=red] ($(midb)-(1,0)$)
    (bot) edge[string=red] (midb)
    ;
    %
    \node [enddot=red] at ($(midt)-(1,0)$) {};
    \node [enddot=red] at ($(midb)-(1,0)$) {};
    \node [enddot=red] at ($(midb)$) {};
    %
    \wallr{4}{4}{1.25}
} - \tikz[vcenter, scale=0.4]{
    \tikzfixsize{(0,0)}{(4,4)}
    \squarecoord
    %
    \path
    ($(top)-(1,0)$) edge[string=red] ($(mid)+(-1,.75)$)
    ($(bot)-(1,0)$) edge[string=red] ($(mid)+(-1,-.75)$)
    ($(mid)+(1,.75)$) edge[string=red] ($(mid)+(1,-.75)$)
    (bot) edge[string=red] ($(mid)+(0,-.75)$)
    ;
    %
    \node [enddot=red] at ($(mid)+(-1,.75)$) {};
    \node [enddot=red] at ($(mid)+(-1,-.75)$) {};
    \node [enddot=red] at ($(mid)+(0,-.75)$) {};
    \node [enddot=red] at ($(mid)+(1,.75)$) {};
    \node [enddot=red] at ($(mid)+(1,-.75)$) {};
    %
    \wallr{5}{4}{1.25}
} \right) - 0
         \\ & = 4 \tikz[vcenter, scale=0.4]{
    \tikzfixsize{(0,0)}{(3,4)}
    \squarecoord
    %
    \path
    ($(top)-(1,0)$) edge[string=red] ($(midt)-(1,0)$)
    ($(bot)-(1,0)$) edge[string=red] ($(midb)-(1,0)$)
    (bot) edge[string=red] (midb)
    ;
    %
    \node [enddot=red] at ($(midt)-(1,0)$) {};
    \node [enddot=red] at ($(midb)-(1,0)$) {};
    \node [enddot=red] at ($(midb)$) {};
    %
    \wallr{4}{4}{1.25}
}
    \end{align*}
\end{example}

