\section{Diagrammatic $\mcal{O}(\fraksl_2)$}
\label{sec:3.2}

\red{A little bit about category $\cO$, and our example of $\mfrak{sl}_2$}

For this section, our category of interest is $\cO$ for the semisimple Lie algebra $\fraksl_2(\C)$. A description of the category $\cO$ can be found in general in {\cite[Sections 3.8--3.10]{humphreys-category-O}} or in {\cite[Section 5.2]{mazorchuk-lectures-sl2-modules}} for the case of $\fraksl_2(\C)$, however we will only give a brief overview. The category $\cO$ is a category of certain modules (or representations) over a semisimple Lie algebra. It is a direct sum of subcategories, where, in the case of $\fraksl_2$ over $\C$, the non-trivial summands are equivalent as abelian categories to the subcategory $\cO_0$. Within this, we look to the full subcategory $\proj(\cO_0)$ of projective modules in $\cO_0$, which, in particular, is additive and contains all direct summands.

In \cite[Section 2.4]{soergel-category-O}, Soergel shows that the category $\cO$, and hence the subcategory $\proj(\cO_0)$, is a Soergel module category, i.e. it has an action of the monoidal category $\sbim$. By the equivalence in \autoref{thm:one-col-sbim-equiv} we will view $\proj(\cO)$ as a $\cH_\C(S_2)$-module category, extending via the additive Karoubi envelope. Since $\cH_\C(S_2)$ is diagrammatic, this action allows us to describe $\proj(\cO_0)$ (thus essentially $\cO_0$ and $\cO$) diagrammatically.

\begin{remark}
    \label{rk:projO_0-to-O_0}
    We can pass from $\proj(\cO_0)$ to $\cO_0$ by observing that $K^b(\proj(\cO_0))$ is equivalent to $D^b(\cO_0)$ as graded $\C$-linear monoidal triangulated categories. This is a standard trick in the field, for example see the introduction of \cite{riche-williamson-tilt-modules-p-canon-basis}\footnote{A self-contained summary of how diagrammatic categories can be related to abelian categories.}. However for our purposes it is not important to understand how this works.
\end{remark}

% \begin{remark}
%     For $\fraksl_2(\C)$, the morphism spaces in $\proj(\cO_0)$ are $\C$-modules so the diagrammatic category $\cH(S_2)$ from \autoref{sec:3.1} must be extended from $\Z$ to $\C$. Formally this is just tensoring the morphism spaces on the left by the $\C$-$\Z$-bimodule $\C$, where the right action is induced by the inclusion $\Z \subset \C$. For the remainder of this section, $\cH(S_2)$ will refer to this $\C$-linear extension. Note that the above process does little to the category. In particular, double leaves in $\cH(S_2)$ become $\C[\barbell]$-bases\footnote{It is easy to see that double leaves tensored with $1 \in \C$ on the left form a basis.} for the morphisms and the equivalence in \autoref{thm:one-col-sbim-equiv} still holds.
% \end{remark}

% For basis, maybe see https://math.stackexchange.com/questions/3399387/basis-of-extension-of-scalars

Although we need to work over $\C$ for $\proj(\cO_0(\fraksl_2))$, the diagrammatic category can be defined more simply, that is over $\Z$.

\begin{definition}
    \label{def:DO_0}
    Let $\mcal{DO}_0 \coloneqq \mcal{DO}_0(\fraksl_2)$ be the $\Z$-linear \red{(Define this in background)} left $\cH(S_2)$-module category with elements generated (\red{Define what this means.}) by the monoidal identity $\vn$ of $\cH(S_2)$ and morphisms generated by the empty diagram $\vn$. The action of $\cH(S_2)$ on the left is left concatenation for both objects and morphisms. In addition to the relations from $\cH(S_2)$, the morphisms have one new relation in which diagrams collapse to $0$ when there are barbells on the right. To depict this we add a wall on the right of the diagram, i.e. embedding the diagrams in the one-sided strip $[0,1] \times \R_{<0}$ instead of in the double-sided strip $[0,1] \times \R$. For example a morphism may be
    \begin{equation*}
        \inputtikz{3.2/wall-example}.
    \end{equation*}
    We impose the local relation that diagrams are related to the wall by
    \begin{equation}\tag{W1}
        \label{rel:barbell-wall}
        \inputtikz{3.2/relation-barbell-wall} = 0.
    \end{equation}
    Note that this local relations applies to any subdiagram involving the wall.
\end{definition}

The objects of this category are identical to objects in $\cH(S_2)$ and the morphisms are the same modulo the wall relation \eqref{rel:barbell-wall}. We can further simplify \autoref{eg:one-col-relations} from the previous section.

\begin{example}
    Using the new relation \eqref{rel:barbell-wall}, we can further simplify the morphism in \autoref{eg:one-col-relations} by
    \begin{align*}
        \inputtikz{3.2/relation-example1}
         & = 2 \inputtikz{3.2/relation-example2.1} - \inputtikz{3.2/relation-example2.2}
        \\ & = 2 \left( 2 \inputtikz{3.2/relation-example3.1} - \inputtikz{3.2/relation-example3.2} \right) - 0
        \\ & = 4 \inputtikz{3.2/relation-example3.1}.
    \end{align*}
\end{example}


A natural question to ask is whether double leaves still form bases for the morphism spaces here. Notice that double leaves appear in $\mcal{DO}_0$ by acting on $\vn$ by double leaves in $\cH(S_2)$. All morphisms in $\mcal{DO}_0$ are morphisms in $\cH(S_2)$ so they can be written as $\Z[\barbell]$-linear combinations of double leaves, though some have collapsed to $0$. Thus double leaves span the morphism spaces of $\mcal{DO}_0$ as (left) $\Z[\barbell]$-modules. However they may not be linearly independent as neither left nor right modules. For example, any pair of double leaves that factor through $\vn$ become $0$ when multiplied by $\barbell$ on either side (by translating the barbell to the right). Although double leaves are not always a basis for its respective morphism space as $\Z[\barbell]$-modules, it turns out they are a basis over $\Z$.

\begin{lemma}
    \label{lem:DO_0-double-leaves}
    Let $\pi: \op{mor}(\cH(S_2)) \to \op{mor}(\mcal{DO}_0)$ be the projection map which takes a morphism to the result of its action on the empty diagram $\vn$. Then the image $\pi(\mathbb{LL}(w,y))$ is a basis for $\Hom_{\mcal{DO}_0}(w,y)$ as a $\Z$-module.
\end{lemma}

\begin{proof}
    We consider morphisms $\Hom(w,y)$ in $\mcal{DO}_0$ for fixed objects $w,y$, and write $\mathbb{LL} \coloneqq \pi(\mathbb{LL}(w,y))$ for the set of double leaves in $\mcal{DO}_0$. Any diagram in $\mcal{DO}_0$ can be written as a $\Z$-linear combination of morphisms without floating diagrams, by simplifying them to barbells, pulling them to the right and killing them with \eqref{rel:barbell-wall}. We can write each of these as a $\Z[\barbell]$-linear combination of double leaves by \eqref{thm:one-col-double-leaves-basis} with the right action, and reduce it to a $\Z$-linear combination by \eqref{rel:barbell-wall}. This implies that $\mathbb{LL}$ spans $\Hom(w,y)$ as a $\Z$-module. Since the barbell-wall relation \eqref{rel:barbell-wall} has no effect on $\Z$-linear combinations of $\mathbb{LL}$, it follows from linear independence over $\Z[\barbell]$ that they are linearly independent over $\Z$ in $\mcal{DO}_0$.
    \red{Check the proof.}
\end{proof}

% TODO: Say some leading words here

Our goal is to prove that this diagrammatic category is equivalent to $\proj(\cO_0)$. To that end, we will shift our focus from $\Z$ to $\C$ for the remainder of this section. From now on we write $\mcal{DO}_0$ for the $\C$-linear $\cH_\C(S_2)$-module category obtained by replacing $\Z$ with $\C$ and $\cH(S_2)$ with $\cH_\C(S_2)$ in \autoref{def:DO_0}. The above discussion and \autoref{lem:DO_0-double-leaves} still apply to $\mcal{DO}_0$ over $\C$.

\red{Maybe put this next bit in section 3.1}

\red{Say more about what this is, and why we say it here}

The next result allows us to reduce all morphisms to a matrix of diagrams only involving $\vn$ and $s$.

\begin{lemma}
    \label{lem:ss-equal-2s}
    In the additive closure of $\cH_\C(S_2)$ we have an explicit isomorphisms $ss \cong s \oplus s$, as detailed in the proof. Particularly, these are isomorphisms in the additive closure of $\mcal{DO}_0$.
\end{lemma}
\begin{proof}
    In $\cH_\C(S_2)$ we have the relation
    \begin{align}
        \inputtikz{3.2/ss-id-calculation1}
         & = \inputtikz{3.2/ss-id-calculation2} \nonumber                                                                                         \\
         & = \frac{1}{2} \inputtikz{3.2/ss-id-calculation3.1} + \frac{1}{2} \inputtikz{3.2/ss-id-calculation3.2} \nonumber                        \\
         & = \frac{1}{2} \inputtikz{3.2/ss-id-calculation4.1} + \frac{1}{2} \inputtikz{3.2/ss-id-calculation4.2}. \label{rel:id-id-to-H}
    \end{align}
    This implies that in $\cH_\C(S_2)^\oplus$, we have maps
    \[
        \pvec{
            \displaystyle \frac{1}{2} \inputtikz{3.2/ss-to-s+s-morphism1} \\
            \displaystyle \frac{1}{2} \inputtikz{3.2/ss-to-s+s-morphism2}
        } : ss \to s \oplus s
        \text{ and }
        \pvec{
            \displaystyle \inputtikz{3.2/s+s-to-ss-morphism1} &
            \displaystyle \inputtikz{3.2/s+s-to-ss-morphism2}
        } : s \oplus s \to ss.
    \]
    It follows from \eqref{rel:one-col-barbell-forcing}, \eqref{rel:one-col-needle} and the calculation \eqref{rel:id-id-to-H}, that these maps are inverses.
    \red{Maybe put the inverse calculation here.}
\end{proof}

\red{Be clear that I don't understand category $\cO$ very well.}

As a shorthand, we write $\proj(\cO_0)$ for $\proj(\cO_0(\fraksl_2))$. The work of Soergel in \cite[Section 2.4]{soergel-category-O} shows that $\proj(\cO_0)$ is a Soergel module, i.e. it has a left action of the category of Soergel bimodules defined by applications of the translation functors $\Theta_\vn, \Theta_s \in \End(\cO)$ (corresponding to elements in $S_2$). \red{Explains what this means, how its related to the $\cH(S_2)$ module category} We construct a functor that maps faithfully into a full subcategory of $\proj(\cO_0)$, which will become the whole of $\proj(\cO_0)$ under the additive Karoubi envelope. This is the same strategy as in the proof for \autoref{thm:one-col-sbim-equiv}.


\begin{definition}
    Let $F: \mcal{DO}_0^\oplus \to \proj(\cO_0)$ be the $\C$-linear $\cH_\C(S_2)$-module functor that sends the empty object $\vn$ to the trivial module $P(\vn)$, and the Soergel module action corresponding to $s$ to the translation functor $\Theta_s$. Then the object $s$ maps to $\Theta_s(P(\vn)) \eqqcolon P(s)$, and for example $s^3$ maps to the composition of three $\Theta_s$ on $P(\vn)$, $\Theta_s^3(P(\vn)) = \Theta_s\Theta_s\Theta_s(P(\vn))$. In order for $F$ to be functorial, it must map identity diagrams $s^n \to s^n$ to $\id_{\Theta_s^n(P(\vn))}$. On non-identity maps, we let $F(\counit)$ be the inclusion $i: P(\vn) \to P(s)$ and $F(\unit)$ be the projection $p: P(s) \to P(\vn)$. We then extend $F$ by composition, additivity and linearity. The mapping of $F$ is depicted by the following picture. \red{Need to talk about compositions, why is F well defined if its generated by compositions of these? Are there any clashes? -- Actually preserves compositions by construction}
    \begin{equation}
        \label{img:osl2-functor}
        \inputtikz{3.2/osl2-proof-functor}
    \end{equation}
    \red{Actually refer to the picture}
\end{definition}

\begin{lemma}
    The functor $F$ is well defined.
\end{lemma}
\begin{proof}
    From \cite[Proposition 5.90]{mazorchuk-lectures-sl2-modules}, there is a natural isomorphism $\Theta_s \Theta_s \cong \Theta_s \oplus \Theta_s$ analogous to the isomorphism $ss \cong s \oplus s$ given in the proof of \autoref{lem:ss-equal-2s}. We consider the additive closure $\mcal{DO}_0^\oplus$, which does not cause problems since we will eventually take this anyway \red{Maybe just do additive karoubi closure in the first place}. Given a morphism in $\mcal{DO}_0$ from $s^n$ to $s^m$, repeated precomposition and postcomposition with $ss \to s \oplus s$ and $ss \oplus s \to s$ from \autoref{lem:ss-equal-2s} results in an isomorphic matrix of diagrams with domain and codomain in $\{\vn, s\}$. By \autoref{lem:DO_0-double-leaves} over $\C$, $\Hom(\vn, \vn)$ has a basis $\{\vn = \id_\vn\}$, $\Hom(s, \vn)$ has a basis $\{\unit\}$, $\Hom(\vn, s)$ has a basis $\{\counit\}$, and $\Hom(s, s)$ has a basis $\{\id_s, \counit \circ \unit\}$. Therefore, extending by linearity, the picture above completely describes the image of $F$.
    
    % We can similarly pull back the image, i.e. matrices of morphisms in $\proj(\cO_0)$, to a morphism between $\Theta_s^n(P(\vn))$ and $\Theta_s^m(P(\vn))$ via the analogous maps defining $\Theta_s \Theta_s \cong \Theta_s \oplus \Theta_s$. Therefore \eqref{img:osl2-functor} is enough to define $F$ for $\mcal{DO}_0$.

    Next we check that all the relations are preserved. From classical results e.g. \cite[Proposition 5.84 and Lemma 5.87]{mazorchuk-lectures-sl2-modules}, it follows that $\Theta_s$ is a Frobenius object in the category of endofunctors of $\cO$. Then there are unit, counit, multiplication and comultiplication natural transformations satisfying coherence relations in the Frobenius object structure. Applying these to $P(\vn)$ result in the same relations in $\proj(\cO_0)$ for $P(\vn), P(s)$ and $\Theta_s^2(P(\vn))$. Note that the projection and inclusion maps above are exactly the unit and counit of $\Theta_s$ evaluated at $P(\vn)$, and the trivalent vertices provided by projecting the isomorphisms in \autoref{lem:ss-equal-2s} map exactly to the multiplication and comultiplication maps. Hence the Frobenius relations \eqref{rel:one-col-frob-unit} and \eqref{rel:one-col-frob-ass} are satisfied. \red{Ref for needle and barbell?} Furthermore, in \cite[Section 2.4]{soergel-category-O} we see that $p \circ i = 0$ in $\proj(\cO_0)$ which is analogous\footnote{This relation extends to the analogue of the local barbell-wall relation, as all `barbell on the right' morphisms in $\proj(\cO_0)$ are linear combinations of applications of $\Theta_s$ to $p \circ i$, which is $0$.} to the barbell-wall relation \eqref{rel:barbell-wall}. Hence all the relations in $\mcal{DO}_0$ are preserved by $F$. By construction, $F$ preserves $\C$-linear combinations and the Soregel module structure in \cite{soergel-category-O}, so $F$ is well defined as a functor between $\C$-linear $\cH(S_2)$-module categories.
\end{proof}

\begin{theorem}[Soergel, {\cite[Endomorhihsmensatz 7, Struktursatz 9 and Section 2.4]{soergel-category-O}}]
    The diagrammatic category $\Kar(\mcal{DO}_0(\fraksl_2))$ and $\proj(\cO_0(\fraksl_2))$ are equivalent as $\C$-linear $\cH(S_2)$-module categories.
\end{theorem}

\red{Check all of this \& Put precise references}

\red{Maybe write description as a soergel module outside the proof}

\begin{proof}
    First we show that $F$ is full and faithful. It follows from \autoref{lem:ss-equal-2s} and the description of $P(\vn)$ and $P(s)$ in \cite[Section 5.2]{mazorchuk-lectures-sl2-modules} that the image of $\unit$ and $\counit$ generate all morphisms of the form $\Theta_s^n(P(\vn)) \to \Theta_s^m(P(\vn))$. Hence $F$ is full. Now the mapping of $F$ on all morphism spaces are determined by those depicted in the above picture. So, for faithfulness, it suffices to compare the $\C$-dimensions of morphism spaces between objects shown in the picture. As mentioned above, the double leaves basis are precisely the diagrams depicted in the image. The bases for the corresponding morphism spaces in $\proj(\cO_0)$ are also those in the image \red{Ref? - that these are actually the bases of the hom spaces}, so the dimensions of $\Hom$ spaces coincide. Therefore $F$ is fully faithful.

    All objects in $\proj(\cO_0)$ appear as direct sums and direct summands of the elements $\Theta_s^n(P(\vn))$ for non-negative integers $n$. Therefore the additive Karoubi envelope induces an equivalence $\Kar(\mcal{DO}_0) \cong \proj(\cO_0)$ as $\C$-linear left $\cH(S_2)$-module categories.
\end{proof}

This result is essentially due to Soergel \cite[Endomorhihsmensatz 7, Struktursatz 9 and Section 2.4]{soergel-category-O} (see also \cite{soergel-combinatorics-of-hcbim}) but this was not its original formulation. Nevertheless we attribute this theorem to Soergel.

\red{Maybe talk about Soergel modules and $\cH(S_2)$-modules vs $\Kar(\cH(S_2))$-modules}

\begin{remark}
    The morphisms spaces in $\mcal{DO}_0$ are graded by the same grading as $\cH(S_2)$ in \autoref{sec:3.1}. The equivalence $\Kar(\mcal{DO}_0) \cong \proj(\cO_0)$ includes a grading of morphisms in $\proj(\cO_0)$ \red{Check!} and hence a grading morphisms of $\cO$, which is otherwise ungraded.
\end{remark}

\red{Some more consequences}

