\section{Diagrammatic $\mcal{O}(\op{SL}(2))$}

With the diagrammatic category $\cH(S_2)$, we can describe diagrammatics for the category $\mcal{O}(\op{SL}(2))$. In particular, we define a modular category $\mcal{DO}(\op{SL}(2))$ with a left-action of $\cH(S_2)$. % TODO: Check the wording and notation here

The category $\mcal{DO}(\op{SL}(2))$ has elements that are generated (\red{Define what this means.}) by the identity element $\varnothing$ of $\cH(S_2)$ and morphisms are generated by the empty diagram $\varnothing$, where $\cH(S_2)$ acts on the left by left concatenation for objects and morphisms. In addition to the relations from $\cH(S_2)$, the morphisms have one new relation in which diagrams collapse to $0$ when there are barbells on the right. To depict this we add a wall on the right of the diagram, i.e. embedding the diagrams in the one-sided strip $[0,1] \times \R_{\geq 0}$ instead of in the double-sided strip $[0,1] \times \R$. For example a morphism may be
\begin{equation*}
    \inputtikz{3.2/wall-example}.
\end{equation*}
We impose the relation that diagrams are related to the wall by
\begin{equation}
    \label{rel:barbell-wall}
    \inputtikz{3.2/relation-barbell-wall} = 0.
\end{equation}

Notice that all the morphisms in $\cH(S_2)$ appear in this modular category, although they may have been annihilated by \eqref{rel:barbell-wall}.

\begin{example}
    We use the new relation \eqref{rel:barbell-wall} to further simplify the morphism in Example \eqref{eg:one-col-relations}.
    \begin{align*}
        \inputtikz{3.2/relation-example1}
         & = 2 \inputtikz{3.2/relation-example2.1} - \inputtikz{3.2/relation-example2.2}
        \\ & = 2 \left( 2 \inputtikz{3.2/relation-example3.1} - \inputtikz{3.2/relation-example3.2} \right) - 0
        \\ & = 4 \inputtikz{3.2/relation-example3.1}
    \end{align*}
\end{example}


Since the objects and morphisms in this category are that of $\cH(S_2)$ modulo the wall relation \eqref{rel:barbell-wall}, double leaves still span the morphism spaces of $\mcal{DO}(\op{SL}(2))$. We see these as left $\Z[\barbell]$-module bases since right multiplication by $\barbell$ annihilates the morphism and adds no new diagrams. \red{Are these basis? They might not be linearly independent e.g. double leaves with a gap means that multiplying with $\barbell$ kills it.}

\red{Only allowing $\Z$-module should make them linearly independent, and they should still span.}

% TODO: Say something here

In the diagrammatic category $\cH(S_2)$ from Section 3.1, we have the relation
\begin{align}
    \inputtikz{3.2/ss-id-calculation1}
     & = \inputtikz{3.2/ss-id-calculation2} \nonumber                                                                  \\
     & = \frac{1}{2} \inputtikz{3.2/ss-id-calculation3.1} + \frac{1}{2} \inputtikz{3.2/ss-id-calculation3.2} \nonumber \\
     & = \frac{1}{2} \inputtikz{3.2/ss-id-calculation4.1} + \frac{1}{2} \inputtikz{3.2/ss-id-calculation4.2}. \label{ss-to-2s-ismorphism}
\end{align}
In the additive closure of this category, this shows there is an isomorphism $s \tensor s \cong s \oplus s$ by
\[
    \pvec{
        \displaystyle \frac{1}{2} \inputtikz{3.2/ss-to-s+s-morphism1} \\
        \displaystyle \frac{1}{2} \inputtikz{3.2/ss-to-s+s-morphism2}
    } : ss \to s \oplus s
    \text{ and }
    \pvec{
        \displaystyle \inputtikz{3.2/s+s-to-ss-morphism1} &
        \displaystyle \inputtikz{3.2/s+s-to-ss-morphism2}
    } : s \oplus s \to ss.
\]
Its is not hard to check that these are inverses.


\begin{theorem}[\red{???}]
    The diagrammatic category $\Kar(\mcal{DO}(\op{SL}(2)))$ and $\mcal{O}(\op{SL}(2))$ are equivalent as categories.
\end{theorem}

\red{Check all of this \& Put precise references}

\begin{proof}
    As a shorthand, we write $\mcal{DO}$ for $\mcal{DO}(\op{SL}(2))$ and $\mcal{O}$ for $\mcal{O}(\op{SL}(2))$. The work of Soergel in \cite{soergel-category-O} shows that $\cO$ is a Soergel module, i.e. it has a left action of the category of Soergel bimodules defined by applications of the translation functors $\Theta_\varnothing, \Theta_s \in \End(\cO)$, corresponding to elements in $S_2$ \red{Check this}. Classical results, e.g. \cite{humphreys-category-O}, show that $\Theta_s$ is a Frobenius object in the category of endofunctors of $\cO$. Then there are unit, counit, multiplication and comultiplication natural transformations satisfying coherence relations in the Frobenius object structure. Additionally, Soergel's work \red{(References?)} shows that there is a relation in $\mcal{O}$ analogous to the barbell-wall relation \eqref{rel:barbell-wall}, and that there is an isomorphism $\Theta_s \Theta_s \cong \Theta_s \oplus \Theta_s$ \red{(Is the direct sum here correct?)} which is analogous to the isomorphism given by \eqref{ss-to-2s-ismorphism}.

    % We consider the \red{full} subcategory \blue{$\cO'$} $\subset \cO$ whose Karoubi envelope produces $\cO$, similar to $\bsbim \subset \sbim$. The objects of $\cO'$ are generated by applying $\Theta_s$ to the trivial module $P_\varnothing$, and the non-identity morphisms are generated by\footnote{The underlying mechanism is dependent on the functor $\Theta_s$.} tensoring $\id_\varnothing$ on the left with unit and counit of $\Theta_s$, evaluated at $P_\varnothing$.

    % In $\cO$, the Soergel module action of $\Theta_s$ corresponds to left tensor of the identity morphism.

    Define the functor $F: \mcal{DO} \to \cO$ that sends the empty object $\varnothing$ to the trivial module $P_\varnothing$, and the Soergel module action corresponding to $s$ to the translation functor $\Theta_s$. Then the object $s$ maps to $\Theta_s(P_\varnothing) \eqqcolon P_s$, and $s^3$ maps to $\Theta_s^3(P_\varnothing) = \Theta_s\Theta_s\Theta_s(P_\varnothing)$.
    Functoriality forces $F$ to map identity diagrams $s^n \to s^n$ to $\id_{\Theta_s^n(P_\varnothing)}$. For non-identity maps, we let $F(\unit) = p$ be the projection map $P_\varnothing \to P_s$, and $F(\counit) = i$ be the inclusion map $P_s \to P_\varnothing$. Note that the projection and inclusion maps are exactly the unit and counit of $\Theta_s$ evaluated at $P_\varnothing$. This is enough to completely determine the image of $F$, since $\Theta_s \Theta_s \cong \Theta_s \oplus \Theta_s$. Since $\Theta_s$ is a Frobenius object and the barbell-wall relation is satisfied in $\cO$, the functor $F$ is well defined.

    Now we show that $F$ is fully faithful. We know \red{(From Soergel, EW, Libedinsky? Explain this more)} that the image of $\unit$ and $\counit$ generate all morphisms of the form $\Theta_s^n(P_\varnothing) \to \Theta_s^m(P_\varnothing)$, so $F$ is full. For the faithfulness of $F$, it suffices to match the dimensions of $\Z$-bases for hom-spaces involving $P_\varnothing$ and $P_s$. By double leaves in $\mcal{DO}$, as $\Z$-modules, $\Hom(\varnothing, \varnothing)$ has a basis $\{\varnothing = \id_\varnothing\}$, $\Hom(s, \varnothing)$ has a basis $\{\unit\}$, $\Hom(\varnothing, s)$ has a basis $\{\counit\}$, and $\Hom(s, s)$ has a basis $\{\id_s, \counit \circ \unit\}$. The dimensions match exactly with the corresponding images of $F$. Therefore $F$ is fully faithful.

    Since objects in $\cO$ are direct sums and direct summands of the elements $\Theta_s^n(P_\varnothing)$ for non-negative integer $n$, taking the Karoubi envelope $\Kar(\mcal{DO})$ induces an equivalence of categories $\Kar(\mcal{DO}) \cong \cO$.
\end{proof}



\begin{proof}[Old Proof]
    As a shorthand, we write $\mcal{DO}$ for $\mcal{DO}(\op{SL}(2))$ and $\mcal{O}$ for $\mcal{O}(\op{SL}(2))$. Let $F: \mcal{DO} \to \mcal{O}$ be a functor that sends the empty object $\varnothing$ to the trivial module $P_\varnothing$ and $s \mapsto P_s$, the indecomposable objects in $\mcal{O}$ corresponding to elements in $S_2$. On morphisms, $F$ sends the identity morphism on $s$ (the red strand) to the translation functor $\Theta_s$ in $\mcal{O}$ corresponding to $s \in S_2$. This completely determines the action of $F$ (\red{Why?}). Due to classical results in \cite{humphreys-category-O}, the translation functors are Frobenius objects, so there have unit, counit, multiplication and comultiplication maps with appropriate relations in $\mcal{O}$. These the image of which are the image of the generators \eqref{eq:one-col-gen} under $F$, that satisfy the analogous relations \eqref{eq:one-col-sbim-rel}. Furthermore, the work of Soergel in \cite{soergel-category-O} shows that there is a relation in $\mcal{O}$ analogous to the barbell-wall relation \eqref{rel:barbell-wall}. This $F$ is well defined as all the generators and relations in $\mcal{DO}$ are accounted for (\red{Word this better}).

    Next we show that $F$ is a fully faithful functor. By results from \cite{elias-williamson-soergel-calculus} and \cite{libedinsky-lightleavesbasis}, the inclusion $\cH(S_2) \to \sbim$ is fully faithful, so we have a copy of double leaves bases in $\sbim$. By the work of Soergel in \cite{soergel-category-O}, the category $\mcal{O}$ is a Soergel module \red{(Explain what this is)} with certain bases for the morphism. Thus (\red{Why?}) it suffices to  compare the dimension of morphism spaces between $\mcal{DO}$ and $\mcal{O}$, as Soergel modules. [\red{Comparison?}]

    The functor $F$ mapped objects of $\mcal{DO}$ to objects \red{???} in $\mcal{O}$, which generate all other objects by direct sums and direct summands \red{Is this right?}. Now $F$ is fully faithful, $\Kar$ preserves equivalences of categories and taking the Karoubi envelope of the image of $\mcal{DO}$ gives exactly $\mcal{O}$ (\red{Is this right?}), we obtain an equivalence of categories between $\Kar(\mcal{DO})$ and $\mcal{O}$.
\end{proof}
