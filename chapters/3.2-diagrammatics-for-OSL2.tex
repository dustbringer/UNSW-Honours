\section{Diagrammatic $\mathcal{O}(\op{SL}(2))$}

With the diagrammatic category $\cH(S_2)$, we can describe diagrammatics for the category $\mathcal{O}(\op{SL}(2))$. In particular, we define a modular category $\mathcal{DO}(\op{SL}(2))$ with a left-action of $\cH(S_2)$. % TODO: Check the wording and notation here

The category $\mathcal{DO}(\op{SL}(2))$ has elements that are generated (\red{Define what this means.}) by the identity element $\varnothing$ of $\cH(S_2)$ and morphisms are generated by the empty diagram $\varnothing$, where $\cH(S_2)$ acts on the left by left concatenation for objects and morphisms. Additionally, the morphisms have one new relation, where diagrams collapse to $0$ when there are barbells on the right. To depict this we add a wall on the right of the diagram, i.e. embedding the diagrams in the one-sided strip $[0,1] \times \R_{\geq 0}$ instead of in the double-sided strip $[0,1] \times \R$. For example a morphism may be
\begin{equation*}
    \inputtikz{3.2/wall-example}.
\end{equation*}
We impose the relation that diagrams are related to the wall by
\begin{equation}
    \label{rel:right-barbell}
    \inputtikz{3.2/relation-barbell-wall} = 0.
\end{equation}

Notice that all the morphisms in $\cH(S_2)$ appear in this modular category, although they may have been annihilated by \eqref{rel:right-barbell}.

\begin{example}
    We use the new relation \eqref{rel:right-barbell} to further simplify the morphism in Example \eqref{eg:one-col-relations}.
    \begin{align*}
        \inputtikz{3.2/relation-example1}
         & = 2 \inputtikz{3.2/relation-example2.1} - \inputtikz{3.2/relation-example2.2}
         \\ & = 2 \left( 2 \inputtikz{3.2/relation-example3.1} - \inputtikz{3.2/relation-example3.2} \right) - 0
         \\ & = 4 \inputtikz{3.2/relation-example3.1}
    \end{align*}
\end{example}

Light leaves for this category is exactly the same as 

