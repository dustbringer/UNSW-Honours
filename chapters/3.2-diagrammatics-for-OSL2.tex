\section{Diagrammatic $\mcal{O}_0(\fraksl_2)$}

\red{A little bit about category $\cO$, and our example of $\mfrak{sl}_2$}

With the diagrammatic category $\cH(S_2)$, we can describe diagrammatics for the category $\mcal{O}_0$ for the Lie algebra $\fraksl_2$. In particular, we define a modular category $\mcal{DO}_0(\fraksl_2)$ with a left-action of $\cH(S_2)$. At the end, we give a useful description of $\mcal{O}_0(\fraksl_2)$ and a proof for an equivalence of these categories (up to idempotent completion) \red{write what type of equivalence}.
% TODO: Check the wording and notation here

\begin{definition}
    Let $\mcal{DO}(\fraksl_2)$ be the left $\cH(S_2)$-module category with elements generated (\red{Define what this means.}) by the monoidal identity $\vn$ of $\cH(S_2)$ and morphisms generated by the empty diagram $\vn$, where $\cH(S_2)$ acts on the left by left concatenation for objects and morphisms. In addition to the relations from $\cH(S_2)$, the morphisms have one new relation in which diagrams collapse to $0$ when there are barbells on the right. To depict this we add a wall on the right of the diagram, i.e. embedding the diagrams in the one-sided strip $[0,1] \times \R_{\geq 0}$ instead of in the double-sided strip $[0,1] \times \R$. For example a morphism may be
    \begin{equation*}
        \inputtikz{3.2/wall-example}.
    \end{equation*}
    We impose the relation that diagrams are related to the wall by
    \begin{equation}
        \label{rel:barbell-wall}
        \inputtikz{3.2/relation-barbell-wall} = 0.
    \end{equation}
\end{definition}

Notice that all the morphisms in $\cH(S_2)$ appear in this modular category, although some may have been annihilated by \eqref{rel:barbell-wall}.

\begin{example}
    Using the new relation \eqref{rel:barbell-wall}, we can further simplify the morphism in Example \eqref{eg:one-col-relations} by
    \begin{align*}
        \inputtikz{3.2/relation-example1}
        & = 2 \inputtikz{3.2/relation-example2.1} - \inputtikz{3.2/relation-example2.2}
        \\ & = 2 \left( 2 \inputtikz{3.2/relation-example3.1} - \inputtikz{3.2/relation-example3.2} \right) - 0
        \\ & = 4 \inputtikz{3.2/relation-example3.1}.
    \end{align*}
\end{example}

% TODO: Say some leading words here

\red{Maybe put this next bit in section 3.1}

\red{Maybe have this as a lemma}

\red{Say more about what this is, and why we say it here}

In the diagrammatic category $\cH(S_2)$ from Section 3.1, we have the relation
\begin{align}
    \inputtikz{3.2/ss-id-calculation1}
     & = \inputtikz{3.2/ss-id-calculation2} \nonumber                                                                                         \\
     & = \frac{1}{2} \inputtikz{3.2/ss-id-calculation3.1} + \frac{1}{2} \inputtikz{3.2/ss-id-calculation3.2} \nonumber                        \\
     & = \frac{1}{2} \inputtikz{3.2/ss-id-calculation4.1} + \frac{1}{2} \inputtikz{3.2/ss-id-calculation4.2}. \label{rel:ss-to-2s-ismorphism}
\end{align}
In the additive closure of this category, this shows there is an isomorphism $s \tensor s \cong s \oplus s$ by
\[
    \pvec{
        \displaystyle \frac{1}{2} \inputtikz{3.2/ss-to-s+s-morphism1} \\
        \displaystyle \frac{1}{2} \inputtikz{3.2/ss-to-s+s-morphism2}
    } : ss \to s \oplus s
    \text{ and }
    \pvec{
        \displaystyle \inputtikz{3.2/s+s-to-ss-morphism1} &
        \displaystyle \inputtikz{3.2/s+s-to-ss-morphism2}
    } : s \oplus s \to ss.
\]
It follows from \eqref{rel:one-col-barbell-forcing}, \eqref{rel:one-col-needle} and the above calculation \eqref{rel:ss-to-2s-ismorphism}, that these maps are inverses.

\grey{
    The objects of this category are identical to objects in $\cH(S_2)$ and the morphisms are the same modulo the wall relation \eqref{rel:barbell-wall}. A natural question to ask is whether double leaves still form a basis for the morphism spaces here. \red{Need to mention that the morphism spaces have $\C$-linear structure again.} \red{How does the left action work if the monoidal category is also a $\C$-linear category?}

    \begin{lemma}
        Something about light leaves being spanning over something and linear over something else
    \end{lemma}

    The objects and morphisms in this category are essentially that of $\cH(S_2)$ modulo the wall relation \eqref{rel:barbell-wall}, so the double leaves bases of $\cH(S_2)$ remain a spanning set of the morphism spaces in $\mcal{DO}(\fraksl_2)$ over $\Z[\barbell]$. Note that is not a $\Z[\barbell]$-basis in $\mcal{DO}(\fraksl_2)$ as linear independence is not preserved by the new relation. For example, any double leaf factoring through $\varnothing$ is $0$ when multiplied by $\barbell$. It turns out that double leaves are a $\Z$-module basis for $\mcal{DO}(\fraksl_2)$. \red{Is this right?}

    Before moving on, we give a useful description of $\cO_0(\fraksl_2)$. This can be found in general in {\cite[Sections 3.8-3.10]{humphreys-category-O}}, or in {\cite[Section 5.2]{mazorchuk-lectures-sl2-modules}} for the specific case of $\fraksl_2$. The category $\cO_0$ for $\fraksl_2$ is an abelian category with exactly two simple modules $L(\vn)$ (the trivial module) and $L(s)$.

    ...that can be decomposed into a direct sum\footnote{\red{Put reference; put quick explanation}} of subcategories called \textit{blocks}. There is a particularly important block called the \textit{principle block}, which we write as $\cO_0$. Over $\fraksl_2$, the principle block of $\cO$ has exactly two simple modules $L(\vn)$ and $L(s)$, corresponding to the elements of $S_2$, and there are projective covers\footnote{A projective cover is a projective module and a surjection onto our module, which is the ``smallest''.} $P(\vn) \twoheadrightarrow L(\vn)$ and $P(s) \twoheadrightarrow L(s)$. Here, all elements of $\cO_0$ are generated from direct sums of $P(s)$ and their filtrations. \green{Note that that simple modules $L(\vn)$ and $L(s)$ appear as factors in the filtration of their respective projective modules, and that $P(\vn) = L(\vn)$ and $P(s) \supset L(s)$}. In this case, all blocks of $\cO$ are isomorphic to either the principle block $\cO_0$ or the block generated by the trivial module $L(\vn)$ (\red{isomorphic to finite dimensional vector spaces over(?) This contains no information about $\fraksl_2$}). This means that a description of the modules $P(\vn)$ and $P(s)$ in $\cO_0$ induces a description of the entire category $\cO(\fraksl_2)$.
}


The following result is essentially due to Soergel \cite[Endomorhihsmensatz 7, Struktursatz 9 and Section 2.4]{soergel-category-O} (see also \cite{soergel-combinatorics-of-hcbim}) but was not originally formulated as such. As the key arguments are in \cite{soergel-category-O}, we contribute this theorem to Soergel.

\begin{theorem}[Soergel, {\cite[Endomorhihsmensatz 7, Struktursatz 9 and Section 2.4]{soergel-category-O}}]
    The diagrammatic category $\Kar(\mcal{DO}_0(\fraksl_2))$ and $\mcal{O}_0(\fraksl_2)$ are equivalent as categories \red{What equivalence of what type of categories?}.
\end{theorem}

\red{Check all of this \& Put precise references}

\begin{proof}
    As a shorthand, we write $\mcal{DO}$ for $\mcal{DO}(\fraksl_2)$ and $\mcal{O}$ for $\mcal{O}(\fraksl_2)$. The work of Soergel in \cite[Section 2.4]{soergel-category-O} shows that $\cO$ is a Soergel module, i.e. it has a left action of the category of Soergel bimodules defined by applications of the translation functors $\Theta_\vn, \Theta_s \in \End(\cO)$, corresponding to elements in $S_2$ \red{Check this}. From classical results e.g. \cite[Proposition 5.8.1 and Lemma 5.8.4]{mazorchuk-lectures-sl2-modules} \red{Whats the numbering? The book numbers it differently}, it follows that $\Theta_s$ is a Frobenius object in the category of endofunctors of $\cO$. Then there are unit, counit, multiplication and comultiplication natural transformations satisfying coherence relations in the Frobenius object structure. Additionally, \cite[Section 2.4]{soergel-category-O} shows that there is a relation in $\mcal{O}$ analogous to the barbell-wall relation \eqref{rel:barbell-wall}, and \cite[Proposition 5.8.4]{mazorchuk-lectures-sl2-modules} shows that there is an isomorphism $\Theta_s \Theta_s \cong \Theta_s \oplus \Theta_s$ \red{(Is the direct sum here correct?)} which is analogous to the isomorphism given by \eqref{rel:ss-to-2s-ismorphism}.

    % We consider the \red{full} subcategory \green{$\cO'$} $\subset \cO$ whose Karoubi envelope produces $\cO$, similar to $\bsbim \subset \sbim$. The objects of $\cO'$ are generated by applying $\Theta_s$ to the trivial module $P(\vn)$, and the non-identity morphisms are generated by\footnote{The underlying mechanism is dependent on the functor $\Theta_s$.} tensoring $\id_\vn$ on the left with unit and counit of $\Theta_s$, evaluated at $P(\vn)$.

    % In $\cO$, the Soergel module action of $\Theta_s$ corresponds to left tensor of the identity morphism.

    Define the functor $F: \mcal{DO} \to \cO$ that sends the empty object $\vn$ to the trivial module $P(\vn)$, and the Soergel module action corresponding to $s$ to the translation functor $\Theta_s$. Then the object $s$ maps to $\Theta_s(P(\vn)) \eqqcolon P(s)$, and $s^3$ maps to $\Theta_s^3(P(\vn)) = \Theta_s\Theta_s\Theta_s(P(\vn))$. In order for $F$ to be functorial, it must map identity diagrams $s^n \to s^n$ to $\id_{\Theta_s^n(P(\vn))}$. For non-identity maps, we let $F(\counit) = i$ be the inclusion $P(\vn) \to P(s)$ and $F(\unit) = p$ be the projection $P(s) \to P(\vn)$. The mapping of $F$ is depicted by the following diagram.
    \begin{center}
        \inputtikz{3.2/osl2-proof-functor}
    \end{center}
    Note that the projection and inclusion maps are exactly the unit and counit of $\Theta_s$ evaluated at $P(\vn)$. This is enough to completely determine the image of $F$, since $\Theta_s \Theta_s \cong \Theta_s \oplus \Theta_s$. Now $\Theta_s$ is a Frobenius object and the barbell-wall relation is satisfied in $\cO$, so the functor $F$ is well defined.

    Now we show that $F$ is fully faithful. It follows from $\Theta_s \Theta_s \cong \Theta_s \oplus \Theta_s$ and the description of $P(\vn)$ and $P(s)$ in \cite[Section 5.2]{mazorchuk-lectures-sl2-modules} that the image of $\unit$ and $\counit$ generate all morphisms of the form $\Theta_s^n(P(\vn)) \to \Theta_s^m(P(\vn))$. Hence $F$ is full. For the faithfulness of $F$, it suffices to match the dimensions of $\Z$-bases for hom-spaces involving $P(\vn)$ and $P(s)$. By double leaves in $\mcal{DO}$, as $\Z$-modules, $\Hom(\vn, \vn)$ has a basis $\{\vn = \id_\vn\}$, $\Hom(s, \vn)$ has a basis $\{\unit\}$, $\Hom(\vn, s)$ has a basis $\{\counit\}$, and $\Hom(s, s)$ has a basis $\{\id_s, \counit \circ \unit\}$. The dimensions coincide with the corresponding images of $F$. Therefore $F$ is fully faithful.

    \grey{
        Since objects in $\cO$ are direct sums and direct summands \red{(How does this fit into the description of $\cO$ we mentioned before the proof?)} of the elements $\Theta_s^n(P(\vn))$ for non-negative integer $n$, taking the Karoubi envelope $\Kar(\mcal{DO})$ induces $\Kar(\mcal{DO}) \cong \cO$ as $\C$-linear left $\cH(S_2)$-module categories.
    }
\end{proof}


% \grey{
%     \begin{proof}[Old Proof]
%         As a shorthand, we write $\mcal{DO}$ for $\mcal{DO}(\fraksl_2)$ and $\mcal{O}$ for $\mcal{O}(\fraksl_2)$. Let $F: \mcal{DO} \to \mcal{O}$ be a functor that sends the empty object $\vn$ to the trivial module $P(\vn)$ and $s \mapsto P(s)$, the indecomposable objects in $\mcal{O}$ corresponding to elements in $S_2$. On morphisms, $F$ sends the identity morphism on $s$ (the red strand) to the translation functor $\Theta_s$ in $\mcal{O}$ corresponding to $s \in S_2$. This completely determines the action of $F$ (\red{Why?}). Due to classical results in \cite{humphreys-category-O}, the translation functors are Frobenius objects, so there have unit, counit, multiplication and comultiplication maps with appropriate relations in $\mcal{O}$. These the image of which are the image of the generators \eqref{eq:one-col-gen} under $F$, that satisfy the analogous relations \eqref{eq:one-col-hecke-rel}. Furthermore, the work of Soergel in \cite{soergel-category-O} shows that there is a relation in $\mcal{O}$ analogous to the barbell-wall relation \eqref{rel:barbell-wall}. This $F$ is well defined as all the generators and relations in $\mcal{DO}$ are accounted for (\red{Word this better}).

%         Next we show that $F$ is a fully faithful functor. By results from \cite{elias-williamson-soergel-calculus} and \cite{libedinsky-lightleavesbasis}, the inclusion $\cH(S_2) \to \sbim$ is fully faithful, so we have a copy of double leaves bases in $\sbim$. By the work of Soergel in \cite{soergel-category-O}, the category $\mcal{O}$ is a Soergel module \red{(Explain what this is)} with certain bases for the morphism. Thus (\red{Why?}) it suffices to  compare the dimension of morphism spaces between $\mcal{DO}$ and $\mcal{O}$, as Soergel modules. [\red{Comparison?}]

%         The functor $F$ mapped objects of $\mcal{DO}$ to objects \red{???} in $\mcal{O}$, which generate all other objects by direct sums and direct summands \red{Is this right?}. Now $F$ is fully faithful, $\Kar$ preserves equivalences of categories and taking the Karoubi envelope of the image of $\mcal{DO}$ gives exactly $\mcal{O}$ (\red{Is this right?}), we obtain an equivalence of categories between $\Kar(\mcal{DO})$ and $\mcal{O}$.
%     \end{proof}
% }


\red{Note on induced grading}

