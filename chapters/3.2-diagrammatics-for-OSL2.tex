\section{Diagrammatic $\mathcal{O}(\op{SL}(2))$}

With the diagrammatic category $\cH(S_2)$, we can describe diagrammatics for the category $\mathcal{O}(\op{SL}(2))$. In particular, we define a modular category $\mathcal{DO}(\op{SL}(2))$ with a left-action of $\cH(S_2)$. % TODO: Check the wording and notation here

The category $\mathcal{DO}(\op{SL}(2))$ has elements that are generated (\red{Define what this means.}) by the identity element $\varnothing$ of $\cH(S_2)$ and morphisms are generated by the empty diagram $\varnothing$, where $\cH(S_2)$ acts on the left by left concatenation for objects and morphisms. In addition to the relations from $\cH(S_2)$, the morphisms have one new relation in which diagrams collapse to $0$ when there are barbells on the right. To depict this we add a wall on the right of the diagram, i.e. embedding the diagrams in the one-sided strip $[0,1] \times \R_{\geq 0}$ instead of in the double-sided strip $[0,1] \times \R$. For example a morphism may be
\begin{equation*}
    \inputtikz{3.2/wall-example}.
\end{equation*}
We impose the relation that diagrams are related to the wall by
\begin{equation}
    \label{rel:barbell-wall}
    \inputtikz{3.2/relation-barbell-wall} = 0.
\end{equation}

Notice that all the morphisms in $\cH(S_2)$ appear in this modular category, although they may have been annihilated by \eqref{rel:barbell-wall}.

\begin{example}
    We use the new relation \eqref{rel:barbell-wall} to further simplify the morphism in Example \eqref{eg:one-col-relations}.
    \begin{align*}
        \inputtikz{3.2/relation-example1}
         & = 2 \inputtikz{3.2/relation-example2.1} - \inputtikz{3.2/relation-example2.2}
         \\ & = 2 \left( 2 \inputtikz{3.2/relation-example3.1} - \inputtikz{3.2/relation-example3.2} \right) - 0
         \\ & = 4 \inputtikz{3.2/relation-example3.1}
    \end{align*}
\end{example}


Since the objects and morphisms in this category are that of $\cH(S_2)$ modulo the wall relation \eqref{rel:barbell-wall}, double leaves still span the morphism spaces of $\mathcal{DO}(\op{SL}(2))$. We see these as left $\Z[\barbell]$-module bases since right multiplication by $\barbell$ annihilates the morphism and adds no new diagrams. \red{Are these basis? They might not be linearly independent e.g. double leaves with a gap means that multiplying with $\barbell$ kills it.}

\red{Only allowing $\Z$-module should make them linearly independent, and they should still span.}

% TODO: Say something here

\begin{theorem}[\red{???}]
    The diagrammatic category $\Kar(\mathcal{DO}(\op{SL}(2)))$ and $\mathcal{O}(\op{SL}(2))$ are equivalent as categories.
\end{theorem}

\red{Check all of this}
\red{Put precise references}
\begin{proof}
    As a shorthand, we write $\mathcal{DO}$ for $\mathcal{DO}(\op{SL}(2))$ and $\mathcal{O}$ for $\mathcal{O}(\op{SL}(2))$. Let $F: \mathcal{DO} \to \mathcal{O}$ be a functor that sends objects \red{???}. On morphisms, $F$ sends $\identity$ (\red{What is non-trivial strand?}) to a relevant translation functor in $\mathcal{O}$. This completely determines the action of $F$ (\red{Why?}). Due to classical results in Humphrey's book \cite{humphreys-category-O}, the translation functors are maps between Frobenius objects, so there are unit, counit, multiplication and comultiplication maps with appropriate relations in $\mathcal{O}$. These are the image of the generators \eqref{eq:one-col-gen} under $F$, that satisfy the analogous relations \eqref{eq:one-col-sbim-rel}. Also, the work of Soergel in \cite{soergel-category-O} imply that there is also a relation analogous to the barbell-wall relation \eqref{rel:barbell-wall} in $\mathcal{O}$. Thus $F$ is well defined as all the generators and relations in $\mathcal{DO}$ are accounted for (\red{Word this better}).

    Next we show that $F$ is a fully faithful functor. By results from \cite{elias-williamson-soergel-calculus} and \cite{libedinsky-lightleavesbasis}, the inclusion $\cH(S_2) \to \sbim$ is fully faithful, so we have a copy of double leaves bases in $\sbim$. By the work of Soergel in \cite{soergel-category-O}, the category $\mathcal{O}$ is a Soergel module \red{(What does this mean?)} with certain bases for the morphism. Thus (\red{Why?}) it suffices to  compare the (\red{graded??}) dimensions of the morphism spaces between $\mathcal{DO}$ and $\mathcal{O}$. [\red{Comparison?}]

    The functor $F$ mapped objects of $\mathcal{DO}$ to objects \red{???} in $\mathcal{O}$, which generate all other objects by direct sums, direct summands and grading shifts (\red{Is this right?}). Now $F$ is fully faithful, $\Kar$ preserves equivalences of categories and taking the Karoubi envelope of the image of $\mathcal{DO}$ gives exactly $\mathcal{O}$ (\red{Is this right?}), we obtain an equivalence of categories between $\Kar(\mathcal{DO})$ and $\mathcal{O}$. 
\end{proof}
