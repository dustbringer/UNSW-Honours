\section{Diagrammatic $\mathcal{O}(\op{SL}(2))$}

With the diagrammatic category $\cH(S_2)$, we can describe diagrammatics for the category $\mathcal{O}(\op{SL}(2))$. In particular, we define a modular category $\red{[what~do~we~call~this~cat?]}$ over $\cH(S_2)$. % TODO: Check the wording and notation here

This module category has elements copied from $\cH(S_2)$ and morphisms are generated by the empty diagram $\varnothing$, with $\cH(S_2)$ acting on the left by left concatenation on objects and morphisms. Additionally, the morphisms have one new relation, where diagrams collapse to $0$ when there are barbells on the right. To depict this we add a wall on the right of the diagram, i.e. embedding the diagrams in the one-sided strip $[0,1] \times \R_{\geq 0}$ instead of in the double-sided strip $[0,1] \times \R$. For example a morphism may be
\begin{equation*}
    \tikz[vcenter, scale=0.4]{
    \tikzfixsize{(0,0)}{(4,4)}
    \squarecoord
    %
    \path
    (midl2) edge[string=red] (topl)
    (midl2) edge[string=red] (mid)
    (mid) edge[string=red] (top)
    (mid) edge[string=red] (midb2)
    (midl2) edge[string=red] (botl);
    %
    \node [enddot=red] at (midb2) {};
    \wallr{4}{4}{1.5}
}
\end{equation*}
Then, diagrams are related to the wall by
\begin{equation}
    \input{tikz/3.2/relation-wall-barbell.tex} = 0.
\end{equation}

% TODO: Maybe simplify this relation to barbell or $\alpha_s$ which generate the non-constant polynomials

