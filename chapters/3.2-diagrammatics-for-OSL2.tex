\section{Diagrammatic $\mathcal{O}(\op{SL}(2))$}

We will describe one-colour diagrammatics for $\mathcal{O}(\op{SL}(2))$ via generators and relations up to isotopy. % TODO: Mention things to do with what isotopy allows

The elements of this category are generated by taking tensor products of an element $s$, coloured \red{red}.

Similar to one colour diagrammatics for $\bsbim$, the morphisms in this category are generated by horizontal concatenation, vertical concatenation, and sums of the following univalent and trivalent vertices, along with boxes where $f$ is a homogeneous polynomial in \red{??}.
\begin{align}
    \tikz[vcenter, scale=0.4]{
        \squarecoord
        %
        \path
        (bot) edge[very thick, red] (mid);
        %
        \node [enddot=red] at (mid) {};
        \draw[rounded corners, draw=black!10, very thick] (-0.25,0) rectangle (4.25,4);
    }
    \quad , \quad
    \tikz[vcenter, scale=0.4]{
        \squarecoord
        %
        \path
        (top) edge[very thick, red] (mid);
        %
        \node [enddot=red] at (mid) {};
        \draw [rounded corners, draw=black!10, very thick] (-0.25,0) rectangle (4.25,4);
    }
    \quad , \quad
    \tikz[vcenter, scale=0.4]{
        \squarecoord
        %
        \path
        (mid) edge[very thick, red] (bot)
        (mid) edge[very thick, red] (topl)
        (mid) edge[very thick, red] (topr);
        %
        \draw[rounded corners, draw=black!10, very thick] (-0.25,0) rectangle (4.25,4);
    }
    \quad , \quad
    \tikz[vcenter, scale=0.4]{
        \squarecoord
        %
        \path
        (mid) edge[very thick, red] (top)
        (mid) edge[very thick, red] (botl)
        (mid) edge[very thick, red] (botr);
        %
        \draw[rounded corners, draw=black!10, very thick] (-0.25,0) rectangle (4.25,4);
    }\quad , \quad
    \tikz[vcenter, scale=0.4]{
        \squarecoord
        %
        \node[draw, rounded corners] at (mid) {$f$};
        \draw [rounded corners, draw=black!10, very thick] (-0.25,0) rectangle (4.25,4);
    }
\end{align}
The morphisms are subject to the following local relations, up to isotopy.
\begin{subequations}
    \begin{align}
        \tikz[vcenter, scale=0.4]{
            \squarecoord
            %
            \path
            (top) edge[very thick, red] (bot)
            (mid) edge[very thick, red] (midr2);
            %
            \node [enddot=red] at (midr2) {};
            \draw[rounded corners, draw=black!10, very thick] (-0.25,0) rectangle (4.25,4);
        }
         & =
        \tikz[vcenter, scale=0.4]{
            \squarecoord
            %
            \path
            (top) edge[very thick, red] (bot);
            %
            \draw[rounded corners, draw=black!10, very thick] (-0.25,0) rectangle (4.25,4);
        }
        \left(=
        \tikz[vcenter, scale=0.4]{
                \squarecoord
                %
                \path
                (top) edge[very thick, red] (bot)
                (mid) edge[very thick, red] (midl2);
                %
                \node [enddot=red] at (midl2) {};
                \draw[rounded corners, draw=black!10, very thick] (-0.25,0) rectangle (4.25,4);
            }
        \right)
        \\
        \tikz[vcenter, scale=0.4]{
            \squarecoord
            %
            \path
            (topl) edge[very thick, red] (midt)
            (topr) edge[very thick, red] (midt)
            (botl) edge[very thick, red] (midb)
            (botr) edge[very thick, red] (midb)
            (midt) edge[very thick, red] (midb);
            %
            \draw[rounded corners, draw=black!10, very thick] (-0.25,0) rectangle (4.25,4);
        }
         & =
        \tikz[vcenter, scale=0.4]{
            \squarecoord
            %
            \path
            (topl) edge[very thick, red] (midl)
            (topr) edge[very thick, red] (midr)
            (botl) edge[very thick, red] (midl)
            (botr) edge[very thick, red] (midr)
            (midl) edge[very thick, red] (midr);
            %
            \draw[rounded corners, draw=black!10, very thick] (-0.25,0) rectangle (4.25,4);
        }
        \\
        \tikz[vcenter, scale=0.4]{
            \squarecoord
            %
            \node[draw, rounded corners] at (midl) {$f$};
            \node[draw, rounded corners] at (midr) {$g$};
            \draw [rounded corners, draw=black!10, very thick] (-0.25,0) rectangle (4.25,4);
        }
         & =
        \tikz[vcenter, scale=0.4]{
            \squarecoord
            %
            \node[draw, rounded corners] at (mid) {$fg$};
            \draw [rounded corners, draw=black!10, very thick] (-0.25,0) rectangle (4.25,4);
        }
        \\
        \tikz[vcenter, scale=0.4]{
            \squarecoord
            %
            \path
            (midt) edge[very thick, red] (midb);
            %
            \node [enddot=red] at (midt) {};
            \node [enddot=red] at (midb) {};
            \draw [rounded corners, draw=black!10, very thick] (-0.25,0) rectangle (4.25,4);
        }
         & =
        \tikz[vcenter, scale=0.4]{
            \squarecoord
            %
            \node[draw, rounded corners] at (mid) {$\alpha_\red{s}$};
            \draw [rounded corners, draw=black!10, very thick] (-0.25,0) rectangle (4.25,4);
        }
        \\
        \tikz[vcenter, scale=0.4]{
            \squarecoord
            %
            \path
            (midt2) edge[very thick, red] (top)
            (midb2) edge[very thick, red] (bot);
            \draw[very thick, red] (midt2) let \p1 = ($0.5*(midt2) - 0.5*(midb2)$) in arc (90 : 360+90 : ({veclen(\x1,\y1)}););
            %
            \draw [rounded corners, draw=black!10, very thick] (-0.25,0) rectangle (4.25,4);
        }
         & = 0
        \\
        \tikz[vcenter, scale=0.4]{
            \squarecoord
            %
            \path
            (top) edge[very thick, red] (bot);
            %
            \node[draw, rounded corners] at (midl2) {$f$};
            \draw [rounded corners, draw=black!10, very thick] (-0.25,0) rectangle (4.25,4);
        }
         & =
        \tikz[vcenter, scale=0.4]{
            \squarecoord
            %
            \path
            (top) edge[very thick, red] (bot);
            %
            \node[draw, rounded corners] at (midr3) {$\red{s}f$};
            \draw [rounded corners, draw=black!10, very thick] (-0.25,0) rectangle (4.25,4);
        } % TODO: Fix the spacing on this image
        +
        \tikz[vcenter, scale=0.4]{
            \squarecoord
            %
            \path
            (top) edge[very thick, red] (midt3)
            (bot) edge[very thick, red] (midb3);
            %
            \node [enddot=red] at (midt3) {};
            \node [enddot=red] at (midb3) {};
            \node[draw, rounded corners] at (mid) {$\partial_\red{s}f$};
            \draw [rounded corners, draw=black!10, very thick] (-0.25,0) rectangle (4.25,4);
        }
    \end{align}
\end{subequations}

Additionally, we must impose a right $R$-module relation following from the lack of a left action. Instead of embedding in the double-sided strip $[0,1] \times \R$, we embed the diagrams in the one-sided strip $[0,1] \times \R_{>0}$ \red{(Check the inequality)}, where the left side is an imaginary wall. For example a morphism may be
\begin{center}
    \tikz[vcenter, scale=0.4]{
        \squarecoord
        %
        \path
        (midr2) edge[very thick, red] (topr)
        (midr2) edge[very thick, red] (mid)
        (mid) edge[very thick, red] (top)
        (mid) edge[very thick, red] (midb2)
        (midr2) edge[very thick, red] (botr);
        %
        \node [enddot=red] at (midb2) {};
        \draw[rounded corners, draw=black!10, very thick] (-0.25,0) rectangle (4.25,4);
        \draw[draw=none, fill=black!40]
        (-0.25,0) --
        ++(1.75,0) --
        ++(0,4) {[rounded corners] --
            ++(-1.75,0) --
            cycle};
    }.
\end{center}
Then, diagrams are related to the wall by


\begin{equation}
    \tikz[vcenter, scale=0.4]{
        \squarecoord
        %
        \node[draw, rounded corners] at (midr) {$f$};
        \draw[rounded corners, draw=black!10, very thick] (-0.25,0) rectangle (4.25,4);
        \draw[draw=none, fill=black!40]
        (-0.25,0) --
        ++(1.75,0) --
        ++(0,4) {[rounded corners] --
            ++(-1.75,0) --
            cycle};
    }
    = 0
\end{equation}
where $f$ is a homogeneous polynomial in $R$ with non-zero degree. That is, if a diagram has a non-constant homogeneous polynomial on its far left, then the entire diagram dies.

% TODO: Maybe simplify this relation to barbell or $\alpha_s$ which generate the non-constant polynomials

