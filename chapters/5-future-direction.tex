\chapter{Future Direction}
\label{chapter:future-direction}

The diagrammatic descriptions we gave for $\cO(\fraksl_2)$ and $\Tilt(\fraksl_2)$ are integral and graded versions of these categories. These can be studied in contexts where the original categories cannot, for example in characteristic $p$. In higher ranks such as $\fraksl_n$, diagrammatics also exist; for example see \cite{riche-williamson-tilt-modules-p-canon-basis}.

A next step is the open problem of constructing diagrammatics for the categorification of the Lusztig--Vogan module by Larson--Romanov \cite{categorification-lusztig-vogan-module}. In the paper, we see that the category is a Soergel module category, so the diagrammatics are likely similar to those in this thesis. Particularly, we suspect the diagrammatics to be similar to $\mcal{DT}_0$ with a new generator connecting to the wall
\begin{center}
    \inputtikz{5/generator-red-wall}
\end{center}
with some unknown relations determining how this generator interacts with other diagrams.

