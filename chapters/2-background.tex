\chapter{Background}

For a category $\mcal{C}$ we write $\op{ob}(\mcal{C})$ for the collection of objects, $\op{mor}(\mcal{C})$ for the collection of all morphisms, and for any pair of objects $A,B$ we write $\Hom(A,B)$ for the collection of morphisms from $A$ to $B$. The collection of endomorphisms of an object $A$ is written $\End(A) \coloneqq \Hom(A,A)$. Our focus of study are particular types of categories, not categories in the abstract, so we may assume that all categories we encounter are locally small. % In fact most of our examples of diagrammatic categories will be small categories.


\red{Put some text here with where to find more details}

\section{Drawing Monoidal Categories}

A monoidal category $\mathcal{C}$ is a category equipped with a bifunctor $\otimes: \mathcal{C} \times \mathcal{C} \to \mathcal{C}$ and a unit object $\bbone$, such that certain associativity and unit relations hold\footnote{For more details see \cite{tensor-categories}.}. We will assume that monoidal categories are strict, since all monoidal categories are monoidally equivalent to a strict one \red{[Reference?]}. The morphisms of $C$ can be drawn as string diagrams, where we read from bottom to top and functions are labelled dots. For example
\begin{center}
    \inputtikz{2.1/string-diagram-example}
\end{center}
describes a morphism $f: a \to b \otimes c$. By convention, $\bbone$ is blank and morphisms to $\bbone$ are blank. For example 

The vertical stacking of diagrams depicts composition.
% \begin{center}
%     \inputtikz{2.1/string-diagram-example}
% \end{center}



\section{Frobenius Objects}
\label{sec:2.2}

\red{Something something about }
\red{Many relations in categorical structures can be written in diagrammatic terms - adjunctions, monoid}

\red{Something about isotopy}

Let $\mcal{C}$ be a (strict) monoidal category. We can define the following objects.

\begin{definition}
    A \textit{monoid object} in $\mcal{C}$ is a triple $(M,\mu,\eta)$ for an object $M \in \mcal{C}$, a \textit{multiplication} map $\mu: M \tensor M \to M$ and a \textit{unit} map $\eta: \mathbbb{1} \to M$, such that
    \begin{gather*}
        \begin{mytikzcd}
    & M \tensor M \tensor M \arrow[dl,"\mu \tensor \id_M"'] \arrow[dr, "\id_M \tensor \mu"] \\
    M \tensor M \arrow[dr, "\mu"']
    && M \tensor M \arrow[dl, "\mu"] \\
    & M
\end{mytikzcd}
    \end{gather*}
    and
    \begin{gather*}
        \begin{mytikzcd}
    \mathbbb{1} \tensor M
    \arrow[r, "\eta \tensor \id_M"]
    \arrow[dr, "\id_M"']
    & M \tensor M
    \arrow[d, "\mu"]
    & M \tensor \mathbbb{1}
    \arrow[l, "\id_M \tensor \eta"']
    \arrow[dl, "\id_M"]
    \\
    & M
\end{mytikzcd}
    \end{gather*}
    commute. The first diagram is the \textit{associativity} relation $\mu \circ (\mu \tensor \id_M) = \mu \circ (\id_M \tensor \mu)$ and the second diagram is the \textit{unit} relation $\id_M = \mu \circ (\eta \tensor \id_M) = \mu \circ (\id_M \tensor \eta)$.

    Dually, a \textit{comonoid object} in $\mcal{C}$ is a triple $(M,\delta,\epsilon)$ for an object $M \in \mcal{C}$, a \textit{comultiplication} map $\delta: M \to M \tensor M$ and a \textit{counit} map $\epsilon: M \to \mathbbb{1}$, satisfying the \textit{coassociativity} relation
    \begin{gather*}
        \begin{mytikzcd}[arrows=<-]
    & M \tensor M \tensor M \arrow[dl,"\delta \tensor \id_M"'] \arrow[dr, "\id_M \tensor \delta"] \\
    M \tensor M \arrow[dr, "\delta"']
    && M \tensor M \arrow[dl, "\delta"] \\
    & M
\end{mytikzcd}
    \end{gather*}
    and \textit{counit} relation
    \begin{gather*}
        \begin{mytikzcd}[arrows=<-]
    \mathbbb{1} \tensor M
    \arrow[r, "\epsilon \tensor \id_M"]
    \arrow[dr, "\id_M"']
    & M \tensor M
    \arrow[d, "\delta"]
    & M \tensor \mathbbb{1}
    \arrow[l, "\id_M \tensor \epsilon"']
    \arrow[dl, "\id_M"]
    \\
    & M
\end{mytikzcd}.
    \end{gather*}
\end{definition}

Monoid objects generalise monoids, i.e. sets with an identity equipped with an associative binary operation.

\begin{definition}
    A \textit{Frobenius object} in $\mcal{C}$ is a quintuple $(A,\mu,\eta,\delta,\epsilon)$ such that $(A,\mu,\eta)$ is a monoid object, $(A,\delta,\epsilon)$ is a comonoid object, and the maps satisfy the \textit{Frobenius relations}
    \begin{gather*}
        \begin{mytikzcd}
    & A \tensor A
    \arrow[dl, "\delta \tensor \id_A"']
    \arrow[d, "\mu"]
    \arrow[dr, "\id_A \tensor \delta"]
    \\
    A \tensor A \tensor A
    \arrow[dr, "\id_A \tensor \mu"']
    & A
    \arrow[d, "\delta"]
    & A \tensor A \tensor A
    \arrow[dl, "\mu \tensor \id_A"]
    \\
    & A \tensor A
\end{mytikzcd},
    \end{gather*}
    that is $(\id_A \tensor \mu) \circ (\delta \tensor \id_A) = \delta \circ \mu = (\mu \tensor \id_A) \circ (\id_A \tensor \delta)$.
\end{definition}


The maps and relations for a Frobenius object $(A,\mu,\eta,\delta,\epsilon)$ have a nice description with the diagrams given in \autoref{sec:2.1}. The structure maps are drawn as
\begin{gather*}
    \inputtikz{2.2/frob-multiplication}
    \quad , \quad
    \inputtikz{2.2/frob-unit}
    \quad , \quad
    \inputtikz{2.2/frob-comultiplication}
    \quad , \quad
    \inputtikz{2.2/frob-counit}.
\end{gather*}

For the rest of this section, we only work with the Frobenius object $A$ and $\mathbbb{1}$. We can stop putting the label $A$ by identifying $A$ with the identity strand $\mathsf{I} = \id_A$. Diagrammatically, the associativity relation $\mu \circ (\mu \tensor \id_M) = \mu \circ (\id_M \tensor \mu)$ is
\begin{gather*}
    \inputtikz{2.2/frob-relation-associativity1}
    =
    \inputtikz{2.2/frob-relation-associativity2},
\end{gather*}
the coassociativity relation $(\delta \tensor \id_A) \circ \delta = (\id_A \tensor \delta) \circ \delta$ is
\begin{gather*}
    \inputtikz{2.2/frob-relation-coassociativity1}
    =
    \inputtikz{2.2/frob-relation-coassociativity2},
\end{gather*}
the unit relation $\id_A = \mu \circ (\eta \tensor \id_A) = \mu \circ (\id_A \tensor \eta)$ is
\begin{gather*}
    \inputtikz{2.2/frob-relation-unit1}
    =
    \inputtikz{2.2/frob-relation-unit2}
    =
    \inputtikz{2.2/frob-relation-unit3},
\end{gather*}
the counit relation $\id_A = (\epsilon \tensor \id_A) \circ \delta = (\id_A \tensor \epsilon) \circ \delta$ is
\begin{gather*}
    \inputtikz{2.2/frob-relation-counit1}
    =
    \inputtikz{2.2/frob-relation-counit2}
    =
    \inputtikz{2.2/frob-relation-counit3},
\end{gather*}
and the Frobenius relation $(\id_A \tensor \mu) \circ (\delta \tensor \id_A) = \delta \circ \mu = (\mu \tensor \id_A) \circ (\id_A \tensor \delta)$ is
\begin{gather*}
    \inputtikz{2.2/frob-relation-frob1}
    =
    \inputtikz{2.2/frob-relation-frob2}
    =
    \inputtikz{2.2/frob-relation-frob3}.
\end{gather*}


To simplify the diagrams, we stop labelling the functions and draw the structure maps as univalent and trivalent vertices
\begin{gather*}
    \inputtikz{2.2/frob-simple-multiplication}
    \quad , \quad
    \inputtikz{2.2/frob-simple-unit}
    \quad , \quad
    \inputtikz{2.2/frob-simple-comultiplication}
    \quad , \quad
    \inputtikz{2.2/frob-simple-counit}.
\end{gather*}
We put a large dot on the unit and counit to indicate that the string stops before reaching the other end. Then the relations become
\begin{gather*}
    \inputtikz{2.2/frob-simple-relation-associativity1}
    = \inputtikz{2.2/frob-simple-relation-associativity2}
    \quad , \quad
    \inputtikz{2.2/frob-simple-relation-coassociativity1}
    = \inputtikz{2.2/frob-simple-relation-coassociativity2}, \\
    \inputtikz{2.2/frob-simple-relation-unit1}
    = \inputtikz{2.2/frob-simple-relation-unit2}
    = \inputtikz{2.2/frob-simple-relation-unit3}
    \quad , \quad
    \inputtikz{2.2/frob-simple-relation-counit1}
    = \inputtikz{2.2/frob-simple-relation-counit2}
    = \inputtikz{2.2/frob-simple-relation-counit3},
\end{gather*}
and
\begin{gather*}
    \inputtikz{2.2/frob-simple-relation-frob1}
    =
    \inputtikz{2.2/frob-simple-relation-frob2}
    =
    \inputtikz{2.2/frob-simple-relation-frob3}.
\end{gather*}

Let us write cups and caps for the diagrams
\begin{gather*}
    \inputtikz{2.2/frob-isotopy-cap1}
    \coloneqq
    \inputtikz{2.2/frob-isotopy-cap2}
    \quad , \quad
    \inputtikz{2.2/frob-isotopy-cup1}
    \coloneqq
    \inputtikz{2.2/frob-isotopy-cup2}.
\end{gather*}
Then the Frobenius object relations admit a more familiar form of (planar) isotopy by the relations
\begin{gather}
    \tag{Iso1}
    \label{rel:frob-isotopy1}
    \inputtikz{2.2/frob-isotopy-bend1}
    = \inputtikz{2.2/frob-isotopy-bend2}
    = \inputtikz{2.2/frob-isotopy-bend3},
\end{gather}
which resemble that we saw in the Temperley-Lieb-Jones category. For instance the first equality follows from the Frobenius relation and the (co)unit relations
\begin{gather*}
    \inputtikz{2.2/frob-isotopy-bend-eg1}
    = \inputtikz{2.2/frob-isotopy-bend-eg2}
    = \inputtikz{2.2/frob-isotopy-bend-eg3}.
\end{gather*}

\red{Remark on self adjointness and the above relation}

Similarly we can deduce more isotopy relations
\begin{gather}
    \tag{Iso2}\label{rel:frob-isotopy2}
    \inputtikz{2.2/frob-isotopy-univalent-rot1.1}
    = \inputtikz{2.2/frob-isotopy-univalent-rot1.2}
    = \inputtikz{2.2/frob-isotopy-univalent-rot1.3}
    \quad , \quad
    \inputtikz{2.2/frob-isotopy-univalent-rot2.1}
    = \inputtikz{2.2/frob-isotopy-univalent-rot2.2}
    = \inputtikz{2.2/frob-isotopy-univalent-rot2.3}
    \\
    \tag{Iso3}\label{rel:frob-isotopy3}
    \inputtikz{2.2/frob-isotopy-trivalent-rot1.1}
    = \inputtikz{2.2/frob-isotopy-trivalent-rot1.2}
    = \inputtikz{2.2/frob-isotopy-trivalent-rot1.3}
    \quad , \quad
    \inputtikz{2.2/frob-isotopy-trivalent-rot2.1}
    = \inputtikz{2.2/frob-isotopy-trivalent-rot2.2}
    = \inputtikz{2.2/frob-isotopy-trivalent-rot2.3}
\end{gather}
which can be thought of as ``rotating of vertices''.

\red{Rotating diagrams}
Using these maps, we can rotate diagrams by putting caps and cups on a diagram.
\begin{example}
    The unit relation can be rotated to the counit map
    \begin{align*}
        \inputtikz{2.2/example-rotation1.1}
        \leadsto
        \inputtikz{2.2/example-rotation1.2}
        & = \inputtikz{2.2/example-rotation1.3}.
    \end{align*}
    where the equality follows from \eqref{rel:frob-isotopy2}.
\end{example}

\begin{example}
    The comultiplication map can be rotated to the multiplication map
    \begin{align*}
        \inputtikz{2.2/example-rotation2.1}
        \leadsto
        \inputtikz{2.2/example-rotation2.2}
        & = \inputtikz{2.2/example-rotation2.3}
        \\ & = \inputtikz{2.2/example-rotation2.4}
        \\ & = \inputtikz{2.2/example-rotation2.5}
        \\ & = \inputtikz{2.2/example-rotation2.6}.
    \end{align*}
    where the equality follows from applying \eqref{rel:frob-isotopy3} three times and \eqref{rel:frob-isotopy1}.
\end{example}

We can therefore consider the diagrams generated by concatenations of Frobenius structure maps up to planar isotopy. That is, we let two diagrams be equal if one diagram can be continuous deformed to the other in the plane without crossing itself. This greatly simplifies calculations as we can now just use our physical intuition rather than the specific isotopy relations \eqref{rel:frob-isotopy1}-\eqref{rel:frob-isotopy3}. \red{aligns with our diagrammatic philosophy} We can also simplify our relations... \red{Fill this in}

\begin{gather}
    \inputtikz{2.2/frob-rel-to-associativity1}
    = \inputtikz{2.2/frob-rel-to-associativity2}
    = \inputtikz{2.2/frob-rel-to-associativity3}
    = \inputtikz{2.2/frob-rel-to-associativity4}
\end{gather}


\begin{gather}
    \inputtikz{2.2/relation1.1}
    = \inputtikz{2.2/relation1.2}
    \left(
    = \inputtikz{2.2/relation1.3}
    \right)
    \quad , \quad
    \inputtikz{2.2/relation2.1}
    = \inputtikz{2.2/relation2.2}
\end{gather}

From now on, when we say `up to isotopy' we mean that isotopic diagrams are equal.

\red{Talk about isotopy}


\red{Maybe something about the (diagrammatic?) category Frob, capturing the data of a frobenius object}


\section{Module Categories}
\label{sec:2.3}

\red{Something about what this is used for}

\begin{definition}
    Let $(\mcal{M}, \tensor, \mathbbb{1})$ be a (strict) monoidal category. A \textit{(left) module category} over $\mcal{M}$ is a category $\mcal{C}$ and a bifunctor $\odot: \mcal{M} \times \mcal{C} \to \mcal{C}$ such that there are natural isomorphisms $(X \tensor Y) \odot A \cong X \odot (Y \odot A)$ and $\mathbbb{1} \odot A \cong A$ for $X,Y \in \mcal{M}$ and $A \in \mcal{C}$, satisfying coherence relations analogous to those for monoidal categories (see \cite{tensor-categories}[Definition 7.1.2]). A (left) module category over $\mcal{M}$ is \textit{strict} if the natural isomorphisms above are identity natural isomorphisms, i.e. $(X \tensor Y) \odot A = X \odot (Y \odot A)$ and $\mathbbb{1} \odot A = A$. We call $\odot$ the action of $\mcal{M}$ or the module product. 
\end{definition}

In the following examples, the module action is essentially the monoidal product, which we may denote by the same symbol $\tensor$. Note that module actions may not always be an underlying monoidal product.

\begin{example}
    A monoidal category is a module category over itself, where the action is its tensor product.
\end{example}

\begin{example}
    Let $G$ be a finite group and $H \subseteq G$ a subgroup. Consider the categories of group representations $\Rep(G)$ and $\Rep(H)$ over a field $\Bbbk$. Recall that $\Rep(G)$ is a category where objects are pairs $(V, \rho)$ for $V$ a $\Bbbk$-vector space and $\rho: G \to \op{GL}(V)$ is a representation of $G$, and morphisms are equivariant maps i.e. linear maps that preserve the group action. There is a monoidal structure on $\Rep(G)$ (and similarly $\Rep(H)$) given by
    \[
        (V, \rho_V) \tensor (W, \rho_W) = (V \tensor W, \rho_{V \tensor W})
    \]
    where $V \tensor V$ is the usual tensor of vector spaces, and $\rho_{V \tensor W}$ is defined such that for $v \in V_1, w \in V_2$ and $g \in G$,
    \[
        (\rho_1 \tensor \rho_2)(g)(v \tensor w) = (\rho_1(g) v) \tensor (\rho_2(g) w)
    \]
    extended linearly. This is well defined by the universal property of tensor products. The monoidal unit is $\Bbbk$ with the trivial representation. The tensor product on morphisms $f$ and $g$ is defined by component-wise application, which is equivariant by equivariance of $f$ and $g$.

    We have that $\Rep(H)$ is a left module category over $\Rep(G)$ with the following action. For an object $(V,\rho)$ in $\Rep(G)$, we can consider it as a representation over $H$ by the restriction
    \[
        \rho|_H: H \hookrightarrow G \xto{\rho} \op{GL}(V).
    \]
    The left action of $(V, \rho)$ is the left tensor of $(V,\rho|_H)$ in $\Rep(H)$. On morphisms we apply a similar restriction of equivariant maps.
\end{example}

\section{Additive Karoubi Envelope}
\label{sec:2.4}

% tilt 4.1

% module category
% generated module category
% k-linear module category & generated version

