\chapter{Background}

For a category $\mcal{C}$ we write $\op{ob}(\mcal{C})$ for the collection of objects, $\op{mor}(\mcal{C})$ for the collection of all morphisms, and for any pair of objects $A,B$ we write $\Hom(A,B)$ for the collection of morphisms from $A$ to $B$. The collection of endomorphisms of an object $A$ is written $\End(A) \coloneqq \Hom(A,A)$. Our focus of study are particular types of categories, not categories in the abstract, so we may assume that all categories we encounter are locally small. % In fact most of our examples of diagrammatic categories will be small categories.


\red{Put some text here with where to find more details}

\section{Drawing Monoidal Categories}

A monoidal category $\mathcal{C}$ is a category equipped with a bifunctor $\tensor: \mathcal{C} \times \mathcal{C} \to \mathcal{C}$ and a unit object $\mathbbb{1}$, such that certain associativity and unit relations hold\footnote{For more details see \cite{tensor-categories}.}. We will assume that monoidal categories are strict, since all monoidal categories are monoidally equivalent to a strict one \red{[Reference?]}. The morphisms of $C$ can be drawn as string diagrams. We can think of maps from bottom to top, and functions that make up the morphism are boxes. For example
\begin{center}
    \inputtikz{2.1/string-diagram-example}
\end{center}
describes a morphism $f: a \to b \tensor c$. For identity morphisms we drop the box and only draw a vertical line, so $\id_a$ is the diagram
\begin{center}
    \inputtikz{2.1/string-diagram-example-identity}.
\end{center}
The tensor product of morphisms is the horizontal concatenation of diagrams, such that strings from separate functions don't interact. For example a morphism $f \tensor g: a \tensor x \to b \tensor c \tensor y$ is drawn as
\begin{center}
    \inputtikz{2.1/string-diagram-example-tensor1}
    = 
    \inputtikz{2.1/string-diagram-example-tensor2}.
\end{center}
By convention, $\mathbbb{1}$ is blank and unlabelled, and strings that would join to $\mathbbb{1}$ are blank. Particularly, $\id_\mathbbb{1}$ is a blank diagram, and we have diagrams such as
\begin{center}
    \inputtikz{2.1/string-diagram-example-unit1}: $a \to \mathbbb{1}$
    \quad and \quad
    \inputtikz{2.1/string-diagram-example-unit2}: $\mathbbb{1} \to b \tensor c$.
\end{center}
The compositions of morphisms is the vertical stacking of diagrams where domains and codomains match. For example a composition $h \circ f: a \to b \tensor c \to a \tensor c$ has the diagram
\begin{center}
    \inputtikz{2.1/string-diagram-example-composition1}
    =
    \inputtikz{2.1/string-diagram-example-composition2}.
\end{center}




% \begin{center}
%     \inputtikz{2.1/string-diagram-example}
% \end{center}




\section{Frobenius Objects}
\label{sec:2.2}

\red{Something something about }
\red{Many relations in categorical structures can be written in diagrammatic terms - adjunctions, monoid}

\red{Something about isotopy}

Let $\mcal{C}$ be a (strict) monoidal category. We can define the following objects.

\begin{definition}
    A \textit{monoid object} in $\mcal{C}$ is a triple $(M,\mu,\eta)$ for an object $M \in \mcal{C}$, a \textit{multiplication} map $\mu: M \tensor M \to M$ and a \textit{unit} map $\eta: \mathbbb{1} \to M$, such that
    \begin{center}
        \begin{mytikzcd}
    & M \tensor M \tensor M \arrow[dl,"\mu \tensor \id_M"'] \arrow[dr, "\id_M \tensor \mu"] \\
    M \tensor M \arrow[dr, "\mu"']
    && M \tensor M \arrow[dl, "\mu"] \\
    & M
\end{mytikzcd}
    \end{center}
    and
    \begin{center}
        \begin{mytikzcd}
    \mathbbb{1} \tensor M
    \arrow[r, "\eta \tensor \id_M"]
    \arrow[dr, "\id_M"']
    & M \tensor M
    \arrow[d, "\mu"]
    & M \tensor \mathbbb{1}
    \arrow[l, "\id_M \tensor \eta"']
    \arrow[dl, "\id_M"]
    \\
    & M
\end{mytikzcd}
    \end{center}
    commute. The first diagram is the \textit{associativity} relation $\mu \circ (\mu \tensor \id_M) = \mu \circ (\id_M \tensor \mu)$ and the second diagram is the \textit{unit} relation $\id_M = \mu \circ (\eta \tensor \id_M) = \mu \circ (\id_M \tensor \eta)$.

    Dually, a \textit{comonoid object} in $\mcal{C}$ is a triple $(M,\delta,\epsilon)$ for an object $M \in \mcal{C}$, a \textit{comultiplication} map $\delta: M \to M \tensor M$ and a \textit{counit} map $\epsilon: M \to \mathbbb{1}$, satisfying the \textit{coassociativity} relation
    \begin{center}
        \begin{mytikzcd}[arrows=<-]
    & M \tensor M \tensor M \arrow[dl,"\delta \tensor \id_M"'] \arrow[dr, "\id_M \tensor \delta"] \\
    M \tensor M \arrow[dr, "\delta"']
    && M \tensor M \arrow[dl, "\delta"] \\
    & M
\end{mytikzcd}
    \end{center}
    and \textit{counit} relation
    \begin{center}
        \begin{mytikzcd}[arrows=<-]
    \mathbbb{1} \tensor M
    \arrow[r, "\epsilon \tensor \id_M"]
    \arrow[dr, "\id_M"']
    & M \tensor M
    \arrow[d, "\delta"]
    & M \tensor \mathbbb{1}
    \arrow[l, "\id_M \tensor \epsilon"']
    \arrow[dl, "\id_M"]
    \\
    & M
\end{mytikzcd}.
    \end{center}
\end{definition}

Monoid objects generalise monoids, i.e. sets with an identity equipped with an associative binary operation.

\begin{definition}
    A \textit{Frobenius object} in $\mcal{C}$ is a quintuple $(A,\mu,\eta,\delta,\epsilon)$ such that $(A,\mu,\eta)$ is a monoid object, $(A,\delta,\epsilon)$ is a comonoid object, and the maps satisfy the \textit{Frobenius relations}
    \begin{center}
        \begin{mytikzcd}
    & A \tensor A
    \arrow[dl, "\delta \tensor \id_A"']
    \arrow[d, "\mu"]
    \arrow[dr, "\id_A \tensor \delta"]
    \\
    A \tensor A \tensor A
    \arrow[dr, "\id_A \tensor \mu"']
    & A
    \arrow[d, "\delta"]
    & A \tensor A \tensor A
    \arrow[dl, "\mu \tensor \id_A"]
    \\
    & A \tensor A
\end{mytikzcd},
    \end{center}
    that is $(\id_A \tensor \mu) \circ (\delta \tensor \id_A) = \delta \circ \mu = (\mu \tensor \id_A) \circ (\id_A \tensor \delta)$.
\end{definition}


The maps and relations for a Frobenius object $(A,\mu,\eta,\delta,\epsilon)$ have a nice description with the diagrams given in \autoref{sec:2.1}. The structure maps are drawn as
\begin{center}
    \tikz[vcenter, scale=0.4]{
    % \tikzfixsize{(0,0)}{(4,4)}
    % \node[below] (dom) at (-3,0) {$A \tensor A$};
    % \node[above] (cod) at (-3,4) {$A$};
    % \path[|->, >=stealth']
    % (dom) edge node[left] {$\mu$} (cod);
    %
    \path
    (0,0) edge[string=black] (1.5,2)
    (3,0) edge[string=black] (1.5,2)
    (1.5,2) edge[string=black] (1.5,4);
    \node[box] at (1.5,2) {$\mu$};
    % labels
    \node[below] at (0,0) {$A$};
    \node[below] at (3,0) {$A$};
    \node[above] at (1.5,4) {$A$};
}
    \quad , \quad
    \tikz[vcenter, scale=0.4]{
    \tikzfixsize{(0,0)}{(2,4)}
    %
    \path
    (1,2) edge[string=black] (1,4);
    \node[box] at (1,2) {$\eta$};
    % labels
    \node[above] at (1,4) {$A$};
    \node[below] at (1,0) {\phantom{$\mathbbb{1}$}};
}
    \quad , \quad
    \tikz[vcenter, scale=0.4]{
    % \tikzfixsize{(0,0)}{(4,4)}
    %
    \path
    (0,4) edge[string=black] (1.5,2)
    (3,4) edge[string=black] (1.5,2)
    (1.5,2) edge[string=black] (1.5,0);
    \node[box] at (1.5,2) {$\delta$};
    % labels
    \node[below] at (1.5,0) {$A$};
    \node[above] at (0,4) {$A$};
    \node[above] at (3,4) {$A$};
}
    \quad , \quad
    \tikz[vcenter, scale=0.4]{
    \tikzfixsize{(0,0)}{(2,4)}
    %
    \path
    (1,2) edge[string=black] (1,0);
    \node[box] at (1,2) {$\epsilon$};
    % labels
    \node[below] at (1,0) {$A$};
    \node[above] at (1,4) {\phantom{$\mathbbb{1}$}};
}.
\end{center}

For the rest of this section, we only work with the Frobenius object $A$ and $\mathbbb{1}$. We can stop putting the label $A$ by identifying $A$ with the identity strand $\mathsf{I} = \id_A$. Diagrammatically, the associativity relation $\mu \circ (\mu \tensor \id_M) = \mu \circ (\id_M \tensor \mu)$ is
\begin{center}
    \tikz[vcenter, scale=0.3]{
    \tikzfixsize{(0,0)}{(6+.5,8)}
    %
    \path
    (0,0) edge[string=black] (2,2)
    (4,0) edge[string=black] (2,2)
    (2,2) edge[string=black] (2,4);
    \node[box] at (2,2) {$\mu$};
    \path
    (2,4) edge[string=black, round] (4,6)
    (6,4) edge[string=black, round] (4,6)
    (4,6) edge[string=black] (4,8);
    \node[box] at (4,6) {$\mu$};
    \path (6,0) edge[string=black] (6,4);
}
    =
    \tikz[vcenter, scale=0.3]{
    \tikzfixsize{(0-.5,0)}{(6,8)}
    %
    \path (0,0) edge[string=black] (0,4);
    \path
    (6,0) edge[string=black] (4,2)
    (2,0) edge[string=black] (4,2)
    (4,2) edge[string=black] (4,4);
    \node[box] at (4,2) {$\mu$};
    \path
    (4,4) edge[string=black, round] (2,6)
    (0,4) edge[string=black, round] (2,6)
    (2,6) edge[string=black] (2,8);
    \node[box] at (2,6) {$\mu$};
},
\end{center}
the coassociativity relation $(\delta \tensor \id_A) \circ \delta = (\id_A \tensor \delta) \circ \delta$ is
\begin{center}
    \tikz[vcenter, scale=0.3]{
    \tikzfixsize{(0,0)}{(6+.5,8)}
    %
    \path
    (0,8) edge[string=black] (2,6)
    (4,8) edge[string=black] (2,6)
    (2,6) edge[string=black] (2,4);
    \node[box] at (2,6) {$\delta$};
    \path
    (2,4) edge[string=black, round] (4,2)
    (6,4) edge[string=black, round] (4,2)
    (4,2) edge[string=black] (4,0);
    \node[box] at (4,2) {$\delta$};
    \path (6,8) edge[string=black] (6,4);
}
    =
    \tikz[vcenter, scale=0.3]{
    \tikzfixsize{(0-.5,0)}{(6,8)}
    %
    \path (0,8) edge[string=black] (0,4);
    \path
    (6,8) edge[string=black] (4,6)
    (2,8) edge[string=black] (4,6)
    (4,6) edge[string=black] (4,4);
    \node[box] at (4,6) {$\delta$};
    \path
    (4,4) edge[string=black, round] (2,2)
    (0,4) edge[string=black, round] (2,2)
    (2,2) edge[string=black] (2,0);
    \node[box] at (2,2) {$\delta$};
},
\end{center}
the unit relation $\id_A = \mu \circ (\eta \tensor \id_A) = \mu \circ (\id_A \tensor \eta)$ is
\begin{center}
    \tikz[vcenter, scale=0.3]{
    \tikzfixsize{(0,0)}{(2,8)}
    %
    \path
    (1,0) edge[string=black] (1,8);
}
    =
    \tikz[vcenter, scale=0.3]{
    \tikzfixsize{(0-.5,0)}{(4+.5,8)}
    %
    \path (4,0) edge[string=black] (4,4);
    \path
    (0,2) edge[string=black] (0,4);
    \node[box] at (0,2) {$\eta$};
    \path
    (0,4) edge[string=black, round] (2,6)
    (4,4) edge[string=black, round] (2,6)
    (2,6) edge[string=black] (2,8);
    \node[box] at (2,6) {$\mu$};
}
    =
    \tikz[vcenter, scale=0.3]{
    \tikzfixsize{(0-.5,0)}{(4,8)}
    %
    \path (0,0) edge[string=black] (0,4);
    \path
    (4,2) edge[string=black] (4,4);
    \node[box] at (4,2) {$\eta$};
    \path
    (0,4) edge[string=black, round] (2,6)
    (4,4) edge[string=black, round] (2,6)
    (2,6) edge[string=black] (2,8);
    \node[box] at (2,6) {$\mu$};
},
\end{center}
the counit relation $\id_A = (\epsilon \tensor \id_A) \circ \delta = (\id_A \tensor \epsilon) \circ \delta$ is
\begin{center}
    \tikz[vcenter, scale=0.3]{
    \tikzfixsize{(0,0)}{(2,8)}
    %
    \path
    (1,0) edge[string=black] (1,8);
}
    =
    \tikz[vcenter, scale=0.3]{
    \tikzfixsize{(0-.5,0)}{(4+.5,8)}
    %
    \path (4,8) edge[string=black] (4,4);
    \path
    (0,6) edge[string=black] (0,4);
    \node[box] at (0,6) {$\epsilon$};
    \path
    (0,4) edge[string=black, round] (2,2)
    (4,4) edge[string=black, round] (2,2)
    (2,2) edge[string=black] (2,0);
    \node[box] at (2,2) {$\delta$};
}
    =
    \tikz[vcenter, scale=0.3]{
    \tikzfixsize{(0-.5,0)}{(4+.5,8)}
    %
    \path (0,8) edge[string=black] (0,4);
    \path
    (4,6) edge[string=black] (4,4);
    \node[box] at (4,6) {$\epsilon$};
    \path
    (0,4) edge[string=black, round] (2,2)
    (4,4) edge[string=black, round] (2,2)
    (2,2) edge[string=black] (2,0);
    \node[box] at (2,2) {$\delta$};
},
\end{center}
and the Frobenius relation $(\id_A \tensor \mu) \circ (\delta \tensor \id_A) = \delta \circ \mu = (\mu \tensor \id_A) \circ (\id_A \tensor \delta)$ is
\begin{center}
    \tikz[vcenter, scale=0.35]{
    \tikzfixsize{(0,0)}{(6,6)}
    %
    \path (6,0) edge[string=black] (6,2);
    \path
    (0,4) edge[string=black] (2,2)
    (4,4) edge[string=black] (2,2)
    (2,2) edge[string=black] (2,0);
    \node[box] at (2,2) {$\delta$};
    %
    \path (0,4) edge[string=black] (0,6);
    \path
    % (2,2) edge[string=black] (4,4)
    (6,2) edge[string=black] (4,4)
    (4,4) edge[string=black] (4,6);
    \node[box] at (4,4) {$\mu$};
}
    =
    \tikz[vcenter, scale=0.35]{
    \tikzfixsize{(0-.5,0)}{(4+.5,7)}
    %
    \path
    (0,0) edge[string=black] (2,2)
    (4,0) edge[string=black] (2,2)
    (2,2) edge[string=black] (2,5)
    (0,7) edge[string=black] (2,5)
    (4,7) edge[string=black] (2,5)
    ;
    \node[box] at (2,2) {$\mu$};
    \node[box] at (2,5) {$\delta$};
}
    =
    \tikz[vcenter, scale=0.35]{
    \tikzfixsize{(0,0)}{(6,6)}
    %
    \path (6,6) edge[string=black] (6,4);
    \path
    (0,2) edge[string=black] (2,4)
    (4,2) edge[string=black] (2,4)
    (2,4) edge[string=black] (2,6);
    \node[box] at (2,4) {$\mu$};
    %
    \path (0,2) edge[string=black] (0,0);
    \path
    % (2,4) edge[string=black] (4,2)
    (6,4) edge[string=black] (4,2)
    (4,2) edge[string=black] (4,0);
    \node[box] at (4,2) {$\delta$};
}.
\end{center}


If we stop labelling the functions and draw the structure maps as
\begin{center}
    \tikz[vcenter, scale=0.4]{
    % \tikzfixsize{(0,0)}{(4,4)}
    % 
    \path
    (0,0) edge[string=black] (2,2)
    (4,0) edge[string=black] (2,2)
    (2,2) edge[string=black] (2,4);
}
    \quad , \quad
    \tikz[vcenter, scale=0.4]{
    \tikzfixsize{(0,0)}{(2,4)}
    %
    \path
    (1,2) edge[string=black] (1,4);
    \node[enddot=black] at (1,2) {};
}
    \quad , \quad
    \tikz[vcenter, scale=0.4]{
    % \tikzfixsize{(0,0)}{(4,4)}
    %
    \path
    (0,4) edge[string=black] (2,2)
    (4,4) edge[string=black] (2,2)
    (2,2) edge[string=black] (2,0);
}
    \quad , \quad
    \tikz[vcenter, scale=0.4]{
    \tikzfixsize{(0,0)}{(2,4)}
    %
    \path
    (1,2) edge[string=black] (1,0);
    \node[enddot=black] at (1,2) {};
},
\end{center}
then the relations become...






\red{Talk about isotopy}


\red{Maybe something about the (diagrammatic?) category Frob, capturing the data of a frobenius object}


\section{Module Categories}
\label{sec:module-cat}

\red{Something about what this is used for}

\red{Something about where this is seen}

\begin{definition}
    Let $(\mcal{M}, \tensor, \mathbbb{1})$ be a (strict) monoidal category. A \textit{(left) module category over $\mcal{M}$} or \textit{$\mcal{M}$-module category} is a category $\mcal{C}$ and a bifunctor $\odot: \mcal{M} \times \mcal{C} \to \mcal{C}$ such that there are natural isomorphisms $(X \tensor Y) \odot A \cong X \odot (Y \odot A)$ and $\mathbbb{1} \odot A \cong A$ for $X,Y \in \mcal{M}$ and $A \in \mcal{C}$, satisfying coherence relations analogous to those for monoidal categories (see \cite[Definition 7.1.2]{tensor-categories}). A (left) $\mcal{M}$-module category is \textit{strict} if the natural isomorphisms above are identity natural isomorphisms, i.e. $(X \tensor Y) \odot A = X \odot (Y \odot A)$ and $\mathbbb{1} \odot A = A$. We call $\odot$ the \textit{action of $\mcal{M}$} or \textit{the module product}. 
\end{definition}

In the following examples, the module action is essentially the monoidal product, which we may denote by the same symbol $\tensor$. Note that module actions may not always be an underlying monoidal product.

\begin{example}
    A monoidal category is a module category over itself, where the action is its tensor product.
\end{example}

\begin{example}
    Let $G$ be a finite group and $H \subseteq G$ a subgroup. Consider the categories of group representations $\Rep(G)$ and $\Rep(H)$ over a field $\Bbbk$. Recall that $\Rep(G)$ is a category where objects are pairs $(V, \rho)$ for $V$ a $\Bbbk$-vector space and $\rho: G \to \op{GL}(V)$ is a representation of $G$, and morphisms are equivariant maps i.e. linear maps that preserve the group action. There is a monoidal structure on $\Rep(G)$ (and similarly $\Rep(H)$) given by
    \[
        (V, \rho_V) \tensor (W, \rho_W) = (V \tensor W, \rho_{V \tensor W})
    \]
    where $V \tensor V$ is the usual tensor of vector spaces, and $\rho_{V \tensor W}$ is defined such that for $v \in V_1, w \in V_2$ and $g \in G$,
    \[
        (\rho_1 \tensor \rho_2)(g)(v \tensor w) = (\rho_1(g) v) \tensor (\rho_2(g) w)
    \]
    extended linearly. This is well defined by the universal property of tensor products. The monoidal unit is $\Bbbk$ with the trivial representation. The tensor product on morphisms $f$ and $g$ is defined by component-wise application, which is equivariant by equivariance of $f$ and $g$.

    We have that $\Rep(H)$ is a left module category over $\Rep(G)$ with the following action. For an object $(V,\rho)$ in $\Rep(G)$, we can consider it as a representation over $H$ by the restriction
    \[
        \rho|_H: H \hookrightarrow G \xto{\rho} \op{GL}(V).
    \]
    The left action of $(V, \rho)$ is the left tensor of $(V,\rho|_H)$ in $\Rep(H)$. On morphisms we apply a similar restriction of equivariant maps.
\end{example}

\begin{definition}
    A (strict) module category $\mcal{C}$ over a monoidal category $\mcal{M}$ is \textit{generated} by finite set $S_o$ of objects and $S_m$ of morphisms in $\mcal{C}$, when all non-unit objects are of the form $X \odot A$ for $X \in \mcal{M}$ and $A \in S_o$, and non-identity morphisms in $\mcal{C}$ are defined similarly.
\end{definition}

\begin{definition}
    Let $\mcal{M}$ be a (strict) $R$-linear monoidal category, and $\mcal{C}$ be a (strict) module category over $\mcal{M}$. We say that $\mcal{C}$ is a \textit{(strict) $R$-linear module category} if $\odot$ is $R$-bilinear on morphisms.
\end{definition}


\section{Additive Karoubi Envelope}
\label{sec:additive-karoubi}

\red{Something about why this is needed}

\red{Mention that this technical and mostly a formal process}

\subsection*{Additive and Karoubian Categories}
% tilt 4.1

\begin{definition}
    A \textit{preadditive category} is a category enriched over the category of abelian groups. That is, for objects $A$ and $B$, $\Hom(A,B)$ has the structure of an abelian group and the composition of morphisms is bilinear (over the abelian group operation).
\end{definition}

\begin{remark}
    In particular $R$-linear categories are preadditive because $R$-modules are defined over abelian groups.
\end{remark}

\begin{definition}
    A \textit{biproduct} of objects of a category is both a product and a coproduct.
    An \textit{additive category} is a preadditive category that admits all finite biproducts.
\end{definition}

Biproducts are a generalisation of direct sums of modules, so we often write $\oplus$ and say ``direct sum''. In other words, additive categories are preadditive categories containing all direct sums.

\begin{definition}
    An \textit{idempotent} is a endomorphism $e$ such that $e \circ e = e$.
    We say that a preadditive category is \textit{Karoubian} or \textit{idempotent complete} if for every idempotent $e: X \to X$ there is a direct sum decomposition $X \cong Y \oplus Z$ such that $e$ is a projection onto $Y$.
\end{definition}

This is a formal way to say that a category contains all direct sums, as every direct summand is an image of an idempotent given by projection.




\subsection*{Additive Closure and Karoubi Envelope}

We can formally add direct sums and direct summands into a preadditive category, by the additive closure and the Karoubi envelope.

\begin{definition}
    Let $\mcal{C}$ be a preadditive category. The \textit{additive closure} $\mcal{C}^\oplus$ of $\mcal{C}$ is the category where objects are finite (possibly empty) formal direct sums $\bigoplus_{i=1}^n A_i$ for $A_i \in \op{ob}(\mcal{C})$. We call the empty direct sum the \textit{zero object} $0$. A morphism $f$ of $\Hom_{\mcal{C}^\oplus}(\bigoplus_{i=1}^n A_i, \bigoplus_{i=1}^m B_i)$ is an $m \times n$ matrix $f = (f_{j,i})$ of morphisms $f_{j,i} \in \Hom_\mcal{C}(A_i, B_j)$.
\end{definition}

% The additive closure is sometimes denoted by $\cat{Mat}(\mcal{C})$.
It is clear that $\mcal{C}$ is a category that embeds in $\mcal{C}^\oplus$ and that $\mcal{C}^\oplus$ is additive. In the case where $\mcal{C}$ is monoidal, $\mcal{C}^\oplus$ is monoidal by extending the monoidal product to be an additive functor in each input. If $\mcal{C}$ is $R$-linear, then $\mcal{C}$ is an $R$-linear category by assuming that the $R$-action on morphisms applies componentwise. If $\mcal{C}$ is a $\mcal{M}$-module category, then $\mcal{C}$ is a $\mcal{M}$-module category by additionally assuming that the module action applies componentwise.

\begin{lemma}
    The additive closure satisfies the following universal property. For every preadditive functor $F: \mcal{C} \to \mcal{D}$ where $\mcal{D}$ is an additive category, there is a unique additive functor $F': \mcal{C}^\oplus \to \mcal{D}$ such that the composition $\mcal{C} \hookrightarrow \mcal{C}^\oplus \xto{F'} \mcal{D}$ is $F$.
\end{lemma}

This is a classical result so we will not provide a proof. It can be observed by extending $F$ to a functor $F^\oplus: \mcal{C}^\oplus \to \mcal{D}^\oplus$ defined componentwise with $F$.

\begin{definition}
    Let $\mcal{C}$ be a category. The \textit{Karoubi envelope} $\op{Kar}(\mcal{C})$ of $\mcal{C}$ is the category where objects are ordered pairs $(A,e)$ for an object $A$ in $\mcal{C}$ and an idempotent $e \in \End_\mcal{C}(A)$.  Morphisms $f: (A, e) \to (A', e')$ are morphisms $f:A \to A'$ in $\mcal{C}$ such that $f = f \circ e = e' \circ f$, where composition is composition in $\mcal{C}$. Equivalently, morphisms $f: (A, e) \to (A', e')$ are of the form $e'\circ f \circ e$ for some (not necessarily unique) morphism $f: A \to A'$. The identity morphism on $(A,e)$ is $e$.
\end{definition}

The objects $(A,e)$ should be seen as ``the image of $e$''. This is sometimes called the \textit{Karoubian closure} or \textit{idempotent completion}. The \textit{additive Karoubi envelope} of a category $\mcal{C}$ is $\op{Kar}(\mcal{C}^\oplus)$ which we may denote $\Kar(\mcal{C})$.

\begin{proposition}
    The Karoubi envelope $\op{Kar}(\mcal{C})$ is Karoubian.
\end{proposition}

A proof can be found at \cite[Lemma 11.17]{intro-soergel-bimodules}.

\begin{lemma}
    Every functor $F: \mcal{C} \to \mcal{D}$ where $\mcal{D}$ is Karoubian, extends uniquely (up to isomorphism) to a functor $F': \op{Kar}(\mcal{C}) \to \mcal{D}$.
\end{lemma}

This is another classical result. See \cite[Proposition 6.5.9 (1)]{borceux-categorical-algebra} for a proof.

\red{What happens when we do this on monoidal, $R$-linear or additive categories? Are the structures preserved?}

\red{Why are these useful?}

\red{Talk about diagrammatics, $\mcal{C}^\oplus$ is easy to describe diagrammatically (just matrices of diagrams) but $\op{Kar}$ is not easy to describe in general (we need to find idempotents and put them before and after a diagram)}



% module category
% generated module category
% k-linear module category & generated version

