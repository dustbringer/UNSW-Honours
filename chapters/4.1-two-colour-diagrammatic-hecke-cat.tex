The previous chapter had its focus on the symmetric group generated by one element $S_2$, which brought about one-colour diagrammatics. We shift our attention to a more complex example by adding an extra generator, that is, another colour. In particular, we consider the case for the affine symmetric group on two elements $\wtilde{S}_2 = \angl{s,t \mid s^2 = t^2 = 1}$.

\section{Two-colour Diagrammatic Hecke Category}

Corresponding to $\wtilde{S}_2$, we define the two-colour (diagrammatic) Hecke category $\cH(\wtilde{S}_2)$. This is a (strict) $\C$-linear monoidal category given by the following isotopy presentation.

Objects in $\cH(\wtilde{S}_2)$ are generated by formal tensor products of the non-identity elements $s,t \in \wtilde{S}_2$. As before, we write objects as words such as $sstttst \eqqcolon s^2t^3st$ where the tensor product is concatenation, and associate the colour \dRed{red} to $s$ and \dBlue{blue} to $t$. The empty word is the monoidal identity, which we write as $\vn$.

The morphisms are generated by the univalent and trivalent vertices
\begin{align} \label{eq:two-col-gen}
    \inputtikz{4.1/generator1}
    \quad , \quad
    \inputtikz{4.1/generator2}
    \quad , \quad
    \inputtikz{4.1/generator3}
    \quad , \quad
    \inputtikz{4.1/generator4}
\end{align}
that are maps $s \to \vn$, $ss \to s$, $t \to \vn$ and $tt \to t$ respectively. As in the one-colour case, tensor product is horizontal concatenation, composition is appropriate vertical stacking, and we denote the empty diagram $\vn \to \vn$ by $\vn$. For each colour, these diagrams have the one-colour relations given by \eqref{eq:one-col-hecke-rel}. Since we have two colours now, we also need to describe how the colours interact. This is given by the \textit{two-colour} relations

\begin{equation} \label{rel:two-col-barbell-forcing}
    % Red barbell, blue line
    \inputtikz{4.1/relation-barbell-line1}
    = \inputtikz{4.1/relation-barbell-line2}
    \quad \text{ and } \quad
    % Blue barbell, red line
    \inputtikz{4.1/relation-barbell-line3}
    = \inputtikz{4.1/relation-barbell-line4}.
\end{equation}

\begin{example}
    The following morphism in $\Hom(ttsts, tst)$ can be simplified using the one-colour and two-colour relations.
    \begin{align*}
        \inputtikz{4.1/relation-example1}
         & = \inputtikz{4.1/relation-example2}
        \\ & = 2 \inputtikz{4.1/relation-example3.1} - \inputtikz{4.1/relation-example3.2}
    \end{align*}
\end{example}


\begin{remark}
    Notice that the red and blue lines never cross as no generators that allow crossings. This is a consequence of working over affine $S_2$ in which the generators $s$ and $t$ have no relation. \red{Mention example of crossing and $S_3$.}
\end{remark}

% This stuff needs to be cleaned up

\begin{definition}
    For a group with a presentation in terms of generators and relations, the \textit{length} of a product of generators is the number of generators in the product. We say that a product of generators is \textit{reduced} if it's length cannot be shortened with relations.
\end{definition}

In $\wtilde{S}_2$ products can be shortened by the relation $s^2 = t^2 = 1$. For instance, $sttsts$ is not reduced because it is equal to $ts$ which is reduced. Notice that for $\wtilde{S}_2$ each element can be written uniquely as a reduced product of generators. This is true since otherwise we have two distinct reduced products for the same element in $\wtilde{S}_2$ so they must be related by $s^2 = t^2$. This means they can be reduced further by $s^2 = t^2 = 1$, which contradicts minimality of their length.

Notice that there is a notational similarity between products in the group and words in $\cH(\wtilde{S}_2)$. This motivates the following definitions. Let $\phi: (\op{ob}(\cH(\wtilde{S}_2)), \tensor) \to (\wtilde{S}_2, *)$ be the monoid homomorphism mapping $\vn \mapsto 1$, $s \mapsto s$ and $t \mapsto t$. Also define $\psi: \wtilde{S}_2 \to \op{ob}(\cH(\wtilde{S}_2))$ to be the function mapping elements $x \in \wtilde{S}_2$ to the tensor product of $s$ and $t$ in $\cH(\wtilde{S}_2)$ corresponding to the reduced product of $x$ in $\wtilde{S}_2$. This is well defined because reduced products are unique and two different reduced products cannot equal the same element of $\wtilde{S}_2$. The composition $\psi\phi: \cH(\wtilde{S}_2) \to \cH(\wtilde{S}_2)$ maps words $w$ to the tensor of $s$ and $t$ corresponding to the reduced product of $\phi(w)$, and $\phi\psi$ is the identity map on $\wtilde{S}_2$.

The following definition is a more general version of \autoref{def:subexpression-S2}.
\begin{definition}[Subexpression]
    Given a word $w$ of length $n$, a \textit{subexpression} $e$ is a binary string of length $n$. A subexpression can be \textit{applied} to produce an word $w(e)$, which is $w$ where terms corresponding to $0$ in $e$ are replaced with $\vn$. For $1 \leq i \leq n$, we write $w(e,i)$ for the result of the first $i$ terms of $e$ applied to the first $i$ terms in $w$. Particularly $w(e,0) = \vn$ and $w(e,n) = w(e)$.
\end{definition}

For example, in $\cH(\wtilde{S}_2)$, if $w = sttts$ and $e=11001$ then $w(e) = st \vn \vn s = sts$ and $w(e,3) = sts(110) = st\vn = st$ in $\cH(\wtilde{S}_2)$.

Let the \textit{length} of a word be the number of generators in its tensor product. As before, given an object $w$ and a subexpression $e$ of $w$, we label each of the $n$ terms by one of $U_0, U_1, D_0, D_1$. Let $i \geq 0$, and write $x$ for the $i$-th term of $w$. We label the $i$-th term $U_*$ if $\psi\phi(w(e,i-1) \tensor x)$ is longer than $\psi\phi(w(e,i-1))$. In other words we write $U_*$ if the next term of $w$ will make $\psi\phi$ applied to the partially evaluated subexpression longer, regardless of the $i$-term of $e$. We label $D_*$ if $\psi\phi(w(e,i-1) \tensor x)$ is longer than $\psi\phi(w(e,i-1))$. The label's subscript is the $i$-th term of $e$. Note that this construction is well defined because $\psi\phi(w(e,i-1) \tensor x) = \psi(\phi(w(e,i-1)) * \phi(x)) = \psi(\phi(w(e,i-1)) * x)$ is always either longer or shorter, since the last element of the reduced product is either the same as $x$ or different. When they are the same, the word is shorter via $s^2 = t^2 = 1$, and when they are different it is longer as no relations can make it shorter.

% Alternate, easier definition
% If last term in $\psi\phi(w(e,i-1))$ is the same as the $i$-th term in $w$, then $U_*$, and otherwise if they are different we label $D_*$.

\begin{remark}
    This description of the labels (via. reduced products) is more akin to the definition for general Coxeter groups than in \autoref{sec:3.1}.
\end{remark}

\begin{example}
    Consider the word $w = sttts$ and subexpression $e = 11001$. The labels can be constructed as in the following table.

    \begin{center}
        \begin{tabular}{ |r||l|l|l|l|l| }
            \hline
            Term $i$  & 1     & 2           & 3             & 4                 & 5                 \\ \hline
            Partial $w$ & $s$   & $st$        & $stt$         & $sttt$            & $sttts$           \\ \hline
            Partial $e$ & $1$   & $11$        & $110$         & $1100$            & $11001$           \\ \hline
            $w(e,i)$    & $s$ & $st$ & $st\vn=st$ & $st \vn\vn = st$  & $st\vn\vn s=sts$  \\ \hline
            Labels      & $U_1$ & $U_1 U_1$   & $U_1 U_1 D_0$ & $U_1 U_1 D_0 D_0$ & $U_1 U_1 D_0 D_0 U_1$ \\ \hline
        \end{tabular}
    \end{center}
\end{example}

\begin{definition}
    The \textit{light leaf} $LL_{w,e} \in \Hom(w, \psi\phi(w(e)))$ for a word $w$ and a subexpression $e$ is defined iteratively as follows. Let $LL_{\vn,\vn} = \vn$ be the empty diagram. Given appropriate subwords $w'$ and $e'$ of $w$ and $e$ respectively, and if the next terms are $x$ in $w$ and $i$ in $e$, the light leaf $LL_{w'x, e'i}$ is one of
    \begin{equation}
        \inputtikz{4.1/light-leaves-def-u0} \:,\:
        \inputtikz{4.1/light-leaves-def-u1} \:,\:
        \inputtikz{4.1/light-leaves-def-d0} \:,\:
        \inputtikz{4.1/light-leaves-def-d1}
    \end{equation}
    corresponding to the next label. The \dPurp{purple} strands represent either \dRed{red} or \dBlue{blue} corresponding to whether the next term $x$ is $s$ or $t$, respectively.
\end{definition}

Notice that the codomain of a light leaf $LL_{w,e}$ is the object $\psi\phi(w(e))$. So if the next label is $U_*$ then the codomain of $LL_{w',e'}$ does not end with the colour corresponding to $x$, and if the next label is $D_*$ the codomain of $LL_{w',e'}$ ends with a strand with the colour corresponding to $x$. This implies the recursive definition in the diagram above is consistent. Note that in the case of $D_*$, one of the black strands in the domain of $LL_{w',e'}$ must have the colour of $x$ in order for the colour to appear in its codomain.


