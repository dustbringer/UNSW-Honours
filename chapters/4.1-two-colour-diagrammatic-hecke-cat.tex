The previous chapter had its focus on the symmetric group generated by one element $S_2$, which brought about one-colour diagrammatics. We shift our attention to a more complex example by adding an extra generator, that is, another colour. In particular, we consider the case for the affine symmetric group on two elements $\wtilde{S}_2 = \angl{s,t \mid s^2 = t^2 = 1}$.

\section{Two-colour Diagrammatic Hecke Category}

Corresponding to $\wtilde{S}_2$, we define the two-colour (diagrammatic) Hecke category $\cH(\wtilde{S}_2)$. This is a (strict) $\Z$-linear monoidal category given by the following isotopy presentation.

Objects in $\cH(\wtilde{S}_2)$ are generated by formal tensor products of the non-identity elements $s,t \in \wtilde{S}_2$. As before, we write objects as words such as $sstttst \eqqcolon s^2t^3st$ where the tensor product is concatenation, and associate the colour \dRed{red} to $s$ and \dBlue{blue} to $t$. The empty word is the monoidal identity, which we write as $\vn$.

The morphisms are generated by the univalent and trivalent vertices
\begin{align} \label{eq:two-col-gen}
    \inputtikz{4.1/generator1}
    \quad , \quad
    \inputtikz{4.1/generator2}
    \quad , \quad
    \inputtikz{4.1/generator3}
    \quad , \quad
    \inputtikz{4.1/generator4}
\end{align}
that are maps $s \to \vn$, $ss \to s$, $t \to \vn$ and $tt \to t$ respectively. As in the one-colour case, tensor product is horizontal concatenation, composition is appropriate vertical stacking, and we denote the empty diagram $\vn \to \vn$ by $\vn$.



% TODO: Remark on crossings