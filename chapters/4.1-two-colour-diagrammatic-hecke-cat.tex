The previous chapter had its focus on the symmetric group generated by one element $S_2$, which brought about one-colour diagrammatics. We shift our attention to a more complex example by adding an extra generator, that is, another colour. In particular, we consider the case for the affine symmetric group on two elements $\wtilde{S}_2 = \angl{s,t \mid s^2 = t^2 = 1}$.

\section{Two-colour Diagrammatic Hecke Category}

Corresponding to $\wtilde{S}_2$, we define the two-colour (diagrammatic) Hecke category $\cH(\wtilde{S}_2)$. This is a (strict) $\C$-linear monoidal category given by the following isotopy presentation.

Objects in $\cH(\wtilde{S}_2)$ are generated by formal tensor products of the non-identity elements $s,t \in \wtilde{S}_2$. As before, we write objects as words such as $sstttst \eqqcolon s^2t^3st$ where the tensor product is concatenation, and associate the colour \dRed{red} to $s$ and \dBlue{blue} to $t$. The empty word is the monoidal identity, which we write as $\vn$.

The morphisms are generated by the univalent and trivalent vertices
\begin{align} \label{eq:two-col-gen}
    \inputtikz{4.1/generator1}
    \quad , \quad
    \inputtikz{4.1/generator2}
    \quad , \quad
    \inputtikz{4.1/generator3}
    \quad , \quad
    \inputtikz{4.1/generator4}
\end{align}
that are maps $s \to \vn$, $ss \to s$, $t \to \vn$ and $tt \to t$ respectively. As in the one-colour case, tensor product is horizontal concatenation, composition is appropriate vertical stacking, and we denote the empty diagram $\vn \to \vn$ by $\vn$. For each colour, these diagrams have the one-colour relations given by \eqref{eq:one-col-hecke-rel}. Since we have two colours now, we also need to describe how the colours interact. This is given by the \textit{two-colour} relations

\begin{equation} \label{rel:two-col-barbell-forcing}
    % Red barbell, blue line
    \inputtikz{4.1/relation-barbell-line1}
    = \inputtikz{4.1/relation-barbell-line2}
    \quad \text{ and } \quad
    % Blue barbell, red line
    \inputtikz{4.1/relation-barbell-line3}
    = \inputtikz{4.1/relation-barbell-line4}.
\end{equation}

\begin{example}
    The following morphism in $\Hom(ttsts, tst)$ can be simplified using the one-colour and two-colour relations.
    \begin{align*}
        \inputtikz{4.1/relation-example1}
         & = \inputtikz{4.1/relation-example2}
        \\ & = 2 \inputtikz{4.1/relation-example3.1} - \inputtikz{4.1/relation-example3.2}
    \end{align*}
\end{example}


\begin{remark}
    Notice that the red and blue lines never cross as no generators that allow crossings. This is a consequence of working over affine $S_2$ in which the generators $s$ and $t$ have no relation. \red{Mention example of crossing and $S_3$.}
\end{remark}


Double leaves can also be defined for this category. Let $w \in \cH(\wtilde{S}_2)$ be an object of length $n$. A subexpression of $w$ is a binary string $e$ of length $n$, which can be applied to produce an element $w(e) \in \wtilde{S}_2$. For example if $w = sstst$ and $e=11001$ then $w(e) = s * s* 1 * 1 * t = t \in \wtilde{S}_2$.

An expression of an element in $\wtilde{S}_s$ is the element written as a proejct of the generators $s$ and $t$. We say an expression is \textit{reduced} if it cannot be shortened by a relation i.e. by $s^2 = 1$ or $t^2 = 1$. For example, $sttst$ is not reduced becayse $sttsts = ssts = ts$ in $\wtilde{S}_2$, but $stst$ is reduced. Note that all reduced expressions in $\wtilde{S}_2$ are either $1$ or alternate $s$ and $t$, and is \textit{unique}. Uniqueness follows from the relations $s^2 = t^2 = 1$, where two expressions cannot be equal unless they can be reduced even further to the same expression. Define the function $\phi: \wtilde{S}_2 \to \op{ob}(\cH(\wtilde{S}_2))$ which sends an object to the tensor product corresponding to its reduced expression. For example, we have $\phi(1) = \vn$, $\phi(s) = s$, $\phi(sts) = s \tensor t \tensor s$ and $\phi(sttsts) = t \tensor s$. This is well defined because the reduced expressions is unique.

Let $w(e,i)$ be the partially evaluated reduced expression up to and including the $i$-th term. So if $w = sttsts$ and $e = 111010$, then

\begin{center}
    \begin{tabular}{ |r||l|l|l|l|l|l|l| }
        \hline
        $i$      & 0   & 1   & 2    & 3         & 4          & 5            & 6              \\ \hline
        $w(e,i)$ & $1$ & $s$ & $st$ & $stt = s$ & $stt1 = s$ & $stt1t = st$ & $stt1ts = sts$ \\ \hline
    \end{tabular}
\end{center}
Particularly if $w$ is length $n$, then $w(e,n) = w(e)$.

As before, given an object $w$ and subexpression $e$ of $w$, we can label each of the $n$ terms in $e$ by one of $U_0, U_1, D_0, D_1$. For $i \geq 1$, let the $i$-th term of $w$ be $x$. Then the $i$-th term is labelled $U_*$ if the reduced expression of $\phi(w(e,i-1) * x)$ is longer than $\phi(w(e,i-1))$. In other words we write $U_*$ if the next term of $w$ will make the evaluated subexpression longer, regardless if it is a $1$ or a $0$ in $e$. We label $D_*$ if $\phi(w(e,i-1) * x)$ is longer than $\phi(w(e,i-1))$.

\green{Note that $\phi(w(e,i-1) * x)$ is always going to be either longer or shorter, because the last element of the reduced expression is either $s$ or $t$, and the next term in $w$ is either $s$ or $t$. When they are the same, it is shorter via $s^2 = t^2 = 1$, and when they are different it is longer because no relations can make it shorter.}

\red{Maybe, instead, use an easier definition like for $S_2$. If last term in reduced expression of $\phi(w(e,i-1))$ is the same as the $i$-th term in $w$, then $U_*$, and otherwise if they are different we label $D_*$.}

\begin{remark}
    This definition of labels via. reduced expressions is more akin to the general definition for Coxeter groups.
\end{remark}

